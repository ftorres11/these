%--------------------------------------------------------------------------------------------------
\section{Object localization}
\label{sec:oc_loc}
Localization metrics are used to measure the precision of saliency maps relative to groundtruth 
bounding boxes of the foreground object of interest. These metrics originate from weakly supervised 
localization (WSOL). However, the objectives of WSOL and explaining the decision of a DNN are not 
necessarily aligned, since context may play an important role in the decision (\cite{shetty2019not}, 
\cite{rao2022towards}).

To investigate the relative importance of the object and its context, we measure classification 
metrics when using the bounding box $B$ itself as a saliency map as well as its complement 
$I \setminus B$, where $I$ is the image. We also evaluate the intersection $B \cap S$ of the 
saliency map $S$ with the bounding box and with its complement ($S \setminus B$).

As shown in \autoref{tab:localization}, the ground truth region of the object is not the only one 
responsible for the network decision. For example, the bounding box fails both when used as a 
saliency map itself and when combined with any saliency map, by harming all classification metrics. 
Even the complement is more effective than the bounding box itself, either alone or when combined. 
These findings support the hypothesis that localization metrics based on the ground truth bounding 
box are not necessarily appropriate for evaluating explanations of network decisions. 
Classification metrics are clearly more appropriate in this sense.

Nevertheless, we report localization metrics in the supplementary material. In summary, although 
its saliency maps are more spread out, Opti-CAM outperforms other methods on a number of metrics.

%--------------------------------------------------------------------------------------------------
%--------------------------------------------------------------------------------------------------
\begin{table}[t]
    \footnotesize
    \centering
    \setlength{\tabcolsep}{4pt}
    \renewcommand{\arraystretch}{0.8}
    \begin{tabular}{lccc|ccc|ccc} \toprule
    \mr{2}{\Th{Method}}                            & \mc{3}{\Th{$\AD\!\downarrow$}} & \mc{3}{\Th{$\AG\!\uparrow$}}& \mc{3}{\Th{$\AI\!\uparrow$}} \\ \cmidrule{2-10}
                                                   & {$S$} & {$B \!\cap\! S$} & {$S \!\setminus\! B$} & {$S$} & {$B \!\cap\! S$} & {$S \!\setminus\! B$}& {$S$} & {$B \!\cap\! S$} & {$S \!\setminus\! B$} \\ \midrule
    $S \defn B$                                    & 67.2 &   -- &   -- &  2.3 &   -- &   -- &  9.2 &   -- &   -- \\
    $S \defn I \setminus B$                        & 44.0 &   -- &   -- &  2.8 &   -- &   -- & 16.3 &   -- &   -- \\ \midrule
    Fake-CAM                                       &  0.5 & 67.2 & 44.1 &  0.7 &  2.3 &  2.8 & 42.0 &  9.2 & 18.9 \\ \midrule
    Grad-CAM                                       & 15.0 & 72.6 & 52.1 & 15.3 &  1.8 &  6.0 & 40.4 &  8.4 & 19.4 \\
    Grad-CAM++                                     & 16.5 & 72.9 & 53.1 & 10.6 &  1.6 &  4.1 & 35.2 &  7.3 & 17.1 \\
    Score-CAM                                      & 12.5 & 71.5 & 50.5 & 16.1 &  2.2 &  6.3 & 42.5 &  8.6 & 20.8 \\
    Ablation-CAM                                   & 15.1 & 72.8 & 52.1 & 13.5 &  1.7 &  5.6 & 39.9 &  7.8 & 19.0 \\
    XGrad-CAM                                      & 14.3 & 72.6 & 51.4 & 15.1 &  1.8 &  6.0 & 42.1 &  8.0 & 20.1 \\
    Layer-CAM                                      & 49.2 & 84.2 & 74.4 &  2.7 &  0.4 &  1.2 & 12.7 &  4.4 &  7.3 \\
    ExPerturbation                                 & 43.8 & 81.6 & 71.0 &  7.1 &  1.4 &  3.2 & 18.9 &  5.6 & 11.1 \\
    \hline
    Opti-CAM (ours)                                & \tb{1.4} & \tb{62.5} & \tb{34.8} & \tb{66.3} & \tb{8.7} & \tb{25.8} & \tb{92.5} & \tb{18.6} & \tb{47.1} \\ \bottomrule
    \end{tabular}
    \caption{\emph{Bounding box} study. Classification metrics on ImageNet validation set using VGG16. $B$: ground-truth box used by localization metrics; $I$: entire image; $S$: saliency map. $\AD$/$\AI$: average drop/increase \autocite{chattopadhay2018grad}; $\AG$: average gain (ours); $\downarrow$ / $\uparrow$: lower / higher is better; bold: best, excluding Fake-CAM.}
    %\caption{}
    \label{tab:localization}
    % \vspace{-0.2cm}
\end{table}
%--------------------------------------------------------------------------------------------------

\subsection{Weakly Supervised Approach}. 
\label{sec:oc_wsol}
Several works measure the localization ability of saliency maps, using metrics from the \emph{weakly
-supervised object localization} (WSOL) task. While we show that localization of the object and 
classifier interpretability are not well aligned as tasks, we still provide localization results. 
We use the \emph{official metric} (OM), \emph{localization error} (LE), \emph{pixel-wise $F_1$ 
score}, \emph{box accuracy} (BoxAcc) \autocite{choe2020evaluating}, standard pointing game (SP) 
\autocite{zhang2018top}, \emph{energy pointing game} (EP) \autocite{wang2020score} and 
\emph{saliency metric} (SM) \autocite{dabkowski2017real} on the ILSVRC2014\footnote{\url{
https://www.image-net.org/challenges/LSVRC/2014/index\#}} dataset. 
The goal of these metrics is to compare the saliency maps with bounding boxes around the object 
of interest. 

%--------------------------------------------------------------------------------------------------
%--------------------------------------------------------------------------------------------------
\begin{table}[H]
    \centering
    \footnotesize
    \setlength{\tabcolsep}{2.5pt}
    %\renewcommand{\arraystretch}{0.6}
    \begin{tabular}{lccc|cccc|ccc|cccc} \toprule
    \mr{2}{\Th{method}} &\mc{7}{\Th{ResNet50}} &\mc{7}{\Th{VGG16}}    \\ \cmidrule{2-15}
    & {OM$\downarrow$} & {LE$\downarrow$} & {F1$\uparrow$}&{BA$\uparrow$}& {SP$\uparrow$} & 
    {EP$\uparrow$} & {SM$\downarrow$} & {OM$\downarrow$} & {LE$\downarrow$} & {F1$\uparrow$}&
    {BA$\uparrow$}& {SP$\uparrow$} & {EP$\uparrow$} & {SM$\downarrow$} \\ \midrule
    
    Fake-CAM               &63.6&54.0&57.7&47.9&99.8&28.5&0.98
    &64.7&54.0&57.7&47.9&99.8&28.5&1.07\\ \midrule
    Grad-CAM~         &72.9&65.8&49.8&\tb{56.2}&69.8&33.3&1.30 
    &71.1&62.3&42.0&54.2&64.8&32.0&1.39\\
    Grad-CAM++     &73.1&66.1&\tb{50.4}&\tb{56.2}&69.9&33.1&1.29   
    &70.8&61.9&44.3&55.2&66.2&32.3&1.38  \\
    Score-CAM            &\tb{72.2}&64.9&49.6&54.5&68.7&32.4&\tb{1.25}   
    &71.2&62.5&\tb{45.3}&\tb{58.5}&\tb{68.2}&33.4&1.40 \\
    Ablation-CAM &72.8&65.7&50.2&56.1&69.9&33.1&1.26      
    &71.3&62.6&43.2&56.2&65.7&32.7&1.39 \\
    XGrad-CAM              &72.9&65.8&49.8&\tb{56.2}&69.8&33.3&1.30  
    &70.8&62.0&41.9&53.5&64.4&31.6&1.41 \\
    Layer-CAM &73.1&66.0&50.1&55.5&\tb{70.0}&33.0&1.29
    &70.5&61.5&28.0&54.7&65.0&32.4&1.45\\
    ExPerturbation  &73.6&66.6&37.5&44.2&64.8&\tb{38.2}&1.59
    &74.1&66.4&37.8&43.3&62.7&\tb{36.1}&1.74\\
    \midrule
    Opti-CAM &\tb{72.2}&\tb{64.8}&47.3&49.2&59.4&30.5&1.34         
    &\tb{69.1}&\tb{59.9}&44.1&51.2&61.4&30.7&\tb{1.34}   \\ 
    \bottomrule
    \end{tabular}
%--------------------------------------------------------------------------------------------------
    \caption{\textbf{Localization metrics} on ImageNet validation set. OM: \emph{official metric}; 
    LE: \emph{localization error}; F1: \emph{pixel-wise $F_1$ score}; BA: box accuracy; SP: standard 
    pointing game; EP: energy pointing game; SM: \emph{saliency metric}. $\downarrow$ / $\uparrow$: 
    lower / higher is better. Bold: best, excluding Fake-CAM.}
    \label{tab:imagenet-loc}
    % \vspace{-0.4cm}
\end{table}
%--------------------------------------------------------------------------------------------------