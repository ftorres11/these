\section{Opti-CAM}
As motivated by \textbf{motif}, we obtain a saliency map as a convex combination of feature maps
 by optimizing a given objective function with respect to the weights.
In particular, following \autocite{wang2020score}, we use channel weights $w_k \defn \softmax(
	\mathbf{u})_k$, where $\mathbf{u} \in \real^{K_\ell}$ is a variable.
We then consider saliency map $S_\ell$ in layer $\ell$ as a function of both the input image $\vx$ 
and variable $\mathbf{u}$:

\begin{equation}
    S_\ell(\vx; \mathbf{u}) \defn \sum_k \softmax(\mathbf{u})_k A^k_\ell.
\label{eq:v-sal}
\end{equation}
Comparing with \eq{sal}, $h$ is the identity mapping, because feature maps are non-negative and
 weights are positive.

%--------------------------------------------------------------------------------------------------

\subsection{Optimization}
Now, given a layer $\ell$ and a class of interest $c$, we find the vector $\mathbf{u}^*$ that
 maximizes the classifier confidence for class $c$, when the input image $\vx$ is masked according 
 to saliency map $S_\ell(\vx; \mathbf{u}^*)$:
\begin{equation}
	\mathbf{u}^* \defn \arg\max_{\mathbf{u}} F^c_\ell(\vx; \mathbf{u}),
\label{eq:opt}
\end{equation}
where we define the objective function
\begin{equation}
	F^c_\ell(\vx; \mathbf{u}) \defn g_c(f(\vx \odot n(\up(S_\ell(\vx; \mathbf{u}))))).
\label{eq:obj}
\end{equation}
Here, the saliency map $S_\ell(\vx; \mathbf{u})$ is adapted to $\vx$ exactly as in~\eq{s-cam} in 
terms of resolution and normalization. For \emph{normalization function} $n$, the default is
 \eq{norm}. 
The \emph{selector function} $g_c$ operates on the logit vector $\vy$; the default is to select the
 logit of class $c$, \ie $g_c(\vy) \defn y_c$. %Other choices, including the definition of
 %$F^c_\ell$ itself, are investigated in \autoref{sec:ablation} \redred{and in the supplementary material.}

%--------------------------------------------------------------------------------------------------
\begin{figure}[t]
    \centering
    % \fig[.8]{method/OptiCAM-1.png}
    %------------------------------------------------------------------------------
    \resizebox{\textwidth}{!}{%
    \begin{tikzpicture}[
        scale=.12,
        font={\small},
        node distance=.2,
        label distance=2pt,
        net/.style={draw,trapezium,trapezium angle=75,inner sep=3pt},
        enc/.style={net,shape border rotate=270},
        txt/.style={inner sep=3pt},
        frame/.style={draw,minimum size=1cm},
        feat/.style={frame},
        sq/.style={minimum size=.15cm},
        elem/.style={draw,sq},
        vec/.style={draw,minimum width=.8cm,minimum height=.15cm},
        var/.style={blue!60},
        B/.style={fill=blue!20},
        R/.style={fill=red!20},
        G/.style={fill=green!20},
        Y/.style={fill=yellow!40},
        P/.style={fill=black!20},
    ]
    \matrix[
        tight,row sep=0,column sep=14,
        cells={scale=.3,},
    ] {
        \&\&\&\&\&\&\&
    % 	\node[var,op] (error) {$-$}; \&
        \node[txt] (loss) {objective \\ $F^c_\ell(\vx; \mathbf{u})$}; \\
        \node[label=90:{input image $\vx$}] (in) {\figah[1.5cm]{opticam/images/idea/input}}; \&
        \node[enc] (net) {network \\ $f$}; \&
        \foreach \s/\c in {-2/B,-1/R,0/G,1/Y,2/P}
            {\node[feat,\c] (feat\s) at ($.4*(\s,-\s)$) {};}
        \node        at (feat2) {\figah[1cm]{opticam/images/idea/27_fea0}};
        \node[frame] at (feat2) {};
        \coordinate[label=90:{feature \\ maps $A^k_\ell$}]
                    (feat-north) at (feat-2.north -| feat0.north);
        \coordinate (feat-west)  at (feat-2.west  |- feat0.west);
        \coordinate (feat-east)  at (feat2.east   |- feat0.east);
        \&
        \node[var,op] (cam) {$\times$};
        \foreach \s/\c in {-2/B,-1/R,0/G,1/Y,2/P}
            {\node[elem,\c] (elem\s) at ($.6*(\s,-6)$) {};}
        \node[sq,label=-90:{weights $\mathbf{u}$}] (weight) at (elem0) {};
        \&
        \node[var,label=90:{saliency map \\ $S_\ell(\vx; \mathbf{u})$}] (sal) {\figah[1.5cm]{opticam/images/idea/saliency}}; \&
        \node[var,op] (mask) {$\odot$}; \&
        \node[label=90:{masked image}] (masked) {\figah[1.5cm]{opticam/images/idea/masked}}; \&
        \node[enc] (net2) {network \\ $f$}; \\[8]
        \&\&\&
        \coordinate (mid); \\
    };
    
    \draw[->]
        (in) edge (net)
        (net) edge (feat-west)
        (feat-east) edge (cam)
        (net2) edge node[pos=.5,right] {class \\ logits} (loss)
    % 	(net) |- node[pos=.3,left] {class \\ logits} (error)
        ;
    
    \draw[var,->]
        (weight) edge (cam)
        (cam) edge (sal)
        (sal) edge (mask)
        (mask) edge (masked)
        (masked) edge (net2)
    % 	(net2) edge (error)
    % 	(error) edge (loss)
        (net2) edge (loss)
        ;
    
    \draw[->]
        (in) |- (mid)
        (mid) -| (mask)
        ;
    
    \end{tikzpicture}
    }
% \vspace{6pt}
\caption{Overview of Opti-CAM. We are given an input image $\vx$, a fixed network $f$, a target layer $\ell$ and a class of interest $c$. We extract the feature maps from layer $\ell$ and obtain a saliency map $S_\ell(\vx; \mathbf{u})$ by forming a convex combination of the feature maps ($\times$) with weights determined by a variable vector $\mathbf{u}$~\eq{v-sal}. After upsampling and normalizing, we element-wise multiply ($\odot$) the saliency map with the input image to form a ``masked'' version of the input, which is fed to $f$. The objective function $F^c_\ell(\vx; \mathbf{u})$ measures the logit of class $c$ for the masked image~\eq{obj}. We find the value of $\mathbf{u}^*$ that maximizes this logit by optimizing along the path highlighted in blue~\eq{opt}, as well as the corresponding optimal saliency map $S_\ell(\vx; \mathbf{u}^*)$~\eq{o-sal}.}
\label{fig:idea}
\vspace{-0.4cm}
\end{figure}
Putting everything together, we define
\begin{equation}
	S^c_\ell(\vx) \defn S_\ell(\vx; \mathbf{u}^*) = S_\ell(\vx; \arg\max_{\mathbf{u}} F^c_\ell(\vx;
	\mathbf{u})),
\label{eq:o-sal}
\end{equation}
where $S_\ell$ and $F^c_\ell$ are defined by~\eq{v-sal} and~\eq{obj} respectively. The objective 
function $F^c_\ell$ ~\eq{obj} depends on variable $\mathbf{u}$ through $S_\ell$~\eq{v-sal}, where
 the feature maps $A^k_\ell = f^k_\ell(\vx)$ are fixed. Then,~\eq{obj} involves masking and a
  forward pass  through the network $f$, which is also fixed.

Figure \ref{fig:idea} is an abstract illustration of our method, \iavr{called Opti-CAM}, without 
details like upsampling and normalization~\eq{obj}. Optimization takes place along the highlighted 
path from variable $\mathbf{u}$ to objective function $F^c_\ell$. The saliency map is real-valued 
and the entire objective function is differentiable in $\mathbf{u}$. We use Adam optimizer 
\autocite{kingma2014adam} to solve the optimization problem ~\eq{opt}.


%--------------------------------------------------------------------------------------------------

%\paragraph{Discussion}

%By maximizing~\eq{obj}, the saliency map focuses on the regions contributing to class $c$, while masked regions contribute less. This way, the influence of background in the average pooling process is reduced.

%The saliency map is expressed as a linear combination of feature maps~\eq{v-sal}, with normalized weights. Hence, the saliency map is discouraged from taking up the entire image, both by the $\softmax$ competition~\eq{v-sal} and by the fact that feature maps only respond to particular locations.

%\iavr{In case $g_c(\vy) \defn y_c$,~\eq{o-sal} takes the form of direct masking~\eq{mask} with $R(\vm) = \vzero$ and
%\begin{equation}
%	\cM \defn \{ S_\ell(\vx; \mathbf{u}) : \mathbf{u} \in \real^{K_\ell} \}.
%\label{eq:mask-m}
%\end{equation}
%This constraint makes ours a CAM-based method. It dispenses the need for regularizers, because we only optimize one vector over the feature dimensions\modify{ (up to 2,048 for ResNet50), which is small compared with the dimensions of input images (50k for ImageNet)}. In addition, it does not complicate the optimization process in any way. It is only a different parametrization.}