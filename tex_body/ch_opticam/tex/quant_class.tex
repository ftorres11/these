\section{Quantitaive Evaluation}
\label{sec:oc_quant}

\subsection{Image classification}
%--------------------------------------------------------------------------------------------------
%------------------------------------------------------------------------------
\begin{table}
\centering
\footnotesize
\setlength{\tabcolsep}{4pt}
\renewcommand{\arraystretch}{0.8}
\begin{tabular}{lrrrr|rrrr} \toprule
\mr{2}{\Th{Method}}                                & \mc{4}{\Th{ResNet50}} & \mc{4}{\Th{VGG16}} \\ \cmidrule{2-9}
                                                   & {$\AD\!\downarrow$} & {$\AG\!\uparrow$} & {$\AI\!\uparrow$} & \mc{1}{T} & {$\AD\!\downarrow$} & {$\AG\!\uparrow$} & {$\AI\!\uparrow$} & \mc{1}{T} \\ \midrule
Fake-CAM                &  0.8 &  1.6 & 46.0 &  0.00 &  0.5 &  0.6 & 42.6 &  0.00 \\ \midrule
Grad-CAM                & 12.2 & 17.6 & 44.4 &  0.03 & 14.2 & 14.7 & 40.6 &  0.02 \\
Grad-CAM++              & 12.9 & 16.0 & 42.1 &  0.03 & 17.1 & 10.2 & 33.4 &  0.02 \\
Score-CAM               &  8.6 & 26.6 & 56.7 & 15.22 & 13.5 & 15.6 & 41.7 &  3.11 \\
Ablation-CAM            & 12.5 & 16.4 & 42.8 & 18.26 & 15.5 & 12.6 & 36.9 &  2.98 \\
XGrad-CAM               & 12.2 & 17.6 & 44.4 &  0.03 & 13.8 & 14.8 & 41.2 &  0.02 \\
Layer-CAM               & 15.6 & 15.0 & 38.8 &  0.08 & 48.9 &  3.1 & 13.5 &  0.07 \\
ExPerturbation          & 38.1 &  9.5 & 22.5 & 152.96 & 43.0 &  7.1 & 20.5 & 83.20 \\
\hline
Opti-CAM                & \tb{ 1.5} & \tb{68.8} & \tb{92.8} &  4.15 &  \tb{1.3} & \tb{71.2} & \tb{92.7} & 3.94 \\
\bottomrule
\end{tabular}
\caption{\emph{Classification metrics} on ImageNet validation set, using CNNs. $\AD$/$\AI$: 
average drop/increase \autocite{chattopadhay2018grad}; $\AG$: average gain (ours); 
$\downarrow$ / $\uparrow$: lower / higher is better; 
T: \iavr{Average time (sec) per batch of 8 images. Bold: best, excluding Fake-CAM.}}
\label{tab:imagenet-cnn}
% \vspace{-0.2cm}
\end{table}
%------------------------------------------------------------------------------
% FakeCAM~\citep{poppi2021revisiting}
% Grad-CAM ~\citep{selvaraju2017grad}
% Grad-CAM++~\cite{chattopadhay2018grad}
% ScoreCAM~\citep{wang2020score} 
% AblationCAM~\citep{ramaswamy2020ablation}
% XGrad-CAM~\citep{fu2020axiom} 
% LayerCAM~\citep{jiang2021layercam} 
% ExPerturbation~\citep{fong2019understanding}
%\modify{HiRes-CAM} &\modify{12.2}&\modify{17.6}&\modify{44.4}&\modify{0.03}&\modify{15.8}&\modify{13.2}&\modify{37.8}&\modify{0.02}\\

%--------------------------------------------------------------------------------------------------
\paragraph{CNN}
\autoref{tab:imagenet-cnn} shows ImageNet classification metrics using \Th{VGG16} and \Th{ResNet50}.
 Our Opti-CAM brings impressive performance in terms of average drop ($\AD$) and Average Increase 
 ($\AI$) metrics. That is, not only impressive improvement over baselines, but near-perfect: 
 near-zero $\AD$ and above 90\% $\AI$. Our new metric $\AG$ is lower, around 70\% 
 for Opti-CAM, but this is still several times higher than for all the other methods.

Interestingly, Fake-CAM \autocite{poppi2021revisiting} is the winner in terms of $\AD$ and second 
or third best in $\AI$ after Opti-CAM and Score-CAM, but fails completely $\AG$. This is expected 
and makes Fake-CAM uninteresting as it should be: By only masking one pixel, the classification 
score can hardly drop (0.8\% on ResNet50) and while it increases very often (on 46\% of images), 
the gain is as little as the drop (0.7\%). This makes the pair ($\AD$, $\AG$) sufficient as primary
 metrics and $\AI$ can be thought of as secondary, if important at all.

%In the supplementary material we report \emph{insertion} (I) and \emph{deletion} (D) metrics along with failure cases of Opti-CAM. The latter indicate that our saliency maps are not incorrect as a whole, but capturing more parts of the object, more instances or more background context results in larger or several disconnected salient regions. This does not let the classifier focus on a single discriminative region when pixels are processed sequentially by increasing saliency. Rather, I/D favor smaller and more compact saliency maps.


\autoref{tab:imagenet-cnn} also includes average execution time per image over the 1000-image 
ImageNet subset for all methods. Opti-CAM is slower than gradient-based methods that require 
only one pass through the network, but on par or faster than gradient-free methods. 
Indeed, we use a maximum of 100 iterations with one forward/backward pass per iteration, 
while Score-CAM and Ablation-CAM perform as many forward passes as channels. Hence they are much 
slower on ResNet50 than VGG16. Extremal Perturbation does not depend on the number of channels but 
is very slow by performing a complex optimization in the image space.

%--------------------------------------------------------------------------------------------------
%------------------------------------------------------------------------------
\begin{table}
    \centering
    \footnotesize
    \setlength{\tabcolsep}{4pt}
    \renewcommand{\arraystretch}{0.8}
    \begin{tabular}{lrrrr|rrrr} \toprule
        \mr{2}{\Th{Method}}& \mc{4}{\Th{ViT-B}} & \mc{4}{\Th{DeiT-B}} \\ \cmidrule{2-9}
        & {$\AD\!\downarrow$} & {$\AG\!\uparrow$} & {$\AI\!\uparrow$} & \mc{1}{T} & {$\AD\!\downarrow$} & {$\AG\!\uparrow$} & {$\AI\!\uparrow$} & \mc{1}{T} \\ \midrule
        Fake-CAM            &  0.3 &  0.4 & 48.3 &  0.00 &  0.6 &  0.3 & 44.6 &  0.00 \\ \midrule
        Grad-CAM            & 69.4 &  2.5 & 12.4 &  0.14 & 33.5 &  1.7 & 12.5 &  0.11 \\
        Grad-CAM            & 86.3 &  1.5 &  1.0 &  0.15 & 50.7 &  0.9 &  7.2 &  0.13 \\
        Score-CAM           & 32.0 &  6.2 & 33.0 & 23.69 & 53.6 &  2.2 & 12.2 & 22.47 \\
        XGrad-CAM           & 88.1 &  0.4 &  4.3 &  0.13 & 80.5 &  0.3 &  4.1 &  0.12 \\
        Layer-CAM           & 82.0 &  0.2 &  2.9 &  0.24 & 88.9 &  0.4 &  2.6 & 0.24\\
        ExPerturbation      &28.8&6.2&24.4&133.52&60.9&2.0&8.5&129.12\\
        RawAtt              & 92.6 &  0.2 &  2.8 &  0.02 & 95.3 &  0.0 &  1.8 &  0.02 \\
        Rollout             & 42.1 &  5.6 & 20.9 &  0.02 & 55.2 &  0.8 &  7.9 &  0.02 \\
        TIBAV               & 81.7 &  0.8 &  5.8 &  0.16 & 62.3 &  0.7 &  7.1 &  0.16 \\\midrule
        Opti-CAM            & \tb{ 0.6} &   \tb{18.0} & \tb{90.1} &    16.05 & \tb{ 0.9} & \tb{26.0} & \tb{83.5} &    15.17 \\ \bottomrule
    \end{tabular}
    \caption{}
    %\caption{\emph{Classification metrics} on ImageNet validation set, using transformers. $\AD$/$\AI$: average drop/increase
    %\autocite{chattopadhay2018grad}; $\AG$: average gain (ours); $\downarrow$ / $\uparrow$: lower / higher is better. \iavr{T: Average time (sec) per batch of 8 images. Bold: best, excluding Fake-CAM.}}
    \label{tab:imagenet-trans}
\end{table}
%------------------------------------------------------------------------------
%Fake-CAM~\citep{poppi2021revisiting}    &  0.3 &  0.4 & 48.3 &  0.00 &  0.6 &  0.3 & 44.6 &  0.00 \\ \midrule
%Grad-CAM~\citep{selvaraju2017grad}      & 69.4 &  2.5 & 12.4 &  0.14 & 33.5 &  1.7 & 12.5 &  0.11 \\
%Grad-CAM++~\cite{chattopadhay2018grad}  & 86.3 &  1.5 &  1.0 &  0.15 & 50.7 &  0.9 &  7.2 &  0.13 \\
%Score-CAM~\citep{wang2020score}         & 32.0 &  6.2 & 33.0 & 23.69 & 53.6 &  2.2 & 12.2 & 22.47 \\
%XGrad-CAM~\citep{fu2020axiom}           & 88.1 &  0.4 &  4.3 &  0.13 & 80.5 &  0.3 &  4.1 &  0.12 \\
%Layer-CAM~\citep{jiang2021layercam}     & 82.0 &  0.2 &  2.9 &  0.24 & 88.9 &  0.4 &  2.6 & 0.24\\
%ExPerturbation~\citep{fong2019understanding}&28.8&6.2&24.4&133.52&60.9&2.0&8.5&129.12\\
%RawAtt~\citep{dosovitskiy2020image}     & 92.6 &  0.2 &  2.8 &  0.02 & 95.3 &  0.0 &  1.8 &  0.02 \\
%Rollout~\citep{abnar2020quantifying}    & 42.1 &  5.6 & 20.9 &  0.02 & 55.2 &  0.8 &  7.9 &  0.02 \\
%TIBAV~\cite{chefer2021transformer}      & 81.7 &  0.8 &  5.8 &  0.16 & 62.3 &  0.7 &  7.1 &  0.16 \\
%\modify{HiRes-CAM~\citep{draelos2020use}} &\modify{98.4}&\modify{0.0}&\modify{0.7}&\modify{0.03}&\modify{97.2}&\modify{0.0}&\modify{1.2}&\modify{0.03} \\
%\hline
%Opti-CAM (ours)                         & \tb{ 0.6} &   \tb{18.0} & \tb{90.1} &    16.05 & \tb{ 0.9} & \tb{26.0} & \tb{83.5} &    15.17 \\ \bottomrule
%--------------------------------------------------------------------------------------------------

\paragraph{Transformers}

\autoref{tab:imagenet-trans} shows ImageNet classification metrics using ViT \iavr{and DeiT}. 
Unlike CAM-based methods that rely on a class-specific linear combination of feature maps, 
raw attention \autocite{dosovitskiy2020image} and rollout \autocite{abnar2020quantifying} use the 
attention map of the [CLS] token from the last attention block and from all blocks respectively. 
\iavr{This attention map depends only on the particular image and not on the target class, hence it
 is not really comparable. TIBAV \autocite{chefer2021transformer} uses both instance-specific and 
 class-specific information.

Opti-CAM outperforms all other methods dramatically, reaching near-zero $\AD$ and $\AI$ above 80 or 
90\%. According to our new $\AG$ metric, Opti-CAM still works while all other methods fail, 
but $\AG$ is much more conservative than $\AI$. On ViT-B for example, the classification score 
increases for 90.1\% of the images by masking with Opti-CAM, but the gain is only 18.0\% on average.}

\paragraph{More metrics}
In this section, we show additional metrics including AOPC \autocite{samek2016evaluating}, Max-Sensitivity
 \autocite{yeh2019fidelity} and ADCC \autocite{poppi2021revisiting}.

We use the code and suggested parameters of package 
Quantus\footnote{\url{https://github.com/understandable-machine-intelligence-lab/Quantus}} to measure AOPC 
and MS. In particular, patch size $14$ and number of evaluation regions $10$ for AOPC; lower bound $0.2$ 
and number of samples $10$ for MS.
For ADCC, we use the official 
code\footnote{\url{https://github.com/aimagelab/ADCC?fbclid=IwAR0YK_93lxp4pZQnt34SlA9aeNCLRX8m0u8yTZPxbTXi80qiyhTiqxWaQ7o}}.
We evaluate these metrics on ImageNet validation set using ResNet50 and VGG16. The results are 
reported in \autoref{tab:more-metrics-asked}. Since AOPC shares the same philosophy as I/D, it is 
not a surprise that Opti-CAM has poor performance on AOPC. Opti-CAM achieves the best performance on MS.


\begin{table}[]
    \centering
    \footnotesize
    \setlength{\tabcolsep}{4pt}
    \begin{tabular}{lrrr rrr} \toprule
        \mr{2}{\Th{Method}} & \mc{3}{\Th{ResNet50}} & \mc{3}{\Th{VGG16}}  \\ \cmidrule{2-7}
        & {{$AOPC\uparrow$}} & {{$MS\downarrow$}}& {{$ADCC\downarrow$}} & {{$AOPC\uparrow$}} 
        & {{$MS\downarrow$}}& {{$ADCC\downarrow$}}  \\ \midrule
        Grad-CAM            &11.7&1.05&74.3&13.1&1.10&73.7        \\
        Grad-CAM++          &11.6&1.04&73.6&11.6&1.09&74.6          \\
        Score-CAM           &10.2&1.04&61.0&11.0&1.09&73.9             \\
        XGrad-CAM           &11.9&1.05&74.3&13.1&1.10&73.9           \\
        Ablation-CAM        &11.1&1.04&71.5&12.5&1.10&75.5          \\
        Layer-CAM           &\tb{13.0}&1.22&61.1&\tb{13.3}&1.25&51.7 \\
        ExPerturbation      &12.0&1.07&\tb{26.0}&11.2&1.09&\tb{42.8}  \\\hline
        Opti-CAM (ours)     &6.3&\tb{1.03}&65.5&8.9&\tb{1.06}&70.0        \\ \bottomrule
    \end{tabular}
    \caption{}
    %\caption{\modify{\emph{AOPC/MS/ADCC} scores on ImageNet validation set.}}
    \label{tab:more-metrics-asked}
\end{table}