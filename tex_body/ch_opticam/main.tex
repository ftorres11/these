\chapter{Opti-CAM: Optimizing saliency maps for interpretability}
\chaptertoc{}
%--------------------------------------------------------------------------------------------------
\label{ch:opticam}
% Introduction, to move to a more verbatum approach
\section{Introduction}
\label{sec:intro}
\noindent Within existing attribution approaches for interpretable saliency map generation, the CAM 
\autocite{zhou2016learning} based family of methods takes special interest given its reliance of
existing  information and properties of a given model to generate explanations. In particular, we 
note that following \autoref{eq:sal}, modifying computation of the weighting coefficient 
$w_k^c$ results in a different attribution being generated. Moreover, this computation can be altered,
 for instance by relying on information found while performing the backward pass 
 (\cite{selvaraju2017grad}, \cite{chattopadhay2018grad}, \cite{axiombased}, 
 \cite{smilkov2017smoothgrad}) and the forward pass \autocite{wang2020score} of the model during 
 inference. Nevertheless, we observe that among existing weighting coefficient computation 
 proposals, none has been directed at maximizing the predicted probability of the generated 
 saliency maps.

Complementary to CAM methods, we observe that within attribution methods based on extremal 
perturbations \autocite{fong2019understanding} or IBA \autocite{schulz2020restricting}, 
their class scores are optimized via gradient descent. In this regard, it can 
be stated that these masks then become variables within input-feature space, and the aforementioned 
scores then become a function of said masking. However it is important to point that optimizing 
these masks ultimately becomes an expensive process, as several constraints are needed 
to be taken into consideration to further control the masking area.

Drawing inspiration from the aforementioned observations, we set ourselves to design \emph{Opti-
CAM}, an attribution method that generates saliency maps with enhanced interpretability. In 
particular, we hypothesize that the weighting coefficient $w_k^c$ can be optimized to attain this 
task; moreover, we also suggest that should the predicted probability of the attribution map is 
optimized, we can gain insight within the regions of the image that appear to be the most important 
for the classifier. For this, we set up the preliminaries in \autoref{sec:oc_prelim} to fully 
define our approach in \autoref{sec:oc_def}\\

\noindent In addition to the proposal of an attribution method in this chapter, we aim towards the 
design of a complementary interpretability evaluation metric of saliency maps. In particular, 
based on the remarks found in Fake-CAM \autocite{poppi2021revisiting}; where it is noted that 
existing metrics such as \gls{ad} (\ref{eq:ad}) and \gls{ai} (\ref{eq:ai}) can be manipulated. As a 
result of this, we argue that a complementary criterion is missing regarding \textit{Objective 
Evaluation for Object Recognition}. In \autoref{sec:av_gain} we define this novel measurement 
under the name \gls{ag}. To support our approach, we demonstrate our generated saliency maps in 
\autoref{sec:oc_qual} and we evaluate them in \autoref{sec:oc_quant} and \autoref{sec:oc_loc}.

\noindent To sum up, with the observations previously mentioned, in this chapter we propose a CAM 
variant that generates saliency maps by optimizing the weighting coefficient $w_k^c$, while also 
introducing a novel metric to complement the existing evaluation of attribution methods.\\

\section{Motivation}
\label{sec:oc_motiv}
From the \gls{cam} methods defined in \autoref{rel:sub_post}, we take particular interest on 
Score-CAM. In particular, Score-CAM considers each feature map as a mask in isolation. But, 
\emph{what about linear combinations?}
Given a vector $\vw \in \real^{K_\ell}$ with $w_k$ its $k$-th element, let
\begin{equation}
	F(\vw) \defn f \left( \mathbf{u} \odot n \left( \operatorname{up} \left(
		\displaystyle\sum_k w_k A^k_\ell
	\right) \right) \right)_c.
\label{eq:s-obj}
\end{equation}
If we assume that $\mathbf{u}_b = \vzero$ in~\eq{s-cam} and define $n(\vzero) \defn \vzero$ 
in~\eq{norm}, then we can rewrite the right-hand side of~\eq{s-cam} as
\begin{equation}	
	\frac{F(\vw_0 + \delta \ve_k) - F(\vw_0)}{\delta},
\label{eq:s-cam2}
\end{equation}
where $\vw_0 = \vzero$, $\delta = 1$ and $\ve_k$ is the $k$-th standard basis vector of 
$\real^{K_\ell}$. This resembles the numerical approximation of the derivative 
$\pder{F}{w_k}(\vw_0)$, except that $\delta$ is not small as usual. One could compute derivatives 
efficiently by standard backpropagation instead. It is then possible to iteratively optimize $F$ 
with respect to $\vw$, starting at any $\vw_0$.\\

\noindent As an alternative, consider masking-based methods relying on optimization in the input 
space, like \emph{meaningful perturbations} (MP) \parencite{fong2017interpretable} or \emph{extremal 
perturbations} \parencite{fong2019understanding}. In general, optimization takes the form
\begin{equation}
	S^c(\mathbf{u}) \defn \arg\max_{\vm \in \cM} f(\mathbf{u} \odot n(\operatorname{up}(\vm)))_c + \lambda R(\vm).
\label{eq:mask}
\end{equation}
Here, a mask $\vm$ is directly optimized and does not rely on feature maps, hence the saliency 
map $S^x(\mathbf{u})$ is not connected to any layer $\ell$. The mask is at the same or lower 
resolution than the input image. In the latter case, upsampling is still necessary.\\

\noindent In this approach, one indeed computes derivatives by backpropagation and iteratively 
optimizes $\vm$. However, because $\vm$ is high-dimensional, there are constraints expressed by 
$\vm \in \cM$, \eg $\vm$ has a certain norm, and regularizers like $R(\vm)$, \eg $\vm$ is smooth in 
a certain way. This makes optimization harder or more expensive and introduces more hyperparameters 
like $\lambda$. One could simply constrain $\vm$ to lie in the linear span of $\{A_\ell^k\}_{k=1}
^{K_\ell}$ instead, like all CAM-based methods.


\newpage
\section{Opti-CAM}
\label{sec:oc_def}
As motivated by \textbf{motif}, we obtain a saliency map as a convex combination of feature maps
 by optimizing a given objective function with respect to the weights.
In particular, following \autocite{wang2020score}, we use channel weights $w_k \defn \softmax(
	\mathbf{u})_k$, where $\mathbf{u} \in \real^{K_\ell}$ is a variable.
We then consider saliency map $S_\ell$ in layer $\ell$ as a function of both the input image $\vx$ 
and variable $\mathbf{u}$:

\begin{equation}
    S_\ell(\vx; \mathbf{u}) \defn \sum_k \softmax(\mathbf{u})_k A^k_\ell.
\label{eq:v-sal}
\end{equation}
Comparing with \eq{sal}, $h$ is the identity mapping, because feature maps are non-negative and
 weights are positive.

%--------------------------------------------------------------------------------------------------

\subsection{Optimization}
Now, given a layer $\ell$ and a class of interest $c$, we find the vector $\mathbf{u}^*$ that
 maximizes the classifier confidence for class $c$, when the input image $\vx$ is masked according 
 to saliency map $S_\ell(\vx; \mathbf{u}^*)$:
\begin{equation}
	\mathbf{u}^* \defn \arg\max_{\mathbf{u}} F^c_\ell(\vx; \mathbf{u}),
\label{eq:opt}
\end{equation}

where we define the objective function
\begin{equation}
	F^c_\ell(\vx; \mathbf{u}) \defn g_c(f(\vx \odot n(\up(S_\ell(\vx; \mathbf{u}))))).
\label{eq:obj}
\end{equation}
Here, the saliency map $S_\ell(\vx; \mathbf{u})$ is adapted to $\vx$ exactly as in~\eq{s-cam} in 
terms of resolution and normalization. For \emph{normalization function} $n$, the default is
 \eq{norm}. 
The \emph{selector function} $g_c$ operates on the logit vector $\vy$; the default is to select the
 logit of class $c$, \ie $g_c(\vy) \defn y_c$. %Other choices, including the definition of
 %$F^c_\ell$ itself, are investigated in \autoref{sec:ablation} \redred{and in the supplementary material.}

%--------------------------------------------------------------------------------------------------
\begin{figure}[t]
    \centering
    % \fig[.8]{method/OptiCAM-1.png}
    %------------------------------------------------------------------------------
    \resizebox{\textwidth}{!}{%
    \begin{tikzpicture}[
        scale=.12,
        font={\small},
        node distance=.2,
        label distance=2pt,
        net/.style={draw,trapezium,trapezium angle=75,inner sep=3pt},
        enc/.style={net,shape border rotate=270},
        txt/.style={inner sep=3pt},
        frame/.style={draw,minimum size=1cm},
        feat/.style={frame},
        sq/.style={minimum size=.15cm},
        elem/.style={draw,sq},
        vec/.style={draw,minimum width=.8cm,minimum height=.15cm},
        var/.style={blue!60},
        B/.style={fill=blue!20},
        R/.style={fill=red!20},
        G/.style={fill=green!20},
        Y/.style={fill=yellow!40},
        P/.style={fill=black!20},
    ]
    \matrix[
        tight,row sep=0,column sep=14,
        cells={scale=.3,},
    ] {
        \&\&\&\&\&\&\&
    % 	\node[var,op] (error) {$-$}; \&
        \node[txt] (loss) {objective \\ $F^c_\ell(\vx; \mathbf{u})$}; \\
        \node[label=90:{input image $\vx$}] (in) {\figah[1.5cm]{opticam/images/idea/input}}; \&
        \node[enc] (net) {network \\ $f$}; \&
        \foreach \s/\c in {-2/B,-1/R,0/G,1/Y,2/P}
            {\node[feat,\c] (feat\s) at ($.4*(\s,-\s)$) {};}
        \node        at (feat2) {\figah[1cm]{opticam/images/idea/27_fea0}};
        \node[frame] at (feat2) {};
        \coordinate[label=90:{feature \\ maps $A^k_\ell$}]
                    (feat-north) at (feat-2.north -| feat0.north);
        \coordinate (feat-west)  at (feat-2.west  |- feat0.west);
        \coordinate (feat-east)  at (feat2.east   |- feat0.east);
        \&
        \node[var,op] (cam) {$\times$};
        \foreach \s/\c in {-2/B,-1/R,0/G,1/Y,2/P}
            {\node[elem,\c] (elem\s) at ($.6*(\s,-6)$) {};}
        \node[sq,label=-90:{weights $\mathbf{u}$}] (weight) at (elem0) {};
        \&
        \node[var,label=90:{saliency map \\ $S_\ell(\vx; \mathbf{u})$}] (sal) {\figah[1.5cm]{opticam/images/idea/saliency}}; \&
        \node[var,op] (mask) {$\odot$}; \&
        \node[label=90:{masked image}] (masked) {\figah[1.5cm]{opticam/images/idea/masked}}; \&
        \node[enc] (net2) {network \\ $f$}; \\[8]
        \&\&\&
        \coordinate (mid); \\
    };
    
    \draw[->]
        (in) edge (net)
        (net) edge (feat-west)
        (feat-east) edge (cam)
        (net2) edge node[pos=.5,right] {class \\ logits} (loss)
    % 	(net) |- node[pos=.3,left] {class \\ logits} (error)
        ;
    
    \draw[var,->]
        (weight) edge (cam)
        (cam) edge (sal)
        (sal) edge (mask)
        (mask) edge (masked)
        (masked) edge (net2)
    % 	(net2) edge (error)
    % 	(error) edge (loss)
        (net2) edge (loss)
        ;
    
    \draw[->]
        (in) |- (mid)
        (mid) -| (mask)
        ;
    
    \end{tikzpicture}
    }
% \vspace{6pt}
\caption{Overview of Opti-CAM. We are given an input image $\vx$, a fixed network $f$, a target layer $\ell$ and a class of interest $c$. We extract the feature maps from layer $\ell$ and obtain a saliency map $S_\ell(\vx; \mathbf{u})$ by forming a convex combination of the feature maps ($\times$) with weights determined by a variable vector $\mathbf{u}$~\eq{v-sal}. After upsampling and normalizing, we element-wise multiply ($\odot$) the saliency map with the input image to form a ``masked'' version of the input, which is fed to $f$. The objective function $F^c_\ell(\vx; \mathbf{u})$ measures the logit of class $c$ for the masked image~\eq{obj}. We find the value of $\mathbf{u}^*$ that maximizes this logit by optimizing along the path highlighted in blue~\eq{opt}, as well as the corresponding optimal saliency map $S_\ell(\vx; \mathbf{u}^*)$~\eq{o-sal}.}
\label{fig:idea}
\vspace{-0.4cm}
\end{figure}
Putting everything together, we define
\begin{equation}
	S^c_\ell(\vx) \defn S_\ell(\vx; \mathbf{u}^*) = S_\ell(\vx; \arg\max_{\mathbf{u}} F^c_\ell(\vx;
	\mathbf{u})),
\label{eq:o-sal}
\end{equation}
where $S_\ell$ and $F^c_\ell$ are defined by~\eq{v-sal} and~\eq{obj} respectively. The objective 
function $F^c_\ell$ ~\eq{obj} depends on variable $\mathbf{u}$ through $S_\ell$~\eq{v-sal}, where
 the feature maps $A^k_\ell = f^k_\ell(\vx)$ are fixed. Then,~\eq{obj} involves masking and a
  forward pass  through the network $f$, which is also fixed.

Figure \ref{fig:idea} is an abstract illustration of our method, \iavr{called Opti-CAM}, without 
details like upsampling and normalization~\eq{obj}. Optimization takes place along the highlighted 
path from variable $\mathbf{u}$ to objective function $F^c_\ell$. The saliency map is real-valued 
and the entire objective function is differentiable in $\mathbf{u}$. We use Adam optimizer 
\autocite{kingma2014adam} to solve the optimization problem ~\eq{opt}.


%--------------------------------------------------------------------------------------------------

%\paragraph{Discussion}

%By maximizing~\eq{obj}, the saliency map focuses on the regions contributing to class $c$, while masked regions contribute less. This way, the influence of background in the average pooling process is reduced.

%The saliency map is expressed as a linear combination of feature maps~\eq{v-sal}, with normalized weights. Hence, the saliency map is discouraged from taking up the entire image, both by the $\softmax$ competition~\eq{v-sal} and by the fact that feature maps only respond to particular locations.

%\iavr{In case $g_c(\vy) \defn y_c$,~\eq{o-sal} takes the form of direct masking~\eq{mask} with $R(\vm) = \vzero$ and
%\begin{equation}
%	\cM \defn \{ S_\ell(\vx; \mathbf{u}) : \mathbf{u} \in \real^{K_\ell} \}.
%\label{eq:mask-m}
%\end{equation}
%This constraint makes ours a CAM-based method. It dispenses the need for regularizers, because we only optimize one vector over the feature dimensions\modify{ (up to 2,048 for ResNet50), which is small compared with the dimensions of input images (50k for ImageNet)}. In addition, it does not complicate the optimization process in any way. It is only a different parametrization.}
\newpage
%--------------------------------------------------------------------------------------------------
\section{Average Gain}
\label{sec:av_gain}
Continuing on with observations of CAM-based Saliency maps, we recall the observation made for 
\emph{Fake-CAM} \cite{poppi2021revisiting}. In particular, we note that traditional 
interpretability measurements such as \gls{ad} and \gls{ai} can be deceiving; as perfect scores can 
be nearly achieved for \gls{ad} by masking all but one pixel in the Saliency Map. This is used to 
motivate the definition of a number of metrics that are orthogonal to the task at hand, \ie 
measuring the effect of masking to the classifier. By contrast, we address the problem by 
introducing a new metric to be paired with $\AD$ as a replacement of $\AI$: 
\emph{Average Gain}.\\

\noindent \emph{\glsfirst{ag}} quantifies how much predictive power, measured as class probability; 
is gained when we mask the image. We define this metric in the following manner, where higher is 
better:
\begin{equation}
	\AG(\%) \defn \frac{1}{N} \sum_{i=1}^N \frac{[o^c_i - p^c_i]_+}{1-p^c_i} \cdot 100.
\label{eq:ag}
\end{equation}
This definition is symmetric to the definition of average drop, in the sense that in absolute value,
the numerator in the sum of $\AD, \AG$ is the positive and negative part of $p^c_i - o^c_i$ 
respectively and the denominator is the maximum value that the numerator can get as a function of 
$o^c_i$, given that $0 < o^c_i < p^c_i$ and $p^c_i < o^c_i < 1$ respectively. The two metrics thus  
compete each other, in the sense that changing $o^c_i$ to improve one leaves the other unchanged or  
harms it. As we shall see, an extreme example is Fake-CAM, which yields near-perfect $\AD$ but 
fails completely on $\AG$.


\newpage
Opti-CAM is evaluated quantitatively using classification metrics and qualitatively by visualizing
 saliency maps.
\section{Qualitative Evaluation}
\label{sec:oc_qual}

\newpage
\section{Quantitaive Evaluation}
\label{sec:oc_quant}

\subsection{Image classification}
%--------------------------------------------------------------------------------------------------
%------------------------------------------------------------------------------
\begin{table}
    \centering
    \footnotesize
    \setlength{\tabcolsep}{4pt}
    \renewcommand{\arraystretch}{0.8}
    \begin{tabular}{lrrrr|rrrr} \toprule
    \mr{2}{\Th{Method}} & \mc{4}{\Th{ResNet50}} & \mc{4}{\Th{VGG16}} \\ \cmidrule{2-9}
    & {$\AD\!\downarrow$} & {$\AG\!\uparrow$} & {$\AI\!\uparrow$} & \mc{1}{T} & {$\AD\!\downarrow$} & {$\AG\!\uparrow$} & {$\AI\!\uparrow$} & \mc{1}{T} \\ \midrule
    Fake-CAM                &  0.8 &  1.6 & 46.0 &  0.00 &  0.5 &  0.6 & 42.6 &  0.00 \\ \midrule
    Grad-CAM                & 12.2 & 17.6 & 44.4 &  0.03 & 14.2 & 14.7 & 40.6 &  0.02 \\
    Grad-CAM++              & 12.9 & 16.0 & 42.1 &  0.03 & 17.1 & 10.2 & 33.4 &  0.02 \\
    Score-CAM               &  8.6 & 26.6 & 56.7 & 15.22 & 13.5 & 15.6 & 41.7 &  3.11 \\
    Ablation-CAM            & 12.5 & 16.4 & 42.8 & 18.26 & 15.5 & 12.6 & 36.9 &  2.98 \\
    XGrad-CAM               & 12.2 & 17.6 & 44.4 &  0.03 & 13.8 & 14.8 & 41.2 &  0.02 \\
    Layer-CAM               & 15.6 & 15.0 & 38.8 &  0.08 & 48.9 &  3.1 & 13.5 &  0.07 \\
    ExPerturbation          & 38.1 &  9.5 & 22.5 & 152.96 & 43.0 &  7.1 & 20.5 & 83.20 \\\midrule
    Opti-CAM                & \tb{ 1.5} & \tb{68.8} & \tb{92.8} &  4.15 &  \tb{1.3} & \tb{71.2} & \tb{92.7} & 3.94 \\
    \bottomrule
    \end{tabular}
    \caption{}
%    \caption{\emph{Classification metrics} on ImageNet validation set, using CNNs. $\AD$/$\AI$: 
%average drop/increase \autocite{chattopadhay2018grad}; $\AG$: average gain (ours); 
%$\downarrow$ / $\uparrow$: lower / higher is better; 
%T: \iavr{Average time (sec) per batch of 8 images. Bold: best, excluding Fake-CAM.}}
    \label{tab:imagenet-cnn}
    \end{table}
%------------------------------------------------------------------------------
% FakeCAM~\citep{poppi2021revisiting}
% Grad-CAM ~\citep{selvaraju2017grad}
% Grad-CAM++~\cite{chattopadhay2018grad}
% ScoreCAM~\citep{wang2020score} 
% AblationCAM~\citep{ramaswamy2020ablation}
% XGrad-CAM~\citep{fu2020axiom} 
% LayerCAM~\citep{jiang2021layercam} 
% ExPerturbation~\citep{fong2019understanding}
%\modify{HiRes-CAM} &\modify{12.2}&\modify{17.6}&\modify{44.4}&\modify{0.03}&\modify{15.8}&\modify{13.2}&\modify{37.8}&\modify{0.02}\\

%--------------------------------------------------------------------------------------------------
\paragraph{CNN}
\autoref{tab:imagenet-cnn} shows ImageNet classification metrics using \Th{VGG16} and \Th{ResNet50}.
 Our Opti-CAM brings impressive performance in terms of average drop ($\AD$) and Average Increase 
 ($\AI$) metrics. That is, not only impressive improvement over baselines, but near-perfect: 
 near-zero $\AD$ and above 90\% $\AI$. Our new metric $\AG$ is lower, around 70\% 
 for Opti-CAM, but this is still several times higher than for all the other methods.

Interestingly, Fake-CAM \autocite{poppi2021revisiting} is the winner in terms of $\AD$ and second 
or third best in $\AI$ after Opti-CAM and Score-CAM, but fails completely $\AG$. This is expected 
and makes Fake-CAM uninteresting as it should be: By only masking one pixel, the classification 
score can hardly drop (0.8\% on ResNet50) and while it increases very often (on 46\% of images), 
the gain is as little as the drop (0.7\%). This makes the pair ($\AD$, $\AG$) sufficient as primary
 metrics and $\AI$ can be thought of as secondary, if important at all.

%In the supplementary material we report \emph{insertion} (I) and \emph{deletion} (D) metrics along with failure cases of Opti-CAM. The latter indicate that our saliency maps are not incorrect as a whole, but capturing more parts of the object, more instances or more background context results in larger or several disconnected salient regions. This does not let the classifier focus on a single discriminative region when pixels are processed sequentially by increasing saliency. Rather, I/D favor smaller and more compact saliency maps.


\autoref{tab:imagenet-cnn} also includes average execution time per image over the 1000-image 
ImageNet subset for all methods. Opti-CAM is slower than gradient-based methods that require 
only one pass through the network, but on par or faster than gradient-free methods. 
Indeed, we use a maximum of 100 iterations with one forward/backward pass per iteration, 
while Score-CAM and Ablation-CAM perform as many forward passes as channels. Hence they are much 
slower on ResNet50 than VGG16. Extremal Perturbation does not depend on the number of channels but 
is very slow by performing a complex optimization in the image space.

%--------------------------------------------------------------------------------------------------
%------------------------------------------------------------------------------
\begin{table}
    \centering
    \footnotesize
    \setlength{\tabcolsep}{4pt}
    \renewcommand{\arraystretch}{0.8}
    \begin{tabular}{lrrrr|rrrr} \toprule
        \mr{2}{\Th{Method}}& \mc{4}{\Th{ViT-B}} & \mc{4}{\Th{DeiT-B}} \\ \cmidrule{2-9}
        & {$\AD\!\downarrow$} & {$\AG\!\uparrow$} & {$\AI\!\uparrow$} & \mc{1}{T} & {$\AD\!\downarrow$} & {$\AG\!\uparrow$} & {$\AI\!\uparrow$} & \mc{1}{T} \\ \midrule
        Fake-CAM            &  0.3 &  0.4 & 48.3 &  0.00 &  0.6 &  0.3 & 44.6 &  0.00 \\ \midrule
        Grad-CAM            & 69.4 &  2.5 & 12.4 &  0.14 & 33.5 &  1.7 & 12.5 &  0.11 \\
        Grad-CAM            & 86.3 &  1.5 &  1.0 &  0.15 & 50.7 &  0.9 &  7.2 &  0.13 \\
        Score-CAM           & 32.0 &  6.2 & 33.0 & 23.69 & 53.6 &  2.2 & 12.2 & 22.47 \\
        XGrad-CAM           & 88.1 &  0.4 &  4.3 &  0.13 & 80.5 &  0.3 &  4.1 &  0.12 \\
        Layer-CAM           & 82.0 &  0.2 &  2.9 &  0.24 & 88.9 &  0.4 &  2.6 & 0.24\\
        ExPerturbation      &28.8&6.2&24.4&133.52&60.9&2.0&8.5&129.12\\
        RawAtt              & 92.6 &  0.2 &  2.8 &  0.02 & 95.3 &  0.0 &  1.8 &  0.02 \\
        Rollout             & 42.1 &  5.6 & 20.9 &  0.02 & 55.2 &  0.8 &  7.9 &  0.02 \\
        TIBAV               & 81.7 &  0.8 &  5.8 &  0.16 & 62.3 &  0.7 &  7.1 &  0.16 \\\midrule
        Opti-CAM            & \tb{ 0.6} &   \tb{18.0} & \tb{90.1} &    16.05 & \tb{ 0.9} & \tb{26.0} & \tb{83.5} &    15.17 \\ \bottomrule
    \end{tabular}
    \caption{}
    %\caption{\emph{Classification metrics} on ImageNet validation set, using transformers. $\AD$/$\AI$: average drop/increase
    %\autocite{chattopadhay2018grad}; $\AG$: average gain (ours); $\downarrow$ / $\uparrow$: lower / higher is better. \iavr{T: Average time (sec) per batch of 8 images. Bold: best, excluding Fake-CAM.}}
    \label{tab:imagenet-trans}
\end{table}
%------------------------------------------------------------------------------
%Fake-CAM~\citep{poppi2021revisiting}    &  0.3 &  0.4 & 48.3 &  0.00 &  0.6 &  0.3 & 44.6 &  0.00 \\ \midrule
%Grad-CAM~\citep{selvaraju2017grad}      & 69.4 &  2.5 & 12.4 &  0.14 & 33.5 &  1.7 & 12.5 &  0.11 \\
%Grad-CAM++~\cite{chattopadhay2018grad}  & 86.3 &  1.5 &  1.0 &  0.15 & 50.7 &  0.9 &  7.2 &  0.13 \\
%Score-CAM~\citep{wang2020score}         & 32.0 &  6.2 & 33.0 & 23.69 & 53.6 &  2.2 & 12.2 & 22.47 \\
%XGrad-CAM~\citep{fu2020axiom}           & 88.1 &  0.4 &  4.3 &  0.13 & 80.5 &  0.3 &  4.1 &  0.12 \\
%Layer-CAM~\citep{jiang2021layercam}     & 82.0 &  0.2 &  2.9 &  0.24 & 88.9 &  0.4 &  2.6 & 0.24\\
%ExPerturbation~\citep{fong2019understanding}&28.8&6.2&24.4&133.52&60.9&2.0&8.5&129.12\\
%RawAtt~\citep{dosovitskiy2020image}     & 92.6 &  0.2 &  2.8 &  0.02 & 95.3 &  0.0 &  1.8 &  0.02 \\
%Rollout~\citep{abnar2020quantifying}    & 42.1 &  5.6 & 20.9 &  0.02 & 55.2 &  0.8 &  7.9 &  0.02 \\
%TIBAV~\cite{chefer2021transformer}      & 81.7 &  0.8 &  5.8 &  0.16 & 62.3 &  0.7 &  7.1 &  0.16 \\
%\modify{HiRes-CAM~\citep{draelos2020use}} &\modify{98.4}&\modify{0.0}&\modify{0.7}&\modify{0.03}&\modify{97.2}&\modify{0.0}&\modify{1.2}&\modify{0.03} \\
%\hline
%Opti-CAM (ours)                         & \tb{ 0.6} &   \tb{18.0} & \tb{90.1} &    16.05 & \tb{ 0.9} & \tb{26.0} & \tb{83.5} &    15.17 \\ \bottomrule
%--------------------------------------------------------------------------------------------------

\paragraph{Transformers}

\autoref{tab:imagenet-trans} shows ImageNet classification metrics using ViT \iavr{and DeiT}. 
Unlike CAM-based methods that rely on a class-specific linear combination of feature maps, 
raw attention \autocite{dosovitskiy2020image} and rollout \autocite{abnar2020quantifying} use the 
attention map of the [CLS] token from the last attention block and from all blocks respectively. 
\iavr{This attention map depends only on the particular image and not on the target class, hence it
 is not really comparable. TIBAV \autocite{chefer2021transformer} uses both instance-specific and 
 class-specific information.

Opti-CAM outperforms all other methods dramatically, reaching near-zero $\AD$ and $\AI$ above 80 or 
90\%. According to our new $\AG$ metric, Opti-CAM still works while all other methods fail, 
but $\AG$ is much more conservative than $\AI$. On ViT-B for example, the classification score 
increases for 90.1\% of the images by masking with Opti-CAM, but the gain is only 18.0\% on average.}

\paragraph{More metrics}
In this section, we show additional metrics including AOPC \autocite{samek2016evaluating}, Max-Sensitivity
 \autocite{yeh2019fidelity} and ADCC \autocite{poppi2021revisiting}.

We use the code and suggested parameters of package 
Quantus\footnote{\url{https://github.com/understandable-machine-intelligence-lab/Quantus}} to measure AOPC 
and MS. In particular, patch size $14$ and number of evaluation regions $10$ for AOPC; lower bound $0.2$ 
and number of samples $10$ for MS.
For ADCC, we use the official 
code\footnote{\url{https://github.com/aimagelab/ADCC}}.
We evaluate these metrics on ImageNet validation set using ResNet50 and VGG16. The results are 
reported in \autoref{tab:more-metrics-asked}. Since AOPC shares the same philosophy as I/D, it is 
not a surprise that Opti-CAM has poor performance on AOPC. Opti-CAM achieves the best performance on MS.


\begin{table}[t]
    \centering
    \footnotesize
    \setlength{\tabcolsep}{4pt}
    \begin{tabular}{lrrr rrr} \toprule
        \mr{2}{\Th{Method}} & \mc{3}{\Th{ResNet50}} & \mc{3}{\Th{VGG16}}  \\ \cmidrule{2-7}
        & {{$AOPC\uparrow$}} & {{$MS\downarrow$}}& {{$ADCC\downarrow$}} & {{$AOPC\uparrow$}} 
        & {{$MS\downarrow$}}& {{$ADCC\downarrow$}}  \\ \midrule
        Grad-CAM            &11.7&1.05&74.3&13.1&1.10&73.7        \\
        Grad-CAM++          &11.6&1.04&73.6&11.6&1.09&74.6          \\
        Score-CAM           &10.2&1.04&61.0&11.0&1.09&73.9             \\
        XGrad-CAM           &11.9&1.05&74.3&13.1&1.10&73.9           \\
        Ablation-CAM        &11.1&1.04&71.5&12.5&1.10&75.5          \\
        Layer-CAM           &\tb{13.0}&1.22&61.1&\tb{13.3}&1.25&51.7 \\
        ExPerturbation      &12.0&1.07&\tb{26.0}&11.2&1.09&\tb{42.8}  \\\hline
        Opti-CAM (ours)     &6.3&\tb{1.03}&65.5&8.9&\tb{1.06}&70.0        \\ \bottomrule
    \end{tabular}
    \caption{}
    %\caption{\modify{\emph{AOPC/MS/ADCC} scores on ImageNet validation set.}}
    \label{tab:more-metrics-asked}
\end{table}

% -------------------------------------------------------------------------------------------------
The experimental results are shown in \autoref{tab:imagenet_cnn_hihd} for CNNs and transformers. 
ExPerturbation \autocite{fong2019understanding} is expected to perform best in insertion because its 
optimization objective is very similar to this evaluation metric, using blurring for masked regions. 
However, ExPerturbation \autocite{fong2019understanding}  only performs best on ResNet50. 
TIBAV \autocite{chefer2021transformer}, which is designed for transformers, outperforms the other 
methods on DeiT and ViT. According to the results of Insertion/Deletion, Opti-CAM has low 
performance but there is no clear winner on either CNNs or transformers.

To further understand the behavior of Opti-CAM, we investigate in \autoref{fig:hihd} examples where 
Score-CAM succeeds (insertion score greater than $90$ and deletion score less than $10$) and 
Opti-CAM fails (insertion score less than $70$ and deletion score greater than $15$). Compared with 
Score-CAM, the saliency maps obtained by Opti-CAM are more spread out and highlight several parts 
of the object and background context. In most of the cases, Opti-CAM fails I/D because it not only 
finds the object but also attaches importance to the background.

%------------------------------------------------------------------------------
\begin{figure}[H]
    \newcommand{\sizeP}{.12}
    \newcommand{\sizeS}{.12}
    \newcommand{\hh}{.175\textwidth}
    \newcommand{\ww}{.200\textwidth}
    \tiny
    \centering
    \setlength{\tabcolsep}{3pt}
    \begin{tabular}{cccccc}
    \centering
    Original & Opti-CAM & Score-CAM & Original & Opti-CAM & Score-CAM\\
    \includegraphics[trim={28mm 8mm 28mm 8mm},clip, width=\sizeP\textwidth]{opticam/images/eval/hihd/ILSVRC2012_val_00045353.JPEG}
    &
    \fig[\sizeS]{opticam/images/eval/hihd/ILSVRC2012_val_00045353JPEG_smap_opticam.png} 
    &  
    \fig[\sizeS]{opticam/images/eval/hihd/ILSVRC2012_val_00045353JPEG_smap_scorecam.png} &
    \includegraphics[trim={32mm 14mm 36mm 1mm},clip, width=\sizeP\textwidth]{opticam/images/eval/hihd/ILSVRC2012_val_00041066.JPEG}
    &
    \fig[\sizeS]{opticam/images/eval/hihd/ILSVRC2012_val_00041066JPEG_smap_opticam.png} 
    &          
    \fig[\sizeS]{opticam/images/eval/hihd/ILSVRC2012_val_00041066JPEG_smap_scorecam.png} \\
    gas pump&I$\uparrow$:66.3, D$\downarrow$:19.4&I$\uparrow$:94.2, D$\downarrow$:9.4&
    worm fence&I$\uparrow$:69.7, D$\downarrow$:16.8&I$\uparrow$:91.9, D$\downarrow$:4.4\\
    &AG$\uparrow$:100.0, AD$\downarrow$:0.0&AG$\uparrow$:0.0, AD$\downarrow$:0.0&
    &AG$\uparrow$:73.2, AD$\downarrow$:0.0&AG$\uparrow$:0.0, AD$\downarrow$:28.8\\
    \includegraphics[trim={10mm 14mm 10mm 4mm},clip, width=\sizeP\textwidth]{opticam/images/eval/hihd/ILSVRC2012_val_00040673.JPEG}
    &        
    \fig[\sizeS]{opticam/images/eval/hihd/ILSVRC2012_val_00040673JPEG_smap_opticam.png} 
    &
    \fig[\sizeS]{opticam/images/eval/hihd/ILSVRC2012_val_00040673JPEG_smap_scorecam.png} &
    \includegraphics[trim={18mm 6mm 10mm 12mm},clip, width=\sizeP\textwidth]{opticam/images/eval/hihd/ILSVRC2012_val_00030507.JPEG}
    &
    \fig[\sizeS]{opticam/images/eval/hihd/ILSVRC2012_val_00030507JPEG_smap_opticam.png} 
    &                
    \fig[\sizeS]{opticam/images/eval/hihd/ILSVRC2012_val_00030507JPEG_smap_scorecam.png} \\
    staffordshire terrier&I$\uparrow$:62.1, D$\downarrow$:32.2&I$\uparrow$:93.4, D$\downarrow$:8.2&
    jacamar&I$\uparrow$:66.3, D$\downarrow$:17.3&I$\uparrow$:94.6, D$\downarrow$:9.9\\
    &AG$\uparrow$:41.3, AD$\downarrow$:0.0&AG$\uparrow$:0.0, AD$\downarrow$:0.3&
    &AG$\uparrow$:91.4, AD$\downarrow$:0.0&AG$\uparrow$:56.5, AD$\downarrow$:0.0\\
    \includegraphics[trim={6mm 1mm 6mm 1mm},clip, width=\sizeP\textwidth]{opticam/images/eval/hihd/ILSVRC2012_val_00029237.JPEG}
    &
    \fig[\sizeS]{opticam/images/eval/hihd/ILSVRC2012_val_00029237JPEG_smap_opticam.png} 
    &     
    \fig[\sizeS]{opticam/images/eval/hihd/ILSVRC2012_val_00029237JPEG_smap_scorecam.png} &
    \includegraphics[trim={28mm 5mm 22mm 5mm},clip, width=\sizeP\textwidth]{opticam/images/eval/hihd/ILSVRC2012_val_00005077.JPEG}
    &
    \fig[\sizeS]{opticam/images/eval/hihd/ILSVRC2012_val_00005077JPEG_smap_opticam.png} 
    &          
    \fig[\sizeS]{opticam/images/eval/hihd/ILSVRC2012_val_00005077JPEG_smap_scorecam.png} \\
    Irish water spaniel&I$\uparrow$:52.6, D$\downarrow$:18.8&I$\uparrow$:90.5, D$\downarrow$:8.6&
    manhole cover&I$\uparrow$:65.8, D$\downarrow$:29.6&I$\uparrow$92.7, D$\downarrow$:9.1\\
    &AG$\uparrow$:86.4, AD$\downarrow$:0.0&AG$\uparrow$:65.1, AD$\downarrow$:0.0&
    &AG$\uparrow$:24.0, AD$\downarrow$:0.0&AG$\uparrow$:0.0, AD$\downarrow$:59.9\\
    % \includegraphics[trim={12mm 5mm 12mm 5mm},clip, width=\sizeP\textwidth]{ opticam/images/eval/hihd/ILSVRC2012_val_00003602.JPEG}
    % &              
    % \fig[\sizeS]{opticam/images/eval/hihd/ILSVRC2012_val_00003602JPEG_smap_opticam.png} 
    % & 
    % \fig[\sizeS]{opticam/images/eval/hihd/ILSVRC2012_val_00003602JPEG_smap_scorecam.png} \\
    % hare&I$\uparrow$:61.3, D$\downarrow$:21.2&I$\uparrow$91.3, D$\downarrow$:8.9\\
    % &AG$\uparrow$:93.7, AD$\downarrow$:0.0&AG$\uparrow$:0.0, AD$\downarrow$:0.6\\
    \end{tabular}
    \caption{Failure examples of Opti-CAM regarding insertion/deletion.}
    \label{fig:hihd}
    \end{figure}
    %------------------------------------------------------------------------------

We argue that this is not a failure. As our localization experiment in \autoref{tab:localization} 
indicates, the background is useful in discriminating a class. Often, the network recognizes the 
background better than the object itself. For example, a gas pump is likely to be seen with a truck, 
and a hare is often seen on grass. Several parts of the object are highlighted by Opti-CAM for the 
worm fence, terrier dog, hare, and manhole cover. Finally, several instances of spaniel dog are found 
by Opti-CAM.

Insertion/Deletion include 224 steps of binarization, with a set of 224 pixels being 
inserted/deleted at each step. If these pixels are all inserted over a single small area, the 
effect on the classifier is more immediate than when sparsely inserting pixels over multiple areas. 
The same observation holds for deletion. By contrast, Opti-CAM attempts to find regions that 
contribute to the classification as a whole. There is no guarantee that those regions are effective 
when used in isolation.

\begin{table}[t]
    \centering
    \footnotesize
    \setlength{\tabcolsep}{8pt}
    \renewcommand{\arraystretch}{0.8}
    \begin{tabular}{lrr rr rr rr} \toprule
    \mr{2}{\Th{Method}} & \mc{2}{\Th{ResNet50}} & \mc{2}{\Th{VGG16}} & \mc{2}{\Th{ViT-B}}& \mc{2}{\Th{DeiT-B}} \\ \cmidrule{2-9}
                        & {{$\I\!\uparrow$}} & {{$\D\!\downarrow$}}& {{$\I\!\uparrow$}} & {{$\D\!\downarrow$}} & {{$\I\!\uparrow$}} & {{$\D\!\downarrow$}}& {{$\I\!\uparrow$}} & {{$\D\!\downarrow$}}\\ \midrule
    Fake-CAM&50.7&28.1&46.1&26.9&57.4&33.3&57.5&34.2\\\midrule
    Grad-CAM&66.3&14.7&\tb{64.1}&11.6&62.9&19.8&61.8&17.5\\
    Grad-CAM++&66.0&14.7&62.9&12.2&56.7&29.3&60.5&21.9\\
    Score-CAM&65.7&16.3&62.5&12.1&\tb{66.5}&15.1 &60.6&24.4\\
    XGrad-CAM&66.3&14.7&\tb{64.1}&11.7&55.6&26.5  &55.2&31.1\\
    %\midrule
    Layer-CAM&67.0&\tb{14.2}&58.3&\tb{6.4}&62.9&14.6 &61.6&21.2\\
    ExPerturbation&\tb{70.7}&15.0&61.1&15.0&64.4&18.4&62.1&27.0\\
    Ablation-CAM&65.9&14.6&63.8&11.4&-&-&-&-\\
    RawAtt&-&-&-&-&62.2&17.9 &56.3&29.3\\
    Rollout&-&-&-&-&64.8&15.2 &56.7&32.8\\
    TIBAV&-&-&-&-&66.1&\tb{14.1} &\tb{63.7}&\tb{16.3}\\
    Opti-CAM (ours)&62.0&19.7&59.2&11.0 &60.5&22.0  &59.2&22.8\\
    \bottomrule
    \end{tabular}
    % \vspace{6pt}
    \caption{
    I/D: insertion/deletion \autocite{petsiuk2018rise} scores on ImageNet validation set; $\downarrow$ / $\uparrow$: lower / higher is better.}% Bold: best, excluding Fake-CAM.}
    \label{tab:imagenet_cnn_hihd}
\end{table}
\newpage
%--------------------------------------------------------------------------------------------------
\section{Object localization}
\label{sec:oc_loc}
Localization metrics are used to measure the precision of saliency maps relative to groundtruth 
bounding boxes of the foreground object of interest. These metrics originate from weakly supervised 
localization (WSOL). However, the objectives of WSOL and explaining the decision of a DNN are not 
necessarily aligned, since context may play an important role in the decision (\cite{shetty2019not}, 
\cite{rao2022towards}).

To investigate the relative importance of the object and its context, we measure classification 
metrics when using the bounding box $B$ itself as a saliency map as well as its complement 
$I \setminus B$, where $I$ is the image. We also evaluate the intersection $B \cap S$ of the 
saliency map $S$ with the bounding box and with its complement ($S \setminus B$).

As shown in \autoref{tab:localization}, the ground truth region of the object is not the only one 
responsible for the network decision. For example, the bounding box fails both when used as a 
saliency map itself and when combined with any saliency map, by harming all classification metrics. 
Even the complement is more effective than the bounding box itself, either alone or when combined. 
These findings support the hypothesis that localization metrics based on the ground truth bounding 
box are not necessarily appropriate for evaluating explanations of network decisions. 
Classification metrics are clearly more appropriate in this sense.

Nevertheless, we report localization metrics in the supplementary material. In summary, although 
its saliency maps are more spread out, Opti-CAM outperforms other methods on a number of metrics.

%------------------------------------------------------------------------------
%--------------------------------------------------------------------------------------------------
\begin{table}[t]
    \footnotesize
    \centering
    \setlength{\tabcolsep}{4pt}
    \renewcommand{\arraystretch}{0.8}
    \begin{tabular}{lccc|ccc|ccc} \toprule
    \mr{2}{\Th{Method}}                            & \mc{3}{\Th{$\AD\!\downarrow$}} & \mc{3}{\Th{$\AG\!\uparrow$}}& \mc{3}{\Th{$\AI\!\uparrow$}} \\ \cmidrule{2-10}
                                                   & {$S$} & {$B \!\cap\! S$} & {$S \!\setminus\! B$} & {$S$} & {$B \!\cap\! S$} & {$S \!\setminus\! B$}& {$S$} & {$B \!\cap\! S$} & {$S \!\setminus\! B$} \\ \midrule
    $S \defn B$                                    & 67.2 &   -- &   -- &  2.3 &   -- &   -- &  9.2 &   -- &   -- \\
    $S \defn I \setminus B$                        & 44.0 &   -- &   -- &  2.8 &   -- &   -- & 16.3 &   -- &   -- \\ \midrule
    Fake-CAM                                       &  0.5 & 67.2 & 44.1 &  0.7 &  2.3 &  2.8 & 42.0 &  9.2 & 18.9 \\ \midrule
    Grad-CAM                                       & 15.0 & 72.6 & 52.1 & 15.3 &  1.8 &  6.0 & 40.4 &  8.4 & 19.4 \\
    Grad-CAM++                                     & 16.5 & 72.9 & 53.1 & 10.6 &  1.6 &  4.1 & 35.2 &  7.3 & 17.1 \\
    Score-CAM                                      & 12.5 & 71.5 & 50.5 & 16.1 &  2.2 &  6.3 & 42.5 &  8.6 & 20.8 \\
    Ablation-CAM                                   & 15.1 & 72.8 & 52.1 & 13.5 &  1.7 &  5.6 & 39.9 &  7.8 & 19.0 \\
    XGrad-CAM                                      & 14.3 & 72.6 & 51.4 & 15.1 &  1.8 &  6.0 & 42.1 &  8.0 & 20.1 \\
    Layer-CAM                                      & 49.2 & 84.2 & 74.4 &  2.7 &  0.4 &  1.2 & 12.7 &  4.4 &  7.3 \\
    ExPerturbation                                 & 43.8 & 81.6 & 71.0 &  7.1 &  1.4 &  3.2 & 18.9 &  5.6 & 11.1 \\
    \hline
    Opti-CAM (ours)                                & \tb{1.4} & \tb{62.5} & \tb{34.8} & \tb{66.3} & \tb{8.7} & \tb{25.8} & \tb{92.5} & \tb{18.6} & \tb{47.1} \\ \bottomrule
    \end{tabular}
    \caption{\emph{Bounding box} study. Classification metrics on ImageNet validation set using VGG16. $B$: ground-truth box used by localization metrics; $I$: entire image; $S$: saliency map. $\AD$/$\AI$: average drop/increase \autocite{chattopadhay2018grad}; $\AG$: average gain (ours); $\downarrow$ / $\uparrow$: lower / higher is better; bold: best, excluding Fake-CAM.}
    %\caption{}
    \label{tab:localization}
    % \vspace{-0.2cm}
\end{table}
%------------------------------------------------------------------------------

%Fake-CAM~\citep{poppi2021revisiting}
%Grad-CAM~\citep{selvaraju2017grad}             
%Grad-CAM++~\cite{chattopadhay2018grad}
%Score-CAM~\citep{wang2020score}
%Ablation-CAM~\citep{ramaswamy2020ablation}
%XGrad-CAM~\citep{fu2020axiom}                 
%Layer-CAM~\citep{jiang2021layercam}            
%ExPerturbation~\citep{fong2019understanding}   






%In this work, we are interested in the interpretability of deep neural networks through the 
%generation of \emph{saliency maps}, highlighting regions of an image that are responsible for the 
%prediction. This originates in \emph{gradient-based} methods \citep{simonyan2013deep, 
%yosinski2015understanding}, including variants of backpropagation \citep{zeiler2014visualizing, 
%springenberg2014striving, bach2015pixel}. CAM~\citep{zhou2016learning} introduced class-specific 
%linear combinations of feature maps, and led to several alternative weighting schemes 
%\citep{ramaswamy2020ablation, wang2020score, muhammad2020eigen}, including the use of gradients 
%\citep{selvaraju2017grad, chattopadhay2018grad}. On the other hand, \emph{occlusion-} or 
%\emph{masking-based} methods \citep{dabkowski2017real, fong2017interpretable, fong2019understanding, 
%schulz2020restricting} remove regions in the image space while improving classification performance.

%Score-CAM~\cite{wang2020score} uses each feature map as a mask and defines a corresponding weight 
%based on the resulting increase of class score; hence, it is both CAM-based and masking-based but does
% not use gradients. It resembles the numerical gradient approximation, in that it needs \emph{one 
% forward pass per weight}. Instead, the analytical approach would be to use a linear combination of feature 
% maps as a mask, express the class score as a function of the weights and measure the gradient analytically,
%  in a \emph{single backward pass}. Then, \emph{why not use gradient descent to maximize the class score?} 
%  The optimal mask should highlight regions for which the network is most confident.

%\emph{Masking-based} methods, such as extremal perturbations~\citep{fong2019understanding} or 
%IBA~\citep{schulz2020restricting}, do use gradient descent to maximize the class score. The mask is now 
%a variable in the input or feature space and the class score is expressed as a function of the mask 
%directly. Because the variable being optimized is a high-dimensional image or tensor, additional constraints 
%or regularizers are needed to control \eg the smoothness and the salient area. This translates to 
%more hyperparameters or more expensive optimization.

%Motivated by the above, we introduce Opti-CAM, illustrated in \autoref{fig:idea}. We form a linear 
%combination of feature maps, where the weights are a variable. Treating it as a saliency map, we 
%form a masked version of the input image that is fed again to the network. Then, the logit of a given 
%class for the masked version of the input is maximized to obtain the optimal weights. Thus, Opti-CAM can 
%be seen as an analytical counterpart of Score-CAM that is optimized iteratively, or as a masking-based 
%method where the mask to be optimized lies in the linear span of the feature maps, like CAM-based methods.