%--------------------------------------------------------------------------------------------------
\section{Interpretable Gradients}
\label{sec:grad_defn}
\begin{figure}[t]
	\resizebox{\textwidth}{!}{
	\begin{tikzpicture}[
		scale=.3,
		font={\footnotesize},
		node distance=.5,
		label distance=3pt,
		wide/.style={yscale=.75},
		net/.style={draw,trapezium,trapezium angle=75,inner sep=3pt},
		enc/.style={net,shape border rotate=270},
		dec/.style={net,shape border rotate=90},
		txt/.style={inner sep=3pt},
		loose/.style={looseness=.6},
		sg/.style={blue!60},
	]
	\matrix[
		tight,row sep=6,column sep=16,
		cells={scale=.3,},
	] {
		\&\&\&\&\&
		\node[dec] (guided-back) {guided \\ backprop}; \&
		\node[wide,label=90:$\ipder{_G L_C}{x}$] (guided) {\fig[.1]{grad/fig/method/guided_gradient}}; \\
		\node[label=90:input image $x$] (in) {\fig[.1]{grad/fig/method/input}}; \&
		\node[enc] (net) {network \\ $f$}; \&\&\&
		\node[box] (class) {classification \\ loss $L_C$}; \&
		\node[dec] (back) {standard \\ backprop}; \&
		\node[wide,label=90:$\ipder{L_C}{x}$] (grad) {\fig[.1]{grad/fig/method/gradient}}; \&
		\node[box] (reg) {regularization \\ loss $L_R$}; \\
		\&\&\&\&\&\&\&
		\node[op] (plus) {$+$}; \&
		\node[txt] (loss) {total loss \\ $L = L_C + $\\ $\lambda L_R$}; \\
	};

	\draw[->]
		(in) edge (net)
		(net) edge node[above] {logits $y$} (class)
		(class) edge (back)
		(back) edge (grad)
		(guided-back) edge[sg] (guided)
		(grad) edge (reg)
		(reg) edge (plus)
		(plus) edge (loss)
		(class) edge[sg,out=90,in=180] (guided-back)
		(class) edge[loose,out=-50,in=180] (plus)
		(guided) edge[sg,out=0,in=90] node[pos=.2,font=\large,label=0:stop gradient] {//} (reg)
		(net) edge[dashed,out=60,in=150] (guided-back)
		(net) edge[dashed,out=30,in=150] (back)
		;
	\end{tikzpicture}
	}
	\caption{\emph{Interpretable gradient learning}. For an input image $x$, we obtain the logit 
	vector $y = f(x; \theta)$ by a forward pass through the network $f$ with parameters $\theta$. 
	We compute the classification loss $L_C$ by softmax and cross-entropy~\eq{class}, \eq{ce}. We 
	obtain the standard gradient $\ipder{L_C}{x}$ and guided gradient $\ipder{_G L_C}{x}$ by two 
	backward passes (dashed) and compute the regularization loss $L_R$ as the error between the 
	two~\eq{reg},\eq{mae}-\eq{cos}. The total loss is $L = L_C + \lambda L_R$~\eq{total}. Learning 
	is based on $\ipder{L}{\theta}$, which involves differentiation of the entire computational 
	graph except the guided backpropagation branch (blue).}
	\label{fig:grad_method}
\end{figure}

\paragraph{Preliminaries}

We consider an image classification network $f$ with parameters $\theta$, which maps an input image 
$x$ to a vector of class logits $y = f(x; \theta)$. At inference, we predict the class label of 
maximum confidence $\arg\max_j y_j$, where $y_j$ is the logit of class $i$. At training, given 
training images $X = \{x_i\}_{i=1}^n$ and target labels $T = \{t_i\}_{i=1}^n$, we compute the 
\emph{classification loss}

\begin{equation}
	L_C(X, \theta, T) = \frac{1}{n} \sum_{i=1}^n \ce(f(x_i; \theta), t_i),
\label{eq:class}
\end{equation}
where $\operatorname{CE}$ is softmax followed by cross-entropy:
\begin{equation}
	\ce(y, t) = -\log \frac{e^{y_t}}{\sum_i e^{y_i}} = -y_t + \sum_i e^{y_i}.
\label{eq:ce}
\end{equation}
Updates of parameters $\theta$ are then performed by an optimizer, based on the standard partial 
derivative (gradient) $\ipder{L_C}{\theta}$ of the classification loss $L_C$ with respect to 
$\theta$, obtained by standard back-propagation.

However, due to non-linearities like ReLU activations and downsampling like max-pooling or 
convolution stride $> 1$, the standard gradient is noisy \autocite{smilkov2017smoothgrad}. This 
is shown by visualizing the gradient $\ipder{L_C}{x}$ with respect to an input image $x$. By 
contrast, the guided gradient $\ipder{_G L_C}{x}$ \autocite{guidedbackprop} does not suffer much 
from noise and preserves sharp details. The difference of the two gradients is illustrated in 
\autoref{fig:grad_method}.

\paragraph{Regularization}
The main idea of this work is to introduce a regularization term during training, which will make 
the standard gradient $\ipder{L_C}{x}$ behave similarly to the corresponding guided gradient 
$\ipder{_G L_C}{x}$, while maintaining the predictive power of the classifier $f$. We hypothesize 
that, if possible, this will improve the quality of all gradients with respect to intermediate 
activations and therefore the quality of saliency maps obtained by CAM-based methods 
(\cite{zhou2016learning}, \cite{selvaraju2017grad}, \cite{chattopadhay2018grad}, 
\cite{wang2020score}) and the interpretability of network $f$. The effect may be 
similar to that of SmoothGrad \autocite{smilkov2017smoothgrad}, but without the need for several 
forward passes at inference.

\noindent In particular, given an input image $x$, we perform a forward pass through $f$ and 
compute the logit vectors $y_i = f(x_i, \theta)$ and the classification loss $L_C(X, \theta, T)$
~\eq{class}. We then obtain the standard gradients $\delta x_i = \ipder{L_C}{x_i}$ and the guided 
gradients $\delta_G x_i = \ipder{_G L_C}{x_i}$ with respect to the input images $x_i$ by two 
separate backward passes. Since the whole process is differentiable (\wrt $\theta$) at training, 
we stop the gradient computation of the latter, so that it only serves as a ``teacher''. We define 
the \emph{regularization loss} 

\begin{equation}
	L_R(X, \theta, T) = \frac{1}{n} \sum_{i=1}^n E(\delta x_i, \delta_G x_i),
\label{eq:reg}
\end{equation}
where $E$ is an error function between the two gradient images, considered below.\\

\noindent Finally, the total loss is defined as
\begin{equation}
	L(x, \theta, t) = L_C(x, \theta, t) + \lambda L_R(x, \theta, t),
\label{eq:total}
\end{equation}
where $\lambda$ is a hyperparameter determining the regularization coefficient. 
$\lambda$ should be large enough to smooth the gradient without decreasing the classification 
accuracy or hurting the training process. 
Updates of the network parameters $\theta$ are now based on the gradient $\ipder{L}{\theta}$ 
\wrt the total loss, using any optimizer. At inference, one may use any interpretability method 
to obtain a saliency map at any layer.

\paragraph{Algorithm}
Our method is summarized in Algorithm \ref{alg:grad} %, where \textsc{detach} stops gradient computation, 
and illustrated in \ref{fig:grad_method}.It is interesting to note that the entire computational 
graph depicted in \ref{fig:grad_method} involves one forward and two backward passes. This graph is 
then differentiated again to compute $\ipder{L}{\theta}$, which involves one more forward and 
backward pass, since the guided backpropagation branch is excluded. Thus, each training iteration 
requires five passes through $f$ instead of two in a standard training.

%------------------------------------------------------------------------------

\begin{algorithm}[H]    
	\SetFuncSty{textsc}
	\SetDataSty{emph}
	\newcommand{\commentsty}[1]{{\color{DarkGreen}#1}}
	\SetCommentSty{commentsty}
	\SetKwComment{Comment}{$\triangleright$ }{}

	\SetKwFunction{Detatch}{detatch}

	\KwIn{network $f$, parameters $\theta$}
	\KwIn{input images $X = \{x_i\}_{i=1}^n$}
	\KwIn{target labels $T = \{t_i\}_{i=1}^n$}
	\KwOut{loss $L$}
	$L_C \gets \frac{1}{n} \sum_i \ce(f(x_i; \theta), t_i)$ \Comment*[f]{class. loss \eq{class}} \\
    \ForEach{$i \in \{1,\dots,n\}$}{
        $\delta x_i \gets \ipder{L_C}{x_i}$ \Comment*[f]{standard grad} \\
	 	$\delta_G x_i \gets \ipder{_G L_C}{x_i}$ \Comment*[f]{guided grad} \\
        $\mbox{\Th{Detach}}(\delta_G x_i)$ \Comment*[f]{detach from graph} \\
	  }
   $L_R \gets \frac{1}{n} \sum_{i=1}^n E(\delta x_i, \delta_G x_i) $ \Comment*[f]{reg. loss \eq{reg}}\\
   $L \gets L_C + \lambda L_R$
 \Comment*[f]{total loss ~\eq{total}} \\

\caption{Interpretable gradient loss}
\label{alg:grad}
\end{algorithm}

%------------------------------------------------------------------------------
\paragraph{Error function}

Given two gradient images $\delta, \delta'$ consisting of $p$ pixels each, we consider the 
following error functions $E$ to compute the regularization loss~\eq{reg}.


%\begin{enumerate}[itemsep=2pt, parsep=0pt, topsep=3pt]
%    \item Sigmoid Masking
%    \begin{equation}
%        E_{\sigma}(\delta,\delta') = - \sum_{i=0}^p \delta'\odot\frac{e^{\delta}}{e^{\delta}+1}.
%    \end{equation}
%    \item Centered Sigmoid Masking
%    \begin{equation}
%        E_{\sigma}(\delta,\delta') = - \sum_{i=0}^p \delta'\odot\frac{e^{\delta-\mu_\delta}}{e^{\delta-\mu_\delta}+1}.
%    \end{equation}
%\end{enumerate}


\begin{enumerate}[itemsep=2pt, parsep=0pt, topsep=3pt]

 	\item Mean absolute error:
 	\begin{equation}
 		E_{\mae}(\delta, \delta') = \frac{1}{p} \norm{\delta - \delta'}_1.
 	\label{eq:mae}
    \end{equation}  
 	\item Mean squared error:
 	\begin{equation}
 		E_{\mse}(\delta, \delta') = \frac{1}{p} \norm{\delta - \delta'}_2^2.
 	\label{eq:mse}
 	\end{equation}

\end{enumerate}

We also consider the following two similarity functions, with a negative sign.
\begin{enumerate}[itemsep=2pt, parsep=0pt, topsep=3pt]
	\setcounter{enumi}{2}
	\item Cosine similarity:
	\begin{equation}
		E_{\cos}(\delta, \delta') = -\frac{\inner{\delta, \delta'}}{\norm{\delta}_2 \norm{\delta'}_2},
	\label{eq:cos}
	\end{equation}
 
	\item Histogram intersection:
	\begin{equation}
		E_{\hi}(\delta, \delta') = -\sum_{i=0}^p 
			\frac{\min(\abs{\delta_i}, \abs{\delta'_i})}{\norm{\delta}_1 \norm{\delta'}_1}.
	\label{eq:hi}
	\end{equation}

	where $\inner{,}$ denotes inner product.

\end{enumerate}

% \begin{enumerate}[itemsep=2pt, parsep=0pt, topsep=3pt]
% 	\item Mean absolute error:
% 	\begin{equation}
% 		E_{\mae}(\delta, \delta') = \frac{1}{p} \norm{\delta - \delta'}_1.
% 	\label{eq:mae}
% 	\end{equation}
% 	\item Mean squared error:
% 	\begin{equation}
% 		E_{\mse}(\delta, \delta') = \frac{1}{p} \norm{\delta - \delta'}_2^2.
% 	\label{eq:mse}
% 	\end{equation}
% \end{enumerate}

% We also consider the following two similarity functions, with a negative sign.

%\green{Do we mention the changes in the architecture so the }
