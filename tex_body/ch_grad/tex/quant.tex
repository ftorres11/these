%--------------------------------------------------------------------------------------------------
\subsection{Quantitative Evaluation}
We evaluate the effect of training a given model using our proposed approach with 
\textit{Faithfulness} and \textit{Causality}. Results are reported in Table \ref{tab:C100_quant}.
\begin{table}[t]
    \centering
    \scriptsize
    \setlength{\tabcolsep}{2.5pt}
        \begin{tabular}{lcccccc}\\\toprule
            \mc{7}{\Th{\textbf{Recognition Metrics}}}\\\midrule
            \Th{Model}&\Th{Error}& &\Th{$\lambda$}&\Th{Acc}$\uparrow$& &\\\midrule
            \mr{2}{\Th{ResNet-18}}&-& &-&\Th{\textbf{73.42}}& &\\
             &\Th{Cosine}& &$7.5\times10^{-3}$&\Th{72.86}& &\\\midrule
             
            \mr{2}{\Th{MobileNet-V2}}&-& &-&\Th{59.43}&\\
             &\Th{Cosine}&  &$1\times10^{-3}$&\Th{\textbf{62.36}}&\\\midrule
            
            \mc{7}{\Th{\textbf{Interpretable Recognition Metrics}}}\\\midrule    
            \mc{7}{\Th{ResNet-18}}\\\midrule
            \Th{Method}&\Th{Error}&\Th{AD$\downarrow$}&\Th{AG$\uparrow$}&\Th{AI$\uparrow$}&\Th{Ins$\uparrow$}&\Th{Del$\downarrow$}\\\hline
            \mr{2}{\Th{Grad-CAM}}&-&30.16&15.23&29.99&58.47&\textbf{17.47}\\
             &\Th{Cosine}&\textbf{28.09}&\textbf{16.19}&\textbf{31.53}&\textbf{58.76}&17.57\\\hline
            \mr{2}{\Th{Grad-CAM++}}&-&31.40&14.17&28.47&58.61&\textbf{17.05}\\
              &\Th{Cosine}&\textbf{29.78}&\textbf{15.07}&\textbf{29.60}&\textbf{58.90}&17.22\\\hline
            \mr{2}{\Th{Score-CAM}}&-&26.49&18.62&33.84&58.42&\textbf{18.31}\\
              &\Th{Cosine}&\textbf{24.82}&\textbf{19.49}&\textbf{35.51}&\textbf{59.11}&18.34\\\hline
            \mr{2}{\Th{Ablation-CAM}}&-&31.96&14.02&28.33&58.36&\textbf{17.14}\\
             &\Th{Cosine}&\textbf{29.90}&\textbf{15.03}&\textbf{29.61}&\textbf{58.70}&17.37\\\hline
            \mr{2}{\Th{Axiom-CAM}}&-&30.16&15.23&29.98&58.47&\textbf{17.47}\\
              &\Th{Cosine}&\textbf{28.09}&\textbf{16.20}&\textbf{31.53}&\textbf{58.76}&17.57\\\midrule
    
            \mc{7}{\Th{MobileNet-V2}}\\\midrule
            \Th{Method}&\Th{Error}&\Th{AD$\downarrow$}&\Th{AG$\uparrow$}&\Th{AI$\uparrow$}&\Th{Ins$\uparrow$}&\Th{Del$\downarrow$}\\\hline
            \mr{2}{\Th{Grad-CAM}}&-&44.64&6.57&25.62&44.64&\textbf{14.34}\\
             &\Th{Cosine}&\textbf{40.89}&\textbf{7.31}&\textbf{27.08}&\textbf{45.57}&15.20\\\hline
            \mr{2}{\Th{Grad-CAM++}}&-&45.98&6.12&24.10&44.72&\textbf{14.76}\\
              &\Th{Cosine}&\textbf{40.76}&\textbf{6.85}&\textbf{26.46}&\textbf{45.51}&14.92\\\hline
            \mr{2}{\Th{Score-CAM}}&-&40.55&7.85&28.57&45.62&\textbf{14.52}\\
              &\Th{Cosine}&\textbf{36.34}&\textbf{9.09}&\textbf{30.50}&\textbf{46.35}&14.72\\\hline
            \mr{2}{\Th{Ablation-CAM}}&-&45.15&6.38&25.32&44.62&\textbf{15.03}\\
              &\Th{Cosine}&\textbf{41.13}&\textbf{7.03}&\textbf{26.10}&\textbf{45.38}&15.12\\\hline
            \mr{2}{\Th{Axiom-CAM}}&-&44.65&6.57&25.62&44.64&15.27\\
              &\Th{Cosine}&\textbf{40.89}&\textbf{7.31}&\textbf{27.08}&\textbf{45.57}&\textbf{15.20}\\\bottomrule
    
        \end{tabular}
        \caption{Experiments with cosine error function on CIFAR-100 with ResNet-18 and MobileNet-V2. Accuracy and interpretability metrics are reported.}
        \label{tab:C100_quant}
\end{table}
\noindent We observe that our method offers a consistent improvement in terms of interpretability 
metrics. Specifically, we obtain improvements on both networks and systematically on five out of 
six metrics. The improvements are higher for AD, AG, and AI. Insertion gets a smaller but consistent 
improvement and Deletion is almost always worse with our method, but with a very small difference.
This decrease in performance of Deletion may be due to some limitations of the metrics as reported 
in Chapter \ref{ch:opticam}.
It is interesting to note that improvements on Score-CAM means that our training not only improves 
gradient for interpretability, but also builds better activation maps.\\

\subsection{Ablation Experiments}
We conduct ablation experiments using ResNet18. In these experiments we analyze the different 
regularization proposals mentioned in Section \ref{sec:grad_defn} and the impact of the 
regularization coefficient.

\paragraph{Regularization proposals} 
To validate our selection of regularization function, we train several models following the same 
training regime while varying the error function. To evaluate these approaches, we focus solely on 
Grad-CAM attributions. Results are reported in Table \ref{tab:Regs}\\
\begin{table}[H]
    \centering
    \scriptsize
    \setlength{\tabcolsep}{3.5pt}
        \begin{tabular}{lcccccc}\\\toprule
        \mc{7}{Regularization Selection}\\\midrule
        \Th{Regularizer}&\Th{Acc}&\Th{AD$\downarrow$}&\Th{AG$\uparrow$}&\Th{AI$\uparrow$}&\Th{Ins$\uparrow$}&\Th{Del$\downarrow$}\\\midrule
        - &73.42&30.16&15.23&29.99&58.47&17.47\\
        Cosine&72,86&\textbf{28.09}&\textbf{16.19}&\textbf{31.53}&58.76&17.57\\
        Histogram &73.88&30.39&14.78&29.38&58.52&17.35\\
        MAE & 73,41& 30,33 & 15,06 &29,61 & 58,13 & 17,95\\
        MSE & 73,86& 29,64 & 15,19 &30,11 & \textbf{59,05} & 18,02\\\bottomrule
        \end{tabular}
        \caption{\textbf{Regularization selection: } Evaluation of the four proposed regularization with ResNet-18 on CIFAR-100.}%}
        \label{tab:Regs}
\end{table}
\noindent Following these results, we observe a consistent improvement on most metrics for all 
regularizer options. We note that the accuracy remains stable within half a percent of the original 
model. However, we note that most options struggle regarding deletion. Cosine Similarity however 
manages to provide improvements in most metrics, while maintaining deletion performance.

\paragraph{Regularization coefficient}
Finally, we study the behavior of the regularization coefficient $\lambda$ in \ref{eq:total}. We 
train multiple models with \textit{Cosine Similarity} and a range of values for $\lambda$, see Table 
\ref{tab:variation}.\\

\begin{table}[H]
    \centering
    \scriptsize
    \setlength{\tabcolsep}{3.5pt}
        \begin{tabular}{lcccccc}\\\toprule
        \mc{7}{Regularization Selection}\\\midrule
        $\lambda$&\Th{Acc}&\Th{AD$\downarrow$}&\Th{AG$\uparrow$}&\Th{AI$\uparrow$}&\Th{Ins$\uparrow$}&\Th{Del$\downarrow$}\\\midrule
        - &73.42&30.16&15.23&29.99&58.47&17.47\\
        $1\times10^{-3}$ &\textbf{73.71}&29.52&15.17&30.03&59.23&\textbf{17.45}\\
        $2.5\times10^{-3}$ &72.99&30.53&15.82&30.56&59.04&17.96\\
        $5\times10^{-3}$ &72.46&30.10&16.06&30.67&57.47&17.80\\
        $7.5\times10^{-3}$ &72.86&\textbf{28.09}&\textbf{16.20}&\textbf{31.53}&58.76&17.57\\
        $1\times10^{-2}$ &73.28&28.97&15.75&31.16&58.99&17.50\\
        $1\times10^{-1}$ &73.00&28.93&16.13&31.55&\textbf{59.66}&17.95\\
        $1$ &73.30&28.44&16.02&31.31&58.64&17.48\\
        $10$ &73.04&29.28&15.23&30.47&58.74&17.47\\\bottomrule
        \end{tabular}
        \caption{\textbf{Regularization coefficient:} Evaluation of the regularization coefficient $\lambda$, using ResNet-18 with \textit{Cosine Similarity} on CIFAR-100.}%}
        \label{tab:variation}
\end{table}    

\noindent We observe that our method is not very sensible to the regularization coefficient and 
that the value of $7.5e^{-3}$ offers better performances and is thus selected as the default value 
for $\lambda$.

