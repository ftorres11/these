\chapter{A learning paradigm for interpretable gradients}
\chaptertoc{}
\label{ch:grad}
%--------------------------------------------------------------------------------------------------
\section{Introduction}
\label{sec:grad_intro}
\noindent A recurring issue faced by both neural networks and transformers is their inherent
lack of interpretability. These models are primarily optimized for high performance in their 
designated tasks. Yet, reflecting upon the information that can be drawn out of a 
model without too much  effort; we observe that the gradient of deep models displays the response 
of its parameters, to a given input. Many current interpretability methods are constructed based 
on this observation.\\

\noindent However,  the effective utilization of gradients in interpretability methods remains a 
pressing question. \textit{How can we leverage gradient better?}, previous interpretability 
approaches have relied on the stand alone gradient information such as Guided Backpropagation 
(\cite{guidedbackprop}) and Smoothgrad (\cite{smilkov2017smoothgrad}). On  another hand as 
seen in previous chapters some CAM variants are based on gradient, like Grad-CAM 
(\cite{selvaraju2017grad}), Grad-CAM++ (\cite{chattopadhay2018grad}) and Axiom-CAM (\cite{axiombased}).
Nevertheless, it is worth reflecting that gradient plays a more prominent role during 
the training phase of a model, particularly as a cornerstone in this process, we can not help but
reflect upon \textit{how can we leverage upon the gradient to improve interpretability during 
training?}.\\

\noindent In this chapter, we propose a modification to the training process of deep models by 
introducing of a regularization term to the error function. This term constrains
the gradient, aligning it with guided backpropagation in the input space.


%--------------------------------------------------------------------------------------------------
\section{Interpretable Gradients}
\label{sec:grad_defn}
\begin{figure}[t]
	\resizebox{\textwidth}{!}{
	\begin{tikzpicture}[
		scale=.3,
		font={\footnotesize},
		node distance=.5,
		label distance=3pt,
		wide/.style={yscale=.75},
		net/.style={draw,trapezium,trapezium angle=75,inner sep=3pt},
		enc/.style={net,shape border rotate=270},
		dec/.style={net,shape border rotate=90},
		txt/.style={inner sep=3pt},
		loose/.style={looseness=.6},
		sg/.style={blue!60},
	]
	\matrix[
		tight,row sep=6,column sep=16,
		cells={scale=.3,},
	] {
		\&\&\&\&\&
		\node[dec] (guided-back) {guided \\ backprop}; \&
		\node[wide,label=90:$\ipder{_G L_C}{x}$] (guided) {\fig[.1]{grad/fig/method/guided_gradient}}; \\
		\node[label=90:input image $x$] (in) {\fig[.1]{grad/fig/method/input}}; \&
		\node[enc] (net) {network \\ $f$}; \&\&\&
		\node[box] (class) {classification \\ loss $L_C$}; \&
		\node[dec] (back) {standard \\ backprop}; \&
		\node[wide,label=90:$\ipder{L_C}{x}$] (grad) {\fig[.1]{grad/fig/method/gradient}}; \&
		\node[box] (reg) {regularization \\ loss $L_R$}; \\
		\&\&\&\&\&\&\&
		\node[op] (plus) {$+$}; \&
		\node[txt] (loss) {total loss \\ $L = L_C + $\\ $\lambda L_R$}; \\
	};

	\draw[->]
		(in) edge (net)
		(net) edge node[above] {logits $y$} (class)
		(class) edge (back)
		(back) edge (grad)
		(guided-back) edge[sg] (guided)
		(grad) edge (reg)
		(reg) edge (plus)
		(plus) edge (loss)
		(class) edge[sg,out=90,in=180] (guided-back)
		(class) edge[loose,out=-50,in=180] (plus)
		(guided) edge[sg,out=0,in=90] node[pos=.2,font=\large,label=0:stop gradient] {//} (reg)
		(net) edge[dashed,out=60,in=150] (guided-back)
		(net) edge[dashed,out=30,in=150] (back)
		;
	\end{tikzpicture}
	}
	\caption{\emph{Interpretable gradient learning}. For an input image $x$, we obtain the logit 
	vector $y = f(x; \theta)$ by a forward pass through the network $f$ with parameters $\theta$. 
	We compute the classification loss $L_C$ by softmax and cross-entropy~\eq{class}, \eq{ce}. We 
	obtain the standard gradient $\ipder{L_C}{x}$ and guided gradient $\ipder{_G L_C}{x}$ by two 
	backward passes (dashed) and compute the regularization loss $L_R$ as the error between the 
	two~\eq{reg},\eq{mae}-\eq{cos}. The total loss is $L = L_C + \lambda L_R$~\eq{total}. Learning 
	is based on $\ipder{L}{\theta}$, which involves differentiation of the entire computational 
	graph except the guided backpropagation branch (blue).}
	\label{fig:grad_method}
\end{figure}

\paragraph{Preliminaries}

We consider an image classification network $f$ with parameters $\theta$, which maps an input image 
$x$ to a vector of class logits $y = f(x; \theta)$. At inference, we predict the class label of 
maximum confidence $\arg\max_j y_j$, where $y_j$ is the logit of class $i$. At training, given 
training images $X = \{x_i\}_{i=1}^n$ and target labels $T = \{t_i\}_{i=1}^n$, we compute the 
\emph{classification loss}

\begin{equation}
	L_C(X, \theta, T) = \frac{1}{n} \sum_{i=1}^n \ce(f(x_i; \theta), t_i),
\label{eq:class}
\end{equation}
where $\operatorname{CE}$ is softmax followed by cross-entropy:
\begin{equation}
	\ce(y, t) = -\log \frac{e^{y_t}}{\sum_i e^{y_i}} = -y_t + \sum_i e^{y_i}.
\label{eq:ce}
\end{equation}
Updates of parameters $\theta$ are then performed by an optimizer, based on the standard partial 
derivative (gradient) $\ipder{L_C}{\theta}$ of the classification loss $L_C$ with respect to 
$\theta$, obtained by standard back-propagation.

However, due to non-linearities like ReLU activations and downsampling like max-pooling or 
convolution stride $> 1$, the standard gradient is noisy \autocite{smilkov2017smoothgrad}. This 
is shown by visualizing the gradient $\ipder{L_C}{x}$ with respect to an input image $x$. By 
contrast, the guided gradient $\ipder{_G L_C}{x}$ \autocite{guidedbackprop} does not suffer much 
from noise and preserves sharp details. The difference of the two gradients is illustrated in 
\autoref{fig:grad_method}.

\paragraph{Regularization}
The main idea of this work is to introduce a regularization term during training, which will make 
the standard gradient $\ipder{L_C}{x}$ behave similarly to the corresponding guided gradient 
$\ipder{_G L_C}{x}$, while maintaining the predictive power of the classifier $f$. We hypothesize 
that, if possible, this will improve the quality of all gradients with respect to intermediate 
activations and therefore the quality of saliency maps obtained by CAM-based methods 
(\cite{zhou2016learning}, \cite{selvaraju2017grad}, \cite{chattopadhay2018grad}, 
\cite{wang2020score}) and the interpretability of network $f$. The effect may be 
similar to that of SmoothGrad \autocite{smilkov2017smoothgrad}, but without the need for several 
forward passes at inference.

\noindent In particular, given an input image $x$, we perform a forward pass through $f$ and 
compute the logit vectors $y_i = f(x_i, \theta)$ and the classification loss $L_C(X, \theta, T)$
~\eq{class}. We then obtain the standard gradients $\delta x_i = \ipder{L_C}{x_i}$ and the guided 
gradients $\delta_G x_i = \ipder{_G L_C}{x_i}$ with respect to the input images $x_i$ by two 
separate backward passes. Since the whole process is differentiable (\wrt $\theta$) at training, 
we stop the gradient computation of the latter, so that it only serves as a ``teacher''. We define 
the \emph{regularization loss} 

\begin{equation}
	L_R(X, \theta, T) = \frac{1}{n} \sum_{i=1}^n E(\delta x_i, \delta_G x_i),
\label{eq:reg}
\end{equation}
where $E$ is an error function between the two gradient images, considered below.\\

\noindent Finally, the total loss is defined as
\begin{equation}
	L(x, \theta, t) = L_C(x, \theta, t) + \lambda L_R(x, \theta, t),
\label{eq:total}
\end{equation}
where $\lambda$ is a hyperparameter determining the regularization coefficient. 
$\lambda$ should be large enough to smooth the gradient without decreasing the classification 
accuracy or hurting the training process. 
Updates of the network parameters $\theta$ are now based on the gradient $\ipder{L}{\theta}$ 
\wrt the total loss, using any optimizer. At inference, one may use any interpretability method 
to obtain a saliency map at any layer.

\paragraph{Algorithm}
Our method is summarized in Algorithm \ref{alg:grad} %, where \textsc{detach} stops gradient computation, 
and illustrated in \ref{fig:grad_method}.It is interesting to note that the entire computational 
graph depicted in \ref{fig:grad_method} involves one forward and two backward passes. This graph is 
then differentiated again to compute $\ipder{L}{\theta}$, which involves one more forward and 
backward pass, since the guided backpropagation branch is excluded. Thus, each training iteration 
requires five passes through $f$ instead of two in a standard training.

%------------------------------------------------------------------------------

\begin{algorithm}[H]    
	\SetFuncSty{textsc}
	\SetDataSty{emph}
	\newcommand{\commentsty}[1]{{\color{DarkGreen}#1}}
	\SetCommentSty{commentsty}
	\SetKwComment{Comment}{$\triangleright$ }{}

	\SetKwFunction{Detatch}{detatch}

	\KwIn{network $f$, parameters $\theta$}
	\KwIn{input images $X = \{x_i\}_{i=1}^n$}
	\KwIn{target labels $T = \{t_i\}_{i=1}^n$}
	\KwOut{loss $L$}
	$L_C \gets \frac{1}{n} \sum_i \ce(f(x_i; \theta), t_i)$ \Comment*[f]{class. loss \eq{class}} \\
    \ForEach{$i \in \{1,\dots,n\}$}{
        $\delta x_i \gets \ipder{L_C}{x_i}$ \Comment*[f]{standard grad} \\
	 	$\delta_G x_i \gets \ipder{_G L_C}{x_i}$ \Comment*[f]{guided grad} \\
        $\mbox{\Th{Detach}}(\delta_G x_i)$ \Comment*[f]{detach from graph} \\
	  }
   $L_R \gets \frac{1}{n} \sum_{i=1}^n E(\delta x_i, \delta_G x_i) $ \Comment*[f]{reg. loss \eq{reg}}\\
   $L \gets L_C + \lambda L_R$
 \Comment*[f]{total loss ~\eq{total}} \\

\caption{Interpretable gradient loss}
\label{alg:grad}
\end{algorithm}

%------------------------------------------------------------------------------
\paragraph{Error function}

Given two gradient images $\delta, \delta'$ consisting of $p$ pixels each, we consider the 
following error functions $E$ to compute the regularization loss~\eq{reg}.


%\begin{enumerate}[itemsep=2pt, parsep=0pt, topsep=3pt]
%    \item Sigmoid Masking
%    \begin{equation}
%        E_{\sigma}(\delta,\delta') = - \sum_{i=0}^p \delta'\odot\frac{e^{\delta}}{e^{\delta}+1}.
%    \end{equation}
%    \item Centered Sigmoid Masking
%    \begin{equation}
%        E_{\sigma}(\delta,\delta') = - \sum_{i=0}^p \delta'\odot\frac{e^{\delta-\mu_\delta}}{e^{\delta-\mu_\delta}+1}.
%    \end{equation}
%\end{enumerate}


\begin{enumerate}[itemsep=2pt, parsep=0pt, topsep=3pt]

 	\item Mean absolute error:
 	\begin{equation}
 		E_{\mae}(\delta, \delta') = \frac{1}{p} \norm{\delta - \delta'}_1.
 	\label{eq:mae}
    \end{equation}  
 	\item Mean squared error:
 	\begin{equation}
 		E_{\mse}(\delta, \delta') = \frac{1}{p} \norm{\delta - \delta'}_2^2.
 	\label{eq:mse}
 	\end{equation}

\end{enumerate}

We also consider the following two similarity functions, with a negative sign.
\begin{enumerate}[itemsep=2pt, parsep=0pt, topsep=3pt]
	\setcounter{enumi}{2}
	\item Cosine similarity:
	\begin{equation}
		E_{\cos}(\delta, \delta') = -\frac{\inner{\delta, \delta'}}{\norm{\delta}_2 \norm{\delta'}_2},
	\label{eq:cos}
	\end{equation}
 
	\item Histogram intersection:
	\begin{equation}
		E_{\hi}(\delta, \delta') = -\sum_{i=0}^p 
			\frac{\min(\abs{\delta_i}, \abs{\delta'_i})}{\norm{\delta}_1 \norm{\delta'}_1}.
	\label{eq:hi}
	\end{equation}

	where $\inner{,}$ denotes inner product.

\end{enumerate}

% \begin{enumerate}[itemsep=2pt, parsep=0pt, topsep=3pt]
% 	\item Mean absolute error:
% 	\begin{equation}
% 		E_{\mae}(\delta, \delta') = \frac{1}{p} \norm{\delta - \delta'}_1.
% 	\label{eq:mae}
% 	\end{equation}
% 	\item Mean squared error:
% 	\begin{equation}
% 		E_{\mse}(\delta, \delta') = \frac{1}{p} \norm{\delta - \delta'}_2^2.
% 	\label{eq:mse}
% 	\end{equation}
% \end{enumerate}

% We also consider the following two similarity functions, with a negative sign.

%\green{Do we mention the changes in the architecture so the }

%--------------------------------------------------------------------------------------------------
\section{Experiments}
This section presents the experimental settings, our evaluation metrics and results.
%Finally, we present qualitative and quantitative results and an ablation study.\\

\subsection{Experimental Set-up}
In the following sections, we evaluate recognition properties and interpretability capabilities of 
our approach. Specifically, we generate explanations following popular attribution methods derived 
from CAM \autocite{zhou2016learning} from the \textbf{pytorch-grad-cam} library from Jacob 
Gildenblat\footnote{https://github.com/jacobgil/pytorch-grad-cam}.

\paragraph{Dataset}
We train and evaluate our models on CIFAR-100 \autocite{krizhevsky2009learning}. This dataset 
contains 60,000 images of 100 categories, split in 50,000 for training and 10,000 for testing. Each 
image has a resolution of $32\times32$ pixels. This dataset is chosen because of its ease of usage 
and prototyping properties. 

\paragraph{Settings}
To obtain competitive performance and ensure the replicability of our method, we follow the 
methodology found in the repository by weiaicunzai 
\footnote{https://github.com/weiaicunzai/pytorch-cifar100}. Thus, we train each model following the 
same training procedure. We perform 200 epochs, with a starting learning rate of $10^{-1}$, a 
batch-size of 128 images, SGD optimizer and a learning rate policy updating said parameter by 
division over 5 on epochs 60, 120 and 160.  

\section{Qualitative Evaluation}
\label{sec:grad_qual}
%--------------------------------------------------------------------------------------------------
\section{Quantitative Evaluation}
\label{sec:grad_quant}
We evaluate the effect of training a given model using our proposed approach with 
\textit{Faithfulness} and \textit{Causality}. Results are reported in Table \ref{tab:C100_quant}.
\begin{table}[t]
    \centering
    \scriptsize
    \setlength{\tabcolsep}{2.5pt}
        \begin{tabular}{lcccccc}\\\toprule
            \mc{7}{\Th{\textbf{Recognition Metrics}}}\\\midrule
            \Th{Model}&\Th{Error}& &\Th{$\lambda$}&\Th{Acc}$\uparrow$& &\\\midrule
            \mr{2}{\Th{ResNet-18}}&-& &-&\Th{\textbf{73.42}}& &\\
             &\Th{Cosine}& &$7.5\times10^{-3}$&\Th{72.86}& &\\\midrule
             
            \mr{2}{\Th{MobileNet-V2}}&-& &-&\Th{59.43}&\\
             &\Th{Cosine}&  &$1\times10^{-3}$&\Th{\textbf{62.36}}&\\\midrule
            
            \mc{7}{\Th{\textbf{Interpretable Recognition Metrics}}}\\\midrule    
            \mc{7}{\Th{ResNet-18}}\\\midrule
            \Th{Method}&\Th{Error}&\Th{AD$\downarrow$}&\Th{AG$\uparrow$}&\Th{AI$\uparrow$}&\Th{Ins$\uparrow$}&\Th{Del$\downarrow$}\\\hline
            \mr{2}{\Th{Grad-CAM}}&-&30.16&15.23&29.99&58.47&\textbf{17.47}\\
             &\Th{Cosine}&\textbf{28.09}&\textbf{16.19}&\textbf{31.53}&\textbf{58.76}&17.57\\\hline
            \mr{2}{\Th{Grad-CAM++}}&-&31.40&14.17&28.47&58.61&\textbf{17.05}\\
              &\Th{Cosine}&\textbf{29.78}&\textbf{15.07}&\textbf{29.60}&\textbf{58.90}&17.22\\\hline
            \mr{2}{\Th{Score-CAM}}&-&26.49&18.62&33.84&58.42&\textbf{18.31}\\
              &\Th{Cosine}&\textbf{24.82}&\textbf{19.49}&\textbf{35.51}&\textbf{59.11}&18.34\\\hline
            \mr{2}{\Th{Ablation-CAM}}&-&31.96&14.02&28.33&58.36&\textbf{17.14}\\
             &\Th{Cosine}&\textbf{29.90}&\textbf{15.03}&\textbf{29.61}&\textbf{58.70}&17.37\\\hline
            \mr{2}{\Th{Axiom-CAM}}&-&30.16&15.23&29.98&58.47&\textbf{17.47}\\
              &\Th{Cosine}&\textbf{28.09}&\textbf{16.20}&\textbf{31.53}&\textbf{58.76}&17.57\\\midrule
    
            \mc{7}{\Th{MobileNet-V2}}\\\midrule
            \Th{Method}&\Th{Error}&\Th{AD$\downarrow$}&\Th{AG$\uparrow$}&\Th{AI$\uparrow$}&\Th{Ins$\uparrow$}&\Th{Del$\downarrow$}\\\hline
            \mr{2}{\Th{Grad-CAM}}&-&44.64&6.57&25.62&44.64&\textbf{14.34}\\
             &\Th{Cosine}&\textbf{40.89}&\textbf{7.31}&\textbf{27.08}&\textbf{45.57}&15.20\\\hline
            \mr{2}{\Th{Grad-CAM++}}&-&45.98&6.12&24.10&44.72&\textbf{14.76}\\
              &\Th{Cosine}&\textbf{40.76}&\textbf{6.85}&\textbf{26.46}&\textbf{45.51}&14.92\\\hline
            \mr{2}{\Th{Score-CAM}}&-&40.55&7.85&28.57&45.62&\textbf{14.52}\\
              &\Th{Cosine}&\textbf{36.34}&\textbf{9.09}&\textbf{30.50}&\textbf{46.35}&14.72\\\hline
            \mr{2}{\Th{Ablation-CAM}}&-&45.15&6.38&25.32&44.62&\textbf{15.03}\\
              &\Th{Cosine}&\textbf{41.13}&\textbf{7.03}&\textbf{26.10}&\textbf{45.38}&15.12\\\hline
            \mr{2}{\Th{Axiom-CAM}}&-&44.65&6.57&25.62&44.64&15.27\\
              &\Th{Cosine}&\textbf{40.89}&\textbf{7.31}&\textbf{27.08}&\textbf{45.57}&\textbf{15.20}\\\bottomrule
    
        \end{tabular}
        \caption{\textbf{Cosine Regularization Experiments: } on CIFAR-100 with ResNet-18 and MobileNet-V2. Accuracy and interpretability metrics are reported.}
        \label{tab:C100_quant}
\end{table}
\noindent We observe that our method offers a consistent improvement in terms of interpretability 
metrics. Specifically, we obtain improvements on both networks and systematically on five out of 
six metrics. The improvements are higher for AD, AG, and AI. Insertion gets a smaller but consistent 
improvement and Deletion is almost always worse with our method, but with a very small difference.
This decrease in performance of Deletion may be due to some limitations of the metrics as reported 
in Chapter \ref{ch:opticam}.
It is interesting to note that improvements on Score-CAM means that our training not only improves 
gradient for interpretability, but also builds better activation maps.\\

\subsection{Ablation Experiments}
We conduct ablation experiments using ResNet18. In these experiments we analyze the different 
regularization proposals mentioned in Section \ref{sec:grad_defn} and the impact of the 
regularization coefficient.

\paragraph{Regularization proposals} 
To validate our selection of regularization function, we train several models following the same 
training regime while varying the error function. To evaluate these approaches, we focus solely on 
Grad-CAM attributions. Results are reported in Table \ref{tab:Regs}\\
\begin{table}[H]
    \centering
    \scriptsize
    \setlength{\tabcolsep}{3.5pt}
        \begin{tabular}{lcccccc}\\\toprule
        \mc{7}{Regularization Selection}\\\midrule
        \Th{Regularizer}&\Th{Acc}&\Th{AD$\downarrow$}&\Th{AG$\uparrow$}&\Th{AI$\uparrow$}&\Th{Ins$\uparrow$}&\Th{Del$\downarrow$}\\\midrule
        - &73.42&30.16&15.23&29.99&58.47&17.47\\
        Cosine&72,86&\textbf{28.09}&\textbf{16.19}&\textbf{31.53}&58.76&17.57\\
        Histogram &73.88&30.39&14.78&29.38&58.52&17.35\\
        MAE & 73,41& 30,33 & 15,06 &29,61 & 58,13 & 17,95\\
        MSE & 73,86& 29,64 & 15,19 &30,11 & \textbf{59,05} & 18,02\\\bottomrule
        \end{tabular}
        \caption{\textbf{Regularization selection: } Evaluation of the four proposed regularization with ResNet-18 on CIFAR-100.}%}
        \label{tab:Regs}
\end{table}
\noindent Following these results, we observe a consistent improvement on most metrics for all 
regularizer options. We note that the accuracy remains stable within half a percent of the original 
model. However, we note that most options struggle regarding deletion. Cosine Similarity however 
manages to provide improvements in most metrics, while maintaining deletion performance.

\paragraph{Regularization coefficient}
Finally, we study the behavior of the regularization coefficient $\lambda$ in \ref{eq:total}. We 
train multiple models with \textit{Cosine Similarity} and a range of values for $\lambda$, see Table 
\ref{tab:variation}.\\

\begin{table}[H]
    \centering
    \scriptsize
    \setlength{\tabcolsep}{3.5pt}
        \begin{tabular}{lcccccc}\\\toprule
        \mc{7}{Regularization Selection}\\\midrule
        $\lambda$&\Th{Acc}&\Th{AD$\downarrow$}&\Th{AG$\uparrow$}&\Th{AI$\uparrow$}&\Th{Ins$\uparrow$}&\Th{Del$\downarrow$}\\\midrule
        - &73.42&30.16&15.23&29.99&58.47&17.47\\
        $1\times10^{-3}$ &\textbf{73.71}&29.52&15.17&30.03&59.23&\textbf{17.45}\\
        $2.5\times10^{-3}$ &72.99&30.53&15.82&30.56&59.04&17.96\\
        $5\times10^{-3}$ &72.46&30.10&16.06&30.67&57.47&17.80\\
        $7.5\times10^{-3}$ &72.86&\textbf{28.09}&\textbf{16.20}&\textbf{31.53}&58.76&17.57\\
        $1\times10^{-2}$ &73.28&28.97&15.75&31.16&58.99&17.50\\
        $1\times10^{-1}$ &73.00&28.93&16.13&31.55&\textbf{59.66}&17.95\\
        $1$ &73.30&28.44&16.02&31.31&58.64&17.48\\
        $10$ &73.04&29.28&15.23&30.47&58.74&17.47\\\bottomrule
        \end{tabular}
        \caption{\textbf{Regularization coefficient:} Evaluation of the regularization coefficient $\lambda$, using ResNet-18 with \textit{Cosine Similarity} on CIFAR-100.}%}
        \label{tab:variation}
\end{table}    

\noindent We observe that our method is not very sensible to the regularization coefficient and 
that the value of $7.5e^{-3}$ offers better performances and is thus selected as the default value 
for $\lambda$.


%--------------------------------------------------------------------------------------------------
\section{Discussion}
\label{sec:grad_discussion}
\paragraph{Guided Backpropagation and Smoothgrad} To begin with, we note that guided 
backpropagation is not the sole updated representation of gradient we can generate for a specific 
model. However, it is important to highlight the requirements for efficiency. On one hand, guided 
backpropagation only requires two passes through the network: one forward pass and one backward 
pass. On the other hand, computation of smoothgrad requires several passes throw the network. By 
default, this approach involves five forward-backward passes; this would mean a noticeable 
increase of training time.

%% Make clear to differentiate from P1. Make clear that it is for the whole thing
\paragraph{Passes through the network} Computational complexity and optimized 
training are the main challenges regarding the scaling of our approach. In particular, 
in \autoref{sec:grad_defn}, we mention that \emph{each training iteration requires five passes 
through $f$ instead of two in a standard training}. We compute the first forward-backward pass 
to generate the guided gradient in the input space. The second forward-backward pass generates the 
standard gradient. Finally, to generate the gradient required to update the weights, we do a final 
backward pass taking into consideration cross entropy and the regularization.\\

%% Make Clear
\noindent Still, \emph{why can't it be done with fewer passes?} In theory, 
guided backpropagation is calculated by changing the activation functions; for instance ReLU. 
In practice, activation functions work as class objects within pytorch. Introducing 
modifications into these objects inherently increases their complexity. For example, a modification 
to ReLU to account for \emph{Guided ReLU} could be achieved with the introduction of an \emph{if-else} 
case. Assuming this would work, the amount of these activations in the model would lead to 
bottlenecks in running time evaluating the condition controlling the gradient behavior. 

\paragraph{Saliency Map Visualization} We validate our claims of interpretability by 
conducting evaluation using saliency maps. However, upon visualization we note a high degree of 
sparsity covering the input images. In this aspect we observe that following the hierarchical nature 
of \glspl{cnn} and the reduction of feature map size leads to deep representations of reduced 
spatial dimensions. Furthermore, since CIFAR-100 contains images with $32\times32$ spatial 
resolution, intermediary activations are chosen to avoid generating attributions, using a 
$1\times1$ patch.

\paragraph{Gradient  Visualization} While we display a comparison between gradients 
computed for models trained with and without our training protocol, a further assessment is not 
considered. In particular, we observe that pure gradient-based interpretability approaches do not 
incur in quantitative measurements to validate interpretability claims. Conversely, we hypothesize 
that since other attribution proposals rely on this information, denoising this leads to 
explainable improvements. This is ultimately confirmed with our evaluation procedure.
%--------------------------------------------------------------------------------------------------
\section{Conclusion}
%% Check
In this chapter, we propose a transparency methodology that improves interpretability properties 
of \glspl{cnn}, by denoising the gradient on the input level. Our approach constrains the 
gradient calculated through backpropagation, aligning this information in the input space with a 
refined representation obtained through guided backpropagation.\\

\noindent We validate our claims on improvement of interpretability properties using a post-hoc 
interpretability evaluation procedure. We find that our methodology not only 
improves these properties but also boosts recognition performance. However, we 
observe that an optimized version of this study better suited for larger collections of data and 
more complex models is desired. Further research is desired to address these limitations.
For instance, an optimization of our training paradigm reducing its computational cost, would 
make the method more scalable.

\newpage

