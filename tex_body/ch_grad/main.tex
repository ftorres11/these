\chapter{A learning paradigm for interpretable gradients}
\chaptertoc{}
\label{ch:grad}
%--------------------------------------------------------------------------------------------------
\section{Introduction}
\noindent A recurring issue faced by both neural networks and transformers is their inherent
lack of interpretability. These models are primarily optimized for high performance in their 
designated tasks. Yet, reflecting upon the information that can be drawn out of a 
model without too much an effort; we observe that the gradient of any given model although noisy,
contains relevant information regarding the behavior after a given input, after all because of this,
several interpretability methods are designed around it.\\

\noindent However,  the effective utilization of gradients in interpretability methods remains a 
pressing question. \textit{How can we leverage gradient better?}, previous interpretability 
approaches have relied on the stand alone gradient information such as Guided Backpropagation 
(\cite{guidedbackprop}) and Smoothgrad (\cite{smilkov2017smoothgrad}). On  another hand as 
seen in previous chapters some CAM variants are designed around it, like Grad-CAM 
(\cite{selvaraju2017grad}), Grad-CAM++ (\cite{chattopadhay2018grad}) and Axiom-CAM (\cite{axiombased}).
Nevertheless, it is worth reflecting that gradient plays a more prominent role during 
the training phase of a model, particularly as a cornerstone in this process, we can not help but
reflect upon \textit{how can we leverage upon the gradient for interpretability during training?}.\\

\noindent In this chapter, we propose a modification to the training process of deep models by 
introducing of a regularization term to the error function. This term is aimed at constraining 
the gradient by aligning it with guided back-propagation in the input space.


%The improvements of deep learning methods in the last decade has led to very large developments and 
%a large diffusion to most sectors. Since these models are computationally complex and opaque, the 
%requirements for explanations or interpretable models receives a lot of attention nowadays 
%\cite{mythos_interp}. Explanations and transparency becomes a legal requirements for diverse types 
%of systems that are used to assist high-stakes decisions.


%In this work, we study deep learning models visual interpretability.
%Model interpretability is often categorized into transparency or post-hoc methods. Transparency 
%aims at producing interpretable models where the inner process or part of it can be understood. 
%Post-hoc methods consider models as black-boxes and interpret decisions mainly based on inputs and 
%outputs. Regarding visual recognition models post-hoc methods producing saliency maps are mostly 
%used as they directly highlight the most important areas of an image related to the decision.
%Initial works focused on gradient information such as guided back-propagation \cite{guidedbackprop}.
%CAM \cite{cam} later presented a class-specific linear combination of feature maps that opened the 
%way to numerous weighting strategies.
%Several methods use gradient information \cite{gradcam}, masking \cite{scorecam}, etc.
%GradCAM \cite{gradcam} and other variants \cite{omeiza2019smooth} are particularly successful 
%gradient methods.Interestingly SmoothGrad \cite{smoothgrad} and SmoothGrad-CAM++ \cite{omeiza2019smooth} 
%improve the precision of saliency maps by denoising gradient.
%These CAM-based methods are the most popular and best performing methods at the moment.

%In this work, we follow a similar idea and propose a learning paradigm for model training that 
%constraint gradient in order to improve the performance of interpretability methods.
%In fact, we add a regularization term to the loss function pulling the gradient in the input space 
%to align with the values obtained by guided back-propagation. 
%Such regularization has a smoothing effect on gradient and is expected to improve the precision of 
%model interpretations.


%More specifically the Figure \ref{fig:method} summarizes our method.
%When training each input image is forwarded to the network to compute the cross-entropy loss. 
%Then a standard and guided back-propagations are computed back to the input image space, where our 
%regularization term is computed. This term is added to the loss and back-propagated only through the 
%standard back-propagation branch.



%The key contributions of this learning interpretable gradients approach follows:
%\begin{itemize}
%    \item We introduce a new learning framework with regularized gradient.
%    \item The proposed method is further evaluated on several architectures both in term of accuracy and interpretability metrics.
%\end{itemize}


\section{Preliminaries}

\paragraph{Double backpropagation}

\paragraph{Guided backpropagation}

\newpage
\section{Interpretable Gradients}
\label{sec:grad_defn}
\begin{figure}[t]
	\resizebox{\textwidth}{!}{
	\begin{tikzpicture}[
		scale=.3,
		font={\footnotesize},
		node distance=.5,
		label distance=3pt,
		wide/.style={yscale=.75},
		net/.style={draw,trapezium,trapezium angle=75,inner sep=3pt},
		enc/.style={net,shape border rotate=270},
		dec/.style={net,shape border rotate=90},
		txt/.style={inner sep=3pt},
		loose/.style={looseness=.6},
		sg/.style={blue!60},
	]
	\matrix[
		tight,row sep=6,column sep=16,
		cells={scale=.3,},
	] {
		\&\&\&\&\&
		\node[dec] (guided-back) {guided \\ backprop}; \&
		\node[wide,label=90:$\ipder{_G L_C}{x}$] (guided) {\fig[.1]{grad/fig/method/guided_gradient}}; \\
		\node[label=90:input image $x$] (in) {\fig[.1]{grad/fig/method/input}}; \&
		\node[enc] (net) {network \\ $f$}; \&\&\&
		\node[box] (class) {classification \\ loss $L_C$}; \&
		\node[dec] (back) {standard \\ backprop}; \&
		\node[wide,label=90:$\ipder{L_C}{x}$] (grad) {\fig[.1]{grad/fig/method/gradient}}; \&
		\node[box] (reg) {regularization \\ loss $L_R$}; \\
		\&\&\&\&\&\&\&
		\node[op] (plus) {$+$}; \&
		\node[txt] (loss) {total loss \\ $L = L_C + $\\ $\lambda L_R$}; \\
	};

	\draw[->]
		(in) edge (net)
		(net) edge node[above] {logits $y$} (class)
		(class) edge (back)
		(back) edge (grad)
		(guided-back) edge[sg] (guided)
		(grad) edge (reg)
		(reg) edge (plus)
		(plus) edge (loss)
		(class) edge[sg,out=90,in=180] (guided-back)
		(class) edge[loose,out=-50,in=180] (plus)
		(guided) edge[sg,out=0,in=90] node[pos=.2,font=\large,label=0:stop gradient] {//} (reg)
		(net) edge[dashed,out=60,in=150] (guided-back)
		(net) edge[dashed,out=30,in=150] (back)
		;
	\end{tikzpicture}
	}
	\caption{}
	\label{fig:grad_method}
\end{figure}
\newpage
\section{Qualitative Evaluation}
\label{sec:grad_qual}
\newpage
%--------------------------------------------------------------------------------------------------
\subsection{Quantitative Evaluation}
We evaluate the effect of training a given model using our proposed approach with 
\textit{Faithfulness} and \textit{Causality}. Results are reported in Table \ref{tab:C100_quant}.
\begin{table}[t]
    \centering
    \scriptsize
    \setlength{\tabcolsep}{2.5pt}
        \begin{tabular}{lcccccc}\\\toprule
            \mc{7}{\Th{\textbf{Recognition Metrics}}}\\\midrule
            \Th{Model}&\Th{Error}& &\Th{$\lambda$}&\Th{Acc}$\uparrow$& &\\\midrule
            \mr{2}{\Th{ResNet-18}}&-& &-&\Th{\textbf{73.42}}& &\\
             &\Th{Cosine}& &$7.5\times10^{-3}$&\Th{72.86}& &\\\midrule
             
            \mr{2}{\Th{MobileNet-V2}}&-& &-&\Th{59.43}&\\
             &\Th{Cosine}&  &$1\times10^{-3}$&\Th{\textbf{62.36}}&\\\midrule
            
            \mc{7}{\Th{\textbf{Interpretable Recognition Metrics}}}\\\midrule    
            \mc{7}{\Th{ResNet-18}}\\\midrule
            \Th{Method}&\Th{Error}&\Th{AD$\downarrow$}&\Th{AG$\uparrow$}&\Th{AI$\uparrow$}&\Th{Ins$\uparrow$}&\Th{Del$\downarrow$}\\\hline
            \mr{2}{\Th{Grad-CAM}}&-&30.16&15.23&29.99&58.47&\textbf{17.47}\\
             &\Th{Cosine}&\textbf{28.09}&\textbf{16.19}&\textbf{31.53}&\textbf{58.76}&17.57\\\hline
            \mr{2}{\Th{Grad-CAM++}}&-&31.40&14.17&28.47&58.61&\textbf{17.05}\\
              &\Th{Cosine}&\textbf{29.78}&\textbf{15.07}&\textbf{29.60}&\textbf{58.90}&17.22\\\hline
            \mr{2}{\Th{Score-CAM}}&-&26.49&18.62&33.84&58.42&\textbf{18.31}\\
              &\Th{Cosine}&\textbf{24.82}&\textbf{19.49}&\textbf{35.51}&\textbf{59.11}&18.34\\\hline
            \mr{2}{\Th{Ablation-CAM}}&-&31.96&14.02&28.33&58.36&\textbf{17.14}\\
             &\Th{Cosine}&\textbf{29.90}&\textbf{15.03}&\textbf{29.61}&\textbf{58.70}&17.37\\\hline
            \mr{2}{\Th{Axiom-CAM}}&-&30.16&15.23&29.98&58.47&\textbf{17.47}\\
              &\Th{Cosine}&\textbf{28.09}&\textbf{16.20}&\textbf{31.53}&\textbf{58.76}&17.57\\\midrule
    
            \mc{7}{\Th{MobileNet-V2}}\\\midrule
            \Th{Method}&\Th{Error}&\Th{AD$\downarrow$}&\Th{AG$\uparrow$}&\Th{AI$\uparrow$}&\Th{Ins$\uparrow$}&\Th{Del$\downarrow$}\\\hline
            \mr{2}{\Th{Grad-CAM}}&-&44.64&6.57&25.62&44.64&\textbf{14.34}\\
             &\Th{Cosine}&\textbf{40.89}&\textbf{7.31}&\textbf{27.08}&\textbf{45.57}&15.20\\\hline
            \mr{2}{\Th{Grad-CAM++}}&-&45.98&6.12&24.10&44.72&\textbf{14.76}\\
              &\Th{Cosine}&\textbf{40.76}&\textbf{6.85}&\textbf{26.46}&\textbf{45.51}&14.92\\\hline
            \mr{2}{\Th{Score-CAM}}&-&40.55&7.85&28.57&45.62&\textbf{14.52}\\
              &\Th{Cosine}&\textbf{36.34}&\textbf{9.09}&\textbf{30.50}&\textbf{46.35}&14.72\\\hline
            \mr{2}{\Th{Ablation-CAM}}&-&45.15&6.38&25.32&44.62&\textbf{15.03}\\
              &\Th{Cosine}&\textbf{41.13}&\textbf{7.03}&\textbf{26.10}&\textbf{45.38}&15.12\\\hline
            \mr{2}{\Th{Axiom-CAM}}&-&44.65&6.57&25.62&44.64&15.27\\
              &\Th{Cosine}&\textbf{40.89}&\textbf{7.31}&\textbf{27.08}&\textbf{45.57}&\textbf{15.20}\\\bottomrule
    
        \end{tabular}
        \caption{Experiments with cosine error function on CIFAR-100 with ResNet-18 and MobileNet-V2. Accuracy and interpretability metrics are reported.}
        \label{tab:C100_quant}
\end{table}
\noindent We observe that our method offers a consistent improvement in terms of interpretability 
metrics. Specifically, we obtain improvements on both networks and systematically on five out of 
six metrics. The improvements are higher for AD, AG, and AI. Insertion gets a smaller but consistent 
improvement and Deletion is almost always worse with our method, but with a very small difference.
This decrease in performance of Deletion may be due to some limitations of the metrics as reported 
in Chapter \ref{ch:opticam}.
It is interesting to note that improvements on Score-CAM means that our training not only improves 
gradient for interpretability, but also builds better activation maps.\\

\subsection{Ablation Experiments}
We conduct ablation experiments using ResNet18. In these experiments we analyze the different 
regularization proposals mentioned in Section \ref{sec:grad_defn} and the impact of the 
regularization coefficient.

\paragraph{Regularization proposals} 
To validate our selection of regularization function, we train several models following the same 
training regime while varying the error function. To evaluate these approaches, we focus solely on 
Grad-CAM attributions. Results are reported in Table \ref{tab:Regs}\\
\begin{table}[H]
    \centering
    \scriptsize
    \setlength{\tabcolsep}{3.5pt}
        \begin{tabular}{lcccccc}\\\toprule
        \mc{7}{Regularization Selection}\\\midrule
        \Th{Regularizer}&\Th{Acc}&\Th{AD$\downarrow$}&\Th{AG$\uparrow$}&\Th{AI$\uparrow$}&\Th{Ins$\uparrow$}&\Th{Del$\downarrow$}\\\midrule
        - &73.42&30.16&15.23&29.99&58.47&17.47\\
        Cosine&72,86&\textbf{28.09}&\textbf{16.19}&\textbf{31.53}&58.76&17.57\\
        Histogram &73.88&30.39&14.78&29.38&58.52&17.35\\
        MAE & 73,41& 30,33 & 15,06 &29,61 & 58,13 & 17,95\\
        MSE & 73,86& 29,64 & 15,19 &30,11 & \textbf{59,05} & 18,02\\\bottomrule
        \end{tabular}
        \caption{\textbf{Regularization selection: } Evaluation of the four proposed regularization with ResNet-18 on CIFAR-100.}%}
        \label{tab:Regs}
\end{table}
\noindent Following these results, we observe a consistent improvement on most metrics for all 
regularizer options. We note that the accuracy remains stable within half a percent of the original 
model. However, we note that most options struggle regarding deletion. Cosine Similarity however 
manages to provide improvements in most metrics, while maintaining deletion performance.

\paragraph{Regularization coefficient}
Finally, we study the behavior of the regularization coefficient $\lambda$ in \ref{eq:total}. We 
train multiple models with \textit{Cosine Similarity} and a range of values for $\lambda$, see Table 
\ref{tab:variation}.\\

\begin{table}[H]
    \centering
    \scriptsize
    \setlength{\tabcolsep}{3.5pt}
        \begin{tabular}{lcccccc}\\\toprule
        \mc{7}{Regularization Selection}\\\midrule
        $\lambda$&\Th{Acc}&\Th{AD$\downarrow$}&\Th{AG$\uparrow$}&\Th{AI$\uparrow$}&\Th{Ins$\uparrow$}&\Th{Del$\downarrow$}\\\midrule
        - &73.42&30.16&15.23&29.99&58.47&17.47\\
        $1\times10^{-3}$ &\textbf{73.71}&29.52&15.17&30.03&59.23&\textbf{17.45}\\
        $2.5\times10^{-3}$ &72.99&30.53&15.82&30.56&59.04&17.96\\
        $5\times10^{-3}$ &72.46&30.10&16.06&30.67&57.47&17.80\\
        $7.5\times10^{-3}$ &72.86&\textbf{28.09}&\textbf{16.20}&\textbf{31.53}&58.76&17.57\\
        $1\times10^{-2}$ &73.28&28.97&15.75&31.16&58.99&17.50\\
        $1\times10^{-1}$ &73.00&28.93&16.13&31.55&\textbf{59.66}&17.95\\
        $1$ &73.30&28.44&16.02&31.31&58.64&17.48\\
        $10$ &73.04&29.28&15.23&30.47&58.74&17.47\\\bottomrule
        \end{tabular}
        \caption{\textbf{Regularization coefficient:} Evaluation of the regularization coefficient $\lambda$, using ResNet-18 with \textit{Cosine Similarity} on CIFAR-100.}%}
        \label{tab:variation}
\end{table}    

\noindent We observe that our method is not very sensible to the regularization coefficient and 
that the value of $7.5e^{-3}$ offers better performances and is thus selected as the default value 
for $\lambda$.


\newpage

