\documentclass[a4paper,twoside]{article}
\usepackage{epsfig}
\usepackage{subcaption}
\usepackage{calc}
\usepackage{amsfonts}
\usepackage{bbm}
\usepackage{amssymb}
\usepackage{amstext}
\usepackage{amsmath}
\usepackage{amsthm}
\usepackage{multicol}
\usepackage{pslatex}
\usepackage{apalike}
\usepackage[ruled,vlined]{algorithm2e}
\usepackage[bottom]{footmisc}
\usepackage{enumitem}
\usepackage{booktabs}
\usepackage{float}
\usepackage{multirow}
\usepackage{algorithmicx}
\input{tex/plots}
\floatstyle{plaintop}
\restylefloat{table}


\usepackage{SCITEPRESS}     % Please add other packages that you may need BEFORE the SCITEPRESS.sty package.


\begin{document}
\input{tex/abbrev}
% \newcommand{\func}[1]{\mathsf{#1}}
% \def \ff {\func{f}}
% \def \fa {\func{a}}
% \def \fb {\func{b}}
% \def \fc {\func{c}}
% \def \fs {\func{s}}
% \def \fP {\func{P}}
% \def \fT {\func{T}}
% \def \fR {\func{R}}
% \def \fL {\func{L}}



\newcommand{\relu}{\operatorname{relu}}
\newcommand{\gap}{\operatorname{GAP}}
\newcommand{\up}{\operatorname{up}}

\newcommand{\cam}{\textrm{CAM}}
\newcommand{\gcam}{\textrm{Grad-CAM}}
\newcommand{\scam}{\textrm{Score-CAM}}

\newcommand{\Fdef}{Mask\xspace}
\newcommand{\Fref}{Diff\xspace}
\newcommand{\MIOFref}{IODiff\xspace}
\newcommand{\MIODref}{IOMask\xspace}


\newcommand{\AG}{\operatorname{AG}}
\newcommand{\AGf}{Average Gain\xspace}
\newcommand{\Agf}{Average gain\xspace}
\newcommand{\agf}{average gain\xspace}

\newcommand{\AC}{\operatorname{AC}}
\newcommand{\ACf}{Average Contract\xspace}

\newcommand{\AD}{\operatorname{AD}}
\newcommand{\I}{\operatorname{I}}
\newcommand{\D}{\operatorname{D}}
\newcommand{\AI}{\operatorname{AI}}
\newcommand{\OM}{\operatorname{OM}}
\newcommand{\LE}{\operatorname{LE}}
\newcommand{\Fo}{\operatorname{F1}}
\newcommand{\prc}{\operatorname{precision}}
\newcommand{\rec}{\operatorname{recall}}
\newcommand{\BA}{\operatorname{BoxAcc}}
\newcommand{\spg}{\operatorname{SP}}
\newcommand{\epg}{\operatorname{EP}}
\newcommand{\SM}{\operatorname{SM}}
\newcommand{\iou}{\operatorname{IoU}}

\title{A Learning Paradigm for Interpretable Gradients}

%\author{\authorname{Felipe Torres Figueroa\sup{1}\orcidAuthor{0000-0000-0000-0000}, Hanwei Zhang\sup{1}\orcidAuthor{0000-0000-0000-0000} and Third Author Name\sup{2}\orcidAuthor{0000-0000-0000-0000}}

%\author{\authorname{Felipe Torres Figueroa\sup{1}, Hanwei Zhang\sup{1}, Ronan Sicre\sup{1}, Yannis Avrithis\sup{2}\sup{3} and Stephane Ayache\sup{1}}
%\affiliation{\sup{1}Centrale Marseille, Aix Marseille Univ, CNRS, LIS, Marseille, France}
%\affiliation{\sup{2}Athena RC}
%\affiliation{\sup{3}Institute of Advanced Research on Artificial Intelligence (IARAI)}
%\email{\{felipe.torres, hanwei.zhang, ronan.sicre, stephane.ayache\}@lis-lab.fr, yannis@avrithis.net}
%}
%\author{\authorname{Authors}\affiliation{Affiliations}\email{email}}

\keywords{Gradient, Class Activation Maps, Interpretability}

\abstract{This paper studies deep neural networks interpretability and more specifically the production of saliency maps to explain a Convolutional Neural Network (CNN) decision. Most of the existing approaches derive from Class Activation Maps (CAM) and GradCAM. These methods combine information from fully connected layers and gradient through variations of backpropagation. 
We present a novel training approach to improve the gradient of a CNN in terms of interpretability.
In particular, we modify the optimization loss so the gradient obtained from back-propagation at the input image level is similar to the gradient coming from guided backpropagation. The resulting gradient is smoother and offer more interpretable power to the CNN, when applying interpretability methods.}

\onecolumn \maketitle \normalsize \setcounter{footnote}{0} \vfill
    \chapter*{Introduction}
    \chaptertoc{}

    \addcontentsline{toc}{chapter}{Introduction}
    \section*{Motivation}
    %\addcontentsline{toc}{section}{Motivation}

    \section*{Dissertation Outline}
    %\addcontentsline{toc}{section}{Dissertation Outline}
    This dissertation is organized in the following manner: First
    we introduce a background on existing approaches towards interpretability of image
    recognition models; for that, we make mention on current architectures dedicated 
    to this approach, while also presenting concepts on interpretability and enunciating 
    current approaches for this study. 

    In Chapter \ref{ch:opticam}, we propose Opti-CAM as a methodology that generates 
    optimized saliency maps highlighting the relevant regions on an image towards image
    classification. On Section \ref{sec:av_gain} we extend existing evaluation metrics 
    with a novel measurement for model coinfidence. extends evaluation metrics with the 
    introduction of a novel measurement yielding improvements on model confidence 
    when using a given attribution approach. On Sections \ref{sec:oc_qual, sec:oc_quant} 
    we evaluate the effect of these contributions towards interpretability assessment.
    
    Chapter \ref{ch:castream} introduces the Cross Attention Stream, an approach that boosts
     existing architectures interpretable properties. We ste up the modulus of this approach 
    on Section \ref{sec:ca_defn} alongside its deployment on Section \ref{sec:ca_design}. 
    On Sections \ref{sec:ca_qual} and \ref{sec:ca_quant} we demonstrate the benefits 
    of using this proposal.
    
    Chapter \ref{ch:grad} characterizes a gradient denoising approch with a gradient denoising
     methodology as an approach to enhance the trainining procedure of current
    models while improving interpretability properties. On Section \ref{sec:grad_defn}, 
    we define the gradient denoising protocol alongside the regularization proposals to do so.
    Sections \ref{sec:grad_qual, sec:grad_quant} illustrate the effects of this paradigmn
    on the trained models and its effects on interpretability.
    
    Chapter \ref{ch:zip} raises the Zero-Information algorithm and its usage as a substitute
    for mask-dependent evaluation proposals. Section \ref{sec:zip_algo} develops this 
    method. Section \ref{sec:zip_insdel} demonstrates its incorporation of this 
    algorithm onto evaluation protocols. Section \ref{sec:zip_qual} displays
    the effect of this approach when applied to mask patches on images. Section 
    \ref{sec:zip_benchmark} displays the results of benchmarking these protocols 
    with this approach. 
    
    Finally, 
    we draw conclusions on our work and detail future research perspectives.


\chapter{Related Work}
\chaptertoc{}

%\addcontentsline{toc}{chapter}{Related Work}

\section{Image Recognition Models}

\subsection{Convolutional Neural Networks}

\subsection{Attention-Based Architectures}


\section{Interpretability}

\subsection{Transparency}

\subsection{Post-Hoc Interpretability}



%\lipsum[1-2]

\uppercase{\section{Background}}
\label{sec:back}

%This section presents some background information regarding guided back propagation and CAM methods

% (Need to elaborate, talk about stuff in verbatum like Mythos of Model interpretability)\cite{mythos_interp} GuidedBackprop\cite{guidedbackprop} Deconvnet\cite{deconvnet} Perturbation Model\cite{perturb_inter}

\subsection{Guided backpropagation \cite{guidedbackprop}}
The derivative of the $\relu$ unit $v = \relu(u) = [u]_+ = \max(u,0)$ with respect to its input $u$ is $\ider{v}{u} = \ind_{u>0}$. By the chain rule, a signal $\delta v = \ipder{L}{v}$ is then propagated backwards through $\relu$ to $\delta u = \ipder{L}{u}$ as $\delta u = \ind_{u>0} \delta v$, where the partial derivative of any scalar quantity of interest is meant, \eg a loss $L$.

Guided backpropagation \cite{guidedbackprop} changes this to $\delta_G u = \ind_{u>0} \ind_{dv>0} \delta v = \ind_{u>0} [\delta v]_+$, essentially masking out values corresponding to negative entries of both the forward ($u$) and the backward ($\delta v$) signals and thus preventing backward flow of negative gradients.

Considering the partial derivative $\ipder{L}{u}$ of some loss $L$ with respect to a variable $u$, \eg through an entire network or part of it, standard backpropagation with this particular change for $\relu$ units is called \emph{guided backpropagation}. We denote the corresponding guided ``partial derivative'' or \emph{guided gradient} by $\ipder{_G L}{u}$. This backpropagation method allows sharp visualization of high level activations conditioned on input images.

%------------------------------------------------------------------------------

\subsection{CAM-based methods}

CAM-based methods build a saliency map as a linear combination of  activation map. 
Specifically, given a target class $c$ and a set of 2D activation maps $\{A^k\}_{k=1}^K$, % outputed from a convolutional layer of a network, 
CAM \cite{cam} is defined as follows:
%ed by $\relu$
\begin{equation}
	S^c = \relu \left( \sum_{k=1}^K \alpha^c_k A^k \right),
\label{eq:cam}
\end{equation}
where the weight $\alpha^c_k$ determines the contribution of channel $k$ to class $c$. This map $S^c$ and the feature maps $A^k$ are non-negative because both are computed with $\relu$ activation functions. Different CAM-based methods differ primarily in the definition of the weights $\alpha^c_k$.

\paragraph{CAM \cite{cam}}
originally defines $\alpha^c_k$ as the weight connecting channel $k$ to class $c$ in the final classification classifier, assuming $\{A^k\}$ are the feature maps of the last convolutional layer, which is followed by \emph{global average pooling} (GAP) and a single dense layer.

\paragraph{Grad-CAM \cite{gradcam}}

is a generalization of CAM for any networks. If $y^c$ is the logit of class $c$, the weights are obtained by GAP of the partial derivatives of $y^c$ with respect to the feature map $A^k$ of a given layer:
\begin{equation}
    \alpha^c_k = \frac{1}{Z} \sum_{i,j} \pder{y^c}{A^k_{ij}},
\label{eq:gcam}
\end{equation}
where $A^k_{ij}$ denotes the value at spatial location $(i,j)$ of feature map $A^k$ and $Z$ is the total number of spatial locations.

Guided Grad-CAM elementwise-multiplies the saliency maps obtained by Grad-CAM and guided backpropagation, after adjusting spatial resolutions. The resulting visualizations are both class-discriminative (by Grad-CAM) and contain fine-grained detail (by guided backpropagation).

\paragraph{Grad-CAM++ \cite{gradcampp}}

is a generalization of Grad-CAM, where partial derivatives of $y^c$ with respect to $A^k$ are followed by $\relu$ as in guided backpropagation \cite{guidedbackprop} and GAP is replaced by a weighted average:
\begin{equation}
    a^c_k = \sum_{i,j} w^{kc}_{ij} \relu \left( \pder{y^c}{A^k_{ij}} \right).
\label{eq:gcamp}
\end{equation}
The weights $w_{ij}^{kc}$ of the linear combination are
\begin{equation}
    w_{ij}^{kc} = \frac{\pder[2]{y^c}{(A^k_{ij})}}
		{2\pder[2]{y^c}{(A^k_{ij})} + \sum_{a,b} A^k_{ab}\pder[3]{y^c}{(A^k_{ij})}}.
\label{eq:gcampw}
\end{equation}

%\paragraph{Smooth Grad-CAM++ \cite{smoothgradcampp}}

%\green{
%\paragraph{SmoothGrad \cite{smoothgrad}} argues that non-linearities like ReLU activations render derivatives noisy. Thus, to improve smoothness of any derivative $D(x)$ with respect to input image $x$, it takes $m$ Gaussian noise images $\epsilon_i \sim \normal(0,\sigma^2)$ and averages the derivatives with respect to $x + \epsilon_i$:
%\begin{equation}
%    \hat{D}(x) = \frac{1}{m} \sum_{i=1}^m D(x + \epsilon_i).
%\label{eq:smooth}
%\end{equation}
%Smooth Grad-CAM++ \cite{smoothgradcampp} applies this idea to Grad-CAM++ \cite{gradcampp}, essentially replacing all partial derivatives in~\eq{gcamp},\eq{gcampw} by smoothed versions according to~\eq{smooth}.
%}


\paragraph{Score-CAM \cite{scorecam}}

%Following the \emph{increase in confidence} metric proposed in Grad-CAM++, 
computes the weights $a^c_k$ based on the increase in confidence \cite{gradcampp} for class $c$ obtained by masking (element-wise multiplying) the input image $x$ with feature map $A^k$:
\begin{equation}
	a^c_k = f(x \circ s(\up(A^k)))_c - f(x_b)_c,
\label{eq:scam}
\end{equation}
where $\up$ is upsampling to the spatial resolution of $x$, $s$ is linear normalization to range $[0,1]$, $\circ$ is the Hadamard product, $f$ is the classifier mapping of input image to class logits and $x_b$ is a baseline image.

While Score-CAM does not require gradients to compute saliency maps, \eq{scam} requires one forward pass through the network $f$ for each channel $k$ and our method modify filters weights during learning.

\section{Method}

%------------------------------------------------------------------------------
\begin{figure*}[t]
\centering
% editable at https://docs.google.com/drawings/d/1nfqVV_dNqSV9g5UBuNopPLvubcEZx-vvg4K0ZCvNdfU/edit
\fig[.7]{CA-CAM}
\vspace{3pt}
\caption{\emph{Visualization of eq.~\eq{connection}.} On the left, a feature tensor $\vF \in \real^{w \times h \times d}$ is multiplied by the vector $\valpha \in \real^d$ in the channel dimension, like in $1 \times 1$ convolution, where $w \times h$ is the spatial resolution and $d$ is the number of channels. This is \emph{cross attention} (CA)~\cite{dosovitskiy2020image} between the query $\valpha$ and the key $\vF$. On the right, a linear combination of feature maps $F^1, \dots, F^d \in \real^{w \times h}$ is taken with weights $\alpha_1, \dots, \alpha_d$. This is a \emph{class activation mapping} (CAM)~\cite{zhou2016learning} with class agnostic weights. Eq.~\eq{connection} expresses the fact that these two quantities are the same, provided that $\valpha = (\alpha_1, \dots, \alpha_d)$ and $\vF$ is reshaped as $F = (\vf^1 \dots \vf^d) \in \real^{p \times d}$, where $p = wh$ and $\vf^k = \vect(F^k) \in \real^{p}$ is the vectorized feature map of channel $k$.}
\label{fig:connection}
\end{figure*}
%------------------------------------------------------------------------------

\subsection{Preliminaries and background}
\label{subsec:prelim}

\paragraph{Notation}

Let $f: \cX \rightarrow \real^C$ be a classifier network that maps an input image $\vx \in \cX$ to a logit vector $\vy= f(\vx) \in \real^C$, where $\cX$ is the image space and $C$ is the number of classes. A class probability vector is obtained by $\vp = \softmax(\vy)$. The logit and probability of class $c$ are respectively denoted by $y^c$ and $p^c = \softmax(\vy)^c \defn e^{y^c} / \sum_j e^{y^j}$. Let $\vF_\ell \in \real^{w_\ell \times h_\ell \times d_\ell}$ be the feature tensor at layer $\ell$ of the network, where $w_\ell \times h_\ell$ is the spatial resolution and $d_\ell$ the embedding dimension, or number of channels. The feature map of channel $k$ is denoted by $F^k_\ell \in \real^{w_\ell \times h_\ell}$. By $\ell$ we may refer to an arbitrary layer of $f$ or a larger compositional block, \eg, a stage.

\paragraph{CAM-based saliency maps}

Given a class of interest $c$ and a layer $\ell$, we consider the saliency maps $S^c_\ell \in \real^{w_\ell \times h_\ell}$ given by the general formula
\begin{equation}
	S^c_\ell \defn h \left( \sum_k \alpha^c_k F^k_\ell \right),
\label{eq:sal}
\end{equation}
where $\alpha^c_k$ are weights defining a linear combination over channels and $h$ is an activation function. Assuming \emph{global average pooling} (GAP) of the last feature tensor $\vF_L$ followed by a linear classifier, CAM~\citep{zhou2016learning} is defined for the last layer $L$ only, with $h$ being the identity mapping and $\alpha^c_k$ the classifier weight connecting channel $k$ with class $c$. Grad-CAM~\citep{DBLP:journals/corr/SelvarajuDVCPB16} is a generalization of CAM defined for any architecture and layer $\ell$, with $h = \relu$ and weights
\begin{equation}
	\alpha^c_k \defn \gap \left( \pder{y^c}{F^k_\ell} \right).
\label{eq:gcam}
\end{equation}

%------------------------------------------------------------------------------

\paragraph{Self-Attention}

Let $X_\ell \in \real^{t_\ell \times d_\ell}$ denote the sequence of token embeddings of a vision transformer~\cite{dosovitskiy2020image} at layer $\ell$, where $t_\ell \defn w_\ell h_\ell + 1$ is the number of tokens, including patch tokens and the \cls token, and $d_\ell$ is the embedding dimension. The \emph{query}, \emph{key} and \emph{value} matrices are defined as $Q = X_\ell W_Q$, $K = X_\ell W_K$, $V = X_\ell W_V \in \real^{t_\ell \times d_\ell}$, where $W_Q, W_K, W_V \in \real^{d_\ell \times d_\ell}$ are learnable linear projections. The \emph{attention matrix} $A \in \real^{t_\ell \times t_\ell}$ expresses pairwise dot-product similarities between queries (rows of $Q$) and keys (rows of $K$), normalized by softmax over rows
\begin{equation}
	A = \softmax \left( \frac{Q K\tran}{\sqrt{d_\ell}} \right).
\label{eq:attention}
\end{equation}
For each token, the \emph{self-attention} operation is then defined as an average of all values (rows of $V$) weighted by attention (the corresponding  row of $A$),
\begin{equation}
	\sa(X_\ell) \defn A V \in \real^{t_\ell \times d_\ell}.
\label{eq:SA}
\end{equation}
At the last layer $L$, the \cls token embedding is used as a global image representation for classification as it gathers information from all patches by weighted averaging, replacing \gap. Thus, at the last layer, it is only cross attention between \cls and the patch tokens that matters.



\subsection{Motivation}
\label{subsec:motiv}

\paragraph{Cross attention}

Let matrix $F_\ell \in \real^{p_\ell \times d_\ell}$ be a reshaping of feature tensor $\vF_\ell$ at layer $\ell$, where $p_\ell \defn w_\ell h_\ell$ is the number of patch tokens without \cls, and let $\vq_\ell \in \real^{d_\ell}$ be the \cls token embedding at layer $\ell$. By focusing on the \emph{cross attention} only between the \cls (query) token $\vq_\ell$ and the patch (key) tokens $F_\ell$ and by ignoring projections $W_Q, W_K, W_V$ for simplicity, attention $A$~\eq{attention} is now a $1 \times p_\ell$ matrix that can be written as a vector $\va \in \real^{p_\ell}$
\begin{equation}
	\va = A\tran = \softmax \left( \frac{F_\ell \vq_\ell}{\sqrt{d_\ell}} \right).
\label{eq:cross-attention}
\end{equation}
Here, $F_\ell \vq_\ell$ expresses the pairwise similarities between the global \cls feature $\vq_\ell$ and the local patch features $F_\ell$. Now, by replacing $\vq_\ell$ by an arbitrary vector $\valpha \in \real^{d_\ell}$ and by writing the feature matrix as $F_\ell = (\vf_\ell^1 \dots \vf_\ell^{d_\ell})$ where $\vf_\ell^k = \vect(F_\ell^k) \in \real^{p_\ell}$ for channel $k$, attention \eq{cross-attention} becomes
\begin{equation}
	\va = h_\ell (F_\ell \valpha) =
		h_\ell \left( \sum_k \alpha_k \vf_\ell^k \right).
\label{eq:connection}
\end{equation}
This takes the same form as~\eq{sal}, with feature maps $F_\ell^k$ being vectorized into $\vf_\ell^k$ and the activation function is defined as $h_\ell(\vx) = \softmax(\vx / \sqrt{d_\ell})$. Eq.~\eq{connection} is visualized in \autoref{fig:connection}. We thus observe the following.

\begin{quote}
	\emph{Pairwise similarities between one query and all patch token embeddings in cross attention are the same as a linear combination of feature maps in CAM-based saliency maps, where the weights are determined by the elements of the query.}
\end{quote}

As it stands, one difference between~\eq{sal} and~\eq{connection} is that~\eq{connection} is class agnostic, although it could be extended by using one query (weight) vector per class. For simplicity, we choose the class agnostic form in the following. We also choose to have no query/key projections. However, we do provide additional experiments with class specific representation as well as projections in \autoref{sec:gen_ablation}. 

\paragraph{Pooling, or masking}

We are thus motivated to integrate an attention mechanism into any network such that making a prediction and explaining (localizing) it are inherently connected. In particular, considering cross attention only between \cls and patch tokens~\eq{cross-attention}, equation~\eq{SA} becomes
\begin{align}
	\ca_\ell(\vq_\ell, F_\ell) \defn F_\ell\tran \va = F_\ell\tran h_\ell(F_\ell \vq_\ell) \in \real^{d_\ell}.
\label{eq:CA}
\end{align}
By writing the transpose of feature matrix as $F_\ell\tran = (\vphi_\ell^1 \dots \vphi_\ell^{p_\ell})$ where $\vphi_\ell^i \in \real^{d_\ell}$ is the feature of patch $i$, this is a weighted average of the local patch features $F_\ell\tran$ with attention vector $\va = (a_1, \dots, a_{p_\ell})$ expressing the weights:
\begin{align}
	\ca_\ell(\vq_\ell, F_\ell) \defn F_\ell\tran \va = \sum_i a_i \vphi_\ell^i.
\label{eq:CA-gap}
\end{align}
We can think of it as as feature \emph{reweighting} or \emph{soft masking} in the feature space, followed by \gap.

Now, considering that $\va$ is obtained exactly as CAM-based saliency maps~\eq{connection}, this operation is similar to occlusion (masking)-based methods~\citep{petsiuk2018rise, fong2017interpretable, fong2019understanding, schulz2020restricting, ribeiro2016should,DBLP:journals/corr/abs-1910-01279, zhang2023opti} and evaluation metrics~\cite{DBLP:journals/corr/abs-1710-11063, petsiuk2018rise}, where a CAM-based saliency map is commonly used to mask the input image. We thus observe the following.

\begin{quote}
	\emph{Attention-based pooling is a form of feature reweighting or soft masking in the feature space followed by \gap, where the weights are given by a class agnostic CAM-based saliency map.}
\end{quote}


%------------------------------------------------------------------------------

\subsection{Cross attention stream}
\label{subsec:CA-base}

Motivated by the observations above, we design a \emph{\OURS} (\emph{\Ours}) in parallel to any network. It takes input features at key locations of the network and uses cross attention to build a global image representation and replace $\gap$ before the classifier. An example is shown in \autoref{fig:fig_method}, applied to a ResNet-based architecture.

\paragraph{Architecture}

More formally, given a network $f$, we consider points between blocks of $f$ where critical operations take place, such as change of spatial resolution or embedding dimension, \eg between stages for ResNet. We decompose $f$ at these points as
\begin{equation}
	f = g \circ \gap \circ f_L \circ \dots \circ f_0
\label{eq:f-decomp}
\end{equation}
such that features $F_\ell \in \real^{p_\ell \times d_\ell}$ of layer (stage) $\ell$ are initialized as $F_{-1} = \vx$ and updated according to
\begin{equation}
	F_\ell = f_\ell(F_{\ell-1})
\label{eq:f-layer}
\end{equation}
for $0 \le \ell \le L$. The last layer features $F_L$ are followed by \gap and $g: \real^{d_L} \to \real^C$ is the classifier, mapping to the logit vector $\vy$. As in \autoref{subsec:motiv}, $p_\ell$ is the number of patch tokens and $d_\ell$ the embedding dimension of stage $\ell$.

In parallel, we initialize a classification token embedding as a learnable parameter $\vq_0 \in \real^{d_0}$ and we build a sequence of updated embeddings $\vq_\ell \in \real^{d_\ell}$ along a stream that interacts with $F_\ell$ at each stage $\ell$. Referring to the global representation $\vq_\ell$ as \emph{query} or \cls and to the local image features $F_\ell$ as \emph{key} or patch embeddings, the interaction consists of cross attention followed by a linear projection $W_\ell \in \real^{d_{\ell+1} \times d_\ell}$ to account for changes of embedding dimension between the corresponding stages of $f$:
\begin{equation}
	\vq_{\ell+1} = W_\ell \cdot \ca_\ell(\vq_\ell, F_\ell),
\label{eq:qk-layer}
\end{equation}
for $0 \le \ell \le L$, where $\ca_\ell$ is defined as in~\eq{CA}. 
% Because of linearity, projection $W_\ell$ is the same as a value projection.

Image features $F_0, \dots, F_L$ do not change by injecting our \Ours into network $f$. However, the final global image representation and hence the prediction do change. In particular, at the last stage $L$, $\vq_{L+1}$ is used as a global image representation for classification, replacing \gap over $F_L$. The final prediction is $g(\vq_{L+1}) \in \real^C$. Unlike \gap, the weights of different image patches in the linear combination are non-uniform, enhancing the contribution of relevant patches in the prediction.

%------------------------------------------------------------------------------
%------------------------------------------------------------------------------
\begin{figure*}[t]
\centering
\begin{tikzpicture}[
	font={\footnotesize},
	trap/.style={trapezium, rotate=-90,trapezium angle=75},
]
	%% CNN branch
	\node(input) at (-5.5, 0) {\includegraphics[width=.1\textwidth]{Images/Method/input.jpg}};
	\node[above] at (input.north) {Input image $\vx$};
	\node[draw, trap] (res0) at (-3.5,0) {\rotatebox{90}{\parbox{1.0cm}{\centering{Res-0}}}};
	\node[draw, trap] (res1) at (-1.5,0) {\rotatebox{90}{\parbox{1.0cm}{\centering{Res-1}}}};
	\node[draw, trap] (res2) at (0.5,0) {\rotatebox{90}{\parbox{1.0cm}{\centering{Res-2}}}};
	\node[draw, trap] (res3) at (2.5,0) {\rotatebox{90}{\parbox{1.0cm}{\centering{Res-3}}}};
	\node[draw, trap] (res4) at (4.5,0) {\rotatebox{90}{\parbox{1.0cm}{\centering{Res-4}}}};
	\node[](empt1) at (6.75, 0){};
	\node[draw, rotate=90, align=center] (class) at (7.5,0) {Classifier};
	\node(logit) at (8.25, 0) {$\vy$};
	%%% CLS stream
	\node[](clsin) at (-4, -1.5) {{$\vq_0$}};
	\node[draw](CA0) at (-2.5, -1.5) {{CA-0}};
	\node[draw](CA1) at (-0.5, -1.5) {{CA-1}};
	\node[draw](CA2) at (1.5, -1.5)  {{CA-2}};
	\node[draw](CA3) at (3.5, -1.5)  {{CA-3}};
	\node[draw](CA4) at (5.5, -1.5)  {{CA-4}};

	%% CNN backbone
	\node(empt0) at (-4.65, 0) {};
	\draw[->] (empt0.center) -- node {} (res0);
	\draw[->] (res0) -- node[above] {$F_0$} (res1);
	\draw[->] (res1) -- node[above] {$F_1$} (res2);
	\draw[->] (res2) -- node[above] {$F_2$} (res3);
	\draw[->] (res3) -- node[above] {$F_3$} (res4);
	\draw[->, blue, dashed] (res4) -- node {\blue{\normalsize//}} (class);
	\node[](GAP) at (6.25,0.25) {\blue{$\gap$}};
	\draw[->] (class) -- node {} (logit);
	%% CLS Stream
	\draw[->] (clsin) -- node {} (CA0);
	\draw[dashed, ->] (res0.north) -|node {} (CA0);
	\draw[->] (CA0) -- node[above] {$\vq_1$} (CA1);
	\draw[dashed, ->] (res1.north) -|node {} (CA1);
	\draw[->] (CA1) -- node[above] {$\vq_2$} (CA2);
	\draw[dashed, ->] (res2.north) -|node {} (CA2);
	\draw[->] (CA2) -- node[above] {$\vq_3$} (CA3);
	\draw[dashed, ->] (res3.north) -|node {} (CA3);
	\draw[->] (CA3) -- node[above] {$\vq_4$} (CA4);
	\draw[dashed, ->] (res4.north) -|node[above] {$F_4$} (CA4);
	\draw[-] (CA4.east) -| node[right] {$\vq_5$} (empt1.center);
	\draw[->] (empt1.center) -- node {} (class);
\end{tikzpicture}
\vspace{3pt}
\caption{\emph{\OURS (\Ours) applied to ResNet-based architectures.} Given a network $f$, we replace global average pooling (\gap) by a learned, attention-based pooling mechanism implemented as a stream in parallel to $f$. The feature tensor $F_\ell \in \real^{p_\ell \times d_\ell}$ (\emph{key}) obtained by stage Res-$\ell$ of $f$ interacts with a \cls token (\emph{query}) embedding $\vq_\ell \in \real^{d_\ell}$ in block CA-$\ell$, which contains cross attention~\eq{CA} followed by a linear projection~\eq{qk-layer} to adapt to the dimension of $F_{\ell+1}$. Here, $p_\ell$ is the number of patches (spatial resolution) and $d_\ell$ the embedding dimension. The query is initialized by a learnable parameter $\vq_0 \in \real^{d_0}$, while the output $\vq_5$ of the last cross attention block is used as a global image representation into the classifier. The network and classifier are pretrained and kept frozen while the parameters of \Ours are learned. At inference, we use existing post-hoc interpretability methods like Grad-CAM~\citep{DBLP:journals/corr/SelvarajuDVCPB16} to obtain saliency maps for both the baseline \gap and our \Ours. We compare interpretability metrics as well as accuracy.}
\label{fig:fig_method}
\end{figure*}

%------------------------------------------------------------------------------


\paragraph{Training}

In this sense, the network $f$ is pretrained and remains frozen while we learn the parameters of our \Ours on the same training set as one used to train $f$. The classifier is kept frozen too. Referring to~\eq{f-decomp}, $f_0, \dots, f_L$ and $g$ are fixed, while \gap is replaced by learned weighted averaging, with the weights obtained by the \Ours.

\paragraph{Inference}

As it stands, \Ours is not an interpretability method, but rather a modification of the baseline architecture, \ie, an attention-based pooling mechanism that replaces \gap to enhance the contribution of relevant image regions in the prediction. We are interested in investigating the interpretability properties of this modification. We therefore employ existing post-hoc, CAM-based interpretability methods to generate saliency maps with both baseline \gap and \Ours. We then compare interpretability metrics as well as classification accuracy.


\uppercase{\section{Experiments}}

\begin{figure*}[t]
\scriptsize
\centering
\setlength{\tabcolsep}{1.5pt}
\begin{tabular}{cccccccccc}
    {}&\mr{2}{\Th{Input}}&\mc{2}{\Th{Grad-CAM}}&\mc{2}{\Th{Grad-CAM++}}&\mc{2}{\Th{Score-CAM}}&\mc{2}{\Th{Ablation-CAM}}\\
    {}&{}&\Th{Baseline}&\Th{Denoised}&\Th{Baseline}&\Th{Denoised}&\Th{Baseline}&\Th{Denoised}&\Th{Baseline}&\Th{Denoised}\\
  %  {\rotatebox{90}{\small Man}}&\includegraphics[width=0.09\textwidth]{fig/fig_cam/original/4862.JPEG}&\includegraphics[width=0.09\textwidth]{fig/fig_cam/baseline/gradcam/4862.JPEG}&\includegraphics[width=0.09\textwidth]{fig/fig_cam/cosine/gradcam/4862.JPEG}&\includegraphics[width=0.09\textwidth]{fig/fig_cam/baseline/gradcampp/4862.JPEG}&\includegraphics[width=0.09\textwidth]{fig/fig_cam/cosine/gradcampp/4862.JPEG}&\includegraphics[width=0.09\textwidth]{fig/fig_cam/baseline/scorecam/4862.JPEG}&\includegraphics[width=0.09\textwidth]{fig/fig_cam/cosine/scorecam/4862.JPEG}&\includegraphics[width=0.09\textwidth]{fig/fig_cam/baseline/ablationcam/4862.JPEG}&\includegraphics[width=0.09\textwidth]{fig/fig_cam/cosine/ablationcam/4862.JPEG}\\
    
    {\rotatebox{90}{\small Girl}}&\includegraphics[width=0.09\textwidth]{fig/fig_cam/original/2641.JPEG}&\includegraphics[width=0.09\textwidth]{fig/fig_cam/baseline/gradcam/2641.JPEG}&\includegraphics[width=0.09\textwidth]{fig/fig_cam/cosine/gradcam/2641.JPEG}&\includegraphics[width=0.09\textwidth]{fig/fig_cam/baseline/gradcampp/2641.JPEG}&\includegraphics[width=0.09\textwidth]{fig/fig_cam/cosine/gradcampp/2641.JPEG}&\includegraphics[width=0.09\textwidth]{fig/fig_cam/baseline/scorecam/2641.JPEG}&\includegraphics[width=0.09\textwidth]{fig/fig_cam/cosine/scorecam/2641.JPEG}&\includegraphics[width=0.09\textwidth]{fig/fig_cam/baseline/ablationcam/2641.JPEG}&\includegraphics[width=0.09\textwidth]{fig/fig_cam/cosine/ablationcam/2641.JPEG}\\
    
    {\rotatebox{90}{\small Woman}}&\includegraphics[width=0.09\textwidth]{fig/fig_cam/original/5257.JPEG}&\includegraphics[width=0.09\textwidth]{fig/fig_cam/baseline/gradcam/5257.JPEG}&\includegraphics[width=0.09\textwidth]{fig/fig_cam/cosine/gradcam/5257.JPEG}&\includegraphics[width=0.09\textwidth]{fig/fig_cam/baseline/gradcampp/5257.JPEG}&\includegraphics[width=0.09\textwidth]{fig/fig_cam/cosine/gradcampp/5257.JPEG}&\includegraphics[width=0.09\textwidth]{fig/fig_cam/baseline/scorecam/5257.JPEG}&\includegraphics[width=0.09\textwidth]{fig/fig_cam/cosine/scorecam/5257.JPEG}&\includegraphics[width=0.09\textwidth]{fig/fig_cam/baseline/ablationcam/5257.JPEG}&\includegraphics[width=0.09\textwidth]{fig/fig_cam/cosine/ablationcam/5257.JPEG}\\
        
    {\rotatebox{90}{\small Lobster}}&\includegraphics[width=0.09\textwidth]{fig/fig_cam/original/5160.JPEG}&\includegraphics[width=0.09\textwidth]{fig/fig_cam/baseline/gradcam/5160.JPEG}&\includegraphics[width=0.09\textwidth]{fig/fig_cam/cosine/gradcam/5160.JPEG}&\includegraphics[width=0.09\textwidth]{fig/fig_cam/baseline/gradcampp/5160.JPEG}&\includegraphics[width=0.09\textwidth]{fig/fig_cam/cosine/gradcampp/5160.JPEG}&\includegraphics[width=0.09\textwidth]{fig/fig_cam/baseline/scorecam/5160.JPEG}&\includegraphics[width=0.09\textwidth]{fig/fig_cam/cosine/scorecam/5160.JPEG}&\includegraphics[width=0.09\textwidth]{fig/fig_cam/baseline/ablationcam/5160.JPEG}&\includegraphics[width=0.09\textwidth]{fig/fig_cam/cosine/ablationcam/5160.JPEG}\\
    
    {\rotatebox{90}{\small Couch}}&\includegraphics[width=0.09\textwidth]{fig/fig_cam/original/2043.JPEG}&\includegraphics[width=0.09\textwidth]{fig/fig_cam/baseline/gradcam/2043.JPEG}&\includegraphics[width=0.09\textwidth]{fig/fig_cam/cosine/gradcam/2043.JPEG}&\includegraphics[width=0.09\textwidth]{fig/fig_cam/baseline/gradcampp/2043.JPEG}&\includegraphics[width=0.09\textwidth]{fig/fig_cam/cosine/gradcampp/2043.JPEG}&\includegraphics[width=0.09\textwidth]{fig/fig_cam/baseline/scorecam/2043.JPEG}&\includegraphics[width=0.09\textwidth]{fig/fig_cam/cosine/scorecam/2043.JPEG}&\includegraphics[width=0.09\textwidth]{fig/fig_cam/baseline/ablationcam/2043.JPEG}&\includegraphics[width=0.09\textwidth]{fig/fig_cam/cosine/ablationcam/2043.JPEG}\\
    
    
\end{tabular}
\vspace{3pt}
\caption{Saliency map comparison of several CAM-based methods with standard vs our models training on CIFAR-100 examples.}
\label{fig:salient}
\end{figure*}

This section presents the experimental settings, the definition of our evaluation metrics and results.
%Finally, we present qualitative and quantitative results and an ablation study.\\



\subsection{Experimental Set-up}

In the following sections, we evaluate recognition properties and interpretability capabilities of our approach. Specifically, we generate explanations following popular attribution methods derived from CAM \cite{cam} from the \textbf{pytorch-grad-cam} library from Jacob Gildenblat\footnote{https://github.com/jacobgil/pytorch-grad-cam}.

\paragraph{Dataset}
We train and evaluate our models on CIFAR-100 \cite{krizhevsky2009learning}. This dataset contains 60.000 images of 100 categories, split in 50.000 for training and 10.000 for testing. Each image has a resolution of $32\times32$ pixels. This dataset is chosen because of its ease of usage and prototyping properties. 

\paragraph{Settings}
To obtain competitive performance and ensure the replicability of our method, we follow the methodology found in the repository by weiaicunzai \footnote{https://github.com/weiaicunzai/pytorch-cifar100}. Thus, we train each model following the same training procedure. We perform 200 epochs, with a starting learning rate of $10^{-1}$, a batch-size of 128 images, SGD optimizer and a learning rate policy updating said parameter by division over 5 on epochs 60, 120 and 160.  

\subsection{Faithfulness Evaluation via Image Recognition}
Faithfulness evaluation as proposed in \cite{gradcampp} offers insight about the regions of an image that are considered important for recognition, as highlighted by the saliency map $S^c$. 
Specifically, given a class $c$, an image \textit{I} and a computed saliency map $S^c$ are element-wise multiplied to obtain a masked image $M^c$:

\begin{equation*}
    M^c = S^c\circ I
\end{equation*}

This masked image is similar to the original image on the salient areas and becomes black on the non-salient ones.
To evaluate the saliency maps capability, we forward both the image and the masked version through the network to obtain the prediction scores \textit{$Y_i^c$} and \textit{$O_i^c$} respectively. We then compute the following metrics:

\begin{itemize}
    \item \textbf{Average Drop (AD)} 
    Aims at quantifying how much predictive power is lost when we take into consideration the masked image compared to the original one. Lower is better.
    \begin{equation}
        AD(\%) = \sum_{i=1}^N \frac{max(0,Y_i^c- O_i^c)}{Y_i^c}
    \end{equation}
    \label{Average Drop}
    
    \item \textbf{Average Increase (AI)}
    Unlike Average Drop, Average Increase, also known as Increase of Confidence, measures the percentage of instances of the dataset where the masked image offers a higher score than the original images for a specific class. Higher is better.
    
    \begin{equation}
        AI(\%) = \frac{1}{N} \sum_i^N \mathbbm{1}{(Y_i^c<O_i^c)}*100
    \end{equation}

    \item \textbf{Average Gain (AG)} 
    Most recently introduced in Opti-CAM, %\cite{zhang2023opti}, 
    this metric is designed to be a complement for Average Drop. This measurement is meant to quantify the gain in predictive power attained when the masked image is taken in comparison compared to the original one.
     Higher is better.
    \begin{equation}
        AG(\%) = \sum_{i=1}^N \frac{max(0, O_i^c-Y_i^c)}{Y_i^c}
    \end{equation}
\end{itemize}


\subsection{Causal Metrics for Explanations}
Causality evaluation as proposed in RISE \cite{petsiuk2018rise}, aims at evaluating the effect of masking certain elements of the image and retrieving the change in predictive power for a model with the given changes. Thus, the authors proposed \textbf{Insertion} and \textbf{Deletion}.

\begin{itemize}
	\item \textbf{Average Drop (AD)} 
	Aims at quantifying how much predictive power is lost when we take into consideration the masked image compared to the original one. Lower is better.
	\begin{equation}
	AD(\%) = \sum_{i=1}^N \frac{max(0,Y_i^c- O_i^c)}{Y_i^c}
	\end{equation}
	\label{Average Drop}
	
	\item \textbf{Average Increase (AI)}
	Unlike Average Drop, Average Increase, also known as Increase of Confidence, measures the percentage of instances of the dataset where the masked image offers a higher score than the original images for a specific class. Higher is better.
	
	\begin{equation}
	AI(\%) = \frac{1}{N} \sum_i^N \mathbbm{1}{(Y_i^c<O_i^c)}*100
	\end{equation}
	
	\item \textbf{Average Gain (AG)} 
	Most recently introduced in Opti-CAM, %\cite{zhang2023opti}, 
	this metric is designed to be a complement for Average Drop. This measurement is meant to quantify the gain in predictive power attained when the masked image is taken in comparison compared to the original one.
	Higher is better.
	\begin{equation}
	AG(\%) = \sum_{i=1}^N \frac{max(0, O_i^c-Y_i^c)}{Y_i^c}
	\end{equation}
\end{itemize}


\subsection{Qualitative Experiments}
We visualize the effect of our approach on Figure \ref{fig:salient} and \ref{fig:grads}, which  presents saliency maps, respectively gradient, obtained for the baseline model and the one trained with our approach.
%Conversely, to study the effect of our methodology, on Figure \ref{fig:grads} we demonstrate the differences between the gradients of models trained with our approach and without it.

Regarding saliency maps, we observe the differences brought by our training method that improves interpretability metrics.
The differences are particularly important fro Grad-CAM which directly averages the gradient to weigh feature maps.
Interestingly the differences are smaller on Score-CAM that is not gradient based and only obtains changes from the differences in feature maps.

%that there is some degree of overlapping regarding the higlighted regions for each method, however those differences can be explained in part because of the training procedure performed, given accuracies are different between approaches;

%however, it is important to notice that while accuracy might not always be better, these differences between image parts highlighted lead to gains on interpretable recognition. 
Gradient visualizations show a bit less noise and magnitude with our method.
%On another part, while it has been acknowledged that standard gradients are hard to explain and the information conveyed by them is overall hard to interpret, we observe that our approach indeed leads to less noisy gradients; 
Moreover, the object of interest is better covered by gradient activations. 
%images is presented in a clearer way, thus proving the intended behavior of our training.

\begin{figure}[t]
    \centering
    \setlength{\tabcolsep}{3.5pt}
    \begin{tabular}{cccc}
        {}&\mr{2}{\Th{Input}}&\mc{2}{\Th{Gradient}}\\
        {}&{}&\Th{Baseline}&\Th{Ours}\\
        {\rotatebox{90}{\small Bed}}&\includegraphics[width=0.125\textwidth]{fig/fig_grad/original/5415.JPEG}&\includegraphics[width=0.125\textwidth]{fig/fig_grad/baseline/5415.JPEG}&\includegraphics[width=0.125\textwidth]{fig/fig_grad/cosine/5415.JPEG}\\
        
        {\rotatebox{90}{\small Lamp}}&\includegraphics[width=0.125\textwidth]{fig/fig_grad/original/1766.JPEG}&\includegraphics[width=0.125\textwidth]{fig/fig_grad/baseline/1766.JPEG}&\includegraphics[width=0.125\textwidth]{fig/fig_grad/cosine/1766.JPEG}\\

        {\rotatebox{90}{\small Lawnmower}}&\includegraphics[width=0.125\textwidth]{fig/fig_grad/original/8311.JPEG}&\includegraphics[width=0.125\textwidth]{fig/fig_grad/baseline/8311.JPEG}&\includegraphics[width=0.125\textwidth]{fig/fig_grad/cosine/8311.JPEG}\\

        {\rotatebox{90}{\small Maple Tree}}&\includegraphics[width=0.125\textwidth]{fig/fig_grad/original/1198.JPEG}&\includegraphics[width=0.125\textwidth]{fig/fig_grad/baseline/1198.JPEG}&\includegraphics[width=0.125\textwidth]{fig/fig_grad/cosine/1198.JPEG}\\

        {\rotatebox{90}{\small Sunflower}}&\includegraphics[width=0.125\textwidth]{fig/fig_grad/original/8881.JPEG}&\includegraphics[width=0.125\textwidth]{fig/fig_grad/baseline/8881.JPEG}&\includegraphics[width=0.125\textwidth]{fig/fig_grad/cosine/8881.JPEG}\\
    \end{tabular}
    \caption{Visualization of standard gradient with models trained with standard and our approach.}
    \label{fig:grads}
\end{figure}

\subsection{Quantitative Experiments}
We evaluate the effect of training a given model using our proposed approach with \textit{Faithfulness} and \textit{Causality}. Results are reported in Table \ref{tab:C100_quant}.

As we see, our method offers a consistent improvement in terms of interpretability metrics. Specifically, we obtain improvements on both networks and systematically on five out of six metrics. The improvements are higher for AD, AG, and AI. Insertion gets a smaller but consistent improvement and Deletion is almost always worst with our method, but with a very small difference.
This decrease in performance of Deletion may be due to some limitations of the metrics as reported in previous works.%\cite{zhang2023opti}.

It is intereting to note that improvements on Score-CAM means that our training not only improves gradient for interpretability, but also builds better activation maps.

\begin{table}[t]
\centering
\scriptsize
\setlength{\tabcolsep}{2.5pt}
    \begin{tabular}{lcccccc}\\\toprule
        \mc{7}{\Th{\textbf{Recognition Metrics}}}\\\midrule
        \Th{Model}&\Th{Error}& &\Th{$\lambda$}&\Th{Acc}& &\\\midrule
        \mr{2}{\Th{ResNet-18}}&-& &-&\Th{73.42}& &\\
         &\Th{Cosine}& &$7.5\times10^{-3}$&\Th{72.86}& &\\\midrule
         
        \mr{2}{\Th{MobileNet-V2}}&-& &-&\Th{59.43}&\\
         &\Th{Cosine}&  &$1\times10^{-3}$&\Th{62.36}&\\\midrule
        
        \mc{7}{\Th{\textbf{Interpretable Recognition Metrics}}}\\\midrule    
        \mc{7}{\Th{ResNet-18}}\\\midrule
        \Th{Method}&\Th{Error}&\Th{AD$\downarrow$}&\Th{AG$\uparrow$}&\Th{AI$\uparrow$}&\Th{Ins$\uparrow$}&\Th{Del$\downarrow$}\\\hline
        \mr{2}{\Th{Grad-CAM}}&-&30.16&15.23&29.99&58.47&17.47\\
         &\Th{Cosine}&28.09&16.19&31.53&58.76&17.57\\\hline
        \mr{2}{\Th{Grad-CAM++}}&-&31.40&14.17&28.47&58.61&17.05\\
          &\Th{Cosine}&29.78&15.07&29.60&58.90&17.22\\\hline
        \mr{2}{\Th{Score-CAM}}&-&26.49&18.62&33.84&58.42&18.31\\
          &\Th{Cosine}&24.82&19.49&35.51&59.11&18.34\\\hline
        \mr{2}{\Th{Ablation-CAM}}&-&31.96&14.02&28.33&58.36&17.14\\
         &\Th{Cosine}&29.90&15.03&29.61&58.70&17.37\\\hline
        \mr{2}{\Th{Axiom-CAM}}&-&30.16&15.23&29.98&58.47&17.47\\
          &\Th{Cosine}&28.09&16.20&31.53&58.76&17.57\\\midrule

        \mc{7}{\Th{MobileNet-V2}}\\\midrule
        \Th{Method}&\Th{Error}&\Th{AD$\downarrow$}&\Th{AG$\uparrow$}&\Th{AI$\uparrow$}&\Th{Ins$\uparrow$}&\Th{Del$\downarrow$}\\\hline
        \mr{2}{\Th{Grad-CAM}}&-&44.64&6.57&25.62&44.64&14.34\\
         &\Th{Cosine}&40.89&7.31&27.08&45.57&15.20\\\hline
        \mr{2}{\Th{Grad-CAM++}}&-&45.98&6.12&24.10&44.72&14.76\\
          &\Th{Cosine}&40.76&6.85&26.46&45.51&14.92\\\hline
        \mr{2}{\Th{Score-CAM}}&-&40.55&7.85&28.57&45.62&14.52\\
          &\Th{Cosine}&36.34&9.09&30.50&46.35&14.72\\\hline
        \mr{2}{\Th{Ablation-CAM}}&-&45.15&6.38&25.32&44.62&15.03\\
          &\Th{Cosine}&41.13&7.03&26.10&45.38&15.12\\\hline
        \mr{2}{\Th{Axiom-CAM}}&-&44.65&6.57&25.62&44.64&15.27\\
          &\Th{Cosine}&40.89&7.31&27.08&45.57&15.20\\\bottomrule

    \end{tabular}
    \caption{Experiments with cosine error function on CIFAR-100 with ResNet-18 and MobileNet-V2. Accuracy and interpretability metrics are reported.}
    \label{tab:C100_quant}
\end{table}


\subsection{Ablation Experiments}
We conduct ablation experiments using ResNet18. In these experiments we analyze the different regularization proposals mentioned in Section \ref{sec:method} and the impact of the regularization coefficient.

\paragraph{Regularization proposals} 
To validate our selection of regularization function, we train several models following the same training regime while varying the error function. To evaluate these approaches, we focus solely on Grad-CAM attributions. Results are reported in Table \ref{tab:Regs}


\begin{table}
\centering
\scriptsize
\setlength{\tabcolsep}{3.5pt}
    \begin{tabular}{lcccccc}\\\toprule
    \mc{7}{Regularization Selection}\\\midrule
    \Th{Regularizer}&\Th{Acc}&\Th{AD$\downarrow$}&\Th{AG$\uparrow$}&\Th{AI$\uparrow$}&\Th{Ins$\uparrow$}&\Th{Del$\downarrow$}\\\midrule
    - &73.42&30.16&15.23&29.99&58.47&17.47\\
    Cosine&72,86&\textbf{28.09}&\textbf{16.19}&\textbf{31.53}&58.76&17.57\\
    Histogram &73.88&30.39&14.78&29.38&58.52&17.35\\
    MAE & 73,41& 30,33 & 15,06 &29,61 & 58,13 & 17,95\\
    MSE & 73,86& 29,64 & 15,19 &30,11 & \textbf{59,05} & 18,02\\\bottomrule
    \end{tabular}
    \caption{\textbf{Regularization selection: } Evaluation of the four proposed regularization with ResNet-18 on CIFAR-100.}%}
    \label{tab:Regs}
\end{table}

Following these results, we observe a consistent improvement on most metrics for all regularizer options. 
We note that the accuracy remains stable within half a percent of the original model.
However, we note that most options struggle regarding deletion. 
Cosine Similarity however manages to provide improvements in most metrics, while maintaining deletion performance.
 
\paragraph{Regularization coefficient}
Finally we study the behavior of the regularization coefficient $\lambda$ in \ref{eq:total}. We train multiple models with \textit{Cosine Similarity} and a range of values for $\lambda$, see Table \ref{tab:variation}.

We observe that our method is not very sensible to the regularization coefficient and that the value of $7.5e^{-3}$ offers better performances and is thus selected as the default value for $\lambda$.


\begin{table}
\centering
\scriptsize
\setlength{\tabcolsep}{3.5pt}
    \begin{tabular}{lcccccc}\\\toprule
    \mc{7}{Regularization Selection}\\\midrule
    $\lambda$&\Th{Acc}&\Th{AD$\downarrow$}&\Th{AG$\uparrow$}&\Th{AI$\uparrow$}&\Th{Ins$\uparrow$}&\Th{Del$\downarrow$}\\\midrule
    - &73.42&30.16&15.23&29.99&58.47&17.47\\
    $1\times10^{-3}$ &\textbf{73.71}&29.52&15.17&30.03&59.23&\textbf{17.45}\\
    $2.5\times10^{-3}$ &72.99&30.53&15.82&30.56&59.04&17.96\\
    $5\times10^{-3}$ &72.46&30.10&16.06&30.67&57.47&17.80\\
    $7.5\times10^{-3}$ &72.86&\textbf{28.09}&\textbf{16.20}&\textbf{31.53}&58.76&17.57\\
    $1\times10^{-2}$ &73.28&28.97&15.75&31.16&58.99&17.50\\
    $1\times10^{-1}$ &73.00&28.93&16.13&31.55&\textbf{59.66}&17.95\\
    $1$ &73.30&28.44&16.02&31.31&58.64&17.48\\
    $10$ &73.04&29.28&15.23&30.47&58.74&17.47\\\bottomrule
    \end{tabular}
    \caption{\textbf{Regularization coefficient:} Evaluation of the regularization coefficient $\lambda$, using ResNet-18 with \textit{Cosine Similarity} on CIFAR-100.}%}
    \label{tab:variation}
\end{table}


\uppercase{\section{conclusion}}

%\green{TO DO}
In this paper, we propose a new training approach to improve the gradient of a CNN in terms of interpretability.
Our methods forces the gradient obtained from back-propagation at the input image level to align with the gradient coming from guided backpropagation. 
The results of our training are presented according to several metrics and interpretability methods. Our method offers consistent improvement on four out of five metrics for two networks.


\bibliographystyle{apalike}
%\tiny
%\scriptsize
\footnotesize

\bibliography{egbib}


\end{document}

