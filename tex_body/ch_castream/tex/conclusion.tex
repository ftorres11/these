\section{Conclusion}

We believe we are the first to observe that attention-based pooling in transformers is the same as forming a class agnostic CAM-based saliency map and then using this map to mask the features before global average pooling, much like we mask inputs to confirm that the prediction is due to a certain object. This observation establishes that transformers have a built-in CAM-based interpretability mechanism and allows us to design a similar mechanism for convolutional networks. Masking in feature space is much more efficient than in the input space as it requires only one forward pass, although of course it is not equivalent because of interactions within the network.

Although the saliency maps obtained with our \Ours are not very different than those obtained with \gap, our approach improves a number of CAM-based interpretability methods on a number of convolutional networks according to most interpretability metrics, while preserving classification accuracy. By doing so, it also enhances the differences in performance between interpretability methods, facilitating their evaluation. Further study may be needed to improve the differentiation of saliency maps themselves, to possibly make a class specific representation more competitive and to apply the approach to more architectures, including transformers.