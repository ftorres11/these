\documentclass[10pt,twocolumn,letterpaper]{article}

\usepackage{wacv}

%------------------------------------------------------------------------------
% needed for externalization of plots (no margins on individual pdfs)
\newcommand{\finalcopy}{\iccvfinalcopy}
%------------------------------------------------------------------------------
% FIGURES: CHOOSE ONE OPTION
%
% plots:       build standalone pdfs for figures, then use them
% plots-ext:   use existing pdfs for figures
% plots-none:  skip figures
%
\input{tex/plots}
% %------------------------------------------------------------------------------
% hack for CVPR/ICCV style
% \makeatletter
% \@namedef{ver@everyshi.sty}{}
% \makeatother

%------------------------------------------------------------------------------
% main packages
\usepackage[dvipsnames,svgnames,x11names]{xcolor}
\usepackage{tikzextern}
\usepackage{pgffor}

%------------------------------------------------------------------------------
% externalization
\newcommand{\extfig}[2]{\tikzsetnextfilename{fig/extern/#1}{#2}}
\newcommand{\noextfig}[1]{!!!}
\newcommand{\extdata}[1]{}

% \newcommand{\extfig}[2]{!!!}
\newcommand{\noextfig}[1]{!!!}
\newcommand{\extdata}[1]{}

%------------------------------------------------------------------------------
% space before \paragraph (default 4.05ex)
\makeatletter
\renewcommand\paragraph{\@startsection{paragraph}{4}{\z@}{1ex}{-1em}{\normalfont\normalsize\bfseries}}
\makeatother
%------------------------------------------------------------------------------

\usepackage{times}
\usepackage{epsfig}
\usepackage{graphicx}
\usepackage{amsmath}
\usepackage{amssymb}

% Include other packages here, before hyperref.

\usepackage{float}
\usepackage{multirow}
\usepackage[numbers]{natbib}
\usepackage{enumitem}
\usepackage{array,booktabs}
\usepackage{bbm}
\usepackage{colortbl}


% If you comment hyperref and then uncomment it, you should delete
% egpaper.aux before re-running latex.  (Or just hit 'q' on the first latex
% run, let it finish, and you should be clear).
\usepackage[pagebackref=true,breaklinks=true,letterpaper=true,colorlinks,bookmarks=false]{hyperref}

% \iccvfinalcopy % *** Uncomment this line for the final submission

\def\wacvPaperID{140} % *** Enter the ICCV Paper ID here
\def\httilde{\mbox{\tt\raisebox{-.5ex}{\symbol{126}}}}

% Pages are numbered in submission mode, and unnumbered in camera-ready
\ifwacvfinal\pagestyle{empty}\fi

\begin{document}

\input{tex/abbrev}
% \newcommand{\func}[1]{\mathsf{#1}}
% \def \ff {\func{f}}
% \def \fa {\func{a}}
% \def \fb {\func{b}}
% \def \fc {\func{c}}
% \def \fs {\func{s}}
% \def \fP {\func{P}}
% \def \fT {\func{T}}
% \def \fR {\func{R}}
% \def \fL {\func{L}}



\newcommand{\relu}{\operatorname{relu}}
\newcommand{\gap}{\operatorname{GAP}}
\newcommand{\up}{\operatorname{up}}

\newcommand{\cam}{\textrm{CAM}}
\newcommand{\gcam}{\textrm{Grad-CAM}}
\newcommand{\scam}{\textrm{Score-CAM}}

\newcommand{\Fdef}{Mask\xspace}
\newcommand{\Fref}{Diff\xspace}
\newcommand{\MIOFref}{IODiff\xspace}
\newcommand{\MIODref}{IOMask\xspace}


\newcommand{\AG}{\operatorname{AG}}
\newcommand{\AGf}{Average Gain\xspace}
\newcommand{\Agf}{Average gain\xspace}
\newcommand{\agf}{average gain\xspace}

\newcommand{\AC}{\operatorname{AC}}
\newcommand{\ACf}{Average Contract\xspace}

\newcommand{\AD}{\operatorname{AD}}
\newcommand{\I}{\operatorname{I}}
\newcommand{\D}{\operatorname{D}}
\newcommand{\AI}{\operatorname{AI}}
\newcommand{\OM}{\operatorname{OM}}
\newcommand{\LE}{\operatorname{LE}}
\newcommand{\Fo}{\operatorname{F1}}
\newcommand{\prc}{\operatorname{precision}}
\newcommand{\rec}{\operatorname{recall}}
\newcommand{\BA}{\operatorname{BoxAcc}}
\newcommand{\spg}{\operatorname{SP}}
\newcommand{\epg}{\operatorname{EP}}
\newcommand{\SM}{\operatorname{SM}}
\newcommand{\iou}{\operatorname{IoU}}


%%%%%%%%% TITLE
\title{\Ours: Attention-based pooling for interpretable image recognition}

\author{First Author\\
Institution1\\
Institution1 address\\
{\tt\small firstauthor@i1.org}
% For a paper whose authors are all at the same institution,
% omit the following lines up until the closing ``}''.
% Additional authors and addresses can be added with ``\and'',
% just like the second author.
% To save space, use either the email address or home page, not both
\and
Second Author\\
Institution2\\
First line of institution2 address\\
{\tt\small secondauthor@i2.org}
}

% \maketitle
% Remove page # from the first page of camera-ready.
\ifwacvfinal\thispagestyle{empty}\fi


% %%%%%%%%% ABSTRACT
% \begin{abstract}
% Deep neural network predictions are often explained by means of class specific saliency maps, highlighting parts of the input responsible for the prediction. Different methods and evaluation metrics are based on masking inputs by such saliency maps. Transformer architectures have a built-in attention mechanism and their pooling is based on a class agnostic raw attention map. But, how exactly are these two concepts connected and can this connection be exploited to improve the interpretability properties of a network?

% In this work, we observe that the raw attention map has the same form as a class agnostic saliency map and that attention-based pooling is a form of masking in the feature space. Motivated by this observation, we design an attention-based pooling mechanism as a replacement of global average pooling in convolutional networks and we study its effect in interpreting the network predictions. This mechanism, called \emph{\OURS (\Ours)}, takes the form of a stream composed of cross attention blocks that interact with features at different network stages. The stream can be trained while the network remains frozen. We show that our approach improves the interpretability properties of different networks, while maintaining classification accuracy.
% \end{abstract}

% %%%%%%%%% BODY TEXT
    \chapter*{Introduction}
    \chaptertoc{}

    \addcontentsline{toc}{chapter}{Introduction}
    \section*{Motivation}
    %\addcontentsline{toc}{section}{Motivation}

    \section*{Dissertation Outline}
    %\addcontentsline{toc}{section}{Dissertation Outline}
    This dissertation is organized in the following manner: First
    we introduce a background on existing approaches towards interpretability of image
    recognition models; for that, we make mention on current architectures dedicated 
    to this approach, while also presenting concepts on interpretability and enunciating 
    current approaches for this study. 

    In Chapter \ref{ch:opticam}, we propose Opti-CAM as a methodology that generates 
    optimized saliency maps highlighting the relevant regions on an image towards image
    classification. On Section \ref{sec:av_gain} we extend existing evaluation metrics 
    with a novel measurement for model coinfidence. extends evaluation metrics with the 
    introduction of a novel measurement yielding improvements on model confidence 
    when using a given attribution approach. On Sections \ref{sec:oc_qual, sec:oc_quant} 
    we evaluate the effect of these contributions towards interpretability assessment.
    
    Chapter \ref{ch:castream} introduces the Cross Attention Stream, an approach that boosts
     existing architectures interpretable properties. We ste up the modulus of this approach 
    on Section \ref{sec:ca_defn} alongside its deployment on Section \ref{sec:ca_design}. 
    On Sections \ref{sec:ca_qual} and \ref{sec:ca_quant} we demonstrate the benefits 
    of using this proposal.
    
    Chapter \ref{ch:grad} characterizes a gradient denoising approch with a gradient denoising
     methodology as an approach to enhance the trainining procedure of current
    models while improving interpretability properties. On Section \ref{sec:grad_defn}, 
    we define the gradient denoising protocol alongside the regularization proposals to do so.
    Sections \ref{sec:grad_qual, sec:grad_quant} illustrate the effects of this paradigmn
    on the trained models and its effects on interpretability.
    
    Chapter \ref{ch:zip} raises the Zero-Information algorithm and its usage as a substitute
    for mask-dependent evaluation proposals. Section \ref{sec:zip_algo} develops this 
    method. Section \ref{sec:zip_insdel} demonstrates its incorporation of this 
    algorithm onto evaluation protocols. Section \ref{sec:zip_qual} displays
    the effect of this approach when applied to mask patches on images. Section 
    \ref{sec:zip_benchmark} displays the results of benchmarking these protocols 
    with this approach. 
    
    Finally, 
    we draw conclusions on our work and detail future research perspectives.


\section{Related Work}

A large number of works study \emph{explainability}, \emph{interpretability} or \emph{attribution} of machine learning models, especially DNN~\citep{guidotti2018survey, montavon2018methods, samek2021explaining, bodria2021benchmarking, li2021interpretable}. These works can be categorized into \emph{transparency} and \emph{post-hoc interpretability}~\citep{lipton2018mythos, guidotti2018survey}. The former addresses how to design an internally understandable model. Here we are interested in the latter, which treats the studied network as a black box and interprets its inner processing~\citep{ribeiro2016should, lundberg2017unified, fong2017interpretable, elliott2021explaining, selvaraju2017grad, petsiuk2018rise}. Among post-hoc methods, LIME~\citep{ribeiro2016should} and SHAP~\citep{lundberg2017unified} are well-known model-agnostic methods that rate feature importance. More specifically, we are interested in the generation of \emph{saliency maps}. These methods are mostly based on gradients, CAM~\citep{zhou2016learning}, occlusion, or a combination.

%------------------------------------------------------------------------------

\paragraph{Gradient-based methods}

Gradient-based methods~\citep{adebayo2018local,springenberg2014striving,baehrens2010explain} use the gradient of a target class score with respect to the input to measure the effect of different image regions on the prediction. In~\citep{simonyan2013deep}, the gradient is directly treated as a saliency map. Inspired by DeconvNet~\citep{zeiler2014visualizing}, \emph{guided backpropagation}~\citep{springenberg2014striving} improves the explanation by setting negative gradients to zero using ReLU units. Other methods~\citep{shrikumar2017learning, zhang2018top, bastings2020elephant} are inspired by Layer-wise Relevance Propagation (LRP)~\citep{bach2015pixel}. SmoothGrad~\citep{smilkov2017smoothgrad} and \emph{integrated gradients}~\cite{sundararajan2017axiomatic} accumulate gradients into saliency maps, while NormGrad~\citep{rebuffi2020there} attempts to unify gradient-based methods. A different approach is to use adversarial attacks~\citep{elliott2021explaining, jalwana2020attack}. Several of these methods do not satisfy the fundamental property of implementation invariance~\cite{sundararajan2017axiomatic}.

%------------------------------------------------------------------------------

\paragraph{CAM-based methods}

\emph{Class activation maps} (CAM)~\citep{zhou2016learning} is a visualization method that highlights the image regions most relevant to a target class by a linear combination of feature maps. A number of variants use different definitions of weights. Many rely on gradients, including GradCAM~\citep{selvaraju2017grad}, GradCAM++~\citep{chattopadhay2018grad}, XGradCAM~\citep{fu2020axiom}, LayerCAM~\citep{jiang2021layercam}, \modify{HiRes-CAM~\citep{draelos2020use}, and Libra-CAM~\citep{lee2022libra}}. Gradient-free methods, including Ablation-CAM~\citep{ramaswamy2020ablation}, Score-CAM~\cite{wang2020score}, SS-CAM~\citep{wang2020ss}, F-CAM~\citep{belharbi2022f}, Abs-CAM~\citep{zeng2023abs}, Poly-CAM~\citep{englebert2022poly} and Shap-CAM~\citep{zheng2022shap}, rather measure the effect on the target class score of each feature map acting as a mask on the input. We inherit the idea of masking but for linear combinations of feature maps and we iteratively optimize the coefficients by analytical gradient computation. Our method is thus faster when the number of iterations is less than the number of channels.

%------------------------------------------------------------------------------

\paragraph{Occlusion (masking)-based methods}

These methods use a number of candidate masks, measure their effect on the prediction, then combine them in a single saliency map. RISE~\citep{petsiuk2018rise} randomly masks input images and uses the class score as a weight to define a linear combination. \emph{Meaningful perturbations} \citep{fong2017interpretable} and \emph{extremal perturbations}~\citep{fong2019understanding} directly optimize the mask in the image space by using gradients. They require a large number of parameters as well as regularizers, \eg for smoothness. \emph{Information bottleneck attribution} (IBA)~\citep{schulz2020restricting} optimizes the mask in the feature space as a tensor instead. Score-CAM~\cite{wang2020score} is also an occlusion-based method, using individual feature maps as candidate masks. The same holds for our Opti-CAM, but for candidate masks constrained in the linear span of the feature maps. Compared with~\citep{fong2019understanding,schulz2020restricting}, we have fewer parameters and do not require a regularizer.

%------------------------------------------------------------------------------

\paragraph{Learning-based methods}

While occlusion-based methods compute or optimize a mask for a particular image at inference, learning-based methods use an additional network or branch and they train it on extra data and image-level labels to predict a saliency map given an input image. This includes for example generators \citep{chang2018explaining} or auto-encoders \citep{dabkowski2017real, phang2020investigating, zolna2020classifier}. This approach may be compared with weakly-supervised object detection~\citep{bilen2016weakly}, segmentation~\citep{KoLa16} or instance segmentation~\citep{AhCK19}. IBA~\citep{schulz2020restricting} includes a learning-based approach in the feature space. Apart from requiring extra data, it is not satisfying in the sense that the learned decoder would need to be explained too. Our method does not need any extra data, network, or training.

%------------------------------------------------------------------------------

\paragraph{Evaluation of attribution methods}

Evaluating saliency maps is challenging because no ground truth attributions exist. \emph{Average drop} ($\AD$) and \emph{average increase} ($\AI$), also known as increase in confidence~\cite{chattopadhay2018grad} are well-established metrics. They consider the effect on the predicted class probabilities by masking the input image with the saliency map. There is a fundamental flaw in using $\AD$, $\AI$ as a pair of metrics, which we fix by replacing $\AI$ by a new metric, \emph{average gain} ($\AG$).

\emph{Insertion} (I) and \emph{deletion} (D) sequentially insert or delete pixels by decreasing order of saliency and observe the effect on the prediction. The resulting images are out-of-distribution (OOD)~\cite{gomez2022metrics} and the metrics favor small and compact regions. Localization metrics measure how the saliency maps are aligned with object bounding boxes, which ignores the importance of background context~\cite{shetty2019not, rao2022towards}. We demonstrate that localization and attribution are not well-aligned as tasks.

\section{Method}

%------------------------------------------------------------------------------
\begin{figure*}[t]
\centering
% editable at https://docs.google.com/drawings/d/1nfqVV_dNqSV9g5UBuNopPLvubcEZx-vvg4K0ZCvNdfU/edit
\fig[.7]{CA-CAM}
\vspace{3pt}
\caption{\emph{Visualization of eq.~\eq{connection}.} On the left, a feature tensor $\vF \in \real^{w \times h \times d}$ is multiplied by the vector $\valpha \in \real^d$ in the channel dimension, like in $1 \times 1$ convolution, where $w \times h$ is the spatial resolution and $d$ is the number of channels. This is \emph{cross attention} (CA)~\cite{dosovitskiy2020image} between the query $\valpha$ and the key $\vF$. On the right, a linear combination of feature maps $F^1, \dots, F^d \in \real^{w \times h}$ is taken with weights $\alpha_1, \dots, \alpha_d$. This is a \emph{class activation mapping} (CAM)~\cite{zhou2016learning} with class agnostic weights. Eq.~\eq{connection} expresses the fact that these two quantities are the same, provided that $\valpha = (\alpha_1, \dots, \alpha_d)$ and $\vF$ is reshaped as $F = (\vf^1 \dots \vf^d) \in \real^{p \times d}$, where $p = wh$ and $\vf^k = \vect(F^k) \in \real^{p}$ is the vectorized feature map of channel $k$.}
\label{fig:connection}
\end{figure*}
%------------------------------------------------------------------------------

\subsection{Preliminaries and background}
\label{subsec:prelim}

\paragraph{Notation}

Let $f: \cX \rightarrow \real^C$ be a classifier network that maps an input image $\vx \in \cX$ to a logit vector $\vy= f(\vx) \in \real^C$, where $\cX$ is the image space and $C$ is the number of classes. A class probability vector is obtained by $\vp = \softmax(\vy)$. The logit and probability of class $c$ are respectively denoted by $y^c$ and $p^c = \softmax(\vy)^c \defn e^{y^c} / \sum_j e^{y^j}$. Let $\vF_\ell \in \real^{w_\ell \times h_\ell \times d_\ell}$ be the feature tensor at layer $\ell$ of the network, where $w_\ell \times h_\ell$ is the spatial resolution and $d_\ell$ the embedding dimension, or number of channels. The feature map of channel $k$ is denoted by $F^k_\ell \in \real^{w_\ell \times h_\ell}$. By $\ell$ we may refer to an arbitrary layer of $f$ or a larger compositional block, \eg, a stage.

\paragraph{CAM-based saliency maps}

Given a class of interest $c$ and a layer $\ell$, we consider the saliency maps $S^c_\ell \in \real^{w_\ell \times h_\ell}$ given by the general formula
\begin{equation}
	S^c_\ell \defn h \left( \sum_k \alpha^c_k F^k_\ell \right),
\label{eq:sal}
\end{equation}
where $\alpha^c_k$ are weights defining a linear combination over channels and $h$ is an activation function. Assuming \emph{global average pooling} (GAP) of the last feature tensor $\vF_L$ followed by a linear classifier, CAM~\citep{zhou2016learning} is defined for the last layer $L$ only, with $h$ being the identity mapping and $\alpha^c_k$ the classifier weight connecting channel $k$ with class $c$. Grad-CAM~\citep{DBLP:journals/corr/SelvarajuDVCPB16} is a generalization of CAM defined for any architecture and layer $\ell$, with $h = \relu$ and weights
\begin{equation}
	\alpha^c_k \defn \gap \left( \pder{y^c}{F^k_\ell} \right).
\label{eq:gcam}
\end{equation}

%------------------------------------------------------------------------------

\paragraph{Self-Attention}

Let $X_\ell \in \real^{t_\ell \times d_\ell}$ denote the sequence of token embeddings of a vision transformer~\cite{dosovitskiy2020image} at layer $\ell$, where $t_\ell \defn w_\ell h_\ell + 1$ is the number of tokens, including patch tokens and the \cls token, and $d_\ell$ is the embedding dimension. The \emph{query}, \emph{key} and \emph{value} matrices are defined as $Q = X_\ell W_Q$, $K = X_\ell W_K$, $V = X_\ell W_V \in \real^{t_\ell \times d_\ell}$, where $W_Q, W_K, W_V \in \real^{d_\ell \times d_\ell}$ are learnable linear projections. The \emph{attention matrix} $A \in \real^{t_\ell \times t_\ell}$ expresses pairwise dot-product similarities between queries (rows of $Q$) and keys (rows of $K$), normalized by softmax over rows
\begin{equation}
	A = \softmax \left( \frac{Q K\tran}{\sqrt{d_\ell}} \right).
\label{eq:attention}
\end{equation}
For each token, the \emph{self-attention} operation is then defined as an average of all values (rows of $V$) weighted by attention (the corresponding  row of $A$),
\begin{equation}
	\sa(X_\ell) \defn A V \in \real^{t_\ell \times d_\ell}.
\label{eq:SA}
\end{equation}
At the last layer $L$, the \cls token embedding is used as a global image representation for classification as it gathers information from all patches by weighted averaging, replacing \gap. Thus, at the last layer, it is only cross attention between \cls and the patch tokens that matters.



\subsection{Motivation}
\label{subsec:motiv}

\paragraph{Cross attention}

Let matrix $F_\ell \in \real^{p_\ell \times d_\ell}$ be a reshaping of feature tensor $\vF_\ell$ at layer $\ell$, where $p_\ell \defn w_\ell h_\ell$ is the number of patch tokens without \cls, and let $\vq_\ell \in \real^{d_\ell}$ be the \cls token embedding at layer $\ell$. By focusing on the \emph{cross attention} only between the \cls (query) token $\vq_\ell$ and the patch (key) tokens $F_\ell$ and by ignoring projections $W_Q, W_K, W_V$ for simplicity, attention $A$~\eq{attention} is now a $1 \times p_\ell$ matrix that can be written as a vector $\va \in \real^{p_\ell}$
\begin{equation}
	\va = A\tran = \softmax \left( \frac{F_\ell \vq_\ell}{\sqrt{d_\ell}} \right).
\label{eq:cross-attention}
\end{equation}
Here, $F_\ell \vq_\ell$ expresses the pairwise similarities between the global \cls feature $\vq_\ell$ and the local patch features $F_\ell$. Now, by replacing $\vq_\ell$ by an arbitrary vector $\valpha \in \real^{d_\ell}$ and by writing the feature matrix as $F_\ell = (\vf_\ell^1 \dots \vf_\ell^{d_\ell})$ where $\vf_\ell^k = \vect(F_\ell^k) \in \real^{p_\ell}$ for channel $k$, attention \eq{cross-attention} becomes
\begin{equation}
	\va = h_\ell (F_\ell \valpha) =
		h_\ell \left( \sum_k \alpha_k \vf_\ell^k \right).
\label{eq:connection}
\end{equation}
This takes the same form as~\eq{sal}, with feature maps $F_\ell^k$ being vectorized into $\vf_\ell^k$ and the activation function is defined as $h_\ell(\vx) = \softmax(\vx / \sqrt{d_\ell})$. Eq.~\eq{connection} is visualized in \autoref{fig:connection}. We thus observe the following.

\begin{quote}
	\emph{Pairwise similarities between one query and all patch token embeddings in cross attention are the same as a linear combination of feature maps in CAM-based saliency maps, where the weights are determined by the elements of the query.}
\end{quote}

As it stands, one difference between~\eq{sal} and~\eq{connection} is that~\eq{connection} is class agnostic, although it could be extended by using one query (weight) vector per class. For simplicity, we choose the class agnostic form in the following. We also choose to have no query/key projections. However, we do provide additional experiments with class specific representation as well as projections in \autoref{sec:gen_ablation}. 

\paragraph{Pooling, or masking}

We are thus motivated to integrate an attention mechanism into any network such that making a prediction and explaining (localizing) it are inherently connected. In particular, considering cross attention only between \cls and patch tokens~\eq{cross-attention}, equation~\eq{SA} becomes
\begin{align}
	\ca_\ell(\vq_\ell, F_\ell) \defn F_\ell\tran \va = F_\ell\tran h_\ell(F_\ell \vq_\ell) \in \real^{d_\ell}.
\label{eq:CA}
\end{align}
By writing the transpose of feature matrix as $F_\ell\tran = (\vphi_\ell^1 \dots \vphi_\ell^{p_\ell})$ where $\vphi_\ell^i \in \real^{d_\ell}$ is the feature of patch $i$, this is a weighted average of the local patch features $F_\ell\tran$ with attention vector $\va = (a_1, \dots, a_{p_\ell})$ expressing the weights:
\begin{align}
	\ca_\ell(\vq_\ell, F_\ell) \defn F_\ell\tran \va = \sum_i a_i \vphi_\ell^i.
\label{eq:CA-gap}
\end{align}
We can think of it as as feature \emph{reweighting} or \emph{soft masking} in the feature space, followed by \gap.

Now, considering that $\va$ is obtained exactly as CAM-based saliency maps~\eq{connection}, this operation is similar to occlusion (masking)-based methods~\citep{petsiuk2018rise, fong2017interpretable, fong2019understanding, schulz2020restricting, ribeiro2016should,DBLP:journals/corr/abs-1910-01279, zhang2023opti} and evaluation metrics~\cite{DBLP:journals/corr/abs-1710-11063, petsiuk2018rise}, where a CAM-based saliency map is commonly used to mask the input image. We thus observe the following.

\begin{quote}
	\emph{Attention-based pooling is a form of feature reweighting or soft masking in the feature space followed by \gap, where the weights are given by a class agnostic CAM-based saliency map.}
\end{quote}


%------------------------------------------------------------------------------

\subsection{Cross attention stream}
\label{subsec:CA-base}

Motivated by the observations above, we design a \emph{\OURS} (\emph{\Ours}) in parallel to any network. It takes input features at key locations of the network and uses cross attention to build a global image representation and replace $\gap$ before the classifier. An example is shown in \autoref{fig:fig_method}, applied to a ResNet-based architecture.

\paragraph{Architecture}

More formally, given a network $f$, we consider points between blocks of $f$ where critical operations take place, such as change of spatial resolution or embedding dimension, \eg between stages for ResNet. We decompose $f$ at these points as
\begin{equation}
	f = g \circ \gap \circ f_L \circ \dots \circ f_0
\label{eq:f-decomp}
\end{equation}
such that features $F_\ell \in \real^{p_\ell \times d_\ell}$ of layer (stage) $\ell$ are initialized as $F_{-1} = \vx$ and updated according to
\begin{equation}
	F_\ell = f_\ell(F_{\ell-1})
\label{eq:f-layer}
\end{equation}
for $0 \le \ell \le L$. The last layer features $F_L$ are followed by \gap and $g: \real^{d_L} \to \real^C$ is the classifier, mapping to the logit vector $\vy$. As in \autoref{subsec:motiv}, $p_\ell$ is the number of patch tokens and $d_\ell$ the embedding dimension of stage $\ell$.

In parallel, we initialize a classification token embedding as a learnable parameter $\vq_0 \in \real^{d_0}$ and we build a sequence of updated embeddings $\vq_\ell \in \real^{d_\ell}$ along a stream that interacts with $F_\ell$ at each stage $\ell$. Referring to the global representation $\vq_\ell$ as \emph{query} or \cls and to the local image features $F_\ell$ as \emph{key} or patch embeddings, the interaction consists of cross attention followed by a linear projection $W_\ell \in \real^{d_{\ell+1} \times d_\ell}$ to account for changes of embedding dimension between the corresponding stages of $f$:
\begin{equation}
	\vq_{\ell+1} = W_\ell \cdot \ca_\ell(\vq_\ell, F_\ell),
\label{eq:qk-layer}
\end{equation}
for $0 \le \ell \le L$, where $\ca_\ell$ is defined as in~\eq{CA}. 
% Because of linearity, projection $W_\ell$ is the same as a value projection.

Image features $F_0, \dots, F_L$ do not change by injecting our \Ours into network $f$. However, the final global image representation and hence the prediction do change. In particular, at the last stage $L$, $\vq_{L+1}$ is used as a global image representation for classification, replacing \gap over $F_L$. The final prediction is $g(\vq_{L+1}) \in \real^C$. Unlike \gap, the weights of different image patches in the linear combination are non-uniform, enhancing the contribution of relevant patches in the prediction.

%------------------------------------------------------------------------------
%------------------------------------------------------------------------------
\begin{figure*}[t]
\centering
\begin{tikzpicture}[
	font={\footnotesize},
	trap/.style={trapezium, rotate=-90,trapezium angle=75},
]
	%% CNN branch
	\node(input) at (-5.5, 0) {\includegraphics[width=.1\textwidth]{Images/Method/input.jpg}};
	\node[above] at (input.north) {Input image $\vx$};
	\node[draw, trap] (res0) at (-3.5,0) {\rotatebox{90}{\parbox{1.0cm}{\centering{Res-0}}}};
	\node[draw, trap] (res1) at (-1.5,0) {\rotatebox{90}{\parbox{1.0cm}{\centering{Res-1}}}};
	\node[draw, trap] (res2) at (0.5,0) {\rotatebox{90}{\parbox{1.0cm}{\centering{Res-2}}}};
	\node[draw, trap] (res3) at (2.5,0) {\rotatebox{90}{\parbox{1.0cm}{\centering{Res-3}}}};
	\node[draw, trap] (res4) at (4.5,0) {\rotatebox{90}{\parbox{1.0cm}{\centering{Res-4}}}};
	\node[](empt1) at (6.75, 0){};
	\node[draw, rotate=90, align=center] (class) at (7.5,0) {Classifier};
	\node(logit) at (8.25, 0) {$\vy$};
	%%% CLS stream
	\node[](clsin) at (-4, -1.5) {{$\vq_0$}};
	\node[draw](CA0) at (-2.5, -1.5) {{CA-0}};
	\node[draw](CA1) at (-0.5, -1.5) {{CA-1}};
	\node[draw](CA2) at (1.5, -1.5)  {{CA-2}};
	\node[draw](CA3) at (3.5, -1.5)  {{CA-3}};
	\node[draw](CA4) at (5.5, -1.5)  {{CA-4}};

	%% CNN backbone
	\node(empt0) at (-4.65, 0) {};
	\draw[->] (empt0.center) -- node {} (res0);
	\draw[->] (res0) -- node[above] {$F_0$} (res1);
	\draw[->] (res1) -- node[above] {$F_1$} (res2);
	\draw[->] (res2) -- node[above] {$F_2$} (res3);
	\draw[->] (res3) -- node[above] {$F_3$} (res4);
	\draw[->, blue, dashed] (res4) -- node {\blue{\normalsize//}} (class);
	\node[](GAP) at (6.25,0.25) {\blue{$\gap$}};
	\draw[->] (class) -- node {} (logit);
	%% CLS Stream
	\draw[->] (clsin) -- node {} (CA0);
	\draw[dashed, ->] (res0.north) -|node {} (CA0);
	\draw[->] (CA0) -- node[above] {$\vq_1$} (CA1);
	\draw[dashed, ->] (res1.north) -|node {} (CA1);
	\draw[->] (CA1) -- node[above] {$\vq_2$} (CA2);
	\draw[dashed, ->] (res2.north) -|node {} (CA2);
	\draw[->] (CA2) -- node[above] {$\vq_3$} (CA3);
	\draw[dashed, ->] (res3.north) -|node {} (CA3);
	\draw[->] (CA3) -- node[above] {$\vq_4$} (CA4);
	\draw[dashed, ->] (res4.north) -|node[above] {$F_4$} (CA4);
	\draw[-] (CA4.east) -| node[right] {$\vq_5$} (empt1.center);
	\draw[->] (empt1.center) -- node {} (class);
\end{tikzpicture}
\vspace{3pt}
\caption{\emph{\OURS (\Ours) applied to ResNet-based architectures.} Given a network $f$, we replace global average pooling (\gap) by a learned, attention-based pooling mechanism implemented as a stream in parallel to $f$. The feature tensor $F_\ell \in \real^{p_\ell \times d_\ell}$ (\emph{key}) obtained by stage Res-$\ell$ of $f$ interacts with a \cls token (\emph{query}) embedding $\vq_\ell \in \real^{d_\ell}$ in block CA-$\ell$, which contains cross attention~\eq{CA} followed by a linear projection~\eq{qk-layer} to adapt to the dimension of $F_{\ell+1}$. Here, $p_\ell$ is the number of patches (spatial resolution) and $d_\ell$ the embedding dimension. The query is initialized by a learnable parameter $\vq_0 \in \real^{d_0}$, while the output $\vq_5$ of the last cross attention block is used as a global image representation into the classifier. The network and classifier are pretrained and kept frozen while the parameters of \Ours are learned. At inference, we use existing post-hoc interpretability methods like Grad-CAM~\citep{DBLP:journals/corr/SelvarajuDVCPB16} to obtain saliency maps for both the baseline \gap and our \Ours. We compare interpretability metrics as well as accuracy.}
\label{fig:fig_method}
\end{figure*}

%------------------------------------------------------------------------------


\paragraph{Training}

In this sense, the network $f$ is pretrained and remains frozen while we learn the parameters of our \Ours on the same training set as one used to train $f$. The classifier is kept frozen too. Referring to~\eq{f-decomp}, $f_0, \dots, f_L$ and $g$ are fixed, while \gap is replaced by learned weighted averaging, with the weights obtained by the \Ours.

\paragraph{Inference}

As it stands, \Ours is not an interpretability method, but rather a modification of the baseline architecture, \ie, an attention-based pooling mechanism that replaces \gap to enhance the contribution of relevant image regions in the prediction. We are interested in investigating the interpretability properties of this modification. We therefore employ existing post-hoc, CAM-based interpretability methods to generate saliency maps with both baseline \gap and \Ours. We then compare interpretability metrics as well as classification accuracy.


\section{Experiments}
\label{sec:exp}

We evaluate the interpretability and recognition capabilities of our approach. In particular, we generate explanations following current state-of-the art post-hoc interpretability methods derived from CAM~\cite{zhou2016learning}. We compare the properties of the backbone network $f$ with and without our \Ours, where $f$ is pretrained and fixed.

\subsection{Experimental setup}
\label{subsec:setup}

\paragraph{Training}

We train and evaluate our models on the ImageNet ILSVRC-2012 dataset~\cite{deng2009imagenet}, on the training and validation splits respectively. Thus, we experiment with ResNet-based architectures~\cite{he2016deep} such as ResNet-18 and ResNet-50, and ConvNeXt based architectures~\cite{liu2022convnet} such as ConvNeXt-Small and ConvNeXt-Base. We aim at learning our \Ours, generating a \cls token that interacts with feature maps at different stages of network $f$, to serve as an attention-based pooling mechanism in order to interpret the predictions of $f$. Therefore, we experiment with pretrained models\footnote{https://pytorch.org/vision/0.8/models.html}, that we keep frozen while the parameters of the \Ours are optimized. Details on training hyperparameters are given in the appendix.

Moreover, we present experiments on the bird dataset: CUB-200-2011 \cite{WahCUB_200_2011} and on PASCAL VOC 2012 dataset \cite{Everingham15}. Here the ResNet-50 network is fine-tuned to these dataset as baseline. Then, our \Ours is learned as for ImageNet.

\paragraph{Evaluation}

We employ existing post-hoc interpretability methods to generate saliency maps with and without \Ours and compare interpretability metrics as well as classification accuracy. Regarding interpretability methods, we use Grad-CAM~\cite{DBLP:journals/corr/SelvarajuDVCPB16}, Grad-CAM++~\cite{DBLP:journals/corr/abs-1710-11063} and ScoreCAM~\cite{DBLP:journals/corr/abs-1910-01279}. We note that the evaluation is performed on the entire validation set, unlike the previous approaches.

Following Opti-CAM~\cite{zhang2023opti}, we use a number of classification metrics for interpretability. In particular, we consider the changes in predictive power measured by \emph{average drop} (AD)~\cite{DBLP:journals/corr/abs-1710-11063} and \emph{average gain} (AG)~\cite{zhang2023opti}, the proportion of better explanations measured by \emph{average increase} (AI)~\cite{DBLP:journals/corr/abs-1710-11063} and the impact of different extent of masking measured by \emph{insertion} (I) and \emph{deletion} (D)~\citep{petsiuk2018rise}.

% For localization, we use metrics from the \emph{weakly-supervised object localization} (WSOL) task to measure the maximum overlap between the saliency map (or corresponding predicted bounding boxes) and ground truth bounding boxes, \ie \emph{official metric} (OM), \emph{localization error} (LE), \emph{pixel-wise $F_1$ score}, \emph{box accuracy} (BoxAcc)~\citep{choe2020evaluating} and \emph{saliency metric} (SM)~\citep{dabkowski2017real}. We also measure the localization of the pixel of maximum saliency by the \emph{standard pointing game} (SP)~\cite{zhang2018top} and the fraction of the saliency map within the ground truth bounding boxes by \emph{energy pointing game} (EP)~\citep{DBLP:journals/corr/abs-1910-01279}. We obtain the ground truth bounding boxes from the ILSVRC2014\footnote{\url{https://www.image-net.org/challenges/LSVRC/2014/index\#}} dataset. 



%------------------------------------------------------------------------------
\begin{figure*}[t]
\scriptsize
\centering
\setlength{\tabcolsep}{1.5pt}
% \resizebox{\textwidth}{!}{%
\begin{tabular}{ccccccccc}
	{}&\multirow{2}{*}{Input image}&\multirow{2}{*}{Raw Attention}&\multicolumn{2}{c}{Grad-CAM}&\multicolumn{2}{c}{Grad-CAM++}&\multicolumn{1}{c}{Score-CAM}\\
	{}&{}&{}&GAP&\Ours&GAP&\Ours&GAP&\Ours\\
    % {\rotatebox{90}{\tiny Loudspeaker}}&\includegraphics[width=0.115\textwidth]{Images/Comparable/figure1_similarities/original/37729.jpeg}&\includegraphics[width=0.115\textwidth]{Images/Comparable/figure1_similarities/raw_att/37729.jpeg}&\includegraphics[width=0.115\textwidth]{Images/Comparable/figure1_similarities/shelf_gradcam/37729.jpeg}&\includegraphics[width=0.115\textwidth]{Images/Comparable/figure1_similarities/gradcam/37729.jpeg}&\includegraphics[width=0.115\textwidth]{Images/Comparable/figure1_similarities/shelf_gradcampp/37729.jpeg}&\includegraphics[width=0.115\textwidth]{Images/Comparable/figure1_similarities/gradcampp/37729.jpeg}&\includegraphics[width=0.115\textwidth]{Images/Comparable/figure1_similarities/scorecam/37729.jpeg}&\includegraphics[width=0.115\textwidth]{Images/Comparable/figure1_similarities/shelf_scorecam/37729.jpeg}\\

    {\rotatebox{90}{\tiny Envelope}}&\includegraphics[width=0.115\textwidth]{Images/Comparable/figure1_similarities/original/23541.jpeg}&\includegraphics[width=0.115\textwidth]{Images/Comparable/figure1_similarities/raw_att/23541.jpeg}&\includegraphics[width=0.115\textwidth]{Images/Comparable/figure1_similarities/shelf_gradcam/23541.jpeg}&\includegraphics[width=0.115\textwidth]{Images/Comparable/figure1_similarities/gradcam/23541.jpeg}&\includegraphics[width=0.115\textwidth]{Images/Comparable/figure1_similarities/shelf_gradcampp/23541.jpeg}&\includegraphics[width=0.115\textwidth]{Images/Comparable/figure1_similarities/gradcampp/23541.jpeg}&\includegraphics[width=0.115\textwidth]{Images/Comparable/figure1_similarities/scorecam/23541.jpeg}&\includegraphics[width=0.115\textwidth]{Images/Comparable/figure1_similarities/shelf_scorecam/23541.jpeg}\\

    {\rotatebox{90}{\tiny Groom}}&\includegraphics[width=0.115\textwidth]{Images/Comparable/figure1_similarities/original/9602.jpeg}&\includegraphics[width=0.115\textwidth]{Images/Comparable/figure1_similarities/raw_att/9602.jpeg}&\includegraphics[width=0.115\textwidth]{Images/Comparable/figure1_similarities/shelf_gradcam/9602.jpeg}&\includegraphics[width=0.115\textwidth]{Images/Comparable/figure1_similarities/gradcam/9602.jpeg}&\includegraphics[width=0.115\textwidth]{Images/Comparable/figure1_similarities/shelf_gradcampp/9602.jpeg}&\includegraphics[width=0.115\textwidth]{Images/Comparable/figure1_similarities/gradcampp/9602.jpeg}&\includegraphics[width=0.115\textwidth]{Images/Comparable/figure1_similarities/scorecam/9602.jpeg}&\includegraphics[width=0.115\textwidth]{Images/Comparable/figure1_similarities/shelf_scorecam/9602.jpeg}\\

    {\rotatebox{90}{\tiny Nematode}}&\includegraphics[width=0.115\textwidth]{Images/Comparable/figure1_similarities/original/12414.jpeg}&\includegraphics[width=0.115\textwidth]{Images/Comparable/figure1_similarities/raw_att/12414.jpeg}&\includegraphics[width=0.115\textwidth]{Images/Comparable/figure1_similarities/shelf_gradcam/12414.jpeg}&\includegraphics[width=0.115\textwidth]{Images/Comparable/figure1_similarities/gradcam/12414.jpeg}&\includegraphics[width=0.115\textwidth]{Images/Comparable/figure1_similarities/shelf_gradcampp/12414.jpeg}&\includegraphics[width=0.115\textwidth]{Images/Comparable/figure1_similarities/gradcampp/12414.jpeg}&\includegraphics[width=0.115\textwidth]{Images/Comparable/figure1_similarities/scorecam/12414.jpeg}&\includegraphics[width=0.115\textwidth]{Images/Comparable/figure1_similarities/shelf_scorecam/12414.jpeg}\\

	% {\rotatebox{90}{\tiny Waffle Iron}}&\includegraphics[width=0.115\textwidth]{Images/Comparable/figure1_revisit/original/15749.jpeg}&\includegraphics[width=0.115\textwidth]{Images/Comparable/figure1_revisit/raw_att/15749.jpeg}&\includegraphics[width=0.115\textwidth]{Images/Comparable/figure1_revisit/shelf_gradcam/15749.jpeg}&\includegraphics[width=0.115\textwidth]{Images/Comparable/figure1_revisit/gradcam/15749.jpeg}&\includegraphics[width=0.115\textwidth]{Images/Comparable/figure1_revisit/shelf_gradcampp/15749.jpeg}&\includegraphics[width=0.115\textwidth]{Images/Comparable/figure1_revisit/gradcampp/15749.jpeg}&\includegraphics[width=0.115\textwidth]{Images/Comparable/figure1_revisit/scorecam/15749.jpeg}&\includegraphics[width=0.115\textwidth]{Images/Comparable/figure1_revisit/scorecam/15749.jpeg}\\

	% {\rotatebox{90}{\tiny Wallaby}}&\multicolumn{1}{c}{\includegraphics[width=0.115\textwidth]{Images/Comparable/figure1_revisit/original/24263.jpeg}}&\multicolumn{1}{c}{\includegraphics[width=0.115\textwidth]{Images/Comparable/figure1_revisit/raw_att/24263.jpeg}}&\multicolumn{1}{c}{\includegraphics[width=0.115\textwidth]{Images/Comparable/figure1_revisit/shelf_gradcam/24263.jpeg}}&\multicolumn{1}{c}{\includegraphics[width=0.115\textwidth]{Images/Comparable/figure1_revisit/gradcam/24263.jpeg}}&\multicolumn{1}{c}{\includegraphics[width=0.115\textwidth]{Images/Comparable/figure1_revisit/shelf_gradcampp/24263.jpeg}}&\multicolumn{1}{c}{\includegraphics[width=0.115\textwidth]{Images/Comparable/figure1_revisit/gradcampp/24263.jpeg}}&\multicolumn{1}{c}{\includegraphics[width=0.115\textwidth]{Images/Comparable/figure1_revisit/scorecam/24263.jpeg}}&\multicolumn{1}{c}{\includegraphics[width=0.115\textwidth]{Images/Comparable/figure1_revisit/scorecam/24263.jpeg}}\\

	% {\rotatebox{90}{\tiny Hoopskirt}}&\multicolumn{1}{c}{\includegraphics[width=0.115\textwidth]{Images/Comparable/figure1_revisit/original/33003.jpeg}}&\multicolumn{1}{c}{\includegraphics[width=0.115\textwidth]{Images/Comparable/figure1_revisit/raw_att/33003.jpeg}}&\multicolumn{1}{c}{\includegraphics[width=0.115\textwidth]{Images/Comparable/figure1_revisit/shelf_gradcam/33003.jpeg}}&\multicolumn{1}{c}{\includegraphics[width=0.115\textwidth]{Images/Comparable/figure1_revisit/gradcam/33003.jpeg}}&\multicolumn{1}{c}{\includegraphics[width=0.115\textwidth]{Images/Comparable/figure1_revisit/shelf_gradcampp/33003.jpeg}}&\multicolumn{1}{c}{\includegraphics[width=0.115\textwidth]{Images/Comparable/figure1_revisit/gradcampp/33003.jpeg}}&\multicolumn{1}{c}{\includegraphics[width=0.115\textwidth]{Images/Comparable/figure1_revisit/scorecam/33003.jpeg}}&\multicolumn{1}{c}{\includegraphics[width=0.115\textwidth]{Images/Comparable/figure1_revisit/scorecam/33003.jpeg}}\\

	% {\rotatebox{90}{\tiny Matchstick}}&\multicolumn{1}{c}{\includegraphics[width=0.115\textwidth]{Images/Comparable/figure1_revisit/original/48096.jpeg}}&\multicolumn{1}{c}{\includegraphics[width=0.115\textwidth]{Images/Comparable/figure1_revisit/raw_att/48096.jpeg}}&\multicolumn{1}{c}{\includegraphics[width=0.115\textwidth]{Images/Comparable/figure1_revisit/shelf_gradcam/48096.jpeg}}&\multicolumn{1}{c}{\includegraphics[width=0.115\textwidth]{Images/Comparable/figure1_revisit/gradcam/48096.jpeg}}&\multicolumn{1}{c}{\includegraphics[width=0.115\textwidth]{Images/Comparable/figure1_revisit/shelf_gradcampp/48096.jpeg}}&\multicolumn{1}{c}{\includegraphics[width=0.115\textwidth]{Images/Comparable/figure1_revisit/gradcampp/48096.jpeg}}&\multicolumn{1}{c}{\includegraphics[width=0.115\textwidth]{Images/Comparable/figure1_revisit/scorecam/48096.jpeg}}&\multicolumn{1}{c}{\includegraphics[width=0.115\textwidth]{Images/Comparable/figure1_revisit/scorecam/48096.jpeg}}\\

	% {\rotatebox{90}{\tiny Sweatshirt}}&\multicolumn{1}{c}{\includegraphics[width=0.115\textwidth]{Images/Comparable/figure1/original/18939.jpeg}}&\multicolumn{1}{c}{\includegraphics[width=0.115\textwidth]{Images/Comparable/figure1/raw_att/18939.jpeg}}&\multicolumn{1}{c}{\includegraphics[width=0.115\textwidth]{Images/Comparable/figure1/shelf_gradcam/18939.jpeg}}&\multicolumn{1}{c}{\includegraphics[width=0.115\textwidth]{Images/Comparable/figure1/gradcam/18939.jpeg}}&\multicolumn{1}{c}{\includegraphics[width=0.115\textwidth]{Images/Comparable/figure1/shelf_gradcampp/18939.jpeg}}&\multicolumn{1}{c}{\includegraphics[width=0.115\textwidth]{Images/Comparable/figure1/gradcampp/18939.jpeg}}&\multicolumn{1}{c}{\includegraphics[width=0.115\textwidth]{Images/Comparable/figure1/shelf_scorecam/18939.jpeg}}&\multicolumn{1}{c}{\includegraphics[width=0.115\textwidth]{Images/Comparable/figure1/scorecam/18939.jpeg}}\\

	{\rotatebox{90}{\tiny CRT screen}}&\multicolumn{1}{c}{\includegraphics[width=0.115\textwidth]{Images/Comparable/figure1/original/43057.jpeg}}&\multicolumn{1}{c}{\includegraphics[width=0.115\textwidth]{Images/Comparable/figure1/raw_att/43057.jpeg}}&\multicolumn{1}{c}{\includegraphics[width=0.115\textwidth]{Images/Comparable/figure1/shelf_gradcam/43057.jpeg}}&\multicolumn{1}{c}{\includegraphics[width=0.115\textwidth]{Images/Comparable/figure1/gradcam/43057.jpeg}}&\multicolumn{1}{c}{\includegraphics[width=0.115\textwidth]{Images/Comparable/figure1/shelf_gradcampp/43057.jpeg}}&\multicolumn{1}{c}{\includegraphics[width=0.115\textwidth]{Images/Comparable/figure1/gradcampp/43057.jpeg}}&\multicolumn{1}{c}{\includegraphics[width=0.115\textwidth]{Images/Comparable/figure1/shelf_scorecam/43057.jpeg}}&\multicolumn{1}{c}{\includegraphics[width=0.115\textwidth]{Images/Comparable/figure1/scorecam/43057.jpeg}}\\ % Checked 

   {\rotatebox{90}{\tiny Snowboard}}&\includegraphics[width=0.115\textwidth]{Images/Comparable/figure1_similarities/original/11376.jpeg}&\includegraphics[width=0.115\textwidth]{Images/Comparable/figure1_similarities/raw_att/11376.jpeg}&\includegraphics[width=0.115\textwidth]{Images/Comparable/figure1_similarities/shelf_gradcam/11376.jpeg}&\includegraphics[width=0.115\textwidth]{Images/Comparable/figure1_similarities/gradcam/11376.jpeg}&\includegraphics[width=0.115\textwidth]{Images/Comparable/figure1_similarities/shelf_gradcampp/11376.jpeg}&\includegraphics[width=0.115\textwidth]{Images/Comparable/figure1_similarities/gradcampp/11376.jpeg}&\includegraphics[width=0.115\textwidth]{Images/Comparable/figure1_similarities/scorecam/11376.jpeg}&\includegraphics[width=0.115\textwidth]{Images/Comparable/figure1_similarities/shelf_scorecam/11376.jpeg}\\

 
\end{tabular}
% }
\vspace{3pt}
\caption{Comparison of saliency maps generated by different CAM-based methods, using GAP and our \Ours, on ImageNet images. The raw attention is the one used for pooling by \Ours.}
\label{fig:compmethods}
\end{figure*}
%------------------------------------------------------------------------------

%------------------------------------------------------------------------------

\subsection{Qualitative evaluation}
\label{subsec:vinspection}

We show saliency maps obtained by different interpretability methods using either \gap or \Ours, as well as the class-agnostic raw attention coming from our \Ours, see \autoref{fig:compmethods}.
%, including the class-agnostic raw attention obtained by \Ours.

We observe that the raw attention focuses on objects of interest in the images. 
%\textcolor{orange}{
In general, saliency maps obtained with \Ours are similar but tend to cover larger regions of the object or more instances compared with \gap.%, see \autoref{fig:compmethods}.
%, for example in the ``CRT screen'' image in \autoref{fig:compmethods}. 
%}
%
Indeed, the differences in saliency maps should not be large, as both methods share the same features maps $F^k_\ell$ and only the weight coefficients $\alpha^c_k$ differ.
Despite the small differences, the following quantitative results show that \Ours has a significant impact on the interpretability metrics.

%
%\textcolor{red}{
%As a further investigation, we compute cosine similarities of saliency maps and obtain 0.998 (or 6.49\% euclidean distance), averaged over the dataset. We then observed a similarity of 0.752 for the pooled features and 0.612 for the gradient.
%Finally, the percentage difference in probability for the ground truth class is of 27.8\% on average.
%These results show that the small changes in the saliency maps are meaningful.
%}


%In some examples, the saliency maps obtained are quite similar.%using \gap and \Ours are not very different. 
%The differences are expected to be low, as both methods share the same features maps $F^k_\ell$ and only the weight coefficients $\alpha^c_k$ differ.
%%, while the CAM-based methods have their own mechanism to focus on what is relevant for the prediction of the class of interest. 
%Despite the small visual differences, the following quantitative results show that \Ours has a significant impact on the interpretability metrics.

%------------------------------------------------------------------------------
\begin{figure}[t]
\scriptsize
\centering
\setlength{\tabcolsep}{1.3pt}
%    \resizebox{\columnwidth}{!}{%
     \begin{tabular}{cccccccc}
           \mc{2}{Corridor}&\mc{2}{Greenhouse}&\mc{2}{Pool Inside}&\mc{2}{Wine Cellar}\\
           Input image&Raw Attention&Input image&Raw Attention&Input image&Raw Attention&Input image&Raw Attention\\
           \includegraphics[width=0.12\textwidth,height=0.08\textwidth]{fig/castream/images/Outdataset/Corridor/Original/c1.jpg}&
           \includegraphics[width=0.12\textwidth,height=0.08\textwidth]{fig/castream/images/Outdataset/Corridor/Attention/c1.jpg}&
           \includegraphics[width=0.12\textwidth,height=0.08\textwidth]{fig/castream/images/Outdataset/Greenhouse/Original/celosie_02.jpg}&
           \includegraphics[width=0.12\textwidth,height=0.08\textwidth]{fig/castream/images/Outdataset/Greenhouse/Attention/celosie_02.jpg}&
           \includegraphics[width=0.12\textwidth,height=0.08\textwidth]{fig/castream/images/Outdataset/Poolinside/Original/003_1b.jpg}&
           \includegraphics[width=0.12\textwidth,height=0.08\textwidth]{fig/castream/images/Outdataset/Poolinside/Attention/003_1b.jpg}&
           \includegraphics[width=0.12\textwidth,height=0.08\textwidth]{fig/castream/images/Outdataset/WineCellar/Original/bodega2.jpg}&
           \includegraphics[width=0.12\textwidth,height=0.08\textwidth]{fig/castream/images/Outdataset/WineCellar/Attention/bodega2.jpg}\\
           
           \includegraphics[width=0.12\textwidth,height=0.08\textwidth]{fig/castream/images/Outdataset/Corridor/Original/1L_10_Corridor_A.jpg}&
           \includegraphics[width=0.12\textwidth,height=0.08\textwidth]{fig/castream/images/Outdataset/Corridor/Attention/1L_10_Corridor_A.jpg}&
           \includegraphics[width=0.12\textwidth,height=0.08\textwidth]{fig/castream/images/Outdataset/Greenhouse/Original/20070417klpcnatun_229_Ies_SCO.jpg}&
           \includegraphics[width=0.12\textwidth,height=0.08\textwidth]{fig/castream/images/Outdataset/Greenhouse/Attention/20070417klpcnatun_229_Ies_SCO.jpg}&
           \includegraphics[width=0.12\textwidth,height=0.08\textwidth]{fig/castream/images/Outdataset/Poolinside/Original/141821195_M.jpg}&
           \includegraphics[width=0.12\textwidth,height=0.08\textwidth]{fig/castream/images/Outdataset/Poolinside/Attention/141821195_M.jpg}&
           \includegraphics[width=0.12\textwidth,height=0.08\textwidth]{fig/castream/images/Outdataset/WineCellar/Original/bodega_45_18_yahoo.jpg}&
           \includegraphics[width=0.12\textwidth,height=0.08\textwidth]{fig/castream/images/Outdataset/WineCellar/Attention/bodega_45_18_yahoo.jpg}\\
           
           \includegraphics[width=0.12\textwidth,height=0.08\textwidth]{fig/castream/images/Outdataset/Corridor/Original/430_Korridor_300.jpg}&
           \includegraphics[width=0.12\textwidth,height=0.08\textwidth]{fig/castream/images/Outdataset/Corridor/Attention/430_Korridor_300.jpg}&
           \includegraphics[width=0.12\textwidth,height=0.08\textwidth]{fig/castream/images/Outdataset/Greenhouse/Original/20070418klpcnaecl_364_Ies_SCO.jpg}&
           \includegraphics[width=0.12\textwidth,height=0.08\textwidth]{fig/castream/images/Outdataset/Greenhouse/Attention/20070418klpcnaecl_364_Ies_SCO.jpg}&
           \includegraphics[width=0.12\textwidth,height=0.08\textwidth]{fig/castream/images/Outdataset/Poolinside/Original/catalogue_piscine_interieur.jpg}&
           \includegraphics[width=0.12\textwidth,height=0.08\textwidth]{fig/castream/images/Outdataset/Poolinside/Attention/catalogue_piscine_interieur.jpg}&
           \includegraphics[width=0.12\textwidth,height=0.08\textwidth]{fig/castream/images/Outdataset/WineCellar/Original/bodega_63_24_flickr.jpg}&
           \includegraphics[width=0.12\textwidth,height=0.08\textwidth]{fig/castream/images/Outdataset/WineCellar/Attention/bodega_63_24_flickr.jpg}\\
           
           \includegraphics[width=0.12\textwidth,height=0.08\textwidth]{fig/castream/images/Outdataset/Corridor/Original/06_Right_corridor_of_the_main_hall.jpg}&
           \includegraphics[width=0.12\textwidth,height=0.08\textwidth]{fig/castream/images/Outdataset/Corridor/Attention/06_Right_corridor_of_the_main_hall.jpg}&
           \includegraphics[width=0.12\textwidth,height=0.08\textwidth]{fig/castream/images/Outdataset/Greenhouse/Original/2026_2006_Grimm_s_Gardens_Greenhouse.jpg}&
           \includegraphics[width=0.12\textwidth,height=0.08\textwidth]{fig/castream/images/Outdataset/Greenhouse/Attention/2026_2006_Grimm_s_Gardens_Greenhouse.jpg}&
           \includegraphics[width=0.12\textwidth,height=0.08\textwidth]{fig/castream/images/Outdataset/Poolinside/Original/connolly_center_pool_inside_lg.jpg}&
           \includegraphics[width=0.12\textwidth,height=0.08\textwidth]{fig/castream/images/Outdataset/Poolinside/Attention/connolly_center_pool_inside_lg.jpg}&
           \includegraphics[width=0.12\textwidth,height=0.08\textwidth]{fig/castream/images/Outdataset/WineCellar/Original/bodega_78_08_flickr.jpg}&
           \includegraphics[width=0.12\textwidth,height=0.08\textwidth]{fig/castream/images/Outdataset/WineCellar/Attention/bodega_78_08_flickr.jpg}\\              
    \end{tabular}
%    }
    \vspace{3pt}
    \caption{\textbf{Raw attention maps} obtained from our \Ours on images of the MIT 67 Scenes dataset \autocite{quattoni2009recognizing} on classes that do not exist in ImageNet. The network sees them at inference for the first time.} 
    %
    \label{fig:enter-label}
\end{figure}
%------------------------------------------------------------------------------

In addition, \autoref{fig:enter-label} shows examples of images from the MIT 67 Scenes dataset~\cite{quattoni2009recognizing} along with raw attention maps obtained by \Ours. These images come from four classes that do not exist in ImageNet and the network sees them at inference for the first time. Nevertheless, the attention maps focus on objects of interest in general.

\subsection{Interpretabity metrics}
\label{subsec:interecon}

%------------------------------------------------------------------------------
\begin{table}
\centering
\scriptsize
\setlength{\tabcolsep}{3.5pt}
% \resizebox{\columnwidth}{!}{%
\begin{tabular}{llcccccc}\toprule
	\mc{8}{\Th{Accuracy and Parameters}}\\\midrule
	\Th{Network}&\mc{1}{\Th{Pool}}&\Th{GFLOPs}&\mc{2}{\Th{$\#$Param}}&\mc{2}{\Th{Param$\%$}}&\Th{Acc$\uparrow$}\\\midrule
	\mr{2}{\Th{ResNet-18}}&\mc{1}{\gap}&3.648&\mc{2}{11.69M}&\mc{2}{\mr{2}{3.71}}&67.28\\
		&\mc{1}{\ours}&3.652&\mc{2}{12.13M}&&&67.54\\\midrule
	\mr{2}{\Th{ResNet-50}}&\mc{1}{\gap}&8.268&\mc{2}{25.56M}&\mc{2}{\mr{2}{27.27}}&74.55\\
		&\mc{1}{\ours}&8.288&\mc{2}{32.53M}&&&74.70\\\midrule
	\mr{2}{\Th{ConvNeXt-S}}&\mc{1}{\gap}&17.395&\mc{2}{50.22M}&\mc{2}{\mr{2}{1.95}}&83.26\\
		&\mc{1}{\ours}&17.400&\mc{2}{51.20M}&&&83.14\\\midrule
	\mr{2}{\Th{ConvNeXt-B}}&\mc{1}{\gap}&30.747&\mc{2}{88.59M}&\mc{2}{\mr{2}{1.96}}&83.72\\
		&\mc{1}{\ours}&30.753&\mc{2}{90.33M}&&&83.51\\\midrule
%	\mr{2}{\Th{ViT-Base}}&&\cls\footnotemark{}&\mc{2}{86.58M}&\mc{2}{\mr{2}{8.18}}&80.01\\
%		&&\ours&\mc{2}{93.66M}&&&74.73\\\midrule
		
	\mc{8}{\Th{Interpretability Metrics}}\\\midrule
	\Th{Network}&\Th{Method}&\Th{Pool}&\Th{AD$\downarrow$}&\Th{AG$\uparrow$}&\Th{AI$\uparrow$}&\Th{I$\uparrow$}&\Th{D$\downarrow$}\\\midrule

	\mr{7}{\Th{ResNet-18}}&\mr{2}{Grad-CAM}&\gap&17.64&12.73&41.21&63.13&\textbf{10.66}\\ %
		& &\ours&\textbf{16.99}&\textbf{17.22}&\textbf{44.95}&\textbf{65.94}&10.68\\\cmidrule{2-8} %
		& \mr{2}{Grad-CAM++}&\gap&19.05&11.16&37.99&62.80&\textbf{10.75}\\ %
		& &\ours&\textbf{19.02}&\textbf{14.76}&\textbf{40.82}&\textbf{65.53}&10.82\\\cmidrule{2-8} %
		& \mr{2}{Score-CAM}&\gap&13.64&12.98&44.53&62.56&\textbf{11.37}\\ %
		& &\ours&\textbf{11.53}&\textbf{18.12}&\textbf{50.32}&\textbf{65.33}&11.51\\\midrule %

	\mr{7}{\Th{ResNet-50}}&\mr{2}{Grad-CAM}&\gap&13.04&17.56&44.47&72.57&\textbf{13.24}\\ %
		& &\ours&\textbf{12.54}&\textbf{22.67}&\textbf{48.56}&\textbf{75.53}&13.50\\\cmidrule{2-8} %
		& \mr{2}{Grad-CAM++}&\gap&\textbf{13.79}&15.87&42.08&72.32&\textbf{13.33}\\ %
		& &\ours&13.99&\textbf{19.29}&\textbf{44.60}&\textbf{75.21}&13.78\\\cmidrule{2-8} %
		& \mr{2}{Score-CAM}&\gap&8.83&17.97&48.46&71.99&\textbf{14.31}\\ %
		& &\ours&\textbf{7.09}&\textbf{23.65}&\textbf{54.20}&\textbf{74.91}&14.68\\\midrule%

	% \multirow{6}{*}{MobileNet-V2}&\mc{2}{Grad-CAM}&\gap&16.11&12.89&40.27&64.47&\textbf{11.92}\\
	%   & &\ours&\textbf{15.53}&\textbf{16.13}&\textbf{42.95}&\textbf{67.80}&12.24\\\cmidrule{2-8}
	%   & \mc{2}{Grad-CAM++}&\gap&17.65&11.12&37.30&64.08&\textbf{11.96}\\
	%   & &\ours&\textbf{17.55}&\textbf{13.52}&\textbf{38.61}&\textbf{67.36}&12.28\\\cmidrule{2-8}
	%   & \mc{2}{Score-CAM}&\gap&12.32&13.52&43.94&64.13&\textbf{12.23}\\
	%   & &\ours&\textbf{10.71}&\textbf{17.35}&\textbf{47.81}&\textbf{67.41}&12.65\\\midrule

	% \multirow{6}{*}{ConvNext Tiny}&\mc{2}{Grad-CAM}&\gap&43.15&2.82&16.58&49.15&\textbf{23.68}\\ % Numbers match, check saliency
	%   & &\ours&\textbf{22.49}&\textbf{14.03}&\textbf{30.65}&\textbf{82.55}&35.22\\\cmidrule{2-8} % Revision running
	%   & \mc{2}{Grad-CAM++}&\gap&43.96&2.45&15.67&48.95&\textbf{23.90}\\ % Pending
	%   & &\ours&\textbf{25.90}&\textbf{12.16}&\textbf{27.70}&\textbf{82.01}&40.42\\\cline{2-8} % Revision running
	%   & \mc{2}{Score-CAM}&\gap&48.25&1.86&13.47&47.12&\textbf{35.38}\\ % Pending
	%   & &\ours&\textbf{19.69}&\textbf{12.42}&\textbf{28.92}&\textbf{80.21}&49.25\\\midrule % Revision running

	\mr{7}{\Th{ConvNeXt-S}}&\mr{2}{Grad-CAM}&\gap&42.99&1.69&12.60&48.42&\textbf{30.12}\\ % Numbers match, check saliency
		& &\ours&\textbf{22.09}&\textbf{14.91}&\textbf{32.65}&\textbf{84.82}&43.02\\\cmidrule{2-8} % Revision running
		& \mr{2}{Grad-CAM++}&\gap&56.42&1.32&10.35&48.28&\textbf{33.41}\\ % Pending
		& &\ours&\textbf{51.87}&\textbf{9.40}&\textbf{20.55}&\textbf{84.28}&52.58\\\cmidrule{2-8} % Revision running
		& \mr{2}{Score-CAM}&\gap&74.79&1.29&10.10&47.40&\textbf{38.21}\\ % Pending
		& &\ours&\textbf{64.21}&\textbf{8.81}&\textbf{18.96}&\textbf{82.92}&57.46\\\midrule % Revision running

	\mr{7}{\Th{ConvNeXt-B}}&\mr{2}{Grad-CAM}&\gap&33.72&2.43&15.25&52.85&\textbf{29.57}\\ % Pending
		& &\ours&\textbf{19.45}&\textbf{13.96}&\textbf{32.89}&\textbf{86.38}&45.29\\\cmidrule{2-8} % Revision running
		& \mr{2}{Grad-CAM++}&\gap&\textbf{34.01}&2.37&15.60&52.83&\textbf{29.17}\\ % Pending
		& &\ours&36.69&\textbf{8.00}&\textbf{21.95}&\textbf{85.39}&53.42\\\cmidrule{2-8} % Revision running
		& \mr{2}{Score-CAM}&\gap&43.55&2.23&15.67&50.96&\textbf{39.49}\\ % Pending
		& &\ours&\textbf{23.51}&\textbf{11.04}&\textbf{27.35}&\textbf{83.41}&60.53\\\midrule% Revision running

%	\mr{7}{\Th{ViT-B}}&\mr{2}{Grad-CAM}&\cls\footnotemark[\value{footnote}]&83.66&1.49&7.37&66.43&34.98\\ % Pending
%		& &\ours&49.88&4.07&16.10&67.12&10.25\\\cmidrule{2-8} % Revision running
%		& \mr{2}{Grad-CAM++}&\cls\footnotemark[\value{footnote}]&97.03&0.01&1.36&\textbf{66.80}&33.35\\ % Pending
%		& &\ours&74.81&1.64&7.43&61.95&28.29\\\cmidrule{2-8} % Revision running
%		& \mr{2}{Score-CAM}&\cls\footnotemark[\value{footnote}]&TBA&TBA&TBA&TBA&TBA\\ % Pending
%		& &\ours&TBA&TBA&TBA&TBA&TBA\\\bottomrule % Revision running
\end{tabular}
% }
%\vspace{3pt}
\caption{\emph{Accuracy, parameters and interpretability metrics} of \Ours \vs baseline \gap for different networks and interpretability methods on ImageNet. \Th{$\#$Param}: total parameters; \Th{Param$\%$}: percentage of \Ours parameters relative to backbone.}
\label{tab:intrecon-all}
\end{table}
%------------------------------------------------------------------------------
% \footnotetext{Built-in \cls token from Vision Transformers.}    
%------------------------------------------------------------------------------

Here we measure the effect of employing our \Ours approach to pool features \vs the baseline \gap on the faithfulness of explanations, using classification metrics for interpretability. Results are reported in \autoref{tab:intrecon-all} for ImageNet and  \autoref{tab:pascal} for CUB and Pascal VOC.

\autoref{tab:intrecon-all} shows that for different networks, CAM-based interpretability methods and dataset, \Ours provides consistent improvements over \gap in terms of AD, AG, AI and I metrics, while performing lower on D. 
%
%\textcolor{red}{
Deletion has raised concerns in previous works \cite{chefer2021transformer, zhang2023opti}. Indeed, it gradually replaces pixels by black, unlike insertion which starts from a blurred image. This poses the problem of \emph{out-of-distribution} (OOD) data~\cite{gomez2022metrics, hase2021outofdistribution, qiu2021resisting}, possibly introducing bias related to the shape of black regions~\cite{rong2022consistent}. Moreover, non-spread saliency maps tend to perform better \cite{zhang2023opti}, which is likely the reason for lower performance. %We suppose that erasing one main object area is more efficient.  
%}

%\textcolor{orange}{Considering that saliency maps of the two pooling methods do not present large differences, the observed improvement in interpretability metrics can be attributed to the predictive power of the learned attention mechanism. 
%}

%That is, class probabilities can change due to attention, even if saliency maps, obtained features and predictions are the same.}
%\textcolor{red}{
Results on CUB in \autoref{tab:pascal} show that our \Ours consistently provides improvements when the model is finetuned on a smaller fine-grained dataset.
%}

%Table \autoref{tab:pascal} present interpretability results for the PASCAL dataset \cite{Everingham15}.
% \textcolor{red}{
Regarding Pascal VOC, the results for Score-CAM are similar to the ones on ImageNet and CUB, with consistent improvements on all metrics but Deletion. 
However, Grad-CAM and Grad-CAM++ only provide improvements on Average Gain and Average Increase. Average Drop, Insertion and Deletion are very similar.
In fact, Pascal VOC is a multi-class dataset and  our \Ours is class agnostic. Thus, the attention-based pooling is the same for different class for a given image, which reduces the benefit of our \Ours.
%We believe the fact that 
% }

It is also interesting to observe the performance of Score-CAM, as it computes channel weights $\alpha_k^c$ in~\eq{sal} without using gradients. 
In gradient-based methods, channel weights are modified by \Ours due to modified backward gradient flow to features through cross attention blocks rather than \gap.
In Score-CAM however, channel weights are only modified in the forward class probabilities computation, due to attention.

% Regarding the results on ViT-Base, we observe the previously reported failure of CAM based methods on vision transformers\cite{zhang2023opti,chefer2021transformer}; in contrast we note that with the addition of our stream, the interpretability properties \textbf{changes.}

%------------------------------------------------------------------------------

% \subsection{Localization Evaluation}
% \label{subsec:loceval}
% 
% In this section, we assess the localization properties of the saliency maps provided by the inclusion of the CLS stream-based representation. We report these results in table \ref{tab:localization}. It shows that all the localization metrics provide similar scores on different methods and networks. CLS helps saliency maps keep localization information while gaining more classification information. 
% \begin{table}[H]
% 	\centering
% 	%\resizebox{\columnwidth}{!}{%
% 	\scriptsize
% 	\setlength{\tabcolsep}{2pt}
% 	\begin{tabular}{lllccc|cccc}\toprule
% 		\Th{Backbone}&\Th{Method}&\Th{Repr}&\Th{OM$\downarrow$}&\Th{LE$\downarrow$}&\Th{F1$\uparrow$}&\Th{BA$\uparrow$}&\Th{SP$\uparrow$}&\Th{EP$\uparrow$}&\Th{SM$\downarrow$}\\\midrule
% 		\mr{6}{\Th{ResNet-18}}&\mr{2}{\Th{Grad-CAM}}&GAP&\tb{80.6}&61.4&49.1&\tb{58.3}&13.7&13.1&\tb{2.93}\\
% 			& &CLS&81.1&\tb{61.3}&49.1&58.2&13.7&13.1&3.10\\ \cmidrule{3-10}
% 			&\mr{2}{\Th{Grad-CAM++}}&GAP&\tb{80.5}&60.9&50.0&\tb{58.3}&13.7&13.1&\tb{2.89}\\
% 			& &CLS&81.0&60.9&\tb{50.1}&58.0&13.7&13.1&3.07\\\cmidrule{3-10}
% 			&\mr{2}{\Th{Score-CAM}}&GAP&\tb{80.5}&60.9&\tb{49.7}&57.7&13.7&13.1&\tb{2.89}\\
% 			& &CLS&81.0&60.9&49.5&57.3&13.7&13.1&3.07\\ \midrule
% 			\mr{6}{\Th{ResNet-50}}&\mr{2}{\Th{Grad-CAM}}&GAP&80.0&66.4&49.5&58.8&13.8&13.2&\tb{2.56}\\
% 			& &CLS&80.0&66.4&\tb{49.6}&58.8&13.8&13.2&2.62\\\cmidrule{3-10}
% 			&\mr{2}{\Th{Grad-CAM++}}&GAP&\tb{80.2}&\tb{66.7}&50.4&\tb{58.7}&13.9&13.3&\tb{2.55}\\
% 			& &CLS&80.3&66.9&\tb{50.6}&58.0&13.9&13.3&2.64\\\cmidrule{3-10}
% 			&\mr{2}{\Th{Score-CAM}}&GAP&\tb{80.0}&66.3&\tb{50.3}&\tb{57.9}&13.8&13.2&\tb{2.55}\\
% 			& &CLS&80.1&66.3&50.0&57.5&13.8&13.2&2.62\\\bottomrule
% 	\end{tabular}
% 	%}
% 	\caption{Comparison of interpretable localization for ResNet on ImageNet.}
% 	\label{tab:localization}
% \end{table}

%------------------------------------------------------------------------------

\subsection{Classification accuracy}
\label{subsec:classification}

%Here we measure the effect of employing our \Ours approach to pool features \vs the baseline \gap on 

Classification accuracy, number of parameters and GFLOPs for both our \Ours and the baselines are reported in \autoref{tab:intrecon-all} (top part).

By adding our \Ours to the network, classification remains on par with the baseline. Importantly, the network including the classifier remains frozen and the features used for the global image representation remain fixed, meaning that any change in accuracy is due to the attention-based pooling mechanism. 
%
% \textcolor{red}{
We further report the number of GFLOPs for one forward pass and the parameters count of both methods.
Our \Ours has little computation cost and the parameter overhead depends on the embedding dimension because of projection $W_\ell$ in~\eq{qk-layer} and is small in general, except for ResNet-50. Thus, with small overhead in resources, \Ours achieves superior explanations of the classifier predictions, while maintaining accuracy.
% }

%------------------------------------------------------------------------------
\begin{table}
\centering
\scriptsize
\setlength{\tabcolsep}{4pt}
%\resizebox{\columnwidth}{!}{%
\begin{tabular}{llccccc}\toprule                    
	\mc{7}{\textbf{\Th{CUB-200-2011 - ResNet-50}}}\\\midrule
	&\Th{Pooling}&\mc{2}{}&\mc{2}{}&\Th{Acc$\uparrow$}\\\midrule
		&\gap&\mc{2}{}&\mc{2}{}&76.96\\
		&\ours&\mc{2}{}&\mc{2}{}&75.90\\\midrule
	
	\mc{7}{\Th{Interpretability Metrics}}\\\midrule
	\Th{Method}&\Th{Pooling}&AD$\downarrow$&AG$\uparrow$&AI$\uparrow$&I$\uparrow$&D$\downarrow$\\\midrule
	% \multirow{3}{*}{GAP}&\multirow{3}{*}{74.55}&Grad-CAM&13.04&17.56&44.47&72.57&13.24\\
	%  & &Grad-CAM++&13.79&15.87&42.08&72.32&13.33\\
	%  & &Score-CAM&13.64&12.98&44.53&62.56&11.37\\\hline %
	\mr{2}{Grad-CAM}&\gap&10.87&10.29&45.81&65.71&\textbf{6.17}\\
		&\ours&\textbf{10.44}&\textbf{17.61}&\textbf{53.54}&\textbf{74.60}&6.56\\\midrule
	\mr{2}{Grad-CAM++}&\gap&11.35&9.68&44.32&65.64&\textbf{5.92}\\
		&\ours&\textbf{11.01}&\textbf{16.50}&\textbf{51.63}&\textbf{74.64}&6.21\\\midrule
	\mr{2}{Score-CAM}&\gap&9.05&10.62&48.90&65.58&5.94\\
		&\ours&\textbf{6.37}&\textbf{19.50}&\textbf{60.41}&\textbf{74.22}&\textbf{2.14}\\

\midrule
  \midrule

	\mc{7}{\textbf{\Th{Pascal VOC 2012 - ResNet-50}}}\\\midrule
	&\Th{Pooling}&\mc{2}{}&\mc{2}{}&\Th{mAP$\uparrow$}\\\midrule
		&\gap&\mc{2}{}&\mc{2}{}&78.32\\
		&\ours&\mc{2}{}&\mc{2}{}&78.35\\\midrule
	
	\mc{7}{\Th{Interpretability Metrics}}\\\midrule
	\Th{Method}&\Th{Pooling}&AD$\downarrow$&AG$\uparrow$&AI$\uparrow$&I$\uparrow$&D$\downarrow$\\\midrule
	% \multirow{3}{*}{GAP}&\multirow{3}{*}{74.55}&Grad-CAM&13.04&17.56&44.47&72.57&13.24\\
	%  & &Grad-CAM++&13.79&15.87&42.08&72.32&13.33\\
	%  & &Score-CAM&13.64&12.98&44.53&62.56&11.37\\\hline %
	\mr{2}{Grad-CAM}&\gap&\textbf{12.61}&9.68&27.88&\textbf{89.10}&59.39\\
		&\ours&12.77&\textbf{15.46}&\textbf{34.53}&88.53&\textbf{59.16}\\\midrule
	\mr{2}{Grad-CAM++}&\gap&\textbf{12.25}&9.68&27.62&\textbf{89.34}&54.23\\
		&\ours&12.28&\textbf{16.76}&\textbf{34.87}&89.02&\textbf{53.34}\\\midrule
	\mr{2}{Score-CAM}&\gap&14.8&6.76&36.41&71.10&\textbf{39.95}\\
		&\ours&\textbf{10.96}&\textbf{21.35}&\textbf{43.82}&\textbf{89.21}&51.44\\\bottomrule
  
\end{tabular}
%}
%\vspace{3pt}
\caption{Accuracy, respectively mean Average Precision, and interpretability metrics of \Ours \vs baseline \gap for ResNet-50 on CUB and Pascal dataset.}
\label{tab:pascal}
\end{table}
%------------------------------------------------------------------------------

\subsection{Ablation}
\label{sec:gen_ablation}

We conduct ablation experiments on ResNet50 because of its modularity and ease of modification. We investigate the effect of the cross attention block design, the placement of the \Ours relative to the backbone network.
The appendix further includes an extra experiments regarding class agnostic \vs class specific representations.

%------------------------------------------------------------------------------

\paragraph{Cross attention block design}

Following transformers~\cite{NIPS2017_3f5ee243,dosovitskiy2020image}, it is possible to add more layers in the cross attention block. We consider a variant referred to as \PO, which uses linear projections $W_\ell^K, W_\ell^V \in \real^{d_\ell \times d_\ell}$ on the key and value
\begin{equation}
	\ca_\ell(\vq_\ell, F_\ell) \defn (F_\ell W^V_\ell)\tran h_\ell(F_\ell W^K_\ell \vq_\ell)\in \real^{d_\ell},
\label{eq:proj_ca}
\end{equation}
while equation~\eq{qk-layer} remains.

% We consider two variants. One, referred to as \PO, uses linear projections $W_\ell^K, W_\ell^V \in \real^{d_\ell \times d_\ell}$ on the key and value
% \begin{equation}
% 	\ca_\ell(\vq_\ell, F_\ell) \defn (F_\ell W^V_\ell)\tran h_\ell(F_\ell W^K_\ell \vq_\ell)\in \real^{d_\ell}.
% \label{eq:proj_ca}
% \end{equation}
% In another variant, referred to as \OM, an MLP follows each cross attention block, defined as
% \begin{equation}
% 	\mlp_\ell(\vq_\ell) = W_\ell^2 \gelu(W_\ell^1 \vq_\ell + b_\ell^1) + b_\ell^2,
% \label{eq:mlp_ca}
% \end{equation}
% where $W_\ell^1 \in \real^{2d_\ell \times d_\ell}$ and $W_\ell^1 \in \real^{d_\ell \times 2d_\ell}$. In both cases, equation~\eq{qk-layer} remains. The combination of the two variants is referred to as \POM.
% 
% We compare the effect on accuracy when adding one of the variants to the backbone (just after the last residual block). \autoref{tab:CA_variations} demonstrates this assessment.
% 
% %------------------------------------------------------------------------------
% \begin{table}
% \centering
% %\resizebox{\columnwidth}{!}{%
% \scriptsize
% \begin{tabular}{lcc}\\\toprule
% 	\Th{Block Type}&\Th{$\#$Params}&\Th{Acc}\\\midrule
% 	CA&4.20M&74.63\\
% 	Proj$\rightarrow$CA&12.58M&74.19\\
% 	CA$\rightarrow$MLP&20.97M&-\\
% 	Proj$\rightarrow$CA$\rightarrow$MLP&29.36M&-\\\bottomrule
% \end{tabular}
% %}
% \vspace{3pt}
% \caption{Comparison of classification performance when one cross attention block is added into the last residual block of ResNet-50.}
% \label{tab:CA_variations}
% \end{table}
% %------------------------------------------------------------------------------
% 
% We observe that the introduction of the basic design of the cross attention module outperforms the classification properties of that which uses projection of the input features to compute the subsequent representation. We hypothesize that maintaining the features without any projection allows the [class] token to better collect global information, as patch information is maintained in the same space that the already trained classifier relies on to perform classification. 
% On another hand, we also note that the introduction of an MLP to update said representation is not feasible given instability during training, for this reason we decide not to pursue further experimentation with this idea.

%------------------------------------------------------------------------------
\begin{table}
\centering
\scriptsize
%\resizebox{\columnwidth}{!}
%\vspace{3pt}
\caption{\emph{Different cross attention block design for \Ours.} Classification accuracy and parameters using ResNet-50 on ImageNet. \Th{$\#$Param}: parameters of \Ours only.}
\label{tab:dif_streams}
\end{table}
%------------------------------------------------------------------------------

Results are reported in \autoref{tab:dif_streams}. We observe that the stream made of vanilla CA blocks~\eq{CA} offers slightly better accuracy than projections~\eq{proj_ca}, while having less parameters. We also note that most of the computation takes place in the last residual stages, where the channel dimension is the largest. To keep our design simple, we choose the vanilla solution without projections~\eq{CA} by default.

%------------------------------------------------------------------------------

\paragraph{\Ours placement}
\label{ab:placement}

To validate the design of \Ours, we measure the effect of its depth on its performance \vs the baseline \gap in terms of both classification accuracy / number of parameters and classification metrics for interpretability. In particular, we place the stream in parallel to the network $f$, starting at stage $\ell$ and running through stage $L$, the last stage of $f$, where $0 \le \ell \le L$. Results are reported in \autoref{tab:intrecog-resnet}.

%------------------------------------------------------------------------------
\begin{table}
\centering
\scriptsize
\setlength{\tabcolsep}{4pt}
%\resizebox{\columnwidth}{!}{%}
\begin{tabular}{lcccccc}\toprule
	\mc{7}{\Th{Accuracy and Parameters}}\\\midrule
	&\Th{Placement}&\mc{2}{\Th{CLS dim}}&\mc{2}{\Th{\#Param}}&\Th{Acc$\uparrow$}\\\midrule
	
	&$S_0-S_4$&\mc{2}{$64$}  &\mc{2}{6.96M}&\textbf{74.70}\\  % 6.963.264
	&$S_1-S_4$&\mc{2}{$256$} &\mc{2}{6.95M}&74.67\\           % 6.947.072
	&$S_2-S_4$&\mc{2}{$512$} &\mc{2}{6.82M}&74.67\\           % 6.816.256
	&$S_3-S_4$&\mc{2}{$1024$}&\mc{2}{6.29M}&74.67\\           % 6.292.480
	&$S_4-S_4$&\mc{2}{$2048$}&\mc{2}{4.20M}&74.63\\\midrule   % 4.196.352
	
	\mc{7}{\Th{Interpretability Metrics}}\\\midrule
	\Th{Method}&\Th{Placement}&\Th{AD$\downarrow$}&\Th{AG$\uparrow$}&\Th{AI$\uparrow$}&\Th{I$\uparrow$}&\Th{D$\downarrow$}\\\midrule
	
	\mr{5}{\Th{Grad-CAM}}&$S_0-S_4$&\textbf{12.54}&\textbf{22.67}&48.56&75.53&13.50\\ % Checked
		&$S_1-S_4$&12.69&22.65&48.31&75.53&13.41\\ % Checked
		&$S_2-S_4$&\textbf{12.54}&21.67&\textbf{48.58}&75.54&13.50\\ % Running
		&$S_3-S_4$&12.69&22.28&47.89&\textbf{75.55}&13.40\\ % Running
		&$S_4-S_4$&12.77&20.65&47.14&74.32&\textbf{13.37}\\\midrule % Checked
		
	\mr{5}{\Th{Grad-CAM++}}&$S_0-S_4$&13.99&19.29&44.60&75.21&13.78\\ % Checked
		&$S_1-S_4$&13.99&19.29&44.62&75.21&13.78\\ % Checked
		&$S_2-S_4$&13.71&\textbf{19.90}&\textbf{45.43}&75.34&13.50\\ % Running
		&$S_3-S_4$&13.69&19.61&45.04&\textbf{75.36}&13.50\\ % Running
		&$S_4-S_4$&\textbf{13.67}&18.36&44.40&74.19&\textbf{13.30}\\\midrule % Checked
		
	\mr{5}{\Th{Score-CAM}}&$S_0-S_4$&\textbf{7.09}&23.65&54.20&74.91&14.68\\ % Checked
		&$S_1-S_4$&\textbf{7.09}&23.65&54.20&74.92&14.68\\ % Checked
		&$S_2-S_4$&\textbf{7.09}&\textbf{23.66}&\textbf{54.21}&74.91&14.68\\ % Running
		&$S_3-S_4$&7.74&23.03&52.92&\textbf{74.97}&14.65\\ % Running
		&$S_4-S_4$&7.52&19.45&50.45&74.19&\textbf{14.46}\\\bottomrule % Checked
\end{tabular}
% }
%\vspace{3pt}
\caption{\emph{Effect of stream placement} on accuracy, parameters and interpretability metrics  for ResNet-50 on ImageNet. $S_\ell-S_L$: \Ours runs from stage $\ell$ to $L$ (last); \Th{$\#$Param}: parameters of \Ours only.}
\label{tab:intrecog-resnet}
\end{table}
%------------------------------------------------------------------------------

From the interpretability metrics as well as accuracy, we observe that stream configurations that allow for iterative interaction with the network features obtain the best performance, although the effect of stream placement is small in general. In many cases, the lightest stream of only one cross attention block ($S_4-S_4$) is inferior to options allowing for more interaction. 
Since starting the stream at early stages has little effect on the number of parameters and performance is stable, we choose to start the stream in the first stage ($S_0-S_4$) by default.

%------------------------------------------------------------------------------



%------------------------------------------------------------------------------

%% Top design, bottom interp metrics
%% More parts for experiments

% Ablation of histogram of diferences with extreme cases

\uppercase{\section{conclusion}}

%\green{TO DO}
In this paper, we propose a new training approach to improve the gradient of a CNN in terms of interpretability.
Our methods forces the gradient obtained from back-propagation at the input image level to align with the gradient coming from guided backpropagation. 
The results of our training are presented according to several metrics and interpretability methods. Our method offers consistent improvement on four out of five metrics for two networks.



% \section{Acknowledgements}
% This publication has recieved funding from the \textbf{Thesis funding}.\\
% Part of this work was performed using HPC resources from GENCI-IDRIS (Grant 2022-AD011012724R1).

%------------------------------------------------------------------------------
%%%%%%%% REFERENCES

\clearpage
{\small
\bibliographystyle{ieee_fullname}
\bibliography{egbib}
}

%------------------------------------------------------------------------------
%%%%%%%%% APPENDIX

\clearpage
\title{\Ours: Attention-based pooling for interpretable image recognition \\ \emph{Supplementary material}}

%------------------------------------------------------------------------------
\maketitle
%------------------------------------------------------------------------------

\appendix
\setcounter{page}{1}
\wacvrulercount=1

% NUMBERING
\renewcommand{\thesection}{A\arabic{section}}
\renewcommand{\theequation}{A\arabic{equation}}
\renewcommand{\thetable}{A\arabic{table}}
\renewcommand{\thefigure}{A\arabic{figure}}
\section{More on experimental setup}

\paragraph{Implementation details}

% %------------------------------------------------------------------------------
% \begin{table}[h]
% \centering
% \scriptsize
% \setlength{\tabcolsep}{3.5pt}
% \begin{tabular}{llcccccc}\toprule
% 	\mc{8}{\Th{Accuracy and Parameters}}\\\midrule
% 	\Th{Network}&&\Th{Pool}&\mc{2}{\Th{$\#$Param}}&\mc{2}{\Th{Param$\%$}}&\Th{Acc$\uparrow$}\\\midrule

% 	\mr{2}{\Th{ResNet-50}}&&\gap&\mc{2}{25.56M}&\mc{2}{\mr{2}{27.27}}&74.55\\
% 		&&\ours&\mc{2}{32.53M}&&&74.70\\\midrule

% 	\mc{8}{\Th{Interpretability Metrics}}\\\midrule
% 	\Th{Network}&\Th{Method}&\Th{Pool}&\Th{AD$\downarrow$}&\Th{AG$\uparrow$}&\Th{AI$\uparrow$}&\Th{I$\uparrow$}&\Th{D$\downarrow$}\\\midrule

% 	\mr{10}{\Th{ResNet-50}}&\mr{3}{Grad-CAM}&\gap&12.31&16.51&44.39&73.06&13.27\\ %
%  		& &\ours&12.37&21.26&47.16&75.82&13.27\\
% 		& &\ours-scratch&23.09&15.21&34.56&73.47&11.65\\\cmidrule{2-8} %
% 	    &\mr{3}{Grad-CAM++}&\gap&13.25&14.42&41.32&72.91&13.28\\ %
%  		& &\ours&14.47&17.04&42.11&75.60&13.48\\
% 		& &\ours-scratch&20.80&13.39&34.26&72.98&12.09\\\cmidrule{2-8} %
% 	    &\mr{3}{Score-CAM}&\gap&TBA&TBA&TBA&TBA&TBA\\ %
%  		& &\ours&TBA&TBA&TBA&TBA&TBA\\
% 		& &\ours-scratch&TBA&TBA&TBA&TBA&TBA\\ % \cmidrule{2-8} %
% 	\bottomrule
% \end{tabular}
% \label{tab:scratch}
% \end{table}
% %------------------------------------------------------------------------------

%\section{More results}


%\paragraph{Saliency differences}
%\alert{
%We observe in \autoref{fig:compmethods} that the saliency maps of \Ours are often not very different from those of GAP. As a further quantitative investigation, we compute cosine similarities of saliency maps and obtain 0.998 
% (or 6.49\% euclidean distance) 
%on average over the dataset. 

%However, the similarity of gradients of class scores with respect to feature tensors (based on which GradCAM weights are computed) is 0.612 on average. This indicates that saliency maps become more similar by spatial pooling (of gradients into weights) and then by weighted average over channels (in the definition of all CAM-based methods). In addition, because of the different pooling process, the similarity of the pooled features is 0.752 and the difference of the ground truth class probabilities is 27.8\% on average. These results explain the differences in accuracy and saliency map evaluation metrics brought by \Ours, despite the small changes in saliency maps.
%}

%------------------------------------------------------------------------------

%------------------------------------------------------------------------------

\section{More ablations}

%\paragraph{Class-specific CLS}

%Table \autoref{tab:TokenvMatrix} presents the results obtained for the class specific CLS described in section 4.5. The performance are similar for both \Ours. Thus, the class agnostic stream is favored as it requires less parameters.

%----------------------------------------------------------------


\paragraph{Class-specific CLS}

%\textcolor{green}{
As discussed in section 3.2, the formulation of single-query cross attention as a CAM-based saliency map (6) is class agnostic (single channel weights $\alpha_k$), whereas the original CAM formulation (1) is class specific (channel weights $\alpha_k^c$ for given class of interest $c$). 
Here we consider a class specific extension of \Ours using one query vector per class. 
In particular, the stream is initialized by one learnable parameter $\vq_0^c$ per class $c$, but only one query (\cls token) embedding is forwarded along the stream. At training, $c$ is chosen according to the target class label, while at inference, the class predicted by the baseline classifier is used instead.
%}

%\textcolor{green}{
%Additional results are reported in supplementary. We observe that the class specific representation for \Ours provides no improvement over the class agnostic representation, despite the additional complexity and parameters. We thus choose the class agnostic representation by default.
%}

Results are resported in \autoref{tab:TokenvMatrix}. We observe that the class specific representation for \Ours provides no improvement over the class agnostic representation, despite the additional complexity and parameters. We thus choose the class agnostic representation by default. The class specific approach is similar to [50] in being able to generate class specific attention maps, although no fine-tuning is required in our case. 





%------------------------------------------------------------------------------
\begin{table}[H]
\centering
\scriptsize
\setlength{\tabcolsep}{4pt}
%\resizebox{\columnwidth}{!}{%
\begin{tabular}{llccccc}\toprule                    
	\mc{7}{\Th{Accuracy and Parameters}}\\\midrule
	&\Th{Representation}&\mc{2}{}&\mc{2}{\Th{\#Param}}&\Th{Acc$\uparrow$}\\\midrule
		&Class agnostic&\mc{2}{}&\mc{2}{32.53M}&74.70\\
		&Class specific&\mc{2}{}&\mc{2}{32.59M}&74.68\\\midrule
	
	\mc{7}{\Th{Interpretability Metrics}}\\\midrule
	\Th{Method}&\Th{Representation}&AD$\downarrow$&AG$\uparrow$&AI$\uparrow$&I$\uparrow$&D$\downarrow$\\\midrule
	% \multirow{3}{*}{GAP}&\multirow{3}{*}{74.55}&Grad-CAM&13.04&17.56&44.47&72.57&13.24\\
	%  & &Grad-CAM++&13.79&15.87&42.08&72.32&13.33\\
	%  & &Score-CAM&13.64&12.98&44.53&62.56&11.37\\\hline %
	\mr{2}{Grad-CAM}&Class agnostic&12.54&22.67&48.56&75.53&13.50\\
		&Class specific&12.53&22.66&48.58&75.54&13.50\\\midrule
	\mr{2}{Grad-CAM++}&Class agnostic&13.99&19.29&44.60&75.21&13.78\\
		&Class specific&13.99&19.28&44.62&75.20&13.78\\\midrule
	\mr{2}{Score-CAM}&Class agnostic&7.09&23.65&54.20&74.91&14.68\\
		&Class specific&7.08&23.64&54.15&74.99&14.53\\\bottomrule
\end{tabular}
%}
\vspace{3pt}
\caption{\emph{Effect of class agnostic \vs class specific representation} on accuracy, parameters and interpretability metrics of \Ours for ResNet-50 and different interpretability methods on ImageNet. \Th{$\#$Param}: parameters of \Ours only.}
\label{tab:TokenvMatrix}
\end{table}
%------------------------------------------------------------------------------




\end{document}
