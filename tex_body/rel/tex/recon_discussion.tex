%--------------------------------------------------------------------------------------------------
\subsection{Discussion}
\label{subsec:rel_recon_discussion}
In computer vision, the developement of image recognition models has been crucial 
towards the advancement of different tasks such as image segmentation and object localization. 
Supported by the theory of the interaction of Malik's three Rs of computer vision 
\autocite{malik2016three}, we observe that progress of one of these three fields, leads to strides 
in complementary areas. As described in the previous section, many of the proposals demonstrated 
therein act as excellent feature extractors. This capacity in turn, facilitates adjacent tasks 
such as segmentation (regrouping) and reconstruction. With this in mind, we suggest that model 
developement and design account for interactions between the Rs (better seen in  
\autoref{fig:malik_rs}); and consequently, model developement being rooted mostly in image 
recognition. We aknowledge image recognition as one foundational task on computer vision.

\begin{figure}[H]
    \centering
    \begin{tikzpicture}[]
        %% CNN branch
        \node(recon) at (3, 3.3541){Recognition};
        \node(reconst) at (0, 0) {Reconstruction};
        \node(reorg) at (6, 0) {Reorganization};

        \draw[-Stealth] ([xshift=-6pt] reconst.north) -- node {} ([xshift=-6pt] recon.south);
        \draw[-Stealth] ([xshift=-2pt] recon.south) -- node {} ([xshift=-2pt] reconst.north);

        \draw[-Stealth] ([xshift=2pt] recon.south) -- node {} ([xshift=2pt] reorg.north);
        \draw[-Stealth] ([xshift=6pt] reorg.north) -- node{} ([xshift=6pt] recon.south);

        \draw[-Stealth] ([yshift=2pt] reconst.east) -- node {} ([yshift=2pt] reorg.west);
        \draw[-Stealth] ([yshift=-2pt] reorg.west) -- node {} ([yshift=-2pt] reconst.east);        
    
    \end{tikzpicture}
    \caption{Malik's three \emph{R} of computer vision}
    \label{fig:malik_rs}
\end{figure}

\noindent Furthermore, this domain has seen constant evolution in recent years. Following the 
resurgence of \glspl{cnn} after the introduction of AlexNet, a plethora of image recognition models 
were proposed. Still, we note that while these models are variate in structural units, complexity, 
and depth; the model formulation  itself is not the solely determining factor of performance.  
In a similar fashion to the points described by \emph{A Metric Learning Reality 
Check} \autocite{musgrave2020metric}, where a revision of metric learning methodologies revealed 
biases in the evaluation of novel methodologies and the enhanced power of 
previous methods when optimized under better considitions; overall performance evaluation of 
architectures and methodologies often lacks fair comparison due to advancement in optimization 
techniques. This is clearly demonstrated in \emph{ResNet Strikes Back} \autocite{wightman2021resnet}. 
Nevertheless, it is also possible to consider that \glspl{cnn} may be approaching a plateau in 
their capabilities, similar to traditional computer vision methods  when applied to ImageNet. As 
a response of this, we take special interest on transfomers, given their recent adoption and 
overall their promising capabilities. \\

\noindent Delvig into transformers, we remark the promise that they display 
given recent advances in tasks such as text recognition, text generation and notably in vision, 
on image generation, captioning and recognition. Comparable to the surge of convolution 
based methods in the early 2010s, overtaking traditional machine learning methods in computer 
vision applications, the paradigm shift previously mentioned is already taking place as 
we can observe on figure \autoref{fig:paradigmn_shift}. However, the enhancement of recognition 
capabilities is highly dependent on the amount and quality of data that is used in their 
design and optimization. At this point, it is important to highlight that recognition capabilities 
on ImageNet are nearing saturation point for the dataset, leading us to question whether it is 
truly a feat of model generalization or a severe case of overfitting.

%--------------------------------------------------------------------------------------------------
\begin{figure}[H]
    \centering
    \pgfplotstableread[col sep=comma,]{fig/rel/data/models_year.csv}\datatable
    \begin{tikzpicture}
        \begin{axis}[
            date coordinates in=x,
            date ZERO=2017-03-31,
            xtick distance=0.25,
            %xtick={\datatable}{Year},
            %xticklabel={\year-\month-\day},
            xtick=data,
            width=\textwidth,
            height=6.85cm,
            %xticklabels from table={\datatable}{Year},
            xticklabel style={rotate=90, anchor=near xticklabel},
            ylabel={Proportion of Papers (Quarterly)},
            legend style={at={(0, 1)},anchor=north west},
            yticklabel style={/pgf/number format/fixed}
            ]
            \addplot [mark=*, red!80] table [x={Year}, y={ResNet}]{\datatable};
            \addlegendentry{{\scriptsize \Th{ResNet}}}

            \addplot [mark=*, black!80] table [x={Year}, y={VGG}]{\datatable};
            \addlegendentry{{\scriptsize \Th{VGG}}}

            \addplot [mark=*, green!80] table [x={Year}, y={DenseNet}]{\datatable};
            \addlegendentry{{\scriptsize\Th{DenseNet}}}

            \addplot [mark=*, cyan!80] table [x={Year}, y={EfficientNet}]{\datatable};
            \addlegendentry{{\scriptsize \Th{EfficientNet}}}

            \addplot [mark=*, blue!80] table [x={Year}, y={ViT}]{\datatable};
            \addlegendentry{{\scriptsize \Th{ViT}}}
        \end{axis}
    \end{tikzpicture}
    \caption{\textbf{Proportion of articles published on ImageNet,} using image recognition models as backbone across the years. 
             Original from \url{https://paperswithcode.com/method/resnet}}
    \label{fig:paradigmn_shift}
\end{figure}