%--------------------------------------------------------------------------------------------------
\section{Evaluating Interpretability}
\subsection{Classification Metrics}
\gls{ad} and \gls{ai} \autocite{chattopadhay2018grad} are 
well-established classification metrics. 
They measure the effect on the predicted class probabilities by masking the input image with the
 saliency map. Let $p^c_i$ and $o^c_i$ be the predicted probability for class $c$ given as input 
 the $i$-th test image $\vx_i$ and its masked version respectively. 
Masking refers to element-wise multiplication with the saliency map, which is at the same 
resolution as the original image with values in $[0,1]$. 
Let $N$ be the number of test images. Class $c$ is taken as the ground truth.

\emph{Average drop} (\gls{ad}) quantifies how much predictive power, measured as class probability, 
is lost when we only mask the image; lower is better:
\begin{equation}
	\AD(\%) \defn \frac{1}{N} \sum_{i=1}^N \frac{[p^c_i - o^c_i]_+}{p^c_i} \cdot 100.
\label{eq:ad}
\end{equation}

\emph{Average increase} (\gls{ai}), also known as \emph{increase in confidence}, measures the 
percentage of images where the masked image yields a higher class probability than the original; 
higher is better:
\begin{equation}
	\AI(\%) \defn \frac{1}{N} \sum_i^N \mathbb{1}_{p^c_i < o^c_i} \cdot 100.
\label{eq:ai}
\end{equation}

$\AD$ and $\AI$ are not defined in a symmetric way. $\AD$ measures changes in class probability 
whereas $\AI$ measures a percentage of images. It is possible that the percentage is high while 
the actual increase is small. Hence, it is possible that an attribution method improves both. 
Indeed, \autocite{poppi2021revisiting} observes that a trivial method called Fake-CAM outperforms 
state-of-the-art methods, including Score-CAM, by a large margin. Fake-CAM simply defines a 
saliency map where the top-left pixel is set to zero and is uniform elsewhere. 
This questions the purpose of $\AD$ and $\AI$.

%--------------------------------------------------------------------------------------------------
\subsection{Localization metrics}
\label{sec:loc-metrics}
Several works measure the localization ability of saliency maps, using metrics from the 
\emph{weakly-supervised object localization} (WSOL) task.

We are given the saliency map $S^c$ obtained from test image $\vx$ for ground truth class $c$. 
We denote by $S^c_{\vp}$ its value at pixel $\vp$. We binarize the saliency map by thresholding at 
its average value and we take the bounding box of the largest connected component of the resulting 
mask as the predicted bounding box $B_p$, represented as a set of pixels. We compare this box 
against the set of ground truth bounding boxes $\cB$, which typically contains 1 or 2 boxes of the 
same class $c$, or with their union $U = \cup \cB$, again represented as a set of pixels. We also 
compare the predicted class label $c_p$ with the ground truth label $c$. All metrics take values in 
$[0,1]$ and are expressed as percentages, except SM~\eq{sm}, which is unbounded.

\emph{Official Metric (OM)}
measures the maximum overlap of the predicted bounding box with any ground truth bounding box, 
requiring that the predicted class label is correct:
\begin{equation}
	\OM \defn 1 - \paren{\max_{B \in \cB} \iou(B, B_p)} \mathbbm{1}_{c_p = c},
\label{eq:om}
\end{equation}
where $\iou$ is intersection over union.
% is defined as $\iou(B, B_p) \defn \frac{B \cap B_p}{B \cup B_p}$.

\emph{Localization Error (LE)} is similar but ignores the predicted class label:
\begin{equation}
	\LE \defn 1 - \max_{B \in \cB} \iou(B, B_p).
\label{eq:le}
\end{equation}

\emph{Pixel-wise $F_1$ score (F1)} is defined as $F_1 = 2 \frac{P R}{P + R}$, where 
\emph{precision} $P$ is the fraction of mass of the saliency map that is within the ground truth 
union
\begin{equation}
	P \defn \frac{\sum_{\vp \in U} S^c_{\vp}}{\sum_{\vp} S^c_{\vp}}
\label{eq:prec}
\end{equation}

and \emph{recall} $R$ is the fraction of the ground truth union that is covered by the saliency map
\begin{equation}
	R \defn \frac{\sum_{\vp \in U} S^c_{\vp}}{\card{U}}.
	\label{eq:rec}
\end{equation}

\emph{Box Accuracy (BA)\autocite{choe2020evaluating}} Given threshold values $\eta$ and $\delta$, 
we find the bounding box $B^\eta_p$ of the largest connected component of the binary mask 
$\set{\vp: S_{\vp} > \eta}$ and require that it overlaps by 
$\delta$ with at least one ground truth box:
\begin{equation}
	\BA(\eta, \delta) \defn \max_{B \in \cB} \mathbbm{1}_{\iou(B^\eta_p, B) \ge \delta}.
\label{eq:ba}
\end{equation}
After averaging over the test images, we take the maximum of this measure over a set of values 
$\eta$ and then the average over a set of values $\delta$.\\
%--------------------------------------------------------------------------------------------------
\emph{Standard Pointing game (SP)\autocite{zhang2018top}} We find the pixel 
$\vp^* \defn \arg\max_{\vp} S^c_{\vp}$ having the maximum saliency value and 
require that it lands in any of the ground truth bounding boxes:
\begin{equation}
	\spg \defn \mathbbm{1}_{\vp^* \in U}.
\label{eq:spg}
\end{equation}\\

\emph{Energy Pointing game (EP)\autocite{wang2020score}} is equivalent to precision~\eq{prec}.\\

\emph{Saliency Metric (SM)\autocite{dabkowski2017real}} penalizes the size of the predicted bounding
 box $B_p$ relative to the image and the cross-entropy
 loss:
\begin{equation}
	\SM \defn \log \max\paren{ 0.05, \frac{\card{B_p}}{hw} } - \log p^c,
\label{eq:sm}
\end{equation}
where $h \times w$ is the input image resolution and $p^c$ is the precicted probability for ground 
truth class label $c$.

\subsection{Discussion}