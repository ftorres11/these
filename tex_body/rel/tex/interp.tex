%--------------------------------------------------------------------------------------------------
\section{Interpretability}
\label{rel:sec_int}


\subsection{Transparency}
\label{rel:sub_transp}

\subsection{Post-Hoc Interpretability}
\label{rel:sub_post}

\paragraph{Notation}
\label{sec:oc_notation}

Consider a classifier network \gls{fx}$: \cX \to \real^C$ that maps an input image $\mathbf{u} \in \cX$ to a 
logit vector $\vy = f(\mathbf{u}) \in \real^C$, where $\cX$ is the image space and $C$ is the number 
of classes. We denote by $y_c = f(\mathbf{u})_c$ the predicted logit and by $p_c = \softmax(\vy)_c 
\defn e^{y_c} / \sum_j e^{y_j}$ the predicted probability for class $c$. For layer $\ell$ 
with $K_\ell$ channels, we denote by $A^k_\ell = f^k_\ell(\mathbf{u}) \in \real^{h_\ell \times w_\ell}$ 
the feature map for channel $k \in \{1,\dots,K_\ell\}$, with spatial resolution $h_\ell \times 
w_\ell$. Because of $\relu$ non-linearities, we assume that feature maps are non-negative. 
Similarly, we denote by $S_\ell \in \real^{h_\ell \times w_\ell}$ a 2D saliency map.

%--------------------------------------------------------------------------------------------------

\paragraph{Background: CAM-based saliency maps}
\label{sec:oc_back}

Given a layer $\ell$ and a class of interest $c$, we consider saliency maps given by the general 
formula
\begin{equation}
	S^c_\ell(\mathbf{u}) \defn h \left( \sum_k w^c_k A^k_\ell \right),
\label{eq:sal}
\end{equation}
where $w^c_k$ are weights defining a linear combination over channels and $h$ is an activation 
function. CAM \parencite{zhou2016learning} is defined for the last layer $L$ only with $h$ being the 
identity mapping and $w^c_k$ being the classifier weight connecting the $k$-th channel with 
class $c$. Grad-CAM \parencite{selvaraju2017grad} is defined for any layer $\ell$ with $h = \relu$ and 
weights
\begin{equation}
	w^c_k \defn \gap \left( \pder{y_c}{A^k_\ell} \right),
\label{eq:gcam}
\end{equation}
where $\gap$ is global average pooling.
% and $\softmax(\vy)_c = e^{y_c} / \sum_j e^{y_j}$ is the predicted probability of class $c$.
The motivation for $\relu$ is that we are only interested in features that have a positive effect 
on the class of interest, \ie pixels whose intensity should be increased in order to increase $y_c$.

Score-CAM \parencite{wang2020score} is also defined for any layer $\ell$ with $h = \relu$ and weights 
$w^c_k \defn \softmax(\mathbf{u}^c)_k$.  Softmax normalization considers positive channel contributions 
only and attends to few feature maps.
%that \alert{produce less highlighted areas in the saliency map}. \iavr{Last part unclear.}
Here, vector $\mathbf{u}^c \in \real^{K_\ell}$ measures the increase in confidence for class $c$ that 
compares a known baseline image $\mathbf{u}_b$ with the input image $\mathbf{u}$ masked according to feature 
map $A^k_\ell$, for all channels $k$:

\begin{equation}
	u^c_k \defn f(\mathbf{u} \odot n(\operatorname{up}( A^k_\ell )))_c - f(\mathbf{u}_b)_c,
\label{eq:s-cam}
\end{equation}

where $\odot$ is the Hadamard product. For this to work, the feature map $A^k_\ell$ is adapted
 to $\mathbf{u}$ first$:\operatorname{up}$ denotes upsampling to the spatial resolution of $\mathbf{u}$ and

\begin{equation}
	n(A) \defn \frac{A - \min A}{\max A - \min A}
\label{eq:norm}
\end{equation}

is a normalization of matrix $A$ into $[0,1]$. While Score-CAM does not need gradients, 
it requires as many forward passes through the network as the number of channels in the chosen layer,
 which is computationally expensive.

%--------------------------------------------------------------------------------------------------

\paragraph{Motivation}
\label{sec:motiv}

\iavr{Score-CAM considers each feature map as a mask in isolation. How about linear combinations?} 
Given a vector $\vw \in \real^{K_\ell}$ with $w_k$ its $k$-th element, let
\begin{equation}
	F(\vw) \defn f \left( \mathbf{u} \odot n \left( \operatorname{up} \left(
		\displaystyle\sum_k w_k A^k_\ell
	\right) \right) \right)_c.
\label{eq:s-obj}
\end{equation}
\ronan{If we assume that $\mathbf{u}_b = \vzero$ in~\eq{s-cam} and define $n(\vzero) \defn \vzero$ 
in~\eq{norm}, then we can rewrite the right-hand side of~\eq{s-cam} as
\begin{equation}
	\frac{F(\vw_0 + \delta \ve_k) - F(\vw_0)}{\delta},
\label{eq:s-cam2}
\end{equation}
where $\vw_0 = \vzero$, $\delta = 1$ and $\ve_k$ is the $k$-th standard basis vector of 
$\real^{K_\ell}$. This resembles the numerical approximation of the derivative $\pder{F}{w_k}(\vw_0)$,
 except that $\delta$ is not small as usual. One could compute derivatives efficiently by 
 standard backpropagation instead. It is then possible to iteratively optimize $F$ with respect
  to $\vw$, starting at any $\vw_0$.}

\iavr{As an alternative, consider masking-based methods relying on optimization in the input space, 
like \emph{meaningful perturbations} (MP) \parencite{fong2017interpretable} or 
\emph{extremal perturbations} \parencite{fong2019understanding}. In general, optimization takes the form
\begin{equation}
	S^c(\mathbf{u}) \defn \arg\max_{\vm \in \cM} f(\mathbf{u} \odot n(\operatorname{up}(\vm)))_c + \lambda R(\vm).
\label{eq:mask}
\end{equation}
Here, a mask $\vm$ is directly optimized and does not rely on feature maps, hence the saliency 
map $S^x(\mathbf{u})$ is not connected to any layer $\ell$. The mask is at the same or lower resolution 
than the input image. In the latter case, upsampling is still necessary.

In this approach, one indeed computes derivatives by backpropagation and indeed iteratively 
optimizes $\vm$. However, because $\vm$ is high-dimensional, there are constraints expressed by 
$\vm \in \cM$, \eg $\vm$ has a certain norm, and regularizers like $R(\vm)$, \eg $\vm$ is smooth in a 
certain way. This makes optimization harder or more expensive and introduces more hyperparameters 
like $\lambda$. One could simply constrain $\vm$ to lie in the linear span of $\{A_\ell^k\}_{k=1}
^{K_\ell}$ instead, like all CAM-based methods.}

\subsection{Discussion}