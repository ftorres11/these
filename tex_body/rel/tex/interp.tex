%--------------------------------------------------------------------------------------------------
\section{Interpretability}
\label{rel:sec_int}
As deep learning based models for Computer Vision have continued to improve in their recognition 
properties, their structure and functioning have become more opaque; in turn making these 
technologies seen as black boxes. A black-box model is defined as a model for which its 
interpretation is not straightforward for humans \autocite{petch2022opening}. In recent years 
with the assimilation of deep learning into everyday tasks, and the implicit effect these models 
are having on human lives; the novel research field of \emph{Interpretability} has been brought 
forth to open up this black-box behavior. Researchers have approached interpretability alongside 
different directions. Starting with the work of \cite{li2018deep}, interpretability is suggested 
to present two categories, \emph{Transparency} and \emph{Post-hoc Interpretability}.
For the former, Lipton argues that it follows modifications of the model or the training process, 
in order to explain the inner-workings of the model. Conversely, for the latter Lipton builds upon 
the black-box behavior of the model, providing explanations instead based on inputs and outputs; 
without adding any further modifications for the model or altering its training process.\\

\noindent Considering these properties, we observe that as machine learning models grew in 
complexity; their transparent properties vanished proportionally to their size. It can be argued 
that traditional models offer themselves to  transparency due to their straightforward formulation 
and inherent properties. Conversely, regarding deep models, we find that it is after their size and 
complexity that their interpretable properties get hindered. Common  \glspl{cnn} 
rely on convolution as their corner stone, coupled with non-linear operations such as 
ReLU (\cite{fukushima1975cognitron}), Sigmoid, and Softmax (\cite{hopfield1985neural}) among others.
This aggregation of convolutions on one hand enables these models to process large quantities of 
data, and to a certain extent generalize; however in the other hand, it also results in an extensive 
parameter count, often reaching of millions, and most recently, even billions (\cite{openai_compute}). 
The computational load required for inference, typically measured in \gls{gflops} further compounds 
complexity. We expand on this in following sections.\\

\noindent Complementary to Lipton's proposal, \cite{guidotti2018survey} points that interpretability is 
contained along different dimensions. For instance, a model can be understood on its entirety 
following a \emph{Global} interpretation; whereas in situations where only the reasons leading to a 
specific prediction, a \emph{Local} interpretation is found. Guidotti also considers time,
more specifically \emph{Time Limitation} as its availability is strictly correlated to the scenario 
where the model is used. Finally, the \emph{Nature of User Expertise} covers the last dimension of 
interpretability; where knowing the experience of a user in a given task is considered to be a 
key aspect describing the interpretability of a model.\\

\noindent In recent years, \cite{zhang2021survey} suggested that three different dimensions encompass this 
study. The first dimension answers to the nature of an approach; where it can be either 
\emph{Passive} or \emph{Active}. The first direction in this dimension correlates to Lipton's 
\emph{Post-hoc interpretability}, the latter follows the aforementioned \emph{Transparency} property.
Zhang's second dimension is addressed towards the type of explanations, where the order of 
explanatory power follows a hierarchy starting on its base level with examples, into attributions, 
leading into hidden semantics and finishing in rules. On one hand, the last dimension that Zhang 
covers explanations in the input space, where a \emph{Local} explanation describes the network  
prediction following individual samples; while in the other hand, a \emph{Global} explanation 
describes the network as a whole. This last dimension is similar to the first point described by 
\cite{guidotti2018survey}. Most importantly, Zhang's dimensions are not used separately to 
categorize an approach, instead according to the methodology properties and the nature of the 
explanation provided. This is further seen in the 3D space found in \autoref{fig:rel_zhangs}

\begin{figure}[H]
    \centering
    \includegraphics[width=0.8\textwidth]{fig/rel/images/ZhangsInterpspace.png}
    \caption{\textbf{Interpretabilty studies} grouped by nature. Source from \url{https://yzhang-gh.github.io/tmp-data/index.html}}
    \label{fig:rel_zhangs}
\end{figure}
 
\noindent In this thesis we study interpretability according to Lipton's properties; nevertheless, as shown 
in the following subsections, we acknowledge their strengths and weaknesses, while denoting some of 
the most prominent works in each dimension. In \autoref{rel:sub_transp} we discuss 
\emph{transparency}, its properties according to Lipton and its difficulties describing deep models, 
alongside a survey following Zhang's active dimension. Conversely, in \autoref{rel:sub_post}, we 
study \emph{post-hoc interpretability} similarly to transparency.

\subsection{Transparency}
\label{rel:sub_transp}
Following Lipton's work, we further define the transparency property of models as the opposite of 
opacity or black-box behavior. Still, similar to the aforementioned works, transparency is 
considered at the model level (\emph{simulatability}), individual components 
(\emph{decomposability}) and the training algorithm (\emph{algorithmic transparency}). Delving into 
detail, \emph{Simulatability} on a strict definition answers to the capability of a model to be 
fully contemplated by an individual on its entirety at once. Currently, with the increase of model 
complexity this understanding level is then redefined as the capacity of a model to be readily 
presented to a user with visual or textual artifacts. This in turn can be demonstrated by 
\emph{post-hoc interpretations}.\\

\noindent Regarding \emph{decomposability}, Lipton suggests that an interpretable model has to present 
an intuitive explanation, regardless of it being an input, parameter or calculation. In this 
dimension it is then stated that a computation follows an association between features, in 
accordance to \emph{intelligibility} as proposed in \autocite{lou2012intelligible}. Still, Lipton 
also points that this notion should not be accepted blindly as some parts of the computation steps 
can be fragile processing and calculation; for instance, image normalization can lead to prediction 
mismatch during inference.\\

\noindent Finally, Lipton argues that \emph{algorithmic transparency} applies on the level of the 
learning algorithm itself. In short, this applies to the error surface and assumption of the 
existence of an optimal and/or unique solution to the problem. Nevertheless this dimension is not 
met for deep learning models as the training process is not completely understood.\\

\noindent While Lipton's transparency properties appear to be applicable and theoretically sound, 
transparency is not fully studied alongside them. As previously observed, designing an approach or 
model covering these properties is a complicated task for deep models. However, with a brief 
modification of their definition, transparent properties can be further applied to deep image 
recognition. \emph{Simulatability}, as mentioned earlier,  can be supported with \emph{post-hoc 
interpretability} explanations. Conversely, deep learning models can be further 
\emph{decomposed} into functional units; for instance, the ResNet architecture consists of
residual blocks, where the structure of each block and its design has been studied thoroughfuly. 
Finally, concerning \emph{algorithmic transparency}, although deep learning models present high 
dimension error surfaces and a high amount of local minima; still, ensuring a deterministic 
behavior leads to replicability, in turn demonstrating part of the training process.\\

\noindent Following the interpretable dimensions described by \autocite{zhang2021survey}, 
transparency is suggested as an active change of the architecture of the network or the training 
process in order to provide explanations. Furthermore, according to the type of explanations 
provided by an interpretable approach, transparency methods can be explained according to logic 
rules, hidden semantics, attributions and explanations by example.\\

\noindent \emph{Rule-based methods.} In this group, rules are used to provide an explanation. 
To do this, Wu et al. proposed training a decision tree on top of a deep learning model, acting as 
a regularization term and an approximation of said deep model (\cite{wu2018beyond}, 
\cite{wu2020regional}). This process involves two steps. First a binary tree is trained, mapping 
inputs to the model prediction. The second part of this approach calculates the average path 
length between root to leaf nodes on the decision tree, covering all data points. Ultimately, Zhang 
argues that this regularization improves interpretability by forcing a model to be described by a 
decision tree.\\

\noindent \emph{Hidden semantics-based methods.} This family of methods focuses on the filters 
and properties found alongside in different depths of the model. These approaches are based in 
observations of filter response to inputs at different depths, and the semantic information 
contained by them(\cite{bau2017network}, \cite{zhou2018interpreting}). 
One particular approach studying this is that of \autocite{zhang2018interpretable}, introducing a 
loss term that encourages deep filters to represent a single concept. More recently, as an answer to 
semantics only being extracted from deep layers, and the information flow not being taken into 
consideration to provide an explanation; the work of B\"ohle et al. (\cite{bohle2022b}, 
\cite{bohle2024b}) introduced the idea of \emph{B-Cos networks} and alignment overall with the 
\emph{Dynamic Alignment Unit}. In these works, alignment between weight-input alignment is 
emphasized by the inclusion of the B-cos transform and dynamic alignment unit. Moreover, the 
interpretations these methods provide insight over the entirety of the network, and they are 
possibly integrated into existing architectures.\\

\noindent \emph{Prototype-based methods.} On this topic, Li et al. designs an architecture that 
incorporates an autoencoder and a standard classification network \autocite{li2018deep}. 
In detail, this model reduces dimensionality of inputs, producing high quality features for 
classification. The encoded features are then used to produce a probability distribution over the 
dataset classes. Interestingly, during the forward pass on the prototype network, these encoded 
features are transformed into prototype vectors; ultimately used to train the classifier. Therefore, 
the classification algorithm is distance-based on the low dimensional learned feature space. 
Another approach showcasing prototype learning is that contained within \emph{This Looks Like That} 
\autocite{chen2019looks}. In this approach, Chen proposes a prototype layer that learns prototypes 
on top of existing architectures, aligning learn a set of prototypes to specific categories.
One difficulty these prototype parts face, is their heavy computation and difficulty to 
train. As a response of this, ProtoPool \autocite{rymarczyk2022interpretable} is recently proposed, 
allowing for prototype reutillization and differentiable assignment of prototypes to classes; 
speeding up the training process\\

\noindent \emph{Attribution-based methods.} Similar to the regularization induced by the inclusion 
of decision trees, local attributes are improved during training through regularization 
(\cite{ismail2021improving}, \cite{Zhou_2022_BMVC}, \cite{ross2017right}, 
\cite{ghaeini2019saliency}, \cite{lee2021lfi}). These methods are often used in conjunction with 
\emph{post-hoc interpretability} approaches, where an observation made on the explanation proposal 
is further taken into consideration to be modified during training. For instance, on LFI-CAM 
\autocite{lee2021lfi},  an attention branch is included to learn attention feature importance 
during training; allowing for a straightforward saliency map computation while improving 
recognition properties. Similarly, saliency-guided training (\cite{ismail2021improving}) 
minimizes the Kullback Leibler divergence between the output of original and masked images. \\


\subsection{Post-Hoc Interpretability}
\label{rel:sub_post}
According to Lipton, post-hoc interpretability provides information directly from learned models. 
Complementary to this, Zhang points out that \emph{passive} interpretability methods extract 
logical rules or understandable patterns from models. As noted in \autoref{fig:rel_zhangs}, a 
discrepancy in the ratio of \emph{active-passive} methods exist, where the \emph{passive} 
dimension presents more studies. Furthermore, according to the nature of 
explanation Lipton groups methods into \emph{textual explanations}, \emph{visualizations}, 
\emph{explanations by example} and \emph{local explanations}. In particular, Lipton argues that 
humans justify decisions verbally, suggesting it is possible to train a model to generate 
explanations describing the predictions of another model. For instance, \gls{nlp}-based approaches 
for image captioning can be utilized to provide textual explanations \autocite{mcauley2013hidden}. 
Nevertheless, in recent years image-captioning approaches have become increasingly complex,
introducing a degree of uncertainty into the explanations provided, as these recent systems 
are now transformer-based and their interpretability is not comprehensively studied yet. Lipton 
closes this discussion addressing that this kind of interpretations are open to scrutiny 
\autocite{chang2009reading}.\\

\noindent To provide interpretations on a visual manner according to Lipton, we observe 
\emph{explanations from visualization}, \emph{explanations by example} and \emph{local explanations}. 
The goal of explanations from visualization is to determine qualitatively what the model has 
learned. One instance of this is found on the work by \cite{mahendran2015understanding}, where an 
image is forwarded through a discriminative \gls{cnn} generating a representation. The authors then 
demonstrate that the original image can be recovered from deep-level representations by performing 
gradient descent on randomly initialized pixels.\\

\noindent Complementary to \emph{explanations by visualization}, Lipton proposes \emph{explanations 
by example}. These explanations are derived from explaining decisions for a model using examples 
deemed the most similar. In detail, once a model is trained it is possible to extract information 
from learned representations; these in turn can be used to explain a new sample by its proximity 
in this representation space to similar observations. On deep image recognition models, we observe 
the works of (\cite{kim2014bayesian}, \cite{doshi2015graph}), where case-based approaches are 
proposed for interpreting generative models. More recently, the work proposed by 
\cite{rombach2020making}, where an Invertible Neural Network is connected at different stages of a 
model, disentangling deep representations and providing an inverse transformation into accessible 
semantic concepts which can be used to point at similar examples of a given input.\\

\noindent Regarding \emph{local explanations}, Lipton acknowledges that describing the entirety of 
the mapping by a \gls{cnn} is a difficult endeavor; and as such, it can be explained on a 
\emph{local} manner. This local behavior is widely considered across interpretability studies, as 
previously mentioned in \autoref{rel:sec_int} (\cite{guidotti2018survey}, \cite{zhang2021survey}). 
Generally speaking, local explanations are often extracted via the computation of a \emph{saliency 
map} highlighting the most important regions on an image describing a prediction. Studies describing 
local explanations are varied and many of them are build on top of transparency approaches; where 
salient properties can be enhanced via training or optimization. Conversely, local explanations 
are grouped into: gradient-based methods, learning-based methods, masked-based methods and 
class activation based-methods.\\

\noindent \emph{Gradient-based methods.} This family of approaches leverages the property of 
image recognition models, wherein forwarding an example through a model allows for the 
visualization of the response of its weights during backpropagation in correlation to a specific 
prediction. This response is visualized in input space, where strong gradient information is 
expected in regions of the image containing information relevant to the corresponding prediction 
(\cite{baehrens2010explain}, \cite{simonyan2013deep}). However, gradient information often contains 
noise, making it challenging to interpret these modifications in a straightforward fashion 
\cite{adebayo2018local}. Consequently, modifications to backpropagation calculation have 
been proposed, including considering only positive values \autocite{guidedbackprop}, adding noise 
on the input space for denoising \autocite{smilkov2017smoothgrad}, introducing rules and axioms to 
constrain the computation \autocite{sundararajan2017axiomatic} and proposal of concepts like 
relevance \autocite{bach2015pixel} to compute the importance pixel interaction leading to a 
prediction \emph{(LRP)}.\\

\noindent \emph{Learning-based methods.} Approaches in this family of methods exhibit a behavior 
intersecting  \emph{transparency} based approaches, where an additional network or branch is 
learned atop an existing architecture to produce an explanation map for a given input. However, 
the lack of modifications on the model for interpretation is the key factor differenting these 
approaches from \emph{active} interpretability methods. Specific examples of learning methods 
comprise modifications inclusion of generative models to fill in information occluded from  the 
classifier \autocite{chang2018explaining}, masking salient points on the map to manipulate scores 
on the classifier \autocite{dabkowski2017real}, inclusion of a \emph{masker} side network to 
collect information on different levels of the network and produce a high resolution explanation 
\autocite{phang2020investigating}, addition of a decoder to generate masks and a different 
classifier predicting the masked inputs \autocite{zolna2020classifier}, and finally, a bottleneck 
on top of intermediary layers of the network to learn a saliency map per sample at the specific 
depth of the network \autocite{schulz2020restricting}.\\

\noindent \emph{Occlusion or masking-based methods.} Continuing on with modifications on the input 
space, by purposefully masking certain regions on the image and measuring the subsequent loss in 
predictive power, a saliency map can be produced. In particular, the masking process can provide 
extremal perturbations by measuring the maximal effect upon the activation of neurons leading to 
predictions (\cite{fong2017interpretable}, \cite{fong2019understanding}), masking in feature space 
to provide saliency maps from intermediate stages \autocite{schulz2020restricting}, learning a 
simple masking model around the prediction to provide saliency maps that are locally faithful to the 
classifier \autocite{ribeiro2016should} and iteratively by random masking the input image, probing 
the model and obtaining a saliency map from the linear combination between prediction weights and 
the random mask that was used to generate them. \\

\noindent \emph{Attention-based techniques.} Given the innate properties of attention based 
architectures, most interpretability methods are ill fitted towards explaining the inference 
process of these models. Nevertheless, with the current design of these models it is possible to 
address salient information, although in a class-agnostic manner. In particular, in the computation 
of self attention, this information is highlighted by the product of \emph{query} and \emph{key} 
vectors by focusing on the \Th{[CLS]} token; this in turn is called \emph{Raw Attention}. 
Conversely, since attention is computed at every layer of a transformer, the flow of information 
can be traced across different layers across the network in an approach named \emph{Rollout 
Attention} \autocite{abnar2020quantifying}. Inspired by \emph{rollout attention} and gradient 
methods such as \emph{LRP} \autocite{bach2015pixel} and the inadequacy property of CAM methods on 
transformers, a novel computation of relevance across layers providing high quality attribution 
maps is introduced, allowing at the same time for a quantitative study of the interpretable 
properties of these methodologies \autocite{chefer2021transformer}.\\

\noindent \emph{\gls{cam}-based methods.} By taking into consideration feature map information 
contained across different layers of a network, it is possible to gain insight into which image 
regions are highlighted at different depths within the model. However, understanding feature maps, 
which are often high-dimensional, present a considerable challenge. Conversely, information 
contained from the classifier layers of the model can be directly associated with classes. For 
example, the product of the weights of the classifier and the last feature map before \gls{gap} 
can generate a saliency map, known as \glsfirst{cam} \autocite{zhou2016learning}. These approaches 
typically utilize feature information from deep layers in conjunction with a set of weights 
correlated with class information to compute explanations (\cite{selvaraju2017grad}, 
\cite{chattopadhay2018grad}, \cite{wang2020score}, \cite{axiombased}, \cite{ablationcam}
\cite{jiang2021layercam}).\\

In this thesis we take particular interest in proposing and evaluating our interpretability 
methodologies based on \gls{cam}. In particular, we find CAM to be an appealing approach because of 
its ease of use and adaptation to interpret existing models. Moreover, it can be argued that while 
it is possible find and use a plethora of the most popular models towards recognition; providing 
explanations ought to be similarly easy to use, a property that CAM-based models share.\\

%--------------------------------------------------------------------------------------------------
\paragraph{CAM-based saliency maps}
One property that \glspl{cnn} display is the presence of semantic information found within the 
deepest layers prior to the classifier. In detail, it has been observed that these models 
operate similarly manner to the visual cortex in the brain; basic textures and their orientation 
is processed in shallow layers, whereas deep layers associate this information into concepts 
\autocite{hubel1959receptive}. This characteristic in turn is the main motivation behind making 
use of CAM methods: by addressing interpretations on the layers prior to the classifier, we 
reconstruct the features deemed salient according to the flow of information within a model. We 
present a representation of filter responses from texture to semantics in \autoref{fig:cnn_depth}.

\paragraph{Notation}
\label{sec:oc_notation}
Consider a classifier network $:f \cX \to \real^C$ that maps an input image $\mathbf{u} \in \cX$ to a 
logit vector $\vy = f(\mathbf{u}) \in \real^C$, where $\cX$ is the image space and $C$ is the number 
of classes. We denote by $y_c = f(\mathbf{u})_c$ the predicted logit and by $p_c = \softmax(\vy)_c 
\defn e^{y_c} / \sum_j e^{y_j}$ the predicted probability for class $c$. For layer $\ell$ 
with $K_\ell$ channels, we denote by $A^k_\ell = f^k_\ell(\mathbf{u}) \in \real^{h_\ell \times w_\ell}$ 
the feature map for channel $k \in \{1,\dots,K_\ell\}$, with spatial resolution $h_\ell \times 
w_\ell$. Because of $\relu$ non-linearities, we assume that feature maps are non-negative. 
Similarly, we denote by $S_\ell \in \real^{h_\ell \times w_\ell}$ a 2D saliency map.\\
%--------------------------------------------------------------------------------------------------

\begin{figure}[t]
    \centering
    \scriptsize
    \includegraphics[width=\textwidth]{fig/rel/images/conv_layers.pdf}
    \caption{Filter response to learnt classes alongside CNN depth \autocite{lee2009convolutional}.}
    \label{fig:cnn_depth}
\end{figure}
\label{sec:oc_back}
\noindent Given a layer $\ell$ and a class of interest $c$, we consider saliency maps given by the 
general formula: \\

\begin{equation}
	S^c_\ell(\mathbf{u}) \defn h \left( \sum_k w^c_k A^k_\ell \right),
\label{eq:sal}
\end{equation}

where $w^c_k$ are weights defining a linear combination over channels and $h$ is an activation 
function, we present a visualization of this process in \autoref{fig:cam_schema}. It is through the 
calculation of these weights or weighting coefficients that a plethora of saliency methods have 
been proposed using CAM. Informally, this can be referred to as \emph{The Many Flavors of CAM}.
Moreover, given the flexibility in defining this coefficient, we outline milestone alternatives for 
this calculation.\\

\begin{figure}[H]
    \centering
        \begin{tikzpicture}[
            font={\footnotesize},
            trap/.style={trapezium, rotate=-90,trapezium angle=75},
        ]
            %% Nodes
            \node(input) at (-6, 0) {\includegraphics[width=.1\textwidth]{fig/rel/images/cat_v2.jpg}};
            \node[above] at (input.north) {Input image $\vumod$};
            \node[draw, trap](cnn) at (-3, 0) {\rotatebox{90}{\parbox{1.0cm}{\centering{\Th{CNN}}}}};
            \node(fmaps) at (-0.5,0) {\includegraphics[width=.1\textwidth]{fig/rel/images/fmaps_side.pdf}};
            \node(fmk) at (-0.5, 1.3) {$A^k_\ell$};
            \node(classtag) at (3, 1.3) {\Th{Classifier}};
            \node(GAP) at (1.25,0.25) {$\gap$};
            \node(gapclass) at (3, 0) {\includegraphics[width=.075\textwidth]{fig/rel/images/gapclass.pdf}};
            \node(logit) at (5, 0) {$\vy_c$};
            \node(wck) at (-0.5, -3) {\includegraphics[width=0.02\textwidth, angle=90]{fig/rel/images/classifier.pdf}};
            \node(comb) at (-0.5, -3.5) {Weighting coefficient $w_k^c$};
            \node[draw, circle] (odot) at (-0.5, -1.8) {$\times$};
            \node(salient) at (-3, -1.8) {\includegraphics[width=.1\textwidth]{fig/rel/images/cat_cam.jpg}};
            \node[below] at (salient.south) {Saliency map $\mathbf{S_\ell}$};
            
            %% Edges

            \draw[->] (input.east) -- node {} (cnn);
            \draw[->] (cnn.north) -- node[above] {$\ell$} (fmaps.west);
            \draw[->] (fmaps.east) -- node {} (gapclass.west);
            \draw[->] (gapclass.east) -- node[above] {} (logit.west);
            \draw[->] (fmaps.south) -- node {} (odot.north);
            \draw[->] (wck.north) -- node {} (odot.south);
            \draw[->] (odot.west) -- node {} (salient);           

    \end{tikzpicture}
    \label{fig:cam_schema}
    \caption{\gls{cam} based methodologies overview.}
\end{figure}

\noindent \emph{CAM} \autocite{zhou2016learning} is the original proposal for this family of 
methods. This approach is defined uniquely for the last layer $L$ before the classifier and $h$ the 
identity mapping and $w^c_k$ is defined as the classifier weights that map the $k$-th channel in 
the feature map with class $c$. This first definition of class activation methods is not general, 
should a classifier be composed of a MLP; $w^c_k$ would have to account for the multiple 
interactions across the stages on the aforementioned layer, which ultimately is not 
straightforward. \\
 
\noindent\emph{Grad-CAM} \autocite{selvaraju2017grad} is proposed as a generalization of \emph{CAM}. 
Following the inconvenience of the calculation of the weighting coefficient on multi layered 
classifiers, this approach considers instead the flow of information on the network following the 
gradient generated by backpropagation following the logit of the class of interest. By doing so, 
the feature map to be used for an explanation can be selected from any
layer $\ell$ at different depths of the network. Moreover, the identity mapping on this approach is 
$h = \relu $ and the weighting coefficient is calculated in the following manner:
\begin{equation}
	w^c_k \defn \gap \left( \pder{y_c}{A^k_\ell} \right),
\label{eq:gcam_coeff}
\end{equation}
where $\gap$ is global average pooling.
The motivation for $\relu$ is that we are only interested in features that have a positive effect 
on the class of interest, \ie pixels whose intensity should be increased in order to increase $y_c$. 
Lastly, this approach also considers the computation of back-propagation using gradient refinements 
such as \emph{guided backpropagation} \autocite{guidedbackprop}, allowing for saliency maps to 
highlight more salient regions within the image.\\

\noindent \emph{Grad-CAM++} \autocite{chattopadhay2018grad} continues the trend of refining the 
generated saliency map via gradient modifications. In particular, the computation of the gradient 
follows partial derivatives as a way to address shortcomings on the original Grad-CAM. Specifically, 
Grad-CAM is not robust to multiple instances of the same class within one image, and overall the 
localization of the attribution usually fails to be located over the entirety of the object of 
the class of interest. The computation of $w^c_k$ for Grad-CAM is defined in the following manner:
\begin{equation}
	w^c_k \defn \gap \left[\frac{\frac{\partial^2 y_c}{(\partial A^k)^2}}{2\frac{\partial^2 y_c}
	{(\partial A^k)^2}+\gap\left(A^k \frac{\partial^3 y_c}{(\partial A^k)^3}\right)}\right] \cdot 
	\relu \left(\pder{y^c}{A^k}\right)
	\label{eq:gcampp_coeff}
\end{equation}
Similar to Grad-CAM, Grad-CAM++ uses $h = \relu$, is not computational expensive, 
provides high quality saliency maps with better localization properties and is yet another 
generalization of prior approaches. Nevertheless, one crucial contribution proposed alongside this 
attribution method consists of an evaluation methodology for saliency maps, explained in detail in 
\autoref{rel:sub_evl}.\\

\noindent\emph{Score-CAM} \autocite{wang2020score} is also defined for any layer $\ell$ with 
$h = \relu$ and weights $w^c_k \defn \softmax(\mathbf{u}^c)_k$.  Softmax normalization considers 
positive channel contributions only and attends to few feature maps. Inspired by the idea of 
comparing the increase in confidence obtained by forwarding an image masked with the saliency map 
relating to class $c$, a vector $\mathbf{u}^c \in \real^{K_\ell}$ compares a known baseline image 
$\mathbf{u}_b$ with the input image $\mathbf{u}$, for all channels $k$:

\begin{equation}
	u^c_k \defn f(\mathbf{u} \odot n(\operatorname{up}( A^k_\ell )))_c - f(\mathbf{u}_b)_c,
\label{eq:s-cam}
\end{equation}

where $\odot$ is the Hadamard product. For this to work, the feature map $A^k_\ell$ is adapted 
to $\mathbf{u}$ first$:\operatorname{up}$ denotes up-sampling to the spatial resolution of 
$\mathbf{u}$ and:

\begin{equation}
	n(A) \defn \frac{A - \min A}{\max A - \min A}
\label{eq:norm}
\end{equation}

is a normalization of matrix $A$ into $[0,1]$. While Score-CAM does not need gradients, it requires 
as many forward passes through the network as the number of channels in the chosen layer, which is 
computationally expensive. Score-CAM is better understood in a visual manner as in 
\autoref{fig:rel_sccam}\\

\begin{figure}[t]
    \centering
    \includegraphics[width=\textwidth]{fig/rel/images/scorecam_schema.pdf}
    \caption{Score-CAM computation process \autocite{wang2020score}}
    \label{fig:rel_sccam}
\end{figure}
\noindent \emph{Axiom-CAM} \autocite{axiombased} Departing from the usage of the increase in 
confidence and gradient information, this approach incorporates logical principles to guide the 
calculation of the saliency map. In particular, two axioms are introduced \emph{sensitivity} and 
\emph{conservation}. On one hand \emph{sensitivity} considers that the importance of each feature 
map has to be equivalent to the score change caused by its removal. In the other hand, \emph{conservation} 
points that changes in the saliency map come from a redistribution of the class score. In other 
words, sensitivity considers the loss of predictive power by the removal of high importance 
feature maps; \emph{conservation} instead ensures that class scores are mainly dominated by feature 
maps rather than external factors. By maintaining the activation function $h = \relu$, the 
formulation of the axiom-based weighting coefficient follows:

\begin{equation}
	w^c_k \defn \gap \left( \frac{A^k}{\gap(A^k)}\pder{y^c}{A^k}\right).
	\label{eq:ax_cam}
\end{equation}

\noindent \emph{Ablation-CAM} \autocite{ablationcam} continuing the trend of computing $w^c_k$ 
without making use of gradients, masking methods can be further incorporated in the production of 
saliency maps. More specifically, this approach determines the weight of individual feature maps 
via ablation perturbation. Going deeper, this approach evaluates the predictive power $y^c_k$ of an 
individual channel $k$ in feature map $A^k$ when all other activations are zeroed out. In one way, 
this could be seen as similar to increase in confidence to produce the linear combination as in 
\emph{Score-CAM}. Nevertheless, in the current approach, modifications are done on feature space 
and the predictive power is compared with that of predictions using $A^k$ with all the feature maps 
unaltered. Retaining activation $h = \relu$, the formulation of this weighting coefficient follows:
\begin{equation}
	w^c_k \defn \frac{y^c-y^c_k}{y^c}
\end{equation}

\noindent \emph{Layer-CAM} \autocite{jiang2021layercam} although \gls{cam} can be extracted at any 
point of a model following the generalities described previously, they do not take into 
consideration the flow of information and relevant features along the depth of a model. 
Nevertheless, with backpropagation gradient can be probed alongside the whole depth of a model; as 
such, this approach computes a saliency map by averaging the up-sampled attribution maps obtained at 
different depths from the model, generating a representation accounting for this information.

%--------------------------------------------------------------------------------------------------
