%--------------------------------------------------------------------------------------------------
\subsection{Discussion}
\label{sec:rel_interp_discussion}
Following the rapid development of image recognition models and the subsequent need to understand 
their behavior, interpretability has become a sought after task within the community. As a result, 
several studies have been introduced over time, as demonstrated in the previous section. Yet, the 
work proposed by Lipton is unique in questioning what interpretability truly entails. 
Instead, most interpretability approaches are designed to address specific needs of current image 
recognition models, ranging from simplifying a model, to describing its components, disentangling 
embeddings on feature space and ultimately explaining predictions by responses on the input space. 
We suggest that some recurring issues that interpretability presents is an outdated vision on some 
properties and a lack of consensus, either in definitions, and evaluation.\\

\noindent Starting with propositions, according to different researchers' interpretability is 
described across various dimensions depending on the nature of the approach. In this thesis we 
chose to follow the original propositions of Lipton, describing interpretability according to the 
properties of \emph{transparency} and the ability to provide \emph{post-hoc interpretations.} We 
acknowledge these properties as base descriptors for interpretability. However, we note that these 
properties are ill-fitted for describing current image recognition models according 
to their original definitions. In particular, transparency on its preliminary definition applies 
mostly to traditional machine learning proposals. Nevertheless, by aligning transparency with 
the active dimension proposed by \cite{zhang2021survey}, the definition holds, and subsequent 
studies adhere it.\\

\noindent Regarding saliency, we also note that this study is-ill formulated. When we obtain a 
saliency map, \emph{how do we define what is important in an image?}. On one hand, although computer  
vision draws heavy inspiration from by human vision, the reasoning process is different between 
human and machine. In particular, a human might identify salient parts to describe an object in 
particular, differently that a machine would. Conversely, when discussing an attribution, 
\emph{who are we considering the explanation for?}. As we mentioned previously, saliency is not 
well aligned for humans and machines, especially when machines derive their knowledge from context. 
Consequently, when evaluating the interpretable properties of an attribution, we find that usually 
those that provide the best results in terms of metrics are not usually the ones that a human 
would consider best. This question remains open up to this day.\\

\noindent Regarding evaluation of interpretability methods, in \autoref{rel:sub_evl} we made mention 
of current evaluation methodologies, but, several questions arise from them. To begin with, and                                                                                                          
relating to the previous paragraph, it is safe to assume that the quantitative metrics presented 
are taking into consideration relevance towards the classifier. In particular \emph{these 
methods are telling us which representation explains the best a given class for the model}. 
And yet, in this case we can also wonder about \emph{which class should we inquire about; 
groundtruth or predictions?}. In most applications researchers extract attributions and provide 
explanations for groundtruth objects. However, in real-world applications we expect models to 
predict incorrectly some instances in the task that they are given; these instances then require 
explanations for the class being predicted. This is crucial given Lipton's desidedata of model 
interpretability. We usually care the most for the instances in which the model fails, and we have 
to provide accountability for the effects these inferences provide.\\

\noindent In addition to this selection of instances to provide explanations for, quantitative 
evaluation suffers from another drawback: a lack of homogeneity in evaluation procedures. 
Complementary to the qualitative evaluation of attribution maps via visual inspection and human 
criteria, quantitative evaluation ought to be more robust, replicable and homogeneous. Yet, we 
observe variations in evaluation procedures across proposals, and while an effort to provide fair 
comparisons, complete insight on the behavior of the proposed methodologies is not attained. 
To exemplify, with the introduction of \emph{objective evaluation for object recognition} in Grad-
CAM++ \autocite{chattopadhay2018grad}, the evaluation procedure only required generation of 
visualizations for the entirety of the validation set of ImageNet and Pascal VOC 2012, whereas 
to assess the performance of Score-CAM \autocite{wang2020score}, a small subset of two thousand 
random images is chosen, thus negating replicability. We note that this is done to circumvent the 
discussion of the high computational cost required to compote Score-CAM attributions, which is  
not clearly discussed in its article. Complementary to this, 
\cite{chattopadhay2018grad} provides analysis for VGG-16, ResNet-50 and AlexNet, while Score-CAM 
presents results only for VGG.  This is not the only occurrence of these phenomena, upon the 
introduction of methodologies such as Integrated Score-CAM \autocite{naidu2020cam}, Ablation-CAM
\autocite{ablationcam} and Layer-CAM \autocite{jiang2021layercam}, the evaluation procedures are 
found not to be standardized between approaches.\\

\noindent A last point we want to highlight is addressed towards claims of interpretability upon 
the proposal of models. To begin, while introducing image recognition models in 
\autoref{rel:sec_imrecon}, we found several approaches such as Conformer 
\autocite{peng2021conformer}, Scouter \autocite{li2021scouter} and LFI-CAM \autocite{lee2021lfi} 
claiming to produce high performance image recognition architectures with built-in interpretability 
properties. However, we note that these claims are sustained only with qualitative results in the 
shape of attribution maps such as \gls{cam} or attention visualization. Taking into consideration 
all the points we have discussed so far in this section, we note that this exemplifies the 
challenges when discussing interpretability. On one hand, visually assessing the quality of 
explanation maps implies the alignment with human reasoning towards describing what is important 
within an image. Conversely, attribution maps are often shown mostly for groundtruth classes, not 
predictions, leaving in turn open the question of what is relevant for these instances. Lastly, 
qualitative measurements are not meaningful to claim the performance of an attribution compared 
with another, in particular given the misalignment between human and machine recognition processes.

%To begin with, 
%adhering to the lack of clarity on the definition of interpretability, we consequently find works 
%claiming to provide interpretability in their approaches but no quantitative results to support 
%them. As a result of this, we wonder around 