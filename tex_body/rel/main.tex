\chapter{Background}
\label{ch:rel}
\chaptertoc{}
\section{Introduction}
\noindent Understanding the processes behind visual recognition has been a prominent research 
question throughout human history. From the preliminary questionings by greek  philosphers 
(\cite{finger2001origins}) to physics based studies like those by Newton and Locke 
(\cite{swenson2010optics}), and more recently with theories like \textit{Unconscious Inference} 
(\cite{gullstrand1909hemholtz}) and \textit{Gestalt} (\cite{wagemans2012century}), many proposals 
to understand and describe this process have been brought forth. Moreover, vision recogniton is not 
only studied in fields like physics, medicine and psychology; 
with the advent of computer science, computational approaches and theories started emerging 
regarding this domain. One such study that proved seminal in this domain is that carried out by 
David Marr (\cite{poggio1981marr}, \cite{marr2010vision}). Most notably, Marr addressed vision on 
three levels: comptational, algorithmic and implementation. In particular, upon the computational 
level, Marr pondered around issues that the visual system answers and their explanation; this 
ultimately led to the formulation of fundamental tasks within computer vision such as object 
recognition and reconstruction.\\

\noindent Following Marr's proposals and the ensuing research on computer vision, researchers 
centered their attention at developing methodologies towards performing these fundamental tasks.
Starting with preliminary works on reconstruction of 3D objects in space, the development of 
computer vision models then followed specialized approaches for specific tasks. In this thesis, 
we are interested on models designed at performing image recognition and, more specially, in 
understanding their functioning and providing explanations for their inner workings.\\

\noindent Image recognition models capabilities have improved over time with constant development, 
closely aligned with the increase in computing power. This evolution has lead to steady 
advancements in both performance and complexity. On its early approaches, image recognition models 
relied on handcrafted feature extraction methods  in conjuction with traditional machine learning 
algorithms. However, this reliance on these features ultimately limits these methodologies' 
capabilities to capture intricate visual patterns.
One particular approach that had a strong initial impact and high performance was \gls{hog} 
\autocite{dalal2005histograms}. In this methodology, gradient information was used to train a 
\gls{svm} to perform pedestrian recognition. While achieving high performance in the dataset it was 
designed, \gls{hog} ultimatately failed in data collections where complexity was higher 
\autocite{5975165}.\\

\noindent It is not only with the increase of computational power that computer vision has improved 
over time. With the developement, popularization and spread of the internet; large collections of 
data are formed. These aggregations can be extremely specific for a given end, or quite general 
representing the common interests of its users. Over time, these compilations have continued to 
grow both in volume and variety; still, several curated collections are introduced by researchers 
to experiment and control the development of models such as MNIST (\cite{lecun1998gradient}), 
BSDS (\cite{MartinFTM01}), Pascal VOC (\cite{pascal-voc-2012}) and most notably, 
ImageNet (\cite{ILSVRC15}) and MS-COCO (\cite{lin2014microsoft}).\\

\noindent Furthermore, with the resurgence of convolutional neural networks and the ensuing advent 
of deep learning, a paradigm shift in this field occurred. This transition led to the elimination 
of the need for handcrafted feature extraction; instead, deep learning allowed \glspl{cnn} to act 
as both feature extractor and classifiers on themselves. Moreover, based on the aggregation of 
convolutions as their fundamental units, a \gls{cnn} is able to learn hierarchichal 
representations of data, extracting intricate features directly. 
Consequently, this shift resulted in improvements in accuracy and performance in various image 
recognition tasks. \\

\noindent However, it was observed that performance of \glspl{cnn} was achieving a plateau, and 
the introduction of novel architectures stagnated over time. Additionally, these models encountered 
challenges capturing long range dependencies, in turn  limiting their capacity to construct global 
representations of data and affecting their generalization capabilities. Transformers however, have 
shown remarkable improvements not only in the image domain but also in language related tasks, 
revolutionizing these fields and paving the groundwork for future research and developments.\\

\noindent In \autoref{rel:sec_imrecon} we explore some of the most important approaches based 
on machine learning designed towards image recognition. In particular, in \autoref{rel:sub_cnn}, we 
introduce the basis of deep learning and \glspl{cnn}. Moving on, in \autoref{rel:sub_att} we make 
mention of the Transformer architecture, its building block and how it has reshaped the landscape 
of image recognition. Finally in \autoref{rel:sub_hybrid} we display approaches that make use of 
combinations of the last two forementioned approaches.\\


\noindent With the adoption of deep image recognition into society; understanding the 
innerworkings of these models has become a top priority. We shine light into some fields where 
this is the case:
\begin{itemize}
    \item \textbf{Facial Recognition} Facial recognition is a fine grained classification task, 
    that can be mostly associated with identification and re-identification of individuals. 
    In this aspect, understanding model predictions is associated with accountability, ethical 
    considerations and safety.
    (\cite{selinger2020inconsentability}, \cite{andrejevic2020facial}).
    \item \textbf{Automated Medical Diagnosis} In medical imaging, the education 
    required to read and provide analysis, often requires experience based 
    on personal expertise \autocite{nakashima2013visual}. Examples of automated diagnosis encompas 
    melanoma detection, bone age assessment and most recently, COVID-19 diagnosis 
    (\cite{yu2016automated}, \cite{BoNet2019hand}, \cite{huang2021artificial}). The medical domain 
    is of special care as human lives are directly at stake, therefore understanding predictions is 
    highly desired.
    \item \textbf{Self-driving Vehicles} Over the past decade, advancements in this field have lead to 
    discussions regarding the impact of the adoption of these kind of vehicles within smart cities  
    (\cite{duarte2018impact}, \cite{millard2018pedestrians}). Nevertheless their 
    navigation is not completely perfect and it is also possible to attack it, leading to possible 
    traffic accidents \autocite{dixit2016autonomous}; in this case, accountability is then again 
    taken into consideration.
\end{itemize}

We observe that interpretability needs in these fields and real world applications follows 
Lipton's discussion on the \textit{Desiderata of Interpretability Research} 
\autocite{mythos_interp}. In this aspect, we expect that the explanations that any given 
approach help us \emph{trust} any model. Conversely, we expect a model to be explained in a 
\emph{causal} manner according to its explanations. On another hand, 
we expect these remarks be \emph{informative} and shine light on similar examples that the 
model processes. Finally, we expect interpretable explanations to guarantee that the outputs 
of a model are \emph{fair} and \emph{ethical}.

With these properties in mind leading the \emph{desiredata of interpretability study}, we further 
investigate in this work some of the most important works on interpretability. In 
\autoref{rel:sec_int} we explore preliminary ideas of this field. Delving deeper, in 
\autoref{rel:sub_transp} we discuss efforts aimed at transparency of machine learning approaches. 
In contrast, in \autoref{rel:sub_post} we explore and study 
some of the most relevant studies on post-hoc interpretability. Conversely, in 
\autoref{rel:sub_evl}, we outline the evaluation metrics used for evaluation the aforementioned 
works. To understand how these proposals are evaluated in their claims of interpretability, we 
introduce and explain evaluation methodologies in \autoref{rel:sub_evl}.
%\addcontentsline{toc}{chapter}{Literature Review}
%--------------------------------------------------------------------------------------------------
\section*{Image Recognition Models}
\addcontentsline{toc}{section}{Image Recognition Models}
\label{rel:sec_imrecon}

\subsection*{Convolutional Neural Networks}
\label{rel:sub_cnn}
%\addcontentsline{toc}{subsection}{Convolutional Neural Networks}

\subsection*{Attention-Based Architectures}
\label{rel:sub_att}
%\addcontentsline{toc}{subsection}{Attention-Based Architectures}
Attention is a powerful mechanism that has been introduced into convolutional networks in several 
ocasions \autocite{bello2019attention, ramachandran2019stand, shen2020global}. With the success of 
vision transformers (ViT) \autocite{dosovitskiy2020image}, fully attention-based architectures are now 
competitive with convolutional networks. To benefit from both self-attention and convolutional 
layers, some hybrid architectures employ convolutional layers before the vision transformer 
\autocite{graham2021levit,xiao2021early}. Others, such as Swin \autocite{liu2021swin} and PiT 
\autocite{heo2021rethinking}, introduces a pyramid structure to share local spatial information 
while reducing the spatial resolution, as in convolutional networks. 

SCOUTER \autocite{li2021scouter} uses slot attention \autocite{locatello2020object} to build a class 
specific pooling mechanism, which does not scale well to more than 20 classes. 

\subsection*{Hybrid Architectures}
Conformer \autocite{peng2021conformer} proposes a dual network structure to retrain and fuse local 
convolutional features with global representations. Our method merely provides a simple 
attention-based pooling mechanism inspired by transformers that can work with any architecture, 
even pretrained and frozen. PatchConvNet \autocite{touvron2021augmenting} replaces global average pooling by an 
attention-based pooling layer.
\label{rel:sub_hybrid}
%--------------------------------------------------------------------------------------------------
\subsection{Discussion}
\label{subsec:rel_recon_discussion}
In computer vision, the developement of image recognition model has been crucial 
towards the advancement of different tasks such as image segmentation and object localization. 
Supported by the theory of the interaction of Malik's three Rs of computer vision 
\autocite{malik2016three}, we observe that progress of one of these three fields, leads to strides 
in complementary fields. As described in the previous section, many of the proposals demonstrated 
therein act as excellent feature extractors. This capacity in turn, facilitates adjacent tasks 
such as segmentation (regrouping) and reconstruction. With this in mind, we suggest that model 
developement and design account for interactions between the Rs (better seen in  
\autoref{fig:malik_rs}), and consequently, model developement being rooted mostly in image 
recognition, we aknowledge image recognition as one foundational task on computer vision.

\begin{figure}[H]
    \centering
    \begin{tikzpicture}[]
        %% CNN branch
        \node(recon) at (3, 3.3541){Recognition};
        \node(reconst) at (0, 0) {Reconstruction};
        \node(reorg) at (6, 0) {Reorganization};

        \draw[-Stealth] ([xshift=-6pt] reconst.north) -- node {} ([xshift=-6pt] recon.south);
        \draw[-Stealth] ([xshift=-2pt] recon.south) -- node {} ([xshift=-2pt] reconst.north);

        \draw[-Stealth] ([xshift=2pt] recon.south) -- node {} ([xshift=2pt] reorg.north);
        \draw[-Stealth] ([xshift=6pt] reorg.north) -- node{} ([xshift=6pt] recon.south);

        \draw[-Stealth] ([yshift=2pt] reconst.east) -- node {} ([yshift=2pt] reorg.west);
        \draw[-Stealth] ([yshift=-2pt] reorg.west) -- node {} ([yshift=-2pt] reconst.east);        
    
    \end{tikzpicture}
    \caption{Malik's three \emph{R} of computer vision}
    \label{fig:malik_rs}
\end{figure}

\noindent Furthermore, this domain has seen constant evolution in recent years. Following the 
resurgence of \glspl{cnn} after the introduction of AlexNet, a plethora of image recognition models 
were proposed. Still, we note that while these models are variate in structural units, complexity, 
and depth; the model formulation  itself is not the solely determining factor of performance.  
In a similar fashion to the points described by \emph{A Metric Learning Reality 
Check} \autocite{musgrave2020metric}, where a revision of metric learning methodologies revealed 
biases in the evaluation of novel methodologies and the enhanced power of 
previous methods when optimized under better considitions, overall performance evaluation of 
architectures and methodologies often lacks fair comparison due to advancement in optimization 
techniques. This is clearly demonstrated in \emph{ResNet Strikes Back} \autocite{wightman2021resnet}. 
Nevertheless, it is also possible to consider that \glspl{cnn} may be approaching a plateau in 
their capabilities, similar to traditional computer vision methods  when applied to ImageNet. As 
a response of this, we take special interest on transfomers, given their recent adoption and 
overall their promising capabilities. \\

\noindent Delvig into transformers, we remark the promise that they display 
given recent advances in tasks such as text recognition, text generation and notably in vision, 
on image generation, captioning and recognition. Comparable to the surge of convolution 
based methods in the early 2010s, overtaking traditional machine learning methods in computer 
vision applications, the paradigm shift previously mentioned is already taking place as 
we can observe on figure \autoref{fig:paradigmn_shift}. However, the enhancement of recognition 
capabilities is highly dependent on the amount and quality of data that is used in their 
design and optimization. At this point, it is important to highlight that recognition capabilities 
on ImageNet are nearing saturation point for the dataset, leading us to question whether it is 
truly a feat of model generalization or a severe case of overfitting.

%--------------------------------------------------------------------------------------------------
\begin{figure}[H]
    \centering
    \pgfplotstableread[col sep=comma,]{fig/rel/data/models_year.csv}\datatable
    \begin{tikzpicture}
        \begin{axis}[
            date coordinates in=x,
            date ZERO=2017-03-31,
            xtick distance=0.25,
            %xtick={\datatable}{Year},
            %xticklabel={\year-\month-\day},
            xtick=data,
            width=\textwidth,
            height=6.85cm,
            %xticklabels from table={\datatable}{Year},
            xticklabel style={rotate=90, anchor=near xticklabel},
            ylabel={Proportion of Papers (Quarterly)},
            legend style={at={(0, 1)},anchor=north west},
            yticklabel style={/pgf/number format/fixed}
            ]
            \addplot [mark=*, red!80] table [x={Year}, y={ResNet}]{\datatable};
            \addlegendentry{{\scriptsize \Th{ResNet}}}

            \addplot [mark=*, black!80] table [x={Year}, y={VGG}]{\datatable};
            \addlegendentry{{\scriptsize \Th{VGG}}}

            \addplot [mark=*, green!80] table [x={Year}, y={DenseNet}]{\datatable};
            \addlegendentry{{\scriptsize\Th{DenseNet}}}

            \addplot [mark=*, cyan!80] table [x={Year}, y={EfficientNet}]{\datatable};
            \addlegendentry{{\scriptsize \Th{EfficientNet}}}

            \addplot [mark=*, blue!80] table [x={Year}, y={ViT}]{\datatable};
            \addlegendentry{{\scriptsize \Th{ViT}}}
        \end{axis}
    \end{tikzpicture}
    \caption{Proportion of articles published on ImageNet, using image recognition models as backbone across the years. 
             Original from \url{https://paperswithcode.com/method/resnet}}
    \label{fig:paradigmn_shift}
\end{figure}
%--------------------------------------------------------------------------------------------------
\section{Interpretability}
\label{rel:sec_int}
As deep learning based models for Computer Vision have continued to improve in their recognition 
properties, their structure and functioning have become more opaque; in turn making these 
technologies seen as black boxes. A black-box model is defined as a model for which its 
interpretation is not straightforward for humans \autocite{petch2022opening}. In recent years 
with the assimilation of deeplearning into everyday tasks, and the implicit effect these models 
are having on human lives; the novel research field of \emph{Interpretability} has been brought 
forth to open up this black-box behaviour. This research field investigated alongside two 
directions(\cite{mythos_interp}, \cite{guidotti2018survey}, \cite{zhang2021survey}):

\begin{enumerate}
	\item \emph{Post-hoc interpretability} considers the model as a black-box and provides 
	explanations based on inputs and outputs, without modifying the model or its training process.
	\item \emph{Transparency} modifies the model or the training process to better explain the 
	behavior of the inner parts of the model.
\end{enumerate}

\subsection{Transparency}
\label{rel:sub_transp}
Transparent 
%Transparency in interpretability research represents a pivotal pursuit within the broader 
%landscape of artificial intelligence. In the ever-evolving field of machine learning, 
%particularly deep neural networks, the inherent complexity of models often gives rise to a lack of 
%transparency, making it challenging to comprehend the decision-making processes. Researchers 
%focusing on transparency in interpretability seek to address this opacity by developing 
%methodologies and frameworks that not only reveal how models arrive at specific outcomes but 
%also provide a comprehensive understanding of the underlying mechanisms. The ultimate goal is to 
%empower users, stakeholders, and policymakers with the knowledge necessary to trust, validate, and 
%navigate the applications of artificial intelligence effectively. This emphasis on transparency not 
%only enhances the accountability of AI systems but also contributes to broader societal acceptance 
%and ethical deployment of these powerful technologies.
Approaches are grouped in a number of categories according to the type of the given explanation. \\
\emph{Rule-based methods} (\cite{wu2018beyond}, \cite{wu2020regional}) train a decision tree as a 
surrogate regularization term to force a network to be easily approximated by a decision tree.
% or learn a set of logit rules as an explanation~\citep{azzolin2022global}.
\emph{Hidden semantics-based methods} (\cite{bau2017network}, \cite{zhou2018interpreting}, 
\cite{zhang2018interpretable}, \cite{zhou2014object},\cite{bohle2022b}. \cite{bohle2024b}) aim to 
make a convolutional network learn disentangled hidden semantics with hierarchical structure or 
object-level concepts.\\
\emph{Prototype-based methods} (\cite{li2018deep}, \cite{chen2019looks}) learn a set of prototypes 
or parts as an intermediate representation in the network, which can be aligned with categories. 
\emph{Attribution-based methods} (\cite{ismail2021improving}, \cite{Zhou_2022_BMVC}, \\
\cite{ross2017right}, \cite{ghaeini2019saliency}) usually modify the architecture of a network or 
the training process to help post-hoc methods produce better saliency maps. Unlike 
(\cite{ross2017right}, \cite{ghaeini2019saliency}), saliency guided localization 
\autocite{Zhou_2022_BMVC} does not need ground truth explanations but replaces them with 
information bottleneck attribution \autocite{schulz2020restricting}. Finally, saliency-guided 
training \autocite{ismail2021improving} minimizes the KL divergence between the output of original 
and masked images.

\subsection{Post-Hoc Interpretability}
\label{rel:sub_post}

Approaches can be grouped into a number of possibly overlapping categories. 
\emph{Gradient-based methods} (\cite{adebayo2018local}, \cite{guidedbackprop}, 
\cite{baehrens2010explain}, \cite{simonyan2013deep}, \cite{smilkov2017smoothgrad}, 
\cite{bach2015pixel}, \cite{sundararajan2017axiomatic}) use the gradient of a target class score 
with respect to the input to compute the contribution of different input regions to the prediction. 
\emph{CAM-based methods} (\cite{wang2020score}, 
\cite{chattopadhay2018grad}, \cite{selvaraju2017grad}, 
\cite{axiombased}, \cite{jiang2021layercam}, \cite{ablationcam}) compute saliency maps 
as a linear combination of feature maps, with different definitions for the weights. 

\emph{Occlusion or masking-based methods} (\cite{petsiuk2018rise}, \cite{fong2017interpretable}, 
\cite{fong2019understanding}, \cite{schulz2020restricting}, \cite{ribeiro2016should}) apply a 
number of candidate masks in the input space, measure their effect on the prediction and then 
combine them into a saliency map. Masking in feature space has been explored too 
\autocite{schulz2020restricting}. 

Finally, \emph{learning-based methods} (\cite{chang2018explaining}, \cite{dabkowski2017real}, 
\cite{phang2020investigating}, \cite{zolna2020classifier}, \cite{schulz2020restricting}) learn an 
additional network or branch on extra data to produce an explanation map for a given input. 

\paragraph{Notation}
\label{sec:oc_notation}

Consider a classifier network $:fx \cX \to \real^C$ that maps an input image $\mathbf{u} \in \cX$ to a 
logit vector $\vy = f(\mathbf{u}) \in \real^C$, where $\cX$ is the image space and $C$ is the number 
of classes. We denote by $y_c = f(\mathbf{u})_c$ the predicted logit and by $p_c = \softmax(\vy)_c 
\defn e^{y_c} / \sum_j e^{y_j}$ the predicted probability for class $c$. For layer $\ell$ 
with $K_\ell$ channels, we denote by $A^k_\ell = f^k_\ell(\mathbf{u}) \in \real^{h_\ell \times w_\ell}$ 
the feature map for channel $k \in \{1,\dots,K_\ell\}$, with spatial resolution $h_\ell \times 
w_\ell$. Because of $\relu$ non-linearities, we assume that feature maps are non-negative. 
Similarly, we denote by $S_\ell \in \real^{h_\ell \times w_\ell}$ a 2D saliency map.

%--------------------------------------------------------------------------------------------------

\paragraph{Background: CAM-based saliency maps}
\label{sec:oc_back}

Given a layer $\ell$ and a class of interest $c$, we consider saliency maps given by the general 
formula
\begin{equation}
	S^c_\ell(\mathbf{u}) \defn h \left( \sum_k w^c_k A^k_\ell \right),
\label{eq:sal}
\end{equation}
where $w^c_k$ are weights defining a linear combination over channels and $h$ is an activation 
function. CAM \parencite{zhou2016learning} is defined for the last layer $L$ only with $h$ being the 
identity mapping and $w^c_k$ being the classifier weight connecting the $k$-th channel with 
class $c$. Grad-CAM \parencite{selvaraju2017grad} is defined for any layer $\ell$ with $h = \relu$ and 
weights
\begin{equation}
	w^c_k \defn \gap \left( \pder{y_c}{A^k_\ell} \right),
\label{eq:gcam}
\end{equation}
where $\gap$ is global average pooling.
% and $\softmax(\vy)_c = e^{y_c} / \sum_j e^{y_j}$ is the predicted probability of class $c$.
The motivation for $\relu$ is that we are only interested in features that have a positive effect 
on the class of interest, \ie pixels whose intensity should be increased in order to increase $y_c$.

Score-CAM \parencite{wang2020score} is also defined for any layer $\ell$ with $h = \relu$ and weights 
$w^c_k \defn \softmax(\mathbf{u}^c)_k$.  Softmax normalization considers positive channel contributions 
only and attends to few feature maps.
%that \alert{produce less highlighted areas in the saliency map}. \iavr{Last part unclear.}
Here, vector $\mathbf{u}^c \in \real^{K_\ell}$ measures the increase in confidence for class $c$ that 
compares a known baseline image $\mathbf{u}_b$ with the input image $\mathbf{u}$ masked according to feature 
map $A^k_\ell$, for all channels $k$:

\begin{equation}
	u^c_k \defn f(\mathbf{u} \odot n(\operatorname{up}( A^k_\ell )))_c - f(\mathbf{u}_b)_c,
\label{eq:s-cam}
\end{equation}

where $\odot$ is the Hadamard product. For this to work, the feature map $A^k_\ell$ is adapted
 to $\mathbf{u}$ first$:\operatorname{up}$ denotes upsampling to the spatial resolution of $\mathbf{u}$ and

\begin{equation}
	n(A) \defn \frac{A - \min A}{\max A - \min A}
\label{eq:norm}
\end{equation}

is a normalization of matrix $A$ into $[0,1]$. While Score-CAM does not need gradients, 
it requires as many forward passes through the network as the number of channels in the chosen layer,
 which is computationally expensive.

%--------------------------------------------------------------------------------------------------

\paragraph{Motivation}
\label{sec:motiv}

\iavr{Score-CAM considers each feature map as a mask in isolation. How about linear combinations?} 
Given a vector $\vw \in \real^{K_\ell}$ with $w_k$ its $k$-th element, let
\begin{equation}
	F(\vw) \defn f \left( \mathbf{u} \odot n \left( \operatorname{up} \left(
		\displaystyle\sum_k w_k A^k_\ell
	\right) \right) \right)_c.
\label{eq:s-obj}
\end{equation}
\ronan{If we assume that $\mathbf{u}_b = \vzero$ in~\eq{s-cam} and define $n(\vzero) \defn \vzero$ 
in~\eq{norm}, then we can rewrite the right-hand side of~\eq{s-cam} as
\begin{equation}
	\frac{F(\vw_0 + \delta \ve_k) - F(\vw_0)}{\delta},
\label{eq:s-cam2}
\end{equation}
where $\vw_0 = \vzero$, $\delta = 1$ and $\ve_k$ is the $k$-th standard basis vector of 
$\real^{K_\ell}$. This resembles the numerical approximation of the derivative $\pder{F}{w_k}(\vw_0)$,
 except that $\delta$ is not small as usual. One could compute derivatives efficiently by 
 standard backpropagation instead. It is then possible to iteratively optimize $F$ with respect
  to $\vw$, starting at any $\vw_0$.}

\iavr{As an alternative, consider masking-based methods relying on optimization in the input space, 
like \emph{meaningful perturbations} (MP) \parencite{fong2017interpretable} or 
\emph{extremal perturbations} \parencite{fong2019understanding}. In general, optimization takes the form
\begin{equation}
	S^c(\mathbf{u}) \defn \arg\max_{\vm \in \cM} f(\mathbf{u} \odot n(\operatorname{up}(\vm)))_c + \lambda R(\vm).
\label{eq:mask}
\end{equation}
Here, a mask $\vm$ is directly optimized and does not rely on feature maps, hence the saliency 
map $S^x(\mathbf{u})$ is not connected to any layer $\ell$. The mask is at the same or lower resolution 
than the input image. In the latter case, upsampling is still necessary.

In this approach, one indeed computes derivatives by backpropagation and indeed iteratively 
optimizes $\vm$. However, because $\vm$ is high-dimensional, there are constraints expressed by 
$\vm \in \cM$, \eg $\vm$ has a certain norm, and regularizers like $R(\vm)$, \eg $\vm$ is smooth in a 
certain way. This makes optimization harder or more expensive and introduces more hyperparameters 
like $\lambda$. One could simply constrain $\vm$ to lie in the linear span of $\{A_\ell^k\}_{k=1}
^{K_\ell}$ instead, like all CAM-based methods.}

\subsection{Discussion}

%--------------------------------------------------------------------------------------------------
\subsection{Evaluating Interpretability}
\label{rel:sub_evl}
\paragraph{Classification Metrics}
\gls{ad} and \gls{ai} \autocite{chattopadhay2018grad} are 
well-established classification metrics. 
They measure the effect on the predicted class probabilities by masking the input image with the
 saliency map. Let $p^c_i$ and $o^c_i$ be the predicted probability for class $c$ given as input 
 the $i$-th test image $\vx_i$ and its masked version respectively. 
Masking refers to element-wise multiplication with the saliency map, which is at the same 
resolution as the original image with values in $[0,1]$. 
Let $N$ be the number of test images. Class $c$ is taken as the ground truth.

\emph{Average drop} (\gls{ad}) quantifies how much predictive power, measured as class probability, 
is lost when we only mask the image; lower is better:
\begin{equation}
	\AD(\%) \defn \frac{1}{N} \sum_{i=1}^N \frac{[p^c_i - o^c_i]_+}{p^c_i} \cdot 100.
\label{eq:ad}
\end{equation}

\emph{Average increase} (\gls{ai}), also known as \emph{increase in confidence}, measures the 
percentage of images where the masked image yiel ds a higher class probability than the original; 
higher is better:
\begin{equation}
	\AI(\%) \defn \frac{1}{N} \sum_i^N \mathbb{1}_{p^c_i < o^c_i} \cdot 100.
\label{eq:ai}
\end{equation}

$\AD$ and $\AI$ are not defined in a symmetric way. $\AD$ measures changes in class probability 
whereas $\AI$ measures a percentage of images. It is possible that the percentage is high while 
the actual increase is small. Hence, it is possible that an attribution method improves both. 
Indeed, \autocite{poppi2021revisiting} observes that a trivial method called Fake-CAM outperforms 
state-of-the-art methods, including Score-CAM, by a large margin. Fake-CAM simply defines a 
saliency map where the top-left pixel is set to zero and is uniform elsewhere. 
This questions the purpose of $\AD$ and $\AI$.

%--------------------------------------------------------------------------------------------------
\paragraph{Localization metrics}
\label{sec:loc-metrics}
Several works measure the localization ability of saliency maps, using metrics from the 
\emph{weakly-supervised object localization} (WSOL) task.

We are given the saliency map $S^c$ obtained from test image $\vx$ for ground truth class $c$. 
We denote by $S^c_{\vp}$ its value at pixel $\vp$. We binarize the saliency map by thresholding at 
its average value and we take the bounding box of the largest connected component of the resulting 
mask as the predicted bounding box $B_p$, represented as a set of pixels. We compare this box 
against the set of ground truth bounding boxes $\cB$, which typically contains 1 or 2 boxes of the 
same class $c$, or with their union $U = \cup \cB$, again represented as a set of pixels. We also 
compare the predicted class label $c_p$ with the ground truth label $c$. All metrics take values in 
$[0,1]$ and are expressed as percentages, except SM~\eq{sm}, which is unbounded.

\emph{Official Metric (OM)}
measures the maximum overlap of the predicted bounding box with any ground truth bounding box, 
requiring that the predicted class label is correct:
\begin{equation}
	\OM \defn 1 - \paren{\max_{B \in \cB} \iou(B, B_p)} \mathbbm{1}_{c_p = c},
\label{eq:om}
\end{equation}
where $\iou$ is intersection over union.
% is defined as $\iou(B, B_p) \defn \frac{B \cap B_p}{B \cup B_p}$.

\emph{Localization Error (LE)} is similar but ignores the predicted class label:
\begin{equation}
	\LE \defn 1 - \max_{B \in \cB} \iou(B, B_p).
\label{eq:le}
\end{equation}

\emph{Pixel-wise $F_1$ score (F1)} is defined as $F_1 = 2 \frac{P R}{P + R}$, where 
\emph{precision} $P$ is the fraction of mass of the saliency map that is within the ground truth 
union
\begin{equation}
	P \defn \frac{\sum_{\vp \in U} S^c_{\vp}}{\sum_{\vp} S^c_{\vp}}
\label{eq:prec}
\end{equation}

and \emph{recall} $R$ is the fraction of the ground truth union that is covered by the saliency map
\begin{equation}
	R \defn \frac{\sum_{\vp \in U} S^c_{\vp}}{\card{U}}.
	\label{eq:rec}
\end{equation}

\emph{Box Accuracy (BA)\autocite{choe2020evaluating}} Given threshold values $\eta$ and $\delta$, 
we find the bounding box $B^\eta_p$ of the largest connected component of the binary mask 
$\set{\vp: S_{\vp} > \eta}$ and require that it overlaps by 
$\delta$ with at least one ground truth box:
\begin{equation}
	\BA(\eta, \delta) \defn \max_{B \in \cB} \mathbbm{1}_{\iou(B^\eta_p, B) \ge \delta}.
\label{eq:ba}
\end{equation}
After averaging over the test images, we take the maximum of this measure over a set of values 
$\eta$ and then the average over a set of values $\delta$.\\
%--------------------------------------------------------------------------------------------------
\emph{Standard Pointing game (SP)\autocite{zhang2018top}} We find the pixel 
$\vp^* \defn \arg\max_{\vp} S^c_{\vp}$ having the maximum saliency value and 
require that it lands in any of the ground truth bounding boxes:
\begin{equation}
	\spg \defn \mathbbm{1}_{\vp^* \in U}.
\label{eq:spg}
\end{equation}\\

\emph{Energy Pointing game (EP)\autocite{wang2020score}} is equivalent to precision~\eq{prec}.\\

\emph{Saliency Metric (SM)\autocite{dabkowski2017real}} penalizes the size of the predicted bounding
 box $B_p$ relative to the image and the cross-entropy
 loss:
\begin{equation}
	\SM \defn \log \max\paren{ 0.05, \frac{\card{B_p}}{hw} } - \log p^c,
\label{eq:sm}
\end{equation}
where $h \times w$ is the input image resolution and $p^c$ is the precicted probability for ground 
truth class label $c$.

\subsection{Discussion}
%--------------------------------------------------------------------------------------------------
\subsection{Discussion}
\label{sec:rel_interp_discussion}
Following the rapid developement of image recognition models and the subsequent need to understand 
their behaviour, interpretability has become a sought after task within the community. As a result, 
several studies have been introduced over time, as demonstrated in the previous section.Yet, the 
work proposed by Lipton is unique in questioning what interpretability truly entails. 
Instead, most interpretability approaches are designed to address specific needs of current image 
recognition models, ranging from simplifying a model, to describing its components, disentangling 
embeddings on feature space and ultimately explaining predictions by responses on the input space. 
We suggest that some recurring issues that interpretability presents is an outdated vision on some 
properties and a lack of of concensus, either in definitions, and evaluation.\\

\noindent Starting with propositions, according to different researchers interpretability is 
described across various dimensions depending on the nature of the approach. In this thesis we 
chose to follow the original propositions of Lipton, describing interpretability according to the 
properties of \emph{transparency} and the ability to provide\emph{post-hoc interpretations.} We 
aknowledge these properties as base descriptors for interpretability. However, we note that these 
properties are ill-fitted for describing current image recognition models according 
to their original definitions. In particular, transparency on its preliminary definition applies 
mostly to traditional machine learning proposals. Nevertheless by aligning transparency with 
the active dimension proposed by \cite{zhang2021survey}, the definition holds and subsequent 
studies adhere it.\\

\noindent Regarding saliency we also note that this study is-ill formulated. When we obtain a 
saliency map, \emph{how do we define what is important in an image?}. On one hand, although computer  
vision draws heavy inspiration from by human vision, the reasoning process is different between 
human and machine. In particular, a human might identify salient parts to describe an object in 
particular in a different fashion that a machine would. Conversely, when discussing an attribution, 
\emph{who are we considering the explanation for?}. As we mentioned previously, saliency is not 
well aligned for humans and machines, especially when machines derive their knowledge from context. 
Consequently, when evaluating the interpretable properties of an attribution, we find that usually 
those that provide the best results in terms of metrics are not usually the ones that a human 
would consider best. This question remains open up to this day.\\

\noindent Regarding evaluation of interpretability methods, in \autoref{rel:sub_evl} we made mention 
of current evaluation methodologies, but, several questions arise from them. To begin with, and                                                                                                          
relating to the previous paragraph, it is safe to assume that the quantitative metrics presented 
are taking into consideration relevance towards the classifier. In particular \emph{these 
methods are telling us which representation explains the best a given class for the model}. 
And yet, in this case we can also wonder about \emph{which class should we inquire about; 
groundtruth or predictions?}. In most applications researchers extract attributions and provide 
explanations for groundtruth objects. However, in real-world applications we expect models to 
predict incorrectly some instances in the task that they are given; these instances then require 
explanations for the class being predicted. This is crucial given Lipton's desiderdata of model 
interpretability. We usually care the most for the instances in which the model fails and we have 
to provide accountability for the effects these inferences provide.\\

\noindent In addition of this selection of instances to provide explanations for, quantitative 
evaluation suffers from another drawback: a lack of homogenenity in evaluation procedures. 
Complementary to the qualitative evaluation of attribution maps via visual inspection and human 
criteria, quantitative evaluation ought to be more robust, replicable and homogeneus. Yet, we 
observe variations in evaluation procedures across proposals, and while an effort to provide fair 
comparisons, complete insight on the behaviour of the proposed methodologies is not attained. 
To exemplify, with the introduction of \emph{objective evaluation for object recognition} in Grad-
CAM++ \autocite{chattopadhay2018grad}, the evaluation procedure only required generation of 
visualizations for the entirety of the validation set of ImageNet and Pascal VOC 2012, whereas 
to assess the performance of Score-CAM \autocite{wang2020score}, a small subset of two thousand 
random images is chosen, thus negating replicability. We note that this is done to circumvent the 
discussion of the high computational cost required to compote Score-CAM attributions, which is  
not clearly discussed in its article. Complementary to this, 
\cite{chattopadhay2018grad} provides analysis for VGG-16, ResNet-50 and AlexNet, while Score-CAM 
presents results only for VGG.  This is not the only occurrence of this phenomenom, upon the 
introduction of methodologies such as Integrated Score-CAM \autocite{naidu2020cam}, Ablation-CAM
\autocite{ablationcam} and Layer-CAM \autocite{jiang2021layercam}, the evaluation procedures are 
found not to be standardized between approaches.\\

\noindent A last point we want to highlight is addressed towards claims of interpretability upon 
the proposal of models. To begin, while introducing image recognition models in 
\autoref{rel:sec_imrecon}, we found several approaches such as Conformer 
\autocite{peng2021conformer}, Scouter \autocite{li2021scouter} and LFI-CAM \autocite{lee2021lfi} 
claiming to produce high performance image recognition architectures with built-in interpretability 
properties. However, we note that these claims are sustained only with qualitative results in the 
shape of attribution maps such as \gls{cam} or attention visualization. Taking into consideration 
all of the points we have discussed so far in this section, we note that this exemplifies the 
challenges when discussing interpretability. On one hand, visually assessing the quality of 
explanation maps implies the alignment with human reasoning towards describing what is important 
within an image. Conversely, attribution maps are often shown mostly for groundtruth classes, not 
predictions, leaving in turn open the question of what is relevant for these instances. Lastly, 
qualitative measurements are not meaningful to claim the performance of an attribution compared 
with another, in particular given the misalignment between human and machine recognition processes.

%To begin with, 
%adhering to the lack of clarity on the definition of interpretability, we consequently find works 
%claiming to provide interpretability in their approaches but no quantitative results to support 
%them. As a result of this, we wonder around 


%\noindent In recent years, a breakthrough in image recognition models was achieved with the development of 
%the transformer architecture. In particular, it was observed that performance of \glspl{cnn} was 
%achieving a plateau, and the introduction of novel architectures stagnated over time. Additionally, 
%these models encountered challenges capturing long range dependencies, in turn  limiting their 
%capacity to construct global representations of data and affecting their generalization capabilities. 
%Transformers however, have shown remarkable improvements not only in the image 
%domain but also in language related tasks, revolutionizing these fields and paving the groundwork 
%for future research and developments.\\