\chapter{Literature Review: Image Recognition Models and Interpretability}
\label{ch:rel}
\chaptertoc{}
\section{Introduction}
The capacities of image recognition models have improved greatly over time, the constant 
development aligned with the increase in computing power has allowed these models to witness 
constant improvements in performance and complexity. Early approaches for this task were built 
using feature extraction methods in conjuction with traditional machine learning methods;
this hand crafted feautre extraction step ultimately limited these methodologies capacity to 
capture rich and intricate visual information. In particular, one approach that had a strong 
initial impact and high performance was that of \gls{hog} \autocite{dalal2005histograms}, 
where gradient information was used to train an \gls{svm} to perform pedestrian recognition. 
This approach while achieving high performance in the dataset it was designed, ultimatately 
failed in data collections where data complexity was higher \autocite{5975165}.\\
Furthermore, with the reappearance of convolutional neural networks and the ensuing advent of 
deep learning, a paradigm shift in this field occurred. In this change, it was observed that 
no longer were hand crafted features needed to be extracted; deep learning allowed CNNs to act 
as both feature extractor and classifiers on themselves. Moreover, based on the aggregation of 
convolutions as their fundamental units, a \gls{cnn} is able to learn hierarchichal 
representations of data, extracting intricate features directly from the data. 
This led to substantial improvements in accuracy and performance in various image recognition 
tasks. In \autoref{rel:sub_background} we explore some of the most important early approaches based 
on machine learning designed towards image recognition. Additionally in \autoref{rel:sub_cnn}, we 
introduce the basis of deep learning and CNNs. Moving on, in \autoref{rel:sub_att} we make mention 
of the Transformer architecture, its basis and how it has reshaped the landscape of image 
recognition. Finally in \autoref{rel:sub_hybrid} we display approaches that make use of combinations 
of the last two forementioned approaches.

\noindent Over time and tied to the increasing compute of power (\cite{schaller1997moore}, 
\cite{theis2017end}), and the improving performance of these models; their adoption into society 
and real world applications has taken place. 

\noindent Despite their remarkable performance, the interpretability of these deep learning models 
became a growing concern. The inherent complexity of deep neural networks often made it 
challenging to understand how these models arrive at their predictions. This lack of 
interpretability posed significant hurdles, especially in critical applications where 
understanding the model's decision-making process is crucial, such as in healthcare or 
autonomous systems.
To address this issue, researchers delved into enhancing the interpretability of image recognition 
models. They explored various techniques and methodologies aimed at making these models more 
transparent and understandable. Methods such as saliency maps, class activation maps, gradient-based 
attribution methods, and attention mechanisms were developed to provide insights into the model's 
decision-making processes and highlight regions of importance within images.

The quest for interpretability in image recognition models continues as researchers strive to 
strike a balance between model complexity and transparency. The ongoing efforts focus on developing 
models that not only achieve high accuracy but also offer explanations and insights into their 
reasoning, fostering trust and understanding in their applications across diverse domains.
%\addcontentsline{toc}{chapter}{Literature Review}
%--------------------------------------------------------------------------------------------------
\section*{Image Recognition Models}
\addcontentsline{toc}{section}{Image Recognition Models}
\label{rel:sec_imrecon}

\subsection*{Convolutional Neural Networks}
\label{rel:sub_cnn}
%\addcontentsline{toc}{subsection}{Convolutional Neural Networks}

\subsection*{Attention-Based Architectures}
\label{rel:sub_att}
%\addcontentsline{toc}{subsection}{Attention-Based Architectures}
Attention is a powerful mechanism that has been introduced into convolutional networks in several 
ocasions \autocite{bello2019attention, ramachandran2019stand, shen2020global}. With the success of 
vision transformers (ViT) \autocite{dosovitskiy2020image}, fully attention-based architectures are now 
competitive with convolutional networks. To benefit from both self-attention and convolutional 
layers, some hybrid architectures employ convolutional layers before the vision transformer 
\autocite{graham2021levit,xiao2021early}. Others, such as Swin \autocite{liu2021swin} and PiT 
\autocite{heo2021rethinking}, introduces a pyramid structure to share local spatial information 
while reducing the spatial resolution, as in convolutional networks. 

SCOUTER \autocite{li2021scouter} uses slot attention \autocite{locatello2020object} to build a class 
specific pooling mechanism, which does not scale well to more than 20 classes. 

\subsection*{Hybrid Architectures}
Conformer \autocite{peng2021conformer} proposes a dual network structure to retrain and fuse local 
convolutional features with global representations. Our method merely provides a simple 
attention-based pooling mechanism inspired by transformers that can work with any architecture, 
even pretrained and frozen. PatchConvNet \autocite{touvron2021augmenting} replaces global average pooling by an 
attention-based pooling layer.
\label{rel:sub_hybrid}
\newpage
%--------------------------------------------------------------------------------------------------
\section{Interpretability}
\label{rel:sec_int}
As deep learning based models for Computer Vision have continued to improve in their recognition 
properties, their structure and functioning have become more opaque; in turn making these 
technologies seen as black boxes. A black-box model is defined as a model for which its 
interpretation is not straightforward for humans \autocite{petch2022opening}. In recent years 
with the assimilation of deeplearning into everyday tasks, and the implicit effect these models 
are having on human lives; the novel research field of \emph{Interpretability} has been brought 
forth to open up this black-box behaviour. This research field investigated alongside two 
directions(\cite{mythos_interp}, \cite{guidotti2018survey}, \cite{zhang2021survey}):

\begin{enumerate}
	\item \emph{Post-hoc interpretability} considers the model as a black-box and provides 
	explanations based on inputs and outputs, without modifying the model or its training process.
	\item \emph{Transparency} modifies the model or the training process to better explain the 
	behavior of the inner parts of the model.
\end{enumerate}

\subsection{Transparency}
\label{rel:sub_transp}
Transparent 
%Transparency in interpretability research represents a pivotal pursuit within the broader 
%landscape of artificial intelligence. In the ever-evolving field of machine learning, 
%particularly deep neural networks, the inherent complexity of models often gives rise to a lack of 
%transparency, making it challenging to comprehend the decision-making processes. Researchers 
%focusing on transparency in interpretability seek to address this opacity by developing 
%methodologies and frameworks that not only reveal how models arrive at specific outcomes but 
%also provide a comprehensive understanding of the underlying mechanisms. The ultimate goal is to 
%empower users, stakeholders, and policymakers with the knowledge necessary to trust, validate, and 
%navigate the applications of artificial intelligence effectively. This emphasis on transparency not 
%only enhances the accountability of AI systems but also contributes to broader societal acceptance 
%and ethical deployment of these powerful technologies.
Approaches are grouped in a number of categories according to the type of the given explanation. \\
\emph{Rule-based methods} (\cite{wu2018beyond}, \cite{wu2020regional}) train a decision tree as a 
surrogate regularization term to force a network to be easily approximated by a decision tree.
% or learn a set of logit rules as an explanation~\citep{azzolin2022global}.
\emph{Hidden semantics-based methods} (\cite{bau2017network}, \cite{zhou2018interpreting}, 
\cite{zhang2018interpretable}, \cite{zhou2014object},\cite{bohle2022b}. \cite{bohle2024b}) aim to 
make a convolutional network learn disentangled hidden semantics with hierarchical structure or 
object-level concepts.\\
\emph{Prototype-based methods} (\cite{li2018deep}, \cite{chen2019looks}) learn a set of prototypes 
or parts as an intermediate representation in the network, which can be aligned with categories. 
\emph{Attribution-based methods} (\cite{ismail2021improving}, \cite{Zhou_2022_BMVC}, \\
\cite{ross2017right}, \cite{ghaeini2019saliency}) usually modify the architecture of a network or 
the training process to help post-hoc methods produce better saliency maps. Unlike 
(\cite{ross2017right}, \cite{ghaeini2019saliency}), saliency guided localization 
\autocite{Zhou_2022_BMVC} does not need ground truth explanations but replaces them with 
information bottleneck attribution \autocite{schulz2020restricting}. Finally, saliency-guided 
training \autocite{ismail2021improving} minimizes the KL divergence between the output of original 
and masked images.

\subsection{Post-Hoc Interpretability}
\label{rel:sub_post}

Approaches can be grouped into a number of possibly overlapping categories. 
\emph{Gradient-based methods} (\cite{adebayo2018local}, \cite{guidedbackprop}, 
\cite{baehrens2010explain}, \cite{simonyan2013deep}, \cite{smilkov2017smoothgrad}, 
\cite{bach2015pixel}, \cite{sundararajan2017axiomatic}) use the gradient of a target class score 
with respect to the input to compute the contribution of different input regions to the prediction. 
\emph{CAM-based methods} (\cite{wang2020score}, 
\cite{chattopadhay2018grad}, \cite{selvaraju2017grad}, 
\cite{axiombased}, \cite{jiang2021layercam}, \cite{ablationcam}) compute saliency maps 
as a linear combination of feature maps, with different definitions for the weights. 

\emph{Occlusion or masking-based methods} (\cite{petsiuk2018rise}, \cite{fong2017interpretable}, 
\cite{fong2019understanding}, \cite{schulz2020restricting}, \cite{ribeiro2016should}) apply a 
number of candidate masks in the input space, measure their effect on the prediction and then 
combine them into a saliency map. Masking in feature space has been explored too 
\autocite{schulz2020restricting}. 

Finally, \emph{learning-based methods} (\cite{chang2018explaining}, \cite{dabkowski2017real}, 
\cite{phang2020investigating}, \cite{zolna2020classifier}, \cite{schulz2020restricting}) learn an 
additional network or branch on extra data to produce an explanation map for a given input. 

\paragraph{Notation}
\label{sec:oc_notation}

Consider a classifier network $:fx \cX \to \real^C$ that maps an input image $\mathbf{u} \in \cX$ to a 
logit vector $\vy = f(\mathbf{u}) \in \real^C$, where $\cX$ is the image space and $C$ is the number 
of classes. We denote by $y_c = f(\mathbf{u})_c$ the predicted logit and by $p_c = \softmax(\vy)_c 
\defn e^{y_c} / \sum_j e^{y_j}$ the predicted probability for class $c$. For layer $\ell$ 
with $K_\ell$ channels, we denote by $A^k_\ell = f^k_\ell(\mathbf{u}) \in \real^{h_\ell \times w_\ell}$ 
the feature map for channel $k \in \{1,\dots,K_\ell\}$, with spatial resolution $h_\ell \times 
w_\ell$. Because of $\relu$ non-linearities, we assume that feature maps are non-negative. 
Similarly, we denote by $S_\ell \in \real^{h_\ell \times w_\ell}$ a 2D saliency map.

%--------------------------------------------------------------------------------------------------

\paragraph{Background: CAM-based saliency maps}
\label{sec:oc_back}

Given a layer $\ell$ and a class of interest $c$, we consider saliency maps given by the general 
formula
\begin{equation}
	S^c_\ell(\mathbf{u}) \defn h \left( \sum_k w^c_k A^k_\ell \right),
\label{eq:sal}
\end{equation}
where $w^c_k$ are weights defining a linear combination over channels and $h$ is an activation 
function. CAM \parencite{zhou2016learning} is defined for the last layer $L$ only with $h$ being the 
identity mapping and $w^c_k$ being the classifier weight connecting the $k$-th channel with 
class $c$. Grad-CAM \parencite{selvaraju2017grad} is defined for any layer $\ell$ with $h = \relu$ and 
weights
\begin{equation}
	w^c_k \defn \gap \left( \pder{y_c}{A^k_\ell} \right),
\label{eq:gcam}
\end{equation}
where $\gap$ is global average pooling.
% and $\softmax(\vy)_c = e^{y_c} / \sum_j e^{y_j}$ is the predicted probability of class $c$.
The motivation for $\relu$ is that we are only interested in features that have a positive effect 
on the class of interest, \ie pixels whose intensity should be increased in order to increase $y_c$.

Score-CAM \parencite{wang2020score} is also defined for any layer $\ell$ with $h = \relu$ and weights 
$w^c_k \defn \softmax(\mathbf{u}^c)_k$.  Softmax normalization considers positive channel contributions 
only and attends to few feature maps.
%that \alert{produce less highlighted areas in the saliency map}. \iavr{Last part unclear.}
Here, vector $\mathbf{u}^c \in \real^{K_\ell}$ measures the increase in confidence for class $c$ that 
compares a known baseline image $\mathbf{u}_b$ with the input image $\mathbf{u}$ masked according to feature 
map $A^k_\ell$, for all channels $k$:

\begin{equation}
	u^c_k \defn f(\mathbf{u} \odot n(\operatorname{up}( A^k_\ell )))_c - f(\mathbf{u}_b)_c,
\label{eq:s-cam}
\end{equation}

where $\odot$ is the Hadamard product. For this to work, the feature map $A^k_\ell$ is adapted
 to $\mathbf{u}$ first$:\operatorname{up}$ denotes upsampling to the spatial resolution of $\mathbf{u}$ and

\begin{equation}
	n(A) \defn \frac{A - \min A}{\max A - \min A}
\label{eq:norm}
\end{equation}

is a normalization of matrix $A$ into $[0,1]$. While Score-CAM does not need gradients, 
it requires as many forward passes through the network as the number of channels in the chosen layer,
 which is computationally expensive.

%--------------------------------------------------------------------------------------------------

\paragraph{Motivation}
\label{sec:motiv}

\iavr{Score-CAM considers each feature map as a mask in isolation. How about linear combinations?} 
Given a vector $\vw \in \real^{K_\ell}$ with $w_k$ its $k$-th element, let
\begin{equation}
	F(\vw) \defn f \left( \mathbf{u} \odot n \left( \operatorname{up} \left(
		\displaystyle\sum_k w_k A^k_\ell
	\right) \right) \right)_c.
\label{eq:s-obj}
\end{equation}
\ronan{If we assume that $\mathbf{u}_b = \vzero$ in~\eq{s-cam} and define $n(\vzero) \defn \vzero$ 
in~\eq{norm}, then we can rewrite the right-hand side of~\eq{s-cam} as
\begin{equation}
	\frac{F(\vw_0 + \delta \ve_k) - F(\vw_0)}{\delta},
\label{eq:s-cam2}
\end{equation}
where $\vw_0 = \vzero$, $\delta = 1$ and $\ve_k$ is the $k$-th standard basis vector of 
$\real^{K_\ell}$. This resembles the numerical approximation of the derivative $\pder{F}{w_k}(\vw_0)$,
 except that $\delta$ is not small as usual. One could compute derivatives efficiently by 
 standard backpropagation instead. It is then possible to iteratively optimize $F$ with respect
  to $\vw$, starting at any $\vw_0$.}

\iavr{As an alternative, consider masking-based methods relying on optimization in the input space, 
like \emph{meaningful perturbations} (MP) \parencite{fong2017interpretable} or 
\emph{extremal perturbations} \parencite{fong2019understanding}. In general, optimization takes the form
\begin{equation}
	S^c(\mathbf{u}) \defn \arg\max_{\vm \in \cM} f(\mathbf{u} \odot n(\operatorname{up}(\vm)))_c + \lambda R(\vm).
\label{eq:mask}
\end{equation}
Here, a mask $\vm$ is directly optimized and does not rely on feature maps, hence the saliency 
map $S^x(\mathbf{u})$ is not connected to any layer $\ell$. The mask is at the same or lower resolution 
than the input image. In the latter case, upsampling is still necessary.

In this approach, one indeed computes derivatives by backpropagation and indeed iteratively 
optimizes $\vm$. However, because $\vm$ is high-dimensional, there are constraints expressed by 
$\vm \in \cM$, \eg $\vm$ has a certain norm, and regularizers like $R(\vm)$, \eg $\vm$ is smooth in a 
certain way. This makes optimization harder or more expensive and introduces more hyperparameters 
like $\lambda$. One could simply constrain $\vm$ to lie in the linear span of $\{A_\ell^k\}_{k=1}
^{K_\ell}$ instead, like all CAM-based methods.}

\subsection{Discussion}

\newpage
%--------------------------------------------------------------------------------------------------
\subsection{Evaluating Interpretability}
\label{rel:sub_evl}
\paragraph{Classification Metrics}
\gls{ad} and \gls{ai} \autocite{chattopadhay2018grad} are 
well-established classification metrics. 
They measure the effect on the predicted class probabilities by masking the input image with the
 saliency map. Let $p^c_i$ and $o^c_i$ be the predicted probability for class $c$ given as input 
 the $i$-th test image $\vx_i$ and its masked version respectively. 
Masking refers to element-wise multiplication with the saliency map, which is at the same 
resolution as the original image with values in $[0,1]$. 
Let $N$ be the number of test images. Class $c$ is taken as the ground truth.

\emph{Average drop} (\gls{ad}) quantifies how much predictive power, measured as class probability, 
is lost when we only mask the image; lower is better:
\begin{equation}
	\AD(\%) \defn \frac{1}{N} \sum_{i=1}^N \frac{[p^c_i - o^c_i]_+}{p^c_i} \cdot 100.
\label{eq:ad}
\end{equation}

\emph{Average increase} (\gls{ai}), also known as \emph{increase in confidence}, measures the 
percentage of images where the masked image yiel ds a higher class probability than the original; 
higher is better:
\begin{equation}
	\AI(\%) \defn \frac{1}{N} \sum_i^N \mathbb{1}_{p^c_i < o^c_i} \cdot 100.
\label{eq:ai}
\end{equation}

$\AD$ and $\AI$ are not defined in a symmetric way. $\AD$ measures changes in class probability 
whereas $\AI$ measures a percentage of images. It is possible that the percentage is high while 
the actual increase is small. Hence, it is possible that an attribution method improves both. 
Indeed, \autocite{poppi2021revisiting} observes that a trivial method called Fake-CAM outperforms 
state-of-the-art methods, including Score-CAM, by a large margin. Fake-CAM simply defines a 
saliency map where the top-left pixel is set to zero and is uniform elsewhere. 
This questions the purpose of $\AD$ and $\AI$.

%--------------------------------------------------------------------------------------------------
\paragraph{Localization metrics}
\label{sec:loc-metrics}
Several works measure the localization ability of saliency maps, using metrics from the 
\emph{weakly-supervised object localization} (WSOL) task.

We are given the saliency map $S^c$ obtained from test image $\vx$ for ground truth class $c$. 
We denote by $S^c_{\vp}$ its value at pixel $\vp$. We binarize the saliency map by thresholding at 
its average value and we take the bounding box of the largest connected component of the resulting 
mask as the predicted bounding box $B_p$, represented as a set of pixels. We compare this box 
against the set of ground truth bounding boxes $\cB$, which typically contains 1 or 2 boxes of the 
same class $c$, or with their union $U = \cup \cB$, again represented as a set of pixels. We also 
compare the predicted class label $c_p$ with the ground truth label $c$. All metrics take values in 
$[0,1]$ and are expressed as percentages, except SM~\eq{sm}, which is unbounded.

\emph{Official Metric (OM)}
measures the maximum overlap of the predicted bounding box with any ground truth bounding box, 
requiring that the predicted class label is correct:
\begin{equation}
	\OM \defn 1 - \paren{\max_{B \in \cB} \iou(B, B_p)} \mathbbm{1}_{c_p = c},
\label{eq:om}
\end{equation}
where $\iou$ is intersection over union.
% is defined as $\iou(B, B_p) \defn \frac{B \cap B_p}{B \cup B_p}$.

\emph{Localization Error (LE)} is similar but ignores the predicted class label:
\begin{equation}
	\LE \defn 1 - \max_{B \in \cB} \iou(B, B_p).
\label{eq:le}
\end{equation}

\emph{Pixel-wise $F_1$ score (F1)} is defined as $F_1 = 2 \frac{P R}{P + R}$, where 
\emph{precision} $P$ is the fraction of mass of the saliency map that is within the ground truth 
union
\begin{equation}
	P \defn \frac{\sum_{\vp \in U} S^c_{\vp}}{\sum_{\vp} S^c_{\vp}}
\label{eq:prec}
\end{equation}

and \emph{recall} $R$ is the fraction of the ground truth union that is covered by the saliency map
\begin{equation}
	R \defn \frac{\sum_{\vp \in U} S^c_{\vp}}{\card{U}}.
	\label{eq:rec}
\end{equation}

\emph{Box Accuracy (BA)\autocite{choe2020evaluating}} Given threshold values $\eta$ and $\delta$, 
we find the bounding box $B^\eta_p$ of the largest connected component of the binary mask 
$\set{\vp: S_{\vp} > \eta}$ and require that it overlaps by 
$\delta$ with at least one ground truth box:
\begin{equation}
	\BA(\eta, \delta) \defn \max_{B \in \cB} \mathbbm{1}_{\iou(B^\eta_p, B) \ge \delta}.
\label{eq:ba}
\end{equation}
After averaging over the test images, we take the maximum of this measure over a set of values 
$\eta$ and then the average over a set of values $\delta$.\\
%--------------------------------------------------------------------------------------------------
\emph{Standard Pointing game (SP)\autocite{zhang2018top}} We find the pixel 
$\vp^* \defn \arg\max_{\vp} S^c_{\vp}$ having the maximum saliency value and 
require that it lands in any of the ground truth bounding boxes:
\begin{equation}
	\spg \defn \mathbbm{1}_{\vp^* \in U}.
\label{eq:spg}
\end{equation}\\

\emph{Energy Pointing game (EP)\autocite{wang2020score}} is equivalent to precision~\eq{prec}.\\

\emph{Saliency Metric (SM)\autocite{dabkowski2017real}} penalizes the size of the predicted bounding
 box $B_p$ relative to the image and the cross-entropy
 loss:
\begin{equation}
	\SM \defn \log \max\paren{ 0.05, \frac{\card{B_p}}{hw} } - \log p^c,
\label{eq:sm}
\end{equation}
where $h \times w$ is the input image resolution and $p^c$ is the precicted probability for ground 
truth class label $c$.

\subsection{Discussion}