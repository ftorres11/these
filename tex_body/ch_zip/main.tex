\chapter{Zero Information in Interpretability}
\chaptertoc{}
\label{ch:zip}
\section{Introduction}
% -------------------------------------------------------------------------------------------------
\noindent Whilst classical approaches for interpretability typically generate attribution maps 
highlighting the most salient regions in an image, alternative approaches  rely on 
feature occlusion. In this variant, the importance of a specific image region is obtained by 
its masking and the ensuing measurement of loss in predictive power (\cite{ribeiro2016should}, 
\cite{dabkowski2017real}, \cite{ablationcam}). An extension of this occlusion-based experimentation
involves a complete masking of the image as a baseline point which can be considered as devoid of 
information (\cite{shrikumar2017just}, \cite{shrikumar2019learning}).

\noindent Interestingly, masking techniques are not only employed for attribution generation but also 
in the development of heuristic thecniques. These methodologies introduce slight perturbations in 
various forms, such as minor noise (\cite{hsieh2021evaluations}), blurry or black masking 
(\cite{petsiuk2018rise}) addition of random noise (\cite{haug2021baselines}) or the introduction 
of maximum distance values for predefined pixels (\cite{sturmfels2020visualizing}). 
Despite efforts to categorize these proposals within different attribution methods 
(\cite{kindermans2017unreliability}), however still, a lack of consistency has been observed in 
the outcomes for each method. Overall there is not a criterion to demonstrate which method provides 
the best outcomes.

%\noindent Upon further scrutinizing upon heuristic techniques, 
%Concerns swiftly emerged regarding the validity of such heuristic filling techniques 
%\citep{hsieh2021evaluations, kindermans2017unreliability}, while perturbation methods failed to 
%yield notable results. These methods prompted critical issues that captured researchers' attention, 
%particularly the challenge of introducing bias to the model's decision [TODO] and the problem of 
%Out-of-Distribution data (OOD) \citep{gomez2022metrics, hase2021outofdistribution, qiu2021resisting}. 
%The first issue involves the potential introduction of bias when replacing one value with another, 
%subsequently affecting the model's decision-making process. When the number of masked features increases,
% this bias can escalate, substantially influencing the model's outcomes. The second issue revolves around 
%the construction of images that lie outside the data distribution in which the model was trained on. A 
%problem which is referred as ``Out of distribution" (OOD) problem in bibliography \citep{qiu2021resisting, 
%janzing20a}. The filled images could feature regions obscured by blurring, blackening, or 
%random noise—elements not encountered during the model's training. As a consequence, a change in the 
%model's decision might be attributed to its incapability to effectively assess such uncharted regions 
%within the data distribution. While various methods have attempted to mitigate the challenge of 
%OOD data by introducing new filling techniques \citep{jethani2021learned, yoon2018invase, chen2018learning}, 
%to the best of our knowledge, none of the methods developed thus far have succeeded in simultaneously 
%addressing both problems. The pursuit of a robust technique to effectively obscure information within an 
%image remains an ongoing challenge.

%Our focus lies in highlighting the issue of concealing information within images and scrutinizing the 
%repercussions of an indiscriminate filling approach. We contend that a robust filling method should obscure
% image segments in a manner that the concealed parts impart no information to the model, while the visible 
%segments maintain an equivalent contribution to the model's decision. We translate those criteria into a loss 
%function and employ an optimization algorithm that simultaneously adheres to both these principles, 
%all while respecting the data distribution. \newline
\section{ZIP definition}
\label{sec:zip_algo}

\newpage
\section{Extension: Insertion-Deletion}
\label{sec:zip_insdel}

\newpage
\section{Qualitative Evaluation}
\label{sec:zip_qual}

\newpage
\section{Benchmark}
\label{sec:zip_benchmark}


%Classical methods in the field of XAI construct an attribution map for each input, which calculates 
%the importance of individual features \citep{sundararajan2017axiomatic, selvaraju2017gradcam, 
%Chattopadhay_2018, Wang_2020_CVPR_Workshops, lundberg2017unified, lrp}. While different methodologies 
%have emerged \citep{BARREDOARRIETA202082}, several promising approaches rely on feature occlusion, 
%assessing the importance of a specific input component by masking its information and quantifying 
%the resulting decline in the model's prediction score \citep{9093360, ribeiro2016why, 
%lundberg2017unified, jung2021better, novello2022making, dabkowski2017real}. Certain methods extend 
%this occlusion to cover the entire input, hence defining a ``baseline" point 
%\footnote{Also defined as ``reference point" \citep{shrikumar2017just, shrikumar2019learning}  or 
%``root point" \citep{MONTAVON2017211}} which is considered to be devoid of any information. In order 
%to occlude information contained in a subset of features, researchers in Computer Vision adopted 
%different heuristic techniques. These techniques encompass subtle perturbations, such as introducing 
%minor random noise—an approach that has given rise to perturbation methods \citep{hsieh2021evaluations}, 
%or applied more intensive interventions like blackening or blurring the concealed image parts, 
%substituting them with random noise, or introducing maximum distance values based on predefined 
%pixels \citep{haug2021baselines, sturmfels2020visualizing}. These assorted heuristic strategies 
%have been systematically categorized \citep{haug2021baselines} and applied across different 
%attribution methods \citep{sturmfels2020visualizing, kindermans2017unreliability}. Nonetheless, 
%researchers have noted a lack of consistency in the outcomes produced by each method, with 
%none demonstrating a decisive advantage over others. 