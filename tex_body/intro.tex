%--------------------------------------------------------------------------------------------------
\addchap{Introduction}
%\addcontentsline{toc}{chapter}{Introduction}


\paragraph{Visual Recognition}  Over the past decade great advancements were made on computer 
vision. In particular with the optimization and popularization of \glspl{gpu}, models requiring 
a strong computational power became accesible for researchers. To be specific, the framework 
developed by Yann LeCun (\cite{lecun1998gradient}) got revitalized in 2012 with the introduction of 
AlexNet (\cite{krizhevsky2012imagenet}) where proper leverage of these machines outclassed that of 
more traditional machine learning models, Chapter \ref{ch:rel} discusses this in more depth. 
In this short span of time, several groundbreaking architectures have been proposed, in 
particular the ResNet family (\cite{he2016deep}) has remained relevant given its properties
(\cite{wightman2021resnet}); furthermore, with the introduction of the transformer architecture 
(\cite{vaswani2017attention}) this field of research recieved a new impulse and more powerful 
models based upon its functional unit are being brought forth.\\

\noindent It is not only with the increase of computational power that computer vision has improved over time. 
With the developement, popularization and spread of the internet; large collections of data have been 
formed. These aggregations can be extremely specific for a given end, or 
quite general representing the common interests of its users. Over time, these compilations have 
continued to grow both in volume and variety; still, several curated collections are introduced by 
researchers to experiment and control the development of models such as MNIST (\cite{lecun1998gradient}),
BSDS (\cite{MartinFTM01}), Pascal VOC (\cite{pascal-voc-2012}) and most notably, 
ImageNet (\cite{ILSVRC15}) and MS-COCO (\cite{lin2014microsoft}). Additionally, data collection is
an ongoing and a never ending process; as such, the idea of a dataset containing all types of 
information is no longer deemded a dream but a reality that might come true aided by Big Data in 
the foreseeable future \autocite{chen2014big}.\\

\noindent Taking into consideration the aforementioned  increase on  both compute power and data 
availability, deep learning based models have been steadily adopted and assimiliated within society; 
nowadays its no longer so much a question \textit{whether can a model achieve a given task}, but 
rather a question on \textit{how can this given model perform this task}. Providing an answer to 
this question is paramount as human lives are now being directly affected by such kind of models. 
The main issue behind understanding deep models, lies within the size and complexity of deep 
architecttures, where providing interpretable explanations has lead to the surge of a novel field 
of research (\cite{guidotti2018survey}, \cite{bodria2021benchmarking}).

\paragraph{Interpretability}
To begin discussing about interpretability, one must ponder around its definition. 
Over the last decade, many authors have attempted to address to this question. One 
of the most notable discussions can be found within \emph{The Mythos of Model Interpretability} 
(\cite{mythos_interp}). In this work, Lipton argues that for a model to be interpretable it must 
display two properties, \emph{Transparency} and \emph{Post-Hoc Interpretability}. In one hand, 
\textit{Transparency} answers questions regarding the model structure, training and inference 
processes; while on another hand, \textit{Post-Hoc Interpretability} relates to the explanations 
and information that can be drawn of the model itself.

\noindent Considering these properties, we observe that as machine learning models grew in 
complexity; their transparent properties vanished proportionaly to their size. It can be argued 
that traditional models offer themselves to  transparency due to their straightforward formulation 
and inherent properties. Conversely, in terms of post-hoc interpretations, methods like decision 
trees \cite{ho1995random} can be pruned to study their performance by removing 
branches (\cite{lakkaraju2016interpretable},\cite{mothilal2020explaining}). 
On another hand, established techniques such as Principal Component Analysis \gls{pca} 
(\cite{wold1987principal}), can be used to gain insight within data leading to a prediction.\\

\noindent When studying deep models, we find that it is after their size and complexity that their 
interpretable propierties get hindered. Common  \glspl{cnn} 
rely convolution as their corner stone, coupled with non-linear operations such as 
ReLU (\cite{fukushima1975cognitron}), Sigmoid, and Softmax (\cite{hopfield1985neural}) among others.
This aggregation of convolutions on one hand enables these models to process large quantities of 
data, and to a certain extent generalize; however, it also results in an extensive parameter count,
often reaching of millions, and most recently, even billions (\cite{openai_compute}). The 
computational load required for inference, typically measured in \gls{gflops} further compounds 
complexity.\\

\noindent Among the properties proposed by Lipton, we observe that offering model transparency is a 
rather challenging task. While understanding the behaviour of convolutions and 
self-attention might seem straightforward, it is their aggregation and subsequent flow of 
information what makes this process intricate. Moreover, transparency encompases aspects related to 
both inference and training. In this regard, to understand the behaviour and functioning of a model, 
it is possible to introduce modifications: attempting  to simplify it (\cite{wu2018beyond}, 
\cite{wu2020regional}), to improve and make sense of the semantic information in deep layers 
(\cite{bau2017network}, \cite{zhou2018interpreting}), to provide prototypes describing the learned
categories (\cite{li2018deep}, \cite{chen2019looks}, \cite{rymarczyk2022interpretable}), and to 
improve the quality of information contained within the network, thereby improving interpretable 
properties from other approximations (\cite{ismail2021improving}, \cite{Zhou_2022_BMVC}, 
\cite{ross2017right}). Nevertheless, these modifications included during the training process may 
modify the built-in properties of the models they are attempting to interpret, all the while 
including another level of uncertainty amongst them.\\

In contrast to transparency, providing post-hoc interpretations from deep models is a 
thriving field, where many of the challenges found within transparency are no longer found.
Within this field, various approaches have been proposed to achieve post-hoc interpretability,
including input masking (\cite{petsiuk2018rise}), attribution generation (\cite{NIPS2017_7062}, 
\cite{zhou2016learning}) and model perturbations 
(\cite{fong2017interpretable}, \cite{fong2019understanding}). Furthermore, the development of 
evaluation methodologies for these approaches has continued in tandem with them 
(\cite{choe2020evaluating}, \cite{chattopadhay2018grad}). Chapter \ref{ch:rel} delves deeper into 
these works. 

\paragraph{Dissertation Outline}
%\addcontentsline{toc}{section}{Dissertation Outline}
\noindent This dissertation is organized in the following manner: In Chapter \ref{ch:rel} we 
introduce a background for image recognition models (Section \ref{rel:sec_imrecon}) and the ensuing 
approaches developed to study the interpretability on them (Section \ref{rel:sec_int}). 
Additionally, we introduce evaluation procedures for these approaches which will be further used to 
evaluate ourproposals.\\

\noindent In Chapter \ref{ch:opticam}, we propose Opti-CAM as a methodology that generates 
optimized saliency maps highlighting the relevant regions on an image towards image classification. 
In Section \ref{sec:av_gain} we extend existing evaluation metrics with a novel measurement for 
model coinfidence. 
In Sections \ref{sec:oc_qual} and \ref{sec:oc_quant} we evaluate the effect of these contributions 
towards interpretability assessment.\\

\noindent Chapter \ref{ch:castream} introduces the Cross Attention Stream, an approach that boosts existing 
architectures interpretable properties. We set up the modulus of this approach in 
Section \ref{sec:ca_defn} alongside its deployment on Section \ref{sec:ca_design}. 
In Sections \ref{sec:ca_qual} and \ref{sec:ca_quant} we demonstrate the benefits of using this
proposal.\\

\noindent Chapter \ref{ch:grad} characterizes a gradient denoising approach with a gradient denoising 
methodology as an approach to enhance the trainining procedure of current models while improving 
interpretability properties. In Section \ref{sec:grad_defn}, we define the gradient denoising 
protocol alongside the regularization proposals to do so.
Sections \ref{sec:grad_qual} and \ref{sec:grad_quant} illustrate the effects of this paradigmn
in the trained models and its effects on interpretability.\\

\noindent Chapter \ref{ch:zip} raises the Zero-Information algorithm and its usage as a substitute
for mask-dependent evaluation proposals. Section \ref{sec:zip_algo} develops this 
method. Section \ref{sec:zip_insdel} demonstrates its incorporation of this 
algorithm onto evaluation protocols. Section \ref{sec:zip_qual} displays
the effect of this approach when applied to mask patches on images. Section 
\ref{sec:zip_benchmark} displays the results of benchmarking these protocols 
with this approach. \\
    
\noindent Finally, we draw conclusions on our work and detail future research perspectives.