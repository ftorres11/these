\begin{figure}[t]
    \centering
    % \fig[.8]{method/OptiCAM-1.png}
    %------------------------------------------------------------------------------
    \begin{tikzpicture}[
        scale=.12,
        font={\small},
        node distance=.2,
        label distance=2pt,
        net/.style={draw,trapezium,trapezium angle=75,inner sep=3pt},
        enc/.style={net,shape border rotate=270},
        txt/.style={inner sep=3pt},
        frame/.style={draw,minimum size=1cm},
        feat/.style={frame},
        sq/.style={minimum size=.15cm},
        elem/.style={draw,sq},
        vec/.style={draw,minimum width=.8cm,minimum height=.15cm},
        var/.style={blue!60},
        B/.style={fill=blue!20},
        R/.style={fill=red!20},
        G/.style={fill=green!20},
        Y/.style={fill=yellow!40},
        P/.style={fill=black!20},
    ]
    \matrix[
        tight,row sep=0,column sep=14,
        cells={scale=.3,},
    ] {
        \&\&\&\&\&\&\&
    % 	\node[var,op] (error) {$-$}; \&
        \node[txt] (loss) {objective \\ $F^c_\ell(\vx; \mathbf{u})$}; \\
        \node[label=90:{input image $\vx$}] (in) {\figah[1.5cm]{opticam/images/idea/input}}; \&
        \node[enc] (net) {network \\ $f$}; \&
        \foreach \s/\c in {-2/B,-1/R,0/G,1/Y,2/P}
            {\node[feat,\c] (feat\s) at ($.4*(\s,-\s)$) {};}
        \node        at (feat2) {\figah[1cm]{opticam/images/idea/27_fea0}};
        \node[frame] at (feat2) {};
        \coordinate[label=90:{feature \\ maps $A^k_\ell$}]
                    (feat-north) at (feat-2.north -| feat0.north);
        \coordinate (feat-west)  at (feat-2.west  |- feat0.west);
        \coordinate (feat-east)  at (feat2.east   |- feat0.east);
        \&
        \node[var,op] (cam) {$\times$};
        \foreach \s/\c in {-2/B,-1/R,0/G,1/Y,2/P}
            {\node[elem,\c] (elem\s) at ($.6*(\s,-6)$) {};}
        \node[sq,label=-90:{weights $\mathbf{u}$}] (weight) at (elem0) {};
        \&
        \node[var,label=90:{saliency map \\ $S_\ell(\vx; \mathbf{u})$}] (sal) {\figah[1.5cm]{opticam/images/idea/saliency}}; \&
        \node[var,op] (mask) {$\odot$}; \&
        \node[label=90:{masked image}] (masked) {\figah[1.5cm]{opticam/images/idea/masked}}; \&
        \node[enc] (net2) {network \\ $f$}; \\[8]
        \&\&\&
        \coordinate (mid); \\
    };
    
    \draw[->]
        (in) edge (net)
        (net) edge (feat-west)
        (feat-east) edge (cam)
        (net2) edge node[pos=.5,right] {class \\ logits} (loss)
    % 	(net) |- node[pos=.3,left] {class \\ logits} (error)
        ;
    
    \draw[var,->]
        (weight) edge (cam)
        (cam) edge (sal)
        (sal) edge (mask)
        (mask) edge (masked)
        (masked) edge (net2)
    % 	(net2) edge (error)
    % 	(error) edge (loss)
        (net2) edge (loss)
        ;
    
    \draw[->]
        (in) |- (mid)
        (mid) -| (mask)
        ;
    
    \end{tikzpicture}

% \vspace{6pt}
\caption{}%Overview of Opti-CAM. We are given an input image $\vx$, a fixed network $f$, a target layer $\ell$ and a class of interest $c$. We extract the feature maps from layer $\ell$ and obtain a saliency map $S_\ell(\vx; \mathbf{u})$ by forming a convex combination of the feature maps ($\times$) with weights determined by a variable vector $\mathbf{u}$~\eq{v-sal}. After upsampling and normalizing, we element-wise multiply ($\odot$) the saliency map with the input image to form a ``masked'' version of the input, which is fed to $f$. The objective function $F^c_\ell(\vx; \mathbf{u})$ measures the logit of class $c$ for the masked image~\eq{obj}. We find the value of $\mathbf{u}^*$ that maximizes this logit by optimizing along the path highlighted in blue~\eq{opt}, as well as the corresponding optimal saliency map $S_\ell(\vx; \mathbf{u}^*)$~\eq{o-sal}.}
\label{fig:idea}
\vspace{-0.4cm}
\end{figure}