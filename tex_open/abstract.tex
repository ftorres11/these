%--------------------------------------------------------------------------------------------------
\chapter*{Abstract}
\addcontentsline{toc}{chapter}{Abstract}
Computer Vision applications using Artificial Intelligence technologies have undergone a remarkable 
evolution in the past decade. Current developments in Computer Vision are a direct result of a 
better utilization of hardware, enabling the construction of models capable of performing more 
complex tasks over time. On image recognition in particular, convolutional neural networks and 
transformers are now able to identify images and their elements, as well as assign semantic value 
to them, even in challenging conditions. These models have completely changed the landscape of deep 
learning and image recognition, permeating deep into society with the automatization of many daily 
tasks. Thus, it is no longer a question of \emph{can a model perform 
this task?}, but more a question of \emph{how does this model perform this task?}. To address these 
questions a new research field has emerged: interpretability and explainable AI.\\
\\
%Moreover, progress in this field has paved the way for further developments in complementary fields 
%of Computer Vision: models developed for image recognition are modified to conduct image 
%segmentation and reconstruction.\\

%\noindent The Deep Learning revolution occurred in two stages: first with the reemergence and 
%popularization of convolutional neural networks, and in second place with the 
%introduction of attention based architectures. On one hand, convolutional neural networks were 
%initially hard to train given their complexity and data volume requirements. 
%Still, early proposals showed promise, such as LeNet with the automatic recognition of handwritten 
%digits in the US postal service in the early 1990s. 
%However, in the early years of the last decade these issues were solved with the widespread of 
%large scale visual recognition datasets, as well as with the effective use of Graphics Processing 
%Units for computation. On the other hand, close to the end of this time period a new style of model 
%emerged: the Transformer architecture. This architecture creates an abstract representation of 
%lasses that interacts with every element of an encoding, generating a representation that learns a 
%class representation. Additionally, Transformers can process larger volumes of data, 
%allowing them to be robust to bias and generalize better.\\

%\noindent Convolutional Neural Networks and Transformers 
\noindent In this thesis, our goal is to understand and further develop interpretability models for 
state-of-the-art image recognition models. We introduce and briefly explain some of the most 
relevant high performance image recognition models for both Convolutional Neural Networks 
and Transformers. Then, current interpretability approaches designed to provide 
explanations, as well as their evaluation protocols. We make observations upon 
these methods and evaluation protocols, highlighting difficulties upon them and suggesting ideas 
to address their limitations.\\

\noindent In particular, a novel attribution method is proposed, ensuring that the highlighted 
regions in an image maximize a given class probability. From this method we observe that the 
most important information correlating to a class prediction, is not only found in the object of 
interest. Instead, context conveys salient information too: it is usually spread all over the image. 
Additionally, an addition to existing architectures is introduced, this method enhances predictive 
confidence, boosting interpretability properties of high performing image recognition models. Lastly, we 
propose a learning paradigm, that denoises gradients on the image space, enhancing 
interpretability properties. This final proposition demonstrates improvements on interpretability 
metrics for small datasets and models, while displaying promise for its scaling on to larger 
datasets and architectures.\\

\vspace{0.5cm}
Keywords: Deep Learning, image recognition, interpretability, explainability.

