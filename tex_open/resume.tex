%--------------------------------------------------------------------------------------------------
\chapter*{Résumé}
\addcontentsline{toc}{chapter}{Résumé}
Les capacités de vision par ordinateur se sont améliorées au cours de la dernière décennie, une 
meilleure utilisation du matériel permettant aux ordinateurs de traiter davantage d'images plus 
rapidement, entraînant l'avènement de l'apprentissage profond. De plus, au cours de cette période, 
des architectures de modèles telles que les réseaux de neurones convolutifs et les transformers 
ont été introduites, permettant aux applications de vision par ordinateur de réaliser des tâches 
plus complexes. En particulier, les modèles de reconnaissance d'images sont désormais capables 
d'identifier et de reconnaître des éléments sur une image, même dans des conditions difficiles. 
Ces facteurs ont contribué à l'introduction de ces modèles dans la société.\\

\noindent Avec la diffusion des technologies de l' apprentissage profond au sein de la société, 
une nouvelle exigence a émergé pour ces méthodologies. Puisqu 'elles interagissent désormais et 
affectent directement les vies humaines, il est impératif de comprendre leur fonctionnement et de 
fournir des explications. Pour répondre à ces questions, un nouveau domaine de recherche a vu le 
jour: l 'interprétabilité et l 'IA explicable.\\

\noindent Dans cette thèse, notre objectif est de comprendre et de développer des modèles 
d'interprétabilité pour les modèles de reconnaissance d'images de pointe. Nous présentons et 
expliquons brièvement certains des modèles de reconnaissance d'images les plus performants et 
pertinents pour les Réseaux de Neurones Convolutifs et les Transformers. Ensuite, nous examinons 
les approches actuelles en matière d' interpr\'etabilit\'e conçues pour fournir des explications, 
ainsi que leurs protocoles d' évaluation. Nous faisons des observations sur ces méthodes et 
protocoles d'évaluation, mettant en évidence les difficultés rencontrées et suggérant des idées 
pour surmonter leurs limitations.\\

\paragraph{Opti-CAM} Notre première contribution, s' appuie sur le raisonnement des Cartes d' 
Activation de Classe. En particulier, cette proposition optimise le coefficient de pondération 
requis pour calculer une carte de saillance, générant une représentation qui maximise la 
probabilité spécifique à la classe. Cette carte de saillance offre les meilleurs résultats selon 
les mesures d'interprétabilité, et met en évidence que le contexte est pertinent pour décrire une 
prédiction. De plus, une nouvelle métrique pour compléter l'évaluation de l'interprétabilité est 
dévoilée, remédiant aux lacunes de cette procédure.\\

\paragraph{Cross Attention Stream} Notre deuxième contribution, est un ajout aux modèles actuels de 
reconnaissance d'images, améliorant les mesures d'interprétabilité. Inspiré par des modèles 
novateurs performants tels que les Transformers, nous construisons un flux qui calcule l'interaction 
d'une représentation de classe abstraite avec les caractéristiques profondes des réseaux neuronaux 
convolutionnels. Cette représentation est finalement utilisée pour effectuer la classification. 
Notre flux affiche des améliorations lors de l'évaluation quantitative, tout en préservant les 
performances de reconnaissance à travers différents modèles.\\

\paragraph{Gradient Denoise} Enfin, notre dernière contribution présente un nouveau paradigme d' 
entraînement pour les réseaux neuronaux profonds. De plus, ce paradigme débruite les informations 
de gradient des modèles profonds dans l'espace d'entrée. La représentation de rétropropagation 
guidée de l'image d'entrée est utilisée pour régulariser les modèles lors de leur phase d' 
entraînement. En conséquence, nos modèles entraînés affichent des améliorations pour l' 
évaluation interprétable. Nous appliquons notre paradigme à de petites architectures dans un cadre 
contraint, ouvrant la voie au développement futur dans des ensembles de données à grande échelle, 
ainsi qu'avec des modèles plus complexes.\\

\vspace{0.5cm}
%Keywords: deep learning, image processing, attributions, interpretability
Mots clés: Apprentissage Profond, reconaissance d'image, interpretabilité, explicabilité.
