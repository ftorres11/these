%--------------------------------------------------------------------------------------------------
\chapter*{Résumé}
\addcontentsline{toc}{chapter}{Résumé}
Les applications de vision par ordinateur utilisant les technologies de l'intelligence artificielle 
ont connu une évolution remarquable au cours de la dernière décennie. Les développements actuels en 
vision par ordinateur sont le résultat direct d'une meilleure utilisation du matériel, permettant 
la construction de modèles capables d'effectuer des tâches plus complexes au fil du temps. En ce 
qui concerne la reconnaissance d'images en particulier, les réseaux neuronaux convolutionnels et 
les transformateurs sont désormais capables d'identifier les images et leurs éléments, ainsi que 
d'attribuer une valeur sémantique à ces derniers, même dans des conditions difficiles. Ces modèles 
ont complètement changé le paysage de l'apprentissage profond et de la reconnaissance d'images, en 
pénétrant profondément dans la société avec l'automatisation de nombreuses tâches quotidiennes. 
Ainsi, il ne s'agit plus de savoir si un modèle peut effectuer cette tâche, mais plutôt de savoir 
comment ce modèle effectue cette tâche. Pour répondre à ces questions, un nouveau domaine de 
recherche a émergé : l'interprétabilité et l'intelligence artificielle explicative.\\

\noindent Dans cette thèse, notre objectif est de comprendre et de développer des modèles 
d'interprétabilité pour les modèles de reconnaissance d'images de pointe. Nous présentons et 
expliquons brièvement certains des modèles de reconnaissance d'images les plus performants et 
pertinents pour les Réseaux de Neurones Convolutifs et les Transformers. Ensuite, nous examinons 
les approches actuelles en matière d' interpr\'etabilit\'e conçues pour fournir des explications, 
ainsi que leurs protocoles d' évaluation. Nous faisons des observations sur ces méthodes et 
protocoles d'évaluation, mettant en évidence les difficultés rencontrées et suggérant des idées 
pour surmonter leurs limitations.\\

\noindent Notre première contribution, Opti-CAM, s'appuie sur le raisonnement des Cartes d' 
Activation de Classe. En particulier, cette proposition optimise le coefficient de pondération 
requis pour calculer une carte de saillance, générant une représentation qui maximise la 
probabilité spécifique à la classe. Cette carte de saillance offre les meilleurs résultats selon 
les mesures d'interprétabilité, et met en évidence que le contexte est pertinent pour décrire une 
prédiction. De plus, une nouvelle métrique pour compléter l'évaluation de l'interprétabilité est 
dévoilée, remédiant aux lacunes de cette procédure.\\

\noindent Notre deuxième contribution, Cross Attention Stream, est un ajout aux modèles actuels de 
reconnaissance d'images, améliorant les mesures d'interprétabilité. Inspiré par des modèles 
novateurs performants tels que les Transformers, nous construisons un flux qui calcule l'interaction 
d'une représentation de classe abstraite avec les caractéristiques profondes des réseaux neuronaux 
convolutionnels. Cette représentation est finalement utilisée pour effectuer la classification. 
Notre flux affiche des améliorations lors de l'évaluation quantitative, tout en préservant les 
performances de reconnaissance à travers différents modèles.\\

\noindent Enfin, notre dernière contribution présente un nouveau paradigme d'entraînement pour les 
réseaux neuronaux profonds. De plus, ce paradigme débruite les informations de gradient des modèles 
profonds dans l'espace d'entrée. La représentation de rétropropagation guidée de l'image d'entrée 
est utilisée pour régulariser les modèles lors de leur phase d'entraînement. En conséquence, nos 
modèles entraînés affichent des améliorations pour l'évaluation interprétable. Nous appliquons 
notre paradigme à de petites architectures dans un cadre contraint, ouvrant la voie au 
développement futur dans des ensembles de données à grande échelle, ainsi qu'avec des modèles plus 
complexes.\\

\vspace{0.5cm}
%Keywords: deep learning, image processing, attributions, interpretability
Mots clés: Apprentissage Profond, reconaissance d'image, interpretabilité, explicabilité.
